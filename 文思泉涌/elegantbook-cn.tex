\documentclass[lang=cn,newtx,14pt,scheme=chinese]{elegantbook}

\title{文思泉涌:如何克服学术写作拖延症}
\subtitle{How to Write a Lot : A Practical Guide to Prod11ctive Academic Writing}

\author{保罗·J.席尔瓦}
\institute{wusir2018@163.com}

\extrainfo{注意:本书为重排版本,仅供个人学习使用!}

\setcounter{tocdepth}{3}

% \logo{logo-blue.png}
\cover{cover.png}

% 本文档命令
\usepackage{array}
\newcommand{\ccr}[1]{\makecell{{\color{#1}\rule{1cm}{1cm}}}}

% 修改标题页的橙色带
\definecolor{customcolor}{RGB}{32,178,170}
\colorlet{coverlinecolor}{customcolor}
\usepackage{cprotect}

% \addbibresource[location=local]{reference.bib} % 参考文献,不要删除

\begin{document}

\maketitle
\frontmatter


\chapter*{前言}
《文思泉涌:如何克服学术写作拖延症》(How to Write a Lot : A Practical Guide to Prod11ctive Academic Writing)不是一本学术著作——这是一本写给学术研究人员看的轻松愉悦的个性化实用指南。大学教授们闷声不响地埋头苦写——写作并非易事,而出版和课题申请的标准总是水涨船高;研究生们则为写作那点事儿大叹苦经——他们总是在为作业或毕业论文发愁,更可悲的是,给他们支招的教授们自己也在写作的泥潭中挣扎。许多人想法设法在忙碌的学期中为写作腾出时间,做好安排,一次次地直面批评和退稿的挑战,却还要强打精神。许多人始终找不到向学术期刊投稿、修改稿件或是与人合著的窍门。

成为一个多产的学术写作者,靠的是后天习得的技能,而不是先天的禀赋,所以,写作这件事是可以学的。这本小册子将告诉你怎样把写作变成一件稀松平常的事。它将告诉你怎样才能在正常的教学周内腾出时间写作,怎样才能在写作时放松心情,以及怎样才能更高效地写作。如果你有积压如山的原始数据,如果你总是担心找不到时间写东西,或者你想让写作变得容易些,那么这本小册子或许能够帮到你。

我很幸运,有一大帮愿意与我分享写作经验的同事,他们对我的不时打扰总是宽以待之 很多人参与了我的非正式调研活动,对本书的初稿提出了各种建议,对这个初看起未有点奇怪的选题给子了充分的肯定和鼓励。特别感谢韦斯利·阿兰 (Wesley Allan) 、珍妮特·波瑟夫斯基 (Janet Boscovski) 、彼得·德兰尼 (Peter Delancy) 、约翰·邓兰斯基 (John Dunlosky) 、迈克·肯特 (Mike Kane) 、汤姆·拿波 (Tom Kwap)、斯科特·劳伦斯(Scott Lawrence)、马克·赖瑞(Mark Leary)、切丽·罗根(Cheryl Logan)、斯图尔特·莫肯维奇(Stuart Marcovitch)、莉莉·萨哈简(Lili Sahakyan)、迈克·塞拉 (Mike Serra) 、里克·沙尔(Rick Shull) 、我的父亲雷蒙德·席尔瓦 (RaymondSilvia) 、杰姬·怀特 (Jackie White) 、贝蒂·温特施泰因(Bcate Winterstein) 、爱德华·维希涅夫斯基 (Ed Wisniewski) 和拉里·赖茨曼 (Larry Wrightsman)。还要特别感谢来自美国心理学会的兰辛·海斯 (Lansing Hays) 和琳达·麦卡特 (Linda McCartcr) ,是你们的不懈努力才使那份皱巴巴的初稿成为现在这本还挺像样的小册子。

正如斯蒂芬·金(Stephen King, 2000: 155) 曾经说过的:“一间作家的书房,真正需要的,仅仅是一扇你想要关上的门。”这本书献给贝蒂 (Bcatc) ,门那边的我的好朋友

\tableofcontents
\mainmatter

\chapter{导言}
《文思泉涌:如何克服学术写作拖延症》是一本关于如何成为一名深思熟虑且训练有素的写作者的书;它不是教你如何粗制滥造,为了积累成果而发表大量“学术垃圾",也不是教你如何故弄玄虚,把一篇干净利落的论文拉扯得又臭又长。大多数心理学家都希望能够写出更多的作品,他们也希望写作的过程能够不那么令人压力山大、负疚不已或忐忑不安。这本书是为他们而写的。我选择了一个实用的、行为主导的角度来讨论写作。我们不会讨论所谓不安全感、回避心理、防御性和人类内在的心理阻隔等阻碍写作进程的因素;我们也不会讨论培养新的技能,因为你已经掌握了所有能让自己更多产的基本技能,虽然你可能还需要进一步的实践;我们更不会说什么释放你“内在的”之类的话:你尽可用一根拴绳把你的“内在写作者”拴起来,最好再给它戴上一个口罩。

相反,我们会把注意力集中在你“外在”的写作者角色上。高效写作的要求其实很简单:制订时间表,设定目标,时时跟踪你的工作,奖励自己和养成良好的习惯。多产的写作者并没有什么特殊技能,他们只是花了更多的时间在写作上,同时,他们的效率更高而已 (Keyes, 2003)。改变你的写作习惯并不会使写作过程变得更有趣,但是会使它变得容易和轻松。

\section{写作的确是件难事}
你做研究的时候觉得很欢喜。做研究的过程有一种奇异的快感。提出一种观点并设法验证自己的观点令人感到满足。数据收集也是有趣的,尤其是当别人帮你完成的时候。甚至数据分析也挺可爱,看看研究是否站得住脚的确挺让人兴奋的。但是,写研究报告就毫无乐趣可言:写作是辛苦的、复杂的和无趣的。威廉·津泽(William Zinsscr, 2001: 12)说:“如果你觉得写作很难,那是因为写作的确很难。”你必须把复杂的理论、研究方法和数据分析都集中在一篇短小的科研论文中,这并不容易,尤其是当你意识到将来那些不知名的审稿人将对你的作品严加拷问,就像拍打一条落满灰尘的旧地毯。

正是因为收集数据比写作来得容易,许多教授都有堆积如山的研究数据。他们想着“总有一天”会发表这些数据,或者更准确地说,是“总有一年”,因为他们总是在为写作而纠结——教授们热切盼望着长周末、春假、法定假日和暑假。但是,每当长周末过后的周二,人们又总是嘟囔着抱怨自己只写了那么一点点。在规模稍大的系里,每个暑假过后的第一周,到处都可以听到吵吵嚷嚷的叹息和自责声。这种可悲的循环周而复始,人们又开始期待下一次长假。心理学家往往发现,在那些所谓周末、晚间或假期的“空余时段”,写作时间总是被其他更为重要的事情侵占,比如朋友聚会、家庭聚餐、炖锅扁豆汤或是给自家的狗织一顶圣诞帽。

与此同时,我们赶上了好时光,对作品的要求达到了前所未有的高度。越来越多的心理学家向越来越多的杂志寄送越来越多的稿件;越来越多的研究人员在互相争夺日益缩减的研究基金。院长和其他院系领导比以前更看重论文发表的数量。过去和颜悦色的教务长们对教员能够申请到研究基金总是颇感意外,备受鼓舞;而现在,他们拉长着脸,甚至希望新晋员工都能够申请到更多的研究经费。有些院系甚至把教员能否申请到经费与他们的晋升挂钩。在研究型大学,如果无法写出更多的论文,就无法升职或获得终身教职。甚至,在一些小型的教学型高校,对于论文发表的要求也日益提高。所以,这年头,要想在科研领域混口饭吃,真的不容易!


\section{我们现在学习写作的方式}
写作是一项技能,而不是什么天赋或特殊才能。所有的高级技能一样,写作技能必须通过系统的指导和实践来培养。人们必须学习相关的规则和策略,并努力实践 (Ericsson, Krampe \& Tesch-Romer, 1993)。心理学家早已发现,有意识的练习有助于技能培养,但是这一理论似乎还没被用于培养写作技能。我们来比较一下写作技能的教育和其他高级技能的教育。教学很难,所以我们有专门的研究生院教学生如何“教学”。学生们通常要先学习一门“教育心理学”,然后通过当助教来练习如何“教学”。很多研究生在研究生阶段的每个学期都做助教,而后才能成为一名合格的教师。统计和研究方法也很难,所以我们要求学生在高年级阶段不断学习这些内容,通常都由在方法论和统计方面富有经验的专家来讲授。通过多个学期的学习,学生们终于成为老练的研究方法高手。

那么,心理学是怎样训练学生学习写作的呢?最常见的模式是指望学生们能够通过向他们的导师学习来掌握写作的技能。问题是,很多导师自己就在“水深火热”中挣扎——他们常常抱怨根本没有时间写作,常常眼巴巴地等待春假或是暑假的来临 这简直就是盲人骑瞎马。但是,这并不是他们的错:正如很多学生所说,大多数教授也都是在“摸爬滚打”中学习写作的。有些系的确开设了写作课,但是这些课程往往忽视了写作动因方面的问题,转而关注教授如何写课题申请报告或是其他各类报告。

研究生毕业以后,就再也没有导师会对学生们才完成了一半的论文给予指导和鼓励,学生们必须自力更生了。我认为这是令人担忧的,我们并没有给下一代学术写作者充分的教育,却期待他们做得更好。


\section{本书的解决之道}
学术写作可以是一部鸡飞狗跳的闹剧。教授们为写了一半的论文忧心忡忡,抱怨又收到了残酷的退稿信,在经费申请最后期限的前一秒才匆匆忙忙提交了申请报告,幻想着平静的夏日午后可以心无旁骛地奋笔疾书,然后埋怨开学日期的临近严重影响了自己的产能。心理学本身就够戏剧化了,我们真的不需要再添加什么戏剧效果。上述所有都是坏习惯。学术写作本应是循规蹈矩、枯燥而平凡的。为了保障本书以一种平凡的视角来看待写作,本书将不会讨论“写作的灵魂”、各种宗派的“写作灵感”或是“写作的精髓”。只有诗人才喜欢讨论“写作的灵魂”。你应该像个普通人一样写作,而不是像个诗人,甚至不应该像个心理学家。同样,本书也不会探讨任何“防御性”或是“回避心理”,关于这些理论,你大可以到书店的自助角自学了解。 《文思泉涌——如何克服学术写作拖延症》视写作为一系列具体的行为,就像:(1)首先在椅子/板凳/高脚凳/长软椅/马桶/草地上坐下来;(2)然后敲打键盘,写出一段文字。你绝对能够通过简单的办法来培养这些行为。让别人尽情拖延、做白日梦和抱怨去吧,你所要做的就是:坐下来,写。

当你阅读本书时,你要记住,写作不是比赛或者游戏。你想写多少就写多少,长短无所谓。千万不要觉得你有责任写更多,也不要为了发表而发表,写一大堆毫无意义的东西。不要误以为那些发表了大量文章的心理学家就有更多的研究成果。心理学家发表文章的目的多种多样,其中最为重要的是用于学术交流。文章的发表是一项科学活动必要的、自然的终结点。科学家们通过文字互相交流,那些印成铅字的文章构成了心理学的基石,它们阐述了人类是怎样的存在以及人类行为背后的原因。我相信大多数心理学家都在写作这件事情上倍感受挫,他们希望能够写得更多,也希望写作变得容易些。这本书献给他们。


\section{各章预览}
这本薄薄的小册子就如何写得更多给出了实用的、个性化的观点。第二章中,我们彻底检查了人们为写不出东西而找的拙劣借口。我们逐一分析这些借口,发现它们对于写作的效率毫无影响。这章将介绍如何用制订写作计划的方式来分配写作任务。第三章介绍了激励你执行写作计划的各种办法。你将学到如何制订好的目标,通过确立优先性原则来同时处理多项任务,以及如何管理你的写作进程。为了帮助培养你的新习惯,你可以和朋友一起建立写作小组。第四章是关于如何组建既有趣又有益的“失写互助组”(agraphia group)——一种有助于培养良好写作习惯的互助小组。第五章教你怎么写得更好。写得好的论文或是开题报告总能在众多平庸之作中脱颖而出,所以你应该努力写得更好。

第六章、第七章主要介绍了写作的原理。第六章剖析了实用的心理学论文写作技巧。我们可能并不喜欢阅读论文,但是我们必须写作论文。多产的写作者告诉过我他们是怎么写论文的,主流期刊的编辑告诉过我他们希望看到怎样的论文。第六章探讨了关于论文发表的几个入门级问题,例如如何给编辑写投稿信,如何与别人合作写作。第七章讲述了怎么写学术著作。心理学界为有抱负的学者们提供的资源实在有限,基于此,我就如何写学术著作及如何与出版商合作提出了一些个人见解。第八章对全书作了总结,还写了很多鼓励的话。



\chapter{妨碍写作的借口都“貌似有理”}
写作是一项严峻的挑战,就好像修理污水管或是经营一家殡仪馆。虽然我从未给死尸穿过衣服,但是我敢确定,给尸体作防腐处理要比写一篇关于此项活动的文章来得容易。写作很难,这就是为什么我们当中有那么多人写得那样少。如果你在阅读这本书,那么也许你能体味那种屡屡受挫的感受。每次我和教授或是研究生聊起写作这桩事,他们总是提到很多阻碍因素。他们相信,如果不是这样或那样的因素妨碍了他们,他们本来能够写得更多。我把这些借口称为“貌似有理”的借口。乍一看,这些妨碍写作的借口挺像模像样的,但是只要稍加深究,就会发现它们根本站不住脚。本章将列举几个最为常见的妨碍写作的借口,并教大家如何用最简单的方法来克服它们。

\section{借口一}
“我找不到时间写作”,也可以称为“如果我有更多整块的时间,我就能写得更多了”。

这个借口简直是学术界的“尚方宝剑”。我们都这么说;有些屡屡受挫的作者甚至把它高高举起作为人生指南。但是这个假设根本靠不住,就好像有些人相信人类只使用了脑容量的10\%。正如所有的假设一样,这一借口得以存在是因为它让人感到舒服。人们总是想当然地认为周遭的环境总是和自己作对,而如果时间表上有更多大块的空余时间,就可以有更多的时间用来写作,自然就能写得更多了。系里的同仁们对此表示理解,因为他们自己也觉得找不到时间写作。与同事们一起共同经历灰暗的挫折,从某种程度来说有一种奇怪的窃窃的甜蜜感。

为什么这个借口是假的?关键问题就在“找”字上。当人们赞成这个说法时,我的脑海里总是浮现一幅画面,他们的眼睛在时间表上游走,就好像自然科学家在努力寻找一种名叫“写作时间”的生物,这种生物隐藏得太深了,根本寻不到它们的踪迹。你需要“找时间教书”吗?当然不用——你有一张课表,你按部就班,从不迟到。如果你认为写作时间藏匿在你每周计划的某个角落,你就永远不会写得更多。如果你认为非等到整块的时间,比如春假或是暑假,否则就不能写作的话,那么你也不会写得更多。“找时间”对于写作来说是毁灭性的。以后再也不要这么说了。

相反,你不应该“找时间”,而是“安排时间”来写作。高效的写作者都会制订一个时间表并严格遵守。就这么简单。现在,花几分钟想一下你想要的写作时间表。好好想想你的一周安排:是否每周总有那么一些相对空闲的时间?如果你周二和周四有课,那么周一和周三的早上可能是最好的写作时间。如果你觉得下午或是晚上精神更好,那么就安排在晚些时候。每个人根据自己的其他安排,会有不同的黄金“写作时间”。关键在于规律性,而不在于天数或是小时数。你每周是花一天还是五天来写作并不重要——只要你腾出一段有规律的时间来,把它标在你的周计划表上,然后在这段时间坚持写作。开始的时候,你可以每周安排四个小时左右。等看到你所写的文章字数突飞猛进时,你可以再适当延长写作时间。

每次讨论到写作安排时,人们总是问我:那么你的安排呢?(有些人的口气里带着挑衅的意味,仿佛他们希望我耸耸肩,回答道: “怎么说呢,制订计划这种事情,说起来容易做起来难啊。”)我每周一到周五上午八点到十点用来写作。我起床,煮咖啡,然后坐在我的办公桌前。为了避免干扰,我写作前不查邮件,不洗澡,也不换衣服。简言之,我起床,然后写作。开始和结束的时间可能会前后调整,不过我每个工作日大概写两个小时。我不是一个喜欢早起的人,不过早上写作有很多优势,我能够在被处理邮件、学生约谈和与同事会面等事务淹没之前,抓紧宝贵时间写点东西。

大多数人都有一个既浪费时间又毫无成效的坏习惯,被称作“突击写作“ (binge writing) (Kellogg, 1994)。想写,拖延,为拖延感到万分内疚和焦躁,“突击写作者”(binge writer)最终选择某个周六什么也不做,只写东西。这样他们的负疚感得以缓解,整个“突击写作”的周期又开始循环。“突击写作者”花在为没有写作而感到内疚和不安上的时间,要远远超过制订计划的人花在写作上的时间。当你执行计划时,你就没有时间担心没有写作,抱怨找不到时间写作和沉溺于夏天你能够写多少东西的幻想上了。相反,你在固定的时间写作,然后彻底忘了它。我们有很多比写作更值得关注的事情。比如我总是担心我是不是喝了太多的咖啡或者我的狗是不是又到后院那个恶臭的水塘里喝水,但是我从不担心我要找时间写这本书:我知道我会在明天早上八点钟继续。

当“突击写作者”因他们的坏习惯而遭到质疑时,他们常常自我辩护:此乃天性使然,“我不是那种喜欢制订计划并能严格执行的人”。这根本就是废话。人们拿天性来说事儿,是因为他们不想改变(Jellison, 1993)。那些宣称自己不是“计划达人”的人在其他方面却成了计划专家:他们总是在固定的时间教书,在固定的时间上床,在固定的时间看自己喜欢的电视节目等。我就遇到过自称没有能力坚持每天写作的人,却无论刮风下雨都能够坚持在固定时间出门慢跑。千万不要还没有开始就选择放弃——制订计划是高效写作的唯一秘诀。如果你不打算制订计划,那么请你小心合上这本书,把它弄得像新的一样,然后当作礼物送给那些想成为一个好的写作者的朋友。

你必须坚决地捍卫你的写作时间。记住,你要安排时间写作,而不是找时间写作。你自己决定了这段时间是用来写作的。在这段时间内,你不能与同事、学生或是导师见面与约谈;也不能批改作业或者备课;更不能查阅电子邮件、看报纸或是查看天气预报。关掉你的网络和电话,关上门。(我以前会在办公室的门上挂一个“请勿打扰”的牌子,但可恶的是,这块牌子常被人误解为“他关上了门,但是他想让我知道他在办公室,所以我应该敲门而入” 。)

我得提醒你们,其他人未必会理解你对写作时间的忠诚。人们会希望在这段时间和你开个会,他们并非存心捣乱,只是无法理解你为什么没有时间。他们会怨你不知变通,认为你顽固不化,甚至揣测是不是有其他什么不便道明的原因使你不愿见他们。对我来说,最常见的问题是研究生们希望我能够早上九点见他们——这个时间对他们来说最方便,但是不巧,这正好在我的写作时间内。同样的,有时候我也不得不在写作时间里参加一些会议,因为这是唯一一个对所有与会者来说都方便的时间。

怎样应对这些无心打扰到你的人呢?对他们说“不”——这个词也许不能让你远离毒品(南希·里根\footnote{南希·里根,美国总统里根的夫人,曾在美国发起“对毒品说不”运动。——译者注}除外),但能够帮助你捍卫你的写作时间。你有两个很好的理由说“不”。首先,只有“失败的写作者”才会把你的拒绝当作是对他的挑战。我遇到的所有认真的写作者都非常尊重我对写作时间的坚持。他们也许会有一点不高兴,因为我无法在他们希望的时间安排会面,但是他们都会表示理解,因为制订计划是写作的唯一出路(这些人也会在他们的写作时间里拒绝我的会面要求)。为此而恼怒和满腹牢骚的人都不是好的写作者,所以别受他们拖累。其次,人们不会想要占用你的上课时间、你的家庭聚会时间或是你的睡觉时间,却想要占用你的写作时间,因为他们认为你的写作时间无关紧要。作为一个学者,你是职业的写作者,就像你是一名职业的教师一样。把你的写作时间当成你的上课时间,对那些无心打扰的人说“不”,并解释为什么你不能(注意:是“不能”而不是“不愿意”)打破你的写作计划。如果你不喜欢说“不”,那就撒谎。如果你也不喜欢撒谎,那就用你在读研究生的时候学过的理论:归咎于一个“常见又固定的职责”或是一个“世俗的拖累”。

在既定的写作时间里奋笔疾书,但是也不要教条化地圃于写作时间。如果你在写作时间结束后仍然文思泉涌或者在其他时间里坚持写作,岂不是更好?我把这称为“意外之作”。一旦你养成了习惯,坐下来写东西就会变得容易了。不过你要提高警惕,不能用“意外写作”来替代正常的写作时间。不管你在放春假的时候写了多少——你都应该遵守你的写作计划,严格执行。如果你发现自己胡说——“我周末已经写得够多了,周一我就轻松一下吧”,那这本书能够帮到你:合上它,用你非惯用的那只手的拇指和食指夹住它,在自己面前狂甩五分钟,以便提醒你怎样才是正确的做法。

或许你对“计划写作”是否有效还有保留意见。“这真的就是秘诀吗?”你也许会问,“难道就没有其他的法子能够多写点?”没有了——做好计划并严格执行是唯一的办法。职业写作者拉夫·凯斯(Ralph Keyes,2003: 49)在对成功写作者们的写作习惯做了大量研究后发现,“使一位写作者得以高产的秘诀就是坐在书桌前日复一日地写作”。如果你每周安排四个小时写作,你会对你能够完成的字数感到惊讶,确切地说是震惊,惊得哑口无言,呆若木鸡。你会发现自已完成了之前无法想象的事,例如提早完成了开题报告。你会收到改稿通知,并在一周内完成。你不敢再和系里的同事交流对写作的恐惧,因为你害怕他们会说“你已经不再是我们的战友了”,而且他们的话千真万确。


\section{借口二}
“我需要先作一些数据分析”,或者“我要先看几篇文章”。

这个借口最为阴险,危害也最大。首先,这个借口看起来合情合理。你也许会说,“我总不能既不看参考文献又不作数据分析,就直接写文章吧”。没错,但是我见过很多人把这个借口当作颂歌每日吟唱。同事们一开始都很尊重他们,相信他们要么是完美主义者,要么就是数据狂人。但是他们写得不多,也从不作所谓的数据分析。“突击写作者”往往也是“突击阅读者”和“突击数据分析员”,阻碍他们写作的坏习惯同样会妨碍他们做其他与写作相关的准备工作(Kellogg, 1994)——阅读、列提纲、提炼观点、分析数据,等等。像其他所有“貌似有理”的借口一样,这个借口也经不起深究。

要破解它很容易:在你安排的写作时间内,做一切你需要做的事情。要做一些数据分析吗?在你的计划时间里做。要看一些参考文献吗?在你的计划时间里做。要校验清样吗?也在你的计划时间里完成。要读一本如何写作的书呢?你知道应该什么时候读。写作的含义远不只是坐在电脑前打字,任何与完成写作任务相关的活动均可称为写作。例如,为了完成一篇论文,我常常会连续花几段写作时间来作数据分析。有时候我会花上一整段写作时间来做一些鸡毛蒜皮的事,比如研究某一杂志的投稿要求、画图表或校样稿。

这也是为什么只有制订计划才能够保障写得更多的又一原因。专业的写作包含很多部分:广泛的文献阅读、仔细的分析、文字严谨的研究方法陈述。我们无法“找时间”来完成所有相关文献的检索与阅读,就如同我们无法“找时间”写下阅读这些文章的笔记。所以安排好你的写作时间来完成这些工作吧。这样你就不会为找不到时间来读文献和作分析而感到惶恐了,因为你知道什么时候你会完成它们。


\section{借口三}
“我需要一台新电脑才能更好地写作”(同理,这里的“电脑”可以替换为“新的激光打印机”“新椅子”“新书桌")。

在所有的借口中,这条是最让人抓狂的。我不确定人们是否真的相信它——与其他的借口不同,这一条其实真的只能勉强算是一个“借口”。我自己的实例可能就能破解这个借口。当我还在读研究生时,我刚开始严肃地写作,我从一个同学的男朋友那里买了一台很旧的电脑。这台电脑即使在1996年也可以算是一件古董:没有鼠标,没有Windows,只有一个键盘和DOS系统下的WordPerfect 5.0软件。这台电脑寿终正寝以后,我的一些文件也因为无法复制而随之陪葬了。我又买了一台手提电脑,常常席地埋头苦写。我现在正在用2001年购买的一台速度奇慢、老得掉牙的东芝笔记本电脑写这本书——在如今电脑更新换代如此迅速的时代,我的这台笔记本电脑老得都可以去领养老金了。

大概有八年的时间,我都坐在一把折叠椅上辛勤笔耕。这把折叠椅退休以后,替代它的是一把稍微时髦,但是同样坚硬的老式埃姆斯椅。它是一把再简单不过的椅子:没有装饰和靠垫,也不能调节高度和角度。为了满足大家的好奇心,附上一张我写这本书的地方的图片。如图2.1,只有一张大而简单的书桌(没有抽屉,没有键盘架,没有豪华的文件收纳体系,等等)、一台激光打印机和一个咖啡杯垫。在我买下这张“蓝点”(注:著名家具品牌)书桌以前,我用的是一张10美元的胶合板折叠书桌,为了显示我的品位,上面铺了一块4美元的桌布。我就坐在那把折叠椅上,在那张折叠书桌旁,写完了我那本关于兴趣的专著的大部分内容(Silvia, 2006)和近20篇学术论文。

\begin{figure}[!htb]
\label{fig2-1}
\centering
\includegraphics[width=0.9\textwidth]{fig2-1.png}
\caption{我写作本书的地方}
\end{figure}


低效的写作者总喜欢悲悲切切地抱怨没有属于自己的哪怕一小块空间用来写作。我对这个老掉牙的借口不屑一顾。我从来没有专属的家庭办公室或是私人写作空间。在狭小的公寓或是房子里,我在客厅、卧室、客卧、主卧甚至浴室里写作,我只需要一张小桌子。我在家里的客卧里完成了这本书的写作。即使现在,我已经写了那么多书和论文,也买了自己的房子,我在家还是没有独立的书房来写作。我也不需要——总有一间浴室是空着的。

我身边的突击写作者曾多次提到打印机问题是阻碍他们写作的原因之一,他们提到这一问题的频率之高让我颇为吃惊。 “如果我家里有一台激光打印机的话……”他们用充满渴望的口气这样抱怨。他们没有意识到自己不能像印钞票一样打印写作稿——打印机只能用来打印你已经写好的稿子。我非常喜爱我的激光打印机,每个认真的写作者都应该买一台,不过打印机真的不是必需的。当雪莱·杜瓦尔(T. Shelley Duval)和我在合写一本关于自我意识的著作时(Duval \& Silvia, 2001),我只有一台石器时代的喷墨打印机,他什么都没有。要用喷墨打印机打印一本书需要很长的时间,到头来我们的部分草稿还是青绿色和红褐色的,因为打印机的黑墨水用完了。

当人们抱怨他们家里没有高速互联网时,我真心祝贺他们有这样正确的判断。如果你们仔细看图2.1,就会发现我的电脑上根本没接网线。我太太在家里的书房里安装了高速网络,我没有。这玩意儿只会让我分散注意力。写作时间就是用来写作的,不是用来查邮件、看新闻,或是浏览最新期刊的。有时候我会觉得下载一些文章可能对写作有点帮助,不过我可以在办公室里下载。最好的自控就是让客观环境不需要自控。

威廉·萨拉扬(William Saroyan, 1952: 42)这样写道,“写作,你只需要一张纸和一支笔。”设备永远不能帮你写作;只有制订写作计划并努力执行才能帮助你成为一名高效的写作者。如果你不相信我所说的,那么就看看比尔·斯顿夫(Bill Stumpf)最新的采访。作为家具设计业的传奇人物,斯顿夫为业界领军者赫尔曼·米勒(Herman Miller)公司设计办公家具。斯顿夫因为参与了艾龙椅(Acron)的设计而闻名,这真的是有史以来最棒的办公椅了。但是作为一名写作者(Stumpf, 2000),他深知家具能做的只有这些了。 “我不确定家具和写作能力这两者之间是否存在必然联系,”他说,“我想赫尔曼·米勒一定不喜欢听到我这样说
(Grawe, 2005: 77) 。”


\section{借口四}
“我只是在等待对的时机”,或者“我在灵感来临的时候才能写出好的文章”。

这最后一个借口是最可笑和最无厘头的。我无数次从那些不知何故拒绝制订写作计划的人那里听到这样的理由。 “好的作品都是在我有灵感的时候一气呵成的,”他们说,“在我没有心情的时候逼着我写也无济于事。我必须感到我想写了才行。”毫无建树的写作者这样说真是可笑。这就好像烟鬼们总是辩解吸烟能够使他们感到放松,而实际上吸入尼古丁压根儿就只会导致精神紧张(Parrott, 1999)。当那些备受折磨的人为不制订计划而辩解时,他们是在支持使他们深受折磨的原因本身。如果你相信你应该只在有灵感的时候写文章,那就请你问自己几个简单的问题:这种策略效果如何?你对自己文章的数量感到满意吗?你是否常常为找时间写作或是为仅仅完成了一半的论文而感到紧张?你是否牺牲了晚上或是周末的时间用来写作?

要驳倒这个借口也不难:研究表明等待灵感是徒劳的。博伊斯(Boice, 1990: 79$\sim$ 81)为那些等待灵感的突击写作者做了一个意义深远的研究。他召集了一批为写作而头痛的大学教授,随机地给他们分配了不同的写作策略。第一组(限制写作组)的教授们被禁止在任何非紧急情况下写作;第二组(顺其自然组)的教授们有50段写作时间,但是仅在他们感到有灵感的时候写;第三组(附加干预组)的教授们有50段写作时间,并且在这些时间内必须写作(如果他们没写,就得向一个他们不喜欢的组织交罚款)。统计项是每天写作的数量和每天提出的创造性观点的数量。图2.2的数据显示了博伊斯的结论。首先,附加干预组的产量最高:他们的产量是顺其自然组的3.5倍,是限制写作组的16倍。那些只在有灵感的时候写作的人比那些被告知没事别写的人多写了一点点——灵感的作用实在是被高估了。其次,那些被逼着写作的人提出了更多创造性的想法,他们平均提出想法的间隔是1天;顺其自然组是2天,而限制写作组是5天。由此可见,写作本身孕育了继续写下去的基础。

\begin{figure}[!htb]
\label{fig2-2}
\centering
\includegraphics[width=0.9\textwidth]{fig2-2.png}
\caption{不同写作策略的效果}
\end{figure}

有些类型的写作实在太不可爱了,以至于没有一个正常人会喜爱它们。什么样的人会对写开题报告感到热情高涨呢?谁会早上醒来,就对写作“具体目标”和“协议/合约安排”感到欢欣鼓舞呢?写课题申请报告就好像报税,而且你还没法请个会计来帮你做。如果你对阅读美国卫生与公共服务部提供的科研基金SF424申请指南有着难以抑制的热情,那么你根本不需要这本书。如果你和其他人一样,那么要完成课题申请报告,你需要的不仅仅是“灵感”。

那些等待灵感的人应该从云端降落,回归到正常的普通大众当中。古希腊人为诗歌、音乐和悲剧都各分配了一个神,但是从未听说他们为按照美国心理学会(APA)格式写作期刊论文安排了什么神灵。作为学者,我们不是在进行文学创作,也没有书迷等在酒店门口,手捧《人格与社会心理学公报》,要求我们在上面签名。我们写的是专业性的、学术性的文章。有些类型的学术写作可能稍微轻松——例如教科书,或是比如这本书——但是即使这样,也要求我们把有用的信息归纳齐整,然后传递给读者。我们的写作很重要,因为它是实用的、清晰的和启发思考的。

拉夫·凯斯(Ralph Keyes, 2003)告诉我们,最杰出的小说家和诗人——在我们看来最应该等待灵感的人——实际上也并非只在有灵感的时候才写作。高产的安东尼·特罗洛普(Anthony Trollope, 1883$\sim$ 1999:121)这样写道:

\begin{lstlisting}
有些人认为他们为灵感而生,所以他们应该允许自己等待灵感来唤醒他们。每当我听到这样的布道时,总是很难掩饰我内心的不屑。在我看来,没有什么比这种说法更荒诞了,就是一个鞋匠说他要等待灵感,或是牛油烛小商贩在等待神的旨意来融化牛油也没有这么荒诞。有一次有人告诉我最可靠的有助写作的方法是在椅子上涂一点鞋线蜡。我宁愿相信鞋线蜡也不相信灵感。
\end{lstlisting}

那么这些杰出的写作者都怎样写作呢?猜猜看。成功的专业写作者,无论他们写的是小说、非小说、诗歌还是戏剧,他们的高产都有赖于有规律地写作(通常是坚持每天写作)。他们都不相信必须等有了感觉才能写作。正如凯斯(Keyes, 2003: 49) 所说,“认真的写作者笔耕不辍,不论有无灵感。随着时间推移,他们发现规律性显然是比灵感更靠谱的朋友”。有人或许会说,那就订个计划并且严格执行吧。

\section{小结}
本章对一些常见的妨碍写作的借口进行了冷静而批判性的审视。我们都喜欢躲在这些温暖的外衣下,但是披着这温暖的外衣还想码字实在是太难了。如果你还对这些借口恋恋不舍,那就把这章多读几遍,直到你被彻底洗脑而坚信只有制订计划才有未来。如果你不相信这一点,那么这本书恐怕就帮不了你了,因为无论你喜不喜欢写作,能够写得更多的秘诀只有保持规律性地写作。制订好了写作计划,你就可以继续读下一章了。它将介绍一些简单而有效的激励工具,让你能够坚持写作并且提高你的写作效率。
\chapter{赘语}
与赘语作斗争就像同杂草作斗争一样,作者总是稍稍滞后。五花八门的新赘语一晚上就冒出来,到中午就成了美国话语的一部分。看看尼克松总统的助手约翰·迪恩在水门事件电视听证会那天的创举吧。第二天美国人人都说“就在当前这个时间点上”,来代替“现在”。

看看披挂在动词后面但并无必要的介词吧。我们不再“领导”委员会。我们“领导起”委员会。我们不再“面对”问题。当我们能“空闲得出来”几分钟,我们“针对问题面对”它。也许你会说,那都是细枝末节,不值得烦扰。但这的确值得烦扰。写作的改进同我们能去除的赘语数量成正比。“空闲得出来”中的“得出来”就不应该有。仔细审查付诸笔端的每一个词,你会发现毫无目的的词语数量惊人。

举形容词“私人/个人的”为例,如:“我的一个私人朋友”、“他的个人感受”,或者“她的私人医生”等。这些只是成百上千可去除词语的典型代表。用“私人朋友”来区分“生意朋友”,结果使语言和友谊都贬值了。某人的感受就是那个人的个人感受——那就是“他的”的含义。至于私人医生,指的也就是被叫进突然病倒的女演员化妆室的男医生或女医生而已,这样该演员就不必由剧院指派的那种公事公办的医生来诊治了。有朝一日我倒愿意看见那人变成“她的医生”。医生就是医生,朋友就是朋友。其余皆赘语。

赘语即佶屈聱牙的词语,它排挤掉简明的同义词。甚至在约翰·迪恩之前,业界人士已经停止说“现在”。他们说“在当下”(“我们所有的操作员在当下都在帮助顾客”),或者说“在目前的这段时间里”,或者说“不久很快”(意思是“一会儿”)。其实这个意思都可以用“现在/马上”来表示目前的时间(“现在我可以见他了”),或者用“现今”来表示历史上的现在(“现今物价高涨”),或者简单用动词进行时(“在下雨”),而没有必要说“在目前的这段时间里我们正在经历降雨”。


“经历/感受”这个词是最糟糕的赘语之一。甚至连牙科医生都会问你是否感受到疼痛。而假如椅子上坐的是他自己的孩子,他就会说,“疼吗?”简而言之,他就会是他自己。他通过在职业角色中用浮夸的词语,使自己不但听起来更重要,而且还钝挫了事实中痛苦的一面。此类语言同样用于航空乘务员演示当机舱内用尽空气时氧气罩会如何落下。“假如飞行器一旦最终遭遇不太可能发生的紧急情况,”乘务员如此开始——其用语本身就够夺人氧气的了,大家都已经准备好了任何灾难的发生。

赘语是乏味的委婉语,贫民窟成为社会经济萧条区、垃圾清扫工成为废物处理人员、城市垃圾场成为减量单位。我想起比尔·莫尔丁的卡通,描述有两个流浪汉在货车车厢里。其中一人说,“我开始只是个简单的盲流,可现在是绝对失业了。”赘语是政治正确性的极端表现。我见过一则男孩夏令营广告,组织者的目的是提供“个别关注给那些极少数与众不同的孩子”。

赘语是公司官方语言,用以掩盖其错误。当DEC公司裁减3000个岗位时,其告示并没提及解雇,而是称之为“不情愿之措施”。在空军发射的火箭爆炸失事后,火箭被称为“提早撞击地面”。当通用汽车公司关闭一家工厂时,那是“与产量计划相关的调整”。破产的公司是处于“负现金流状态”。

赘语是五角大楼的语言,称侵略是“加强保护性反应打击”,对其庞大预算需求的解释是为了“反威慑力”。正像乔治·奥威尔在《政治与英语语言》一文中指出的那样——该文写于1964年,但在柬埔寨、越南以及伊拉克战争期间经常被引用——“政治性演讲与写作主要是为不可辩护者辩护……因而政治性语言不得不由委婉语、设问句以及纯粹的云山雾罩式的模糊语构成”。奥威尔警告说,赘语不只是恼人,而且是致命的工具。这在最近几十年间的美国军事冒险中得以证明。在乔治·布什任总统期间,伊拉克的“平民伤亡”成立“间接破坏”。

语言伪装在亚历山大·黑格将军任里根总统国务卿期间达到新高峰。在黑格之前,没人想到会说“在此成熟之机的关键时刻”来表示“现在”。他告知美国人民与恐怖主义作斗争可以用“有意义的制裁性强制手段”,中程核导弹正处于“关键性漩涡之中”。至于公众对此心存的忧虑,他的意思是“交给艾尔”,但他实际说的是:“我们必须将此推到公众关注度更低的分贝。我认为在这一领域的情况没多少学习曲线可得。”

我可以继续从各行各业引出例证——每个行业都有不断增长的行话库,它们扬起尘埃,迷住大众的眼晴。但都列出来会很烦人。在此提出来的目的是引起大家注意:赘语是写作的敌人。因此要警惕并不比短词强的长词:assistance/help(帮助),numerous/many(许多),facilitate/ease(促进),individual/man or woman(个人),remainder/rest(剩余),initial/first(首先),implement/do(实施),sufficient/enough(足够的),attempt/try(试图),referred to as/called(被称为),还有成百上千更多的词。警惕所有含糊的时髦新词:paradigm(范例)与parameter(参数),prioritize(优先考虑)与potentialize(使成为潜力)。这些都是窒息写作的杂草。能写“与什么人交谈”,就不要写“与什么人对话”。不要写“与什么人协调配合”。

同样阴险的还有人们用于解释自己打算如何解释的各种词组:“我也许可以补充”、“应该指出的是”、“使人有兴趣注意的是”。假如你也许可以补充,就补充吧。假如有什么应该指出,就指出吧。假如使人有兴趣注意,就使它有趣好了。当有人说“那会使你感兴趣吗”之时,我们对下面所说的到底是什么不都会感到疑惑不解吗?不要胀大本无须胀大之事,如:可能除外的情形是/除外,由 于这样一个事实/因为,他完全缺乏这样一种能力/他不能,直到那样一个时刻/直到,目的就是/为了⋯⋯

有什么一眼就能认出赘语的办法吗?有一个办法,我在耶鲁的学生发现其行之有效。我会在一篇文章中用括号括上任何一个无用的成分。通常情况下只有一个词被括上:动词后赘的不必要的介词(命令起来),或者同动词意思一样的副词(高兴地笑),或者描述已经很清楚的事实的形容词(高高的摩天楼)。我的括号经常括上削弱句子力度的小小修饰语(有一点儿),或者诸如“在某种意义上”之类的毫无意义的词组。有时候我的括号会括上整个句子——就是那种基本上重复前一句,或者读者无须知道,或读者自己能明白的句子。大多数初稿可以砍掉一半而不损失任何信息或作者的语气。

我括上学生的浮夸词而不划掉这些词的原因,是避免冒犯其视若神明的散文体。我要完好无损地保留他们的句子,以便他们自己分析。我对学生说,“我也可能错,但我认为去掉这个地方并不影响意思。你自己决定。读一下不带括号的部分,看是否通顺。”在学期的前几周,我发还给学生的文章都是括号。整段整段都被括上。但很快学生们就学会在心里给赘语加上括号,到了期末,他们的文章就几乎没有赘语了。现今,当初的许多学生已成为专业作家,他们对我说,“我至今还能看见你的括号——它们会跟我一辈子。”

你也可以培养同样的眼力。从文章里找出赘语,无情地修剪。对所有可以去除的部分都要心存感激。重新审校你写的每一句话。每一个单词都起新作用吗?有没有什么地方虚夸、做作或者赶时髦?你是否只是因为自己觉得写得漂亮,而对那无用的部分不能割舍?

简洁,再简洁。
\chapter{风格}
作者努力写出简洁的句子,但总有那肿胀的怪物埋伏在暗处捣乱。有关写作初期的告诫就谈到此。

“但是,”你也许会说,“如果我去除你认为是赘语的所有部分,如果我将每一句话都剥到只剩骨头,那我还能剩下什么?”这个问题问得合理。简洁推到极致也可能会指向一种不比“狄克喜欢珍妮”和“看斯波特跑”这样的句子更复杂多少的风格。

我将首先从木匠手艺的层面回答这个问题。然后涉及更大的问题,即作者是谁,如何保护作者的身份。

很少有人意识到自己写得多么糟糕。没人向他们指出有多少过剩或含糊的词语溜进自己的写作风格中,以及它们如何阻碍了作者想说的话。假如你给我一篇八页的文章,我叫你删减到四页,你会叫喊说那做不到。然后你回家去做,结果会好得多。之后便是难的部分:删减到三页。

要点是你必须先将自己的写作拆开,然后再搭建起来。你必须知道基本工具是什么,以及其预设的作用是什么。再拿那个木匠手艺比喻为列,首先必须锯好木头,然后钉钉子。之后,你如果有雅兴,再切修棱角、添加别致的顶部。但千万不要忘记你是在练习一项基于一定原则之下的技能。如果钉子不牢,房子就会坍塌。如果动词不牢、句子结构摇晃,句子就会分崩离析。

我得承认,有一些非虚构作家,如汤姆·沃尔夫和诺曼·梅勒,他们建构起了不起的房屋。但这些作家花了许多年来学习技艺,最终才搭建起神奇的塔楼和空中花园,使我们这些从未梦想过如此装饰的人惊叹不已。他们知道自己在做什么。没人一夜间就成为汤姆·沃尔夫,就连他本人也不能。

那么首先要学会钉钉子。如果你所建房屋既结实又好用,就该对具简洁之力感到满意。

但你没有耐心去成就一种“风格”——去修饰简朴的词语,这样读者就会把你认作一种特殊的人。你只会寻觅花哨的比喻、华而不实的形容词,就好像“风格”是什么能在风格店买到的东西,是可以用装饰漆那鲜亮的颜色装饰词语的东西。(装饰漆是装修工用的彩色漆。)风格店并不存在,风格对于写作中的人是有机体,就像头发是他自己身体的一部分,或者假如他秃顶,那么就是他身体缺乏的那部分。试图添加风格就像加假发在秃头上。瞧第一眼时,之前秃顶的人看起来年轻,甚至还帅气,但瞧第二眼时——看假发时人们总会瞧第二眼——那人看起来就不对劲了。这里的问题并非是他看起来没有梳理好,他梳理得很好,我们真得敬佩假发匠人的高超工艺,问题是他看起来不像自己了。

这个问题是那些特意装点自己文体的作家的通病。你丧失的是使你自己独一无二之处。读者会看出你是否在装腔作势。读者要的是,与他们交谈之人听起来是真挚的。因此,写作的一个基本准则是:做你自己。

然而,没有什么准则比这一条更难遵循。这要求作者做到两件事,而按其新陈代谢的本能来说,这些都是难以做到的。他们必须放松,必须有信心。

叫作者放松就像在检查疝气时叫人放松一样;至于信心,看,他多么僵硬地坐着,眼睛直勾勾地盯着等待他造词儿的电脑屏幕。看,他多么频繁地站起来找吃的或喝的。作者会想方设法躲避写作实践。我可以证明我在报社工作期间,作为记者,我每小时去饮水机的次数大大超过身体对水的需求。

如何拯救作者出苦海呢?很不幸,并没有拯救的办法。我只能安慰大家说并非只有你的遭遇如此。有些日子好过一些,另一些日子则难熬到让你绝望,不再想写作。大家都有过这些日子,而且还会有更多这样的日子。

当然,最好还是尽量减少难熬的日子。这就使我回到如何放松这个问题上来。

假如你就是那位坐下来准备写作的作者。你想到你写的文章必须具有一定的长度,不然就不会显得重要。你想到文章印出来有多么庄严。你想到那么多人阅读你的文章。你想到文章必须具有重重的权威性。你想到文章的风格必须炫目。怪不得你会浑身发紧:你在忙于想着自己甚至还未开始的文章写完之后所要承担的巨大责任。而你发誓要对这一任务称职,于是四处找大词,找那些如果你不是想刻意给人突出印象就根本想不到的词,并且一头栽进去。

第一段是一场灾难——整段话成了似乎来自机器的一系列笼统词语的组合。人是不会那样写的。第二段也好不了多少。但是第三段开始有点儿人情味,而到了第四段你开始听起来像自己了。你开始放松了。令人称奇的是,编辑会常常删掉文章的前三四段,甚至前几页,而始于作者听起来开始像自己的部分。前几段的问题不只是缺乏人情味和浮夸华丽,而是根本就没说什么,只是刻意地要写出一个花哨的前言。作为编辑,我总是在寻找说类似以下这样话的句子:“我永远不会忘记那一天……”这时我就想,“啊哈!真人说话啦!”

作者用第一人称写作显然是最自然的。写作是把两个人之间密切的交往付诸笔端,写作保持人情味才会顺畅。因此我敦促人们用第一人称写作,用“我”或“我们”。但大家表示抗拒。

“说我认为什么,或者我感觉什么,那个我是谁啊?”他们问。

“不说你认为什么,那么你又是谁呢?”我回答他们,“只有一个你。没有任何其他人想的和感觉的一模一样。”

“可是没人会在乎我的观点,”他们说,“那样会让我觉得太突出了。”

“如果你对他们讲述有趣的事,他们会在乎的,”我说,“而且要用自然而然的词语讲述。”

然而,要作者用“我”并不容易。他们认为必须先赢得袒露自己情感和思想的权利,不然就会显得自以为是,或者不庄重。此类恐惧也影响了学术界,因而出现了学术性的“某人/有人”的用法(“有人不能苟同于莫尔特比博士对于人类状况的观点”),或者无人称的用法(“希望费尔特教授的专著会理所当然地拥有更广泛的读者”)。我可不想见用“有人/某人”这样的人——很乏味。我要的是对自己的话题有激情的教授来向我诉说该话题为何使他着迷。

我意识到写作中有广大领域不允许用“我”。报纸不要第一人称代词“我”用于新闻报道中;许多杂志也不要它出现在文章中;商业、社会机构不要它用在大量发送到美国家庭的报告中;大学不让把“我”用在学期论文或学位论文中;还有英语教师也不赞成用第一人称代词,除非是作为书面语的“我们”(“我们在麦尔维尔对于白鲸的象征用法中看到……”)。以上那些禁用是合理的。报纸文章就应该由客观报道的新闻构成。在学生没经过一番挣扎学会从其内在优点和外在评论评价一部作品之前,教师不希望学生避重就轻地发表意见——“我觉得哈姆雷特挺愚蠢”。我也理解教师的这些想法。“我”这个词有可能变成某种自我陶醉和自我逃避的工具。

然而,我们的社会已经变得害怕袒露自己的心声。向我们发送宣传品、寻求支持的社会机构听起来惊人地相似,但所有这些机构——医院、学校、图书馆、博物馆、动物园等——当然是由那些有不同梦想和展望的男男女女创建和管理的。这些人都在哪儿?在所有那些非人称的被动语态句子中,如“已经采取行动”和“重点已经确定”,很难瞧见他们的真面目。

即使是在不允许用“我”的时候,也有可能传达一种个性化的我的意思。政治专栏作家詹姆斯·赖斯顿在专栏写作中并没用“我”,但我却很清楚他是什么样的人,对其他很多随笔作家和记者,我也可以这么说。好作家在字里行间是看得见的。如果不允许你用“我”,写作时至少要用“我”思考,或者第一稿用第一人称写,然后再把“我”去掉。这样能预热你的非人称风格。

风格与心理绑在一起,写作的心理机制根深蒂固。我们表达自己的特有方式,或由于“作者心理阻滞”而没能表达好自己的原因,部分埋藏于潜意识心理中。作者心理阻滞的种类就像作家种类一样多,我也没有捋清它们的意图。这是一本薄书,我的名字也不叫西格蒙德·弗洛伊德。

但是我也注意到避免“我”的新理由:美国人在言辞上不愿意冒任何风险。我们的上一辈领袖们直言不讳自己的立场和信仰,现今的领袖们却千方百计地巧用词语逃避这一宿命。看看他们如何在电视采访中拐弯抹角,就是不立场鲜明。我记得有一次福特总统向一行来访的商人保证他的财政政策会奏效。他说:“每月我们所看见的是不断增亮的云彩,此外别无他物。”我对此的理解是那云彩还相当黑。福特的句子含义模糊不清,等于什么也没说,却给他的选民打了镇静剂。

后来的当局者们也并没有起色。国防部长卡斯帕·温伯格在1984年评价波兰危机时说:“有继续严重关注的余地,而且形势仍然严重。严重的形势持续越长,严重关注的余地也就越多。”老布什总统被问及他有关突击步枪问题的立场时说:“有不同的群体认为可以禁止某种枪支。我不在其列。我是在那些深刻关注者之列。”

不过我的全能冠军当属70年代身为四任主要内阁成员的艾略特·理查森。很难确定从他那模棱两可的句子宝库中选哪一句,还是看这句吧:“但是呢,均衡地来讲,我认为少数族裔及妇女维权行动还是取得了一定的成果。”\footnote{原文如下: And yet, on balance, affirmative action has, I think, been a qualified success.}一句13个单词的话中就有5个词含义模糊。在现代公共话语中,我给它最空泛句子一等奖,但与其媲美的还有他分析如何消除生产线工人单调乏味的句子: “这样呢,最后,我形成一个坚定的信念,我在开始提到过:就是这个问题太新,无法最终判断。”


那就是坚定的信念吗?像摇摇晃晃来回摆动的年迈拳击手这样的领袖不能鼓舞信心——也不配鼓舞信心。作家也是如此。推销你自己,你的题材就会发挥自己的吸引力。相信自我身份,相信自己的见解。写作是一种自我行为,你得承认这一点。要全力以赴使自己不断向前。
\chapter{写作常见问题汇总}
以下检查清单(checklist)中所说的这些问题,虽然看似细节,但并不是小题大做、吹毛求疵。我们非常有必要在投稿前,仔细检查并尽可能避免这些问题。这么做的原因主要如下。

(1)论文的发表是需要经过同行评审的。站在审稿人的角度,你可以想象下,如果有两份稿件在你面前,第一眼望过去的时候,映入你眼帘的往往是图片和文字的整体感觉。在这种情况下,整体感觉本质上是由格式和一些细节来传达的。如果其中一份看上去格式很混乱,比如有些地方多输入空格,有些地方没有空格;上下标不注意;不同段落之间行距间距不同;文字中夹杂中文全角字符;许多拼写和语法错误;参考文献格式混乱;图片作得不规范等等。另一份看着特别舒服,没太多问题。这个时候,你会有什么感觉?你会不会感觉那个写作格式混乱的人,他做实验的时候是不是也会比较粗糙、态度不够严谨认真呢?那么既然如此,他设计的实验方案以及获得的数据结果和结论可靠吗?当然,我知道这两者之间不能画等号,科学工作者本身是要基于证据来得出结论,不能带有倾向性,但是大量出现以下检查清单中的问题,会给人带来很不好的心理暗示,因为这反映出作者的态度。毕竟我们是在向某一个期刊投稿,期刊本身就是有明确格式要求的。期刊这么做也是有原因的,是为了能够让审稿人和读者清晰阅读论文,理解论文,并便于论文的传播、交流和分享。而且我在过去的审稿中也确实发现,一般作图和文字格式比较混乱的文章,整体实验设计和结果推导方面也确实相对来说问题会更多。

(2)站在审稿人和读者的角度,太多的格式错误使人无法流畅地阅读,这样就影响了他们清晰理解作者的写作意图。可能作者的工作不错,但由于大量格式错误,分散了读者的注意力,读者没有把握到你实际想阐述的逻辑线索等,进而低估了你论文的贡献。

(3)站在导师的角度,看到一份格式错误太多的稿件,同样很容易干扰导师对于文章核心内容的聚焦,因为之前提到的格式错误问题,会干扰注意力。比如每句话都存在基本格式或语法错误,这样就很难把所有句子都衔接在一起连贯地看它们的观点展现、逻辑表达是否合理。

(4)同样的原因,这样反而很干扰创造力的发挥。检查清单上列的这些要点都属于没太多创造性的工作,我们在这个上面花费的时间和精力越多,越耽误我们投入在创造性工作上的时间和精力。所以我们如果在第一次成稿中就能注意这些问题,那么之后就可以专心地集中在具有更高创造性的事情上了。

此外,以下给出的所有案例都是我实际指导和修改学生论文过程中,所遇到的具体问题。

\section{影响表达的问题}

\vspace{0.5cm}
{\kaishu 一、定语太长,影响理解句子含义}
\vspace{0.5cm}

案例:The sandwich structured flexible Zn-air battery device were assembled with the flexible electrodes of the cotton textile waste Zn plated and the NiFe hydroxide face-to-face separated by the poly (vinyl alcohol) (PVA)-KOH hydrogel polymer electrolyte.

说明:案例中的“cotton textile waste Zn plated”这里定语太长,影响阅读。

\vspace{0.5cm}
{\kaishu 二、句子太长,影响含义表达和读者的阅读感受}
\vspace{0.5cm}

案例:By using energy storage systems (ESSs), the power system can shift part of the peak load to low power consumption period, thus utilizing surplus power during low power consumption period, improving the load rate of the power grid, in order to achieve the purpose of energy saving, which can save resources, reduce pollution, and be more friendlyto our environment.

说明:这个案例整段就一句话构成,句子过长了,导致句子中间停顿太多。一方面非常影响含义的有效表达,另一方面也使读者的阅读感受不佳,可以考虑改造为:By using energy storage systems (ESSs), the power system can shift part of the peak load to low power consumption period. Thus, surplus power during low power consumption period can be utilized to improve the load rate of the power grid, achieving the purpose of energy saving. As a summary, using ESSs in power grid can save resources, reduce pollution, and be more friendly to our environment.

案例:Figure2 shows the SEM image and EDS results of NMCTW. It can be seen that the waste cotton textile is uniformly covered by Ni after the deposition (Figure 2a), and energy dispersive spectrum (EDS) mapping of Ni element further indicates Ni metal exist which are evenly dispersed on the surface of the waste cotton textiles substrate (Figure 2b and c), which is conducive to function as an conductive electrode substrate.

说明:第二句跨度过长,也是存在同样的问题。例如可以改造为:Figure 2 shows the scanning electron microscopy (SEM) image and energy-dispersive spectroscopy (EDS) spectrum of NMCTW. It can be seen that the cotton textile waste is uniformly covered by Ni after the deposition (Figure 2a). The EDS mapping of Ni element further indicates the presence of metallic Ni particles. The Ni particles evenly dispersed on the surface of the cotton textiles waste substrate (Figure 2b and c) act as a flexible conductive electrode substrate.



\section{格式问题}

\vspace{0.5cm}
{\kaishu 一、英文论文中所有符号应为英文字符和半角字符}
\vspace{0.5cm}

这个是经常出现的问题,在以往我检查的稿件中,大部分都出现过文中夹杂使用中文全角字符这个问题,这可能是由于输入法没及时切换产生的问题。

案例:I would like to submit the manuscript entitled “Long-battery-life flexible zinc-air battery...
说明:其中的双引号是中文全角字符,应改为英文半角字符。

案例:…is stirring at 90 ℃ for about…

说明:案例中的摄氏度为宋体格式,正确的应该是英文格式,如Times New Roman格式:… is stirring at 90℃ for about...

\vspace{0.5cm}
{\kaishu 二、检查标点符号是否正确,包括句点、空格等}
\vspace{0.5cm}

案例:… by morphological regulation, which could enhance the performance of WS2 [9,10] Through the construction...

说明:第一句缺少句点。

案例:Reproduced with permission from Ref.[32] , Copyright 1998, Springer Nature.

说明:文中“Ref.[32]”后多输入了一个空格。这也是经常犯的错误。还包括少输入空格,此类格式错误包括语法错误等,可以通过一些软件(如grammarly)很好地解决,也可以在word软件中打开标点符号的标记,便于识别。

案例:The voltage decreases sharply at the end of discharge, dem-onstrating the discharge failure of the battery. The reactions occurring on the Zn electrode during the discharge can be expressed as follows: 

\begin{equation}
 Zn^{2+} + 4OH^- + 4OH EN)_4^{2-}
 Zn(OH)_4^{2-} - ZnO + H_2O + 2OH^-
\end{equation}

说明:方程格式混乱,存在多处错误。修改后的形式为

\begin{equation}
 Zn^{2+} + 4OH^- \rightarrow Zn(OH)_4^{2-}
 Zn(OH)_4^{2-} \rightarrow ZnO + H_2O + 2OH^-
\end{equation}

此外,我制作了一份文档,提供了一些常用的但经常容易输入错误的符号,下载地址:https://pan.baidu.com/s/1KedVanP6z9dHjcz0Ov-seQ。

\vspace{0.5cm}
{\kaishu 三、字体格式不统一}
\vspace{0.5cm}

(一)上下标是否正确

案例:Co3O4@NCNTS

说明:Co3O4应改为$Co_3O_4$。同时掌握批量搜索功能,比如Word中可以采用Ctrl+F调用出批量搜索功能,检查是否还有其他地方的Co3O4没有注意上下标问题。

(二)缩写格式需要统一

案例:The morphology and composition of the as-obtained Zn anodic electrode and the air electrode with nickel iron hydroxide catalyst electrodes were characterized by field emission electron microscope (FESEM, S-4800, HITACHI, Japan) equipped with... Figure 2 shows the SEM image and EDS of NMCTW. It can be seen that the waste cotton textile is uniformly covered by Ni after the deposition (Figure 2a), and energy dispersive spectrum (EDS) mapping of Ni element further indicates Ni metal...

说明:第一次在实验中出现缩写的时候,将扫描电子显微镜缩写为FESEM,但随后Results部分是以SEM来指代。改动方式之一:将之前的FESEM改为SEM。而且此处还存在第二个问题,原文写的是“electron microscope”,而非“scanning electron microscope”,所以简称也无法和SEM对应。

案例:2.3 Assembly and tests of the cotton textile-based flexible zinc-air battery

The sandwich structured flexible Zn-air battery device were assembled with the flexible electrodes of the cotton textile waste Zn plated and...

说明:表述不统一,文中有的地方表达为zinc-air,有的地方则是Zn-air,而且注意Zn/zinc和air之间的连接符号也不统一。

(三)仪器表达方式统一

案例:...field emission scanning electron microscope (FESEM, S4800,HITACHI, Japan) equipped with X-ray energy dispersive spectroscopy system (EDS). The crystalline structures of the as-obtained electrodes were measured by X-ray diffraction (XRD, D/max 2200/PC,Cu Ka radiation). Electrochemical tests were conducted in a three-electrode cell configuration using the electrochemical workstation (CHI760E and PARSTAT 4000).

说明:SEM(扫描电子显微镜)仪器给出了型号、厂家和产地(国家),但XRD(X射线衍射设备)没有厂家和产地等信息,之后的电化学工作站描述也是同样问题,


\section{参考文献问题}

\vspace{0.5cm}
{\kaishu 一、文献的规范引用}
\vspace{0.5cm}

全文引用的每一篇参考文献都要下载到全文,然后仔细阅读引用的这篇文章是不是真的提到过你在正文中所说的相关内容。如果提到过,作者是不是也是引用其他文献的。如果是,那么你还得继续找到被引用的这个更原始的文献,找到最终的源头。否则有可能别人引的文献,根本没提到这个内容,然后一旦跟着引了,就会出现明显错误了。当然还有一种情况就是由于引文习惯不良(比如没有很好掌握文献管理工具的使用)等,导致引文错误,引的论文根本没提到所要表达的内容。

文中表述的每一句话,尤其关于结论、观点类的,都要准备好相关的证据。仔细从源头出发,思考这句话是否正确。不能仅仅因为文献是这么阐述的,所以就不假思索地认为这是对的。毕竟文献中也只是阐述了作者的观点,科学不断在进步,过去的观点有错误或者实验有瑕疵,那也是很有可能的。

案例:“In addition, side $reactions^9$ related to Li electrodes or oxidative products and electrolyte decomposition are usually involved in battery electrochemistry, particularly during the charging process.”

说明:学生初稿中提到锂电极的副反应,然后在副反应这里引了文献9。但是文献9没有提到任何关于锂电极的副反应。

\vspace{0.5cm}
{\kaishu 二、参考文献在正文中的格式错误}
\vspace{0.5cm}

这也是常见的问题,解决方案很简单,如果使用Endnote之类的文献管理工具,一般期刊都会提供相应的参考文献style文件,只要导入相应的style,就可以生成正确的格式了。

案例:...performance of $WS_2$[9][10].

说明:学生初稿中正文的文献引用格式是[9][10],但基本上很少期刊是这种方式的引用格式。常见的比如[9,10]、9,10(上标)等。

案例:... which both deteriorate the catalytic property of the as-obtained electrodes. The transparent NiFe HUFs electrode electrodeposited for 500s have the large number of active sites and the good conductivity, resulting in the high catalytic performance of as-obtained electrode, which have also been reported in our previous studies.

说明:学生原稿件是拟投ACS(美国化学学会)旗下的一个期刊。文献引用应该在标点之后的,而不是在句号之前。此外,该版本中还存在不少语法错误。

\vspace{0.5cm}
{\kaishu 三、参考文献列表中的错误}
\vspace{0.5cm}

(一)期刊术语缩写不正确或者全称不正确

解决方式:以Endnote文献管理工具为例,Endnote中的Tools可以找到Term Lists选项,这个Term Lists 给出了常见期刊的缩写方式,如图\ref{fig5-1}所示。

\begin{figure}[!htb]
\centering
\includegraphics[width=0.9\textwidth]{fig5-1.png}
\caption{在文献管理工具中,正确编辑期刊的全称以及缩写}
\label{fig5-1}
\end{figure}



但是,这个Term Lists通常是不完整的,所以有一些期刊包括新期刊,没有对应的缩写方式。针对期刊全称输入有误的情况,将期刊全称输入正确后,检查Term Lists中是否有对应缩写;对于期刊全称正确,但无对应缩写的,可以自己新建Term Lists 中的Term,这样以后修改论文时,会非常方便,每次可以自动更新参考文献列表,而不需要每次手工去编辑参考文献列表。

案例:
[102] D.Su, S.Dou, G.Wang, Chem Commun (Camb) 50 (2014) 4192.

[103] D.B. Kong, X.Y.Qiu, B. Wang, Z.C.Xiao, X.H. Zhang, R.Y.Guo, Y.Gao, Q.H. Yang, L.J.Zhi, Science China-Materials 61 (2018) 671.

说明:学生初稿中,参考文献[102]和[103]中的期刊名称的缩写不规范。

(二)参考文献中的作者拼写是否正确

如果发现作者名字输入有误这些情况,也是直接在文献管理工具中进行修改。

(三)参考文献标题错误,包括上下标等

解决方式:如果发现 Endnote导入的文献,存在标题错误(如空格问题、上下标问题)或者缩写方式不统一等问题,不要直接在Word中修改参考文献中的错误,这样一来工作量大,二来后期一旦再次刷新文献,之前的改动就无效了,下次还要重新再改一次。正确方式是直接在Endnote中对文献的格式进行修改,比如Endnote中是可以直接对文献的上下标进行改动的(图\ref{fig5-2})。

\begin{figure}[!htb]
\centering
\includegraphics[width=0.9\textwidth]{fig5-2.png}
\caption{在文献管理工具中,正确编辑期刊的各项信息,包括标题和作者信息等}
\label{fig5-2}
\end{figure}



案例:以下就是以后容易出问题的改动方式,是在Endnote生成的Reference list(参考文献列表)中,直接调整上下标。导致的一个显著问题就是当以后参考文献有变化,重新利用Endnote生成参考文献列表后,这些调整都会失效,然后还需要再次操作一下。一方面增加工作量,另一方面也非常容易遗漏。

\begin{figure}[!htb]
\centering
\includegraphics[width=0.9\textwidth]{fig5-3.png}
\label{fig5-3}
\end{figure}

\section{改动某处时,是否将其他牵连到的相关部位也进行了相应的改动}

每当我们修改了一个地方的时候,还需要考虑到这个部位的改动,是否还牵连文中其他部位的改动,包括关键词、图片、表格、正文所有部位、补充材料等。

\begin{figure}[!htb]
\centering
\includegraphics[width=0.9\textwidth]{fig5-4.png}
\label{fig5-4}
\end{figure}

说明:原稿中,图中的“new Zn anode and electrolyte replacement”等表达方式不佳,在导师指导后,正文部分的文字已修改为“replacement of Zn anode and electrolyte”,但是相应的Figure(图片)中的注释没改动。

\section{全文前后是否有矛盾之处}

家例:However,the most current research focuses are the single component (zinc anode, electrolyte and air electrode) of zinc-air batteries,there is little research available on integral battery configuration. Zinc-air battery is prone to the leakage and volatilization...There are still many studies which induce the componentized cell structures after studies of certain single component on zinc-air batteries to illustrate the application prospect of the target research [6].

说明:本文之前说相关研究少(“there is little research available”),之后又说相关研究还是很多的(“There are still many studies”),前后矛盾。特别是在写作大型综述时,由于前后文相隔比较远,此外可能会借鉴不同作者观点,容易导致出现前后表达矛盾的问题。

\section{第六节检查清单(checklist)的使用方式}

根据过去指导学生的反馈来看,关于使用检查清单进行检查的推荐方法之一是每次只检查一种类型的错误。不要试图一次性地通读文章,把各类型错误都找出来。因为试图一次性地找出各种错误类型,最后的结果往往是每种类型的错误都会有遗漏。

我暂时还没有条件去仔细地查证思考背后的原因。我初步猜测是不是因为人类大脑进化至今,其实还并没特别擅长高通量的并行事件处理,所以通过一次性的阅读,试图发现不同的错误类型,最后的结果是每种错误类型的检查都会有所遗漏。比如,我们可以这次只检查图片中的(a),(b)等标记是不是格式一致,至于其他图片错误或者文字拼写错误、格式错误等,我们可以分解到第二步去检查。当然如果能力提升上来后,也可以逐步增加合并检查的事项。所以我们经常看到的检查清单类型,都是一个个打勾的框,检查完一个类型,打一个勾,再检查下一个类型的问题。检查的时候,注意举一反三,思考同类型的错误,是不是会出现在其他地方,包括标题、关键词、摘要补充材料、图片和表格等等。

在我观察这个检查清单效果时,我发现男性学生一般出错率要高于女性学生。讨论这些,是为了帮助我们可以更好地认识自己,不断提升自己工作的方式。这个我猜测会不会是由进化所决定的。男性远古主要是以狩猎方式生存的,他们需要长时间跟踪猎物,在合适的时候出击。因此,大脑的思维方式是要锁定猎物,尽可能排除猎物以外的一切干扰,以降低能量损耗,将能量用于最重要的那件事情上。而不擅长这个技能的男性,可能被淘汰了,没法繁衍至今。所以,这也是为什么现代大部分男性逛商场,一般而言会直奔主题,直接找到要买的那件物品,买完走人,而不是随意闲逛,也不太会被不想买的东西干扰。这种思维方式就导致一次只执行检查清单中的一个任务是比较好的,不要试图同时执行太多的任务。而远古女性接受信息有“面”接受的特征,她们的生存并不是依赖特定的猎物,而是要在野外尽可能地搜寻到用于生存或改善生活的东西。比如,她们会同时留意到地上的作物、树上的果实,同时还能注意到一些可供穿着、可供装饰的东西。所以,她们才可以发现除了猎物以外的生存方式,这要求她们不是聚焦在一个点,而是要非常发散地捕捉到各个可能的“细节”。所以逛商场的时候,她们可能会被许多东西所吸引。

福特汽车公司也是相对较早发现这一类似规律的组织,并将这种方式用于汽车的流水线作业,从而极大提升了效率。也就是说,每个工人只负责一个工位,只负责重复一件相对简单的事情,通过许多工人和工位的流水线组合完成一辆整车的制造,而不是每个人负责一辆车的制造。

这个概念转化到检查清单上来说,相当于不要试图一次性检查出所有错误,而是每一次的检查只检查特定的错误类型。比如,这次检查语法错误中的时态错误,那你就只看时态,直到时态问题对了后,再来看单复数问题(当然,随着能力的提升,可以将几项不同类型的错误合并检查);又如,检查参考文献时,这次就只检查正文中引用的参考文献是否正确,下一次只看参考文献中作者拼写、卷期号和题目输入等是否正确。

同时,也要学会不断寻找甚至自己开发软件,来高效准确地检查出这些错误,比如借助Grammarly等工具。






\chapter{撰写期刊论文}
心理学期刊就好像20世纪80年代高中生电影里目中无人的运动健将或高傲的富家小姐一样——他们拒绝所有人,只有那些长得好看和坚持不懈的人才能获得他们的垂青。撰写期刊论文对自信心真是一种全方位的打击:成功概率很低;受到批评和拒绝的可能性很大;即便成功了,换来的结果也常常并不丰硕。做研究很有趣,但是写研究报告就毫无乐趣可言。尽管如此,我们还是必须撰写期刊论文,因为科学界通过期刊进行沟通。学术研讨会只是会会老朋友和互相切磋一下最近都在干些什么的地方,研讨会上的发言既没有同行审阅也不存档。 只有发表,才是研究过程自然的终结点。

文件柜里藏满了未能写出的文章。我认识很多科研人员,他们有整橱的数据;有些人还存着1980年以来未能发表的数据,他们希望“有一天能够发表”。他们当然这样希望。因为心理学界对在期刊上发表文章尤其热衷,所以学界提供了很好的资源,帮助初学者学习如何发表论文如:APA, 2010; Sternberg, 2000)。然而,大多数的资源都没能提供关于如何提高撰写论文动力的解决方案。本章将给出一些实用的个人建议。我会提供一些关于如何写作强有力的论文以及如何在面对不可避免的批评或失败的情况下坚持写作的建议。本章的这些建议不会使你爱上论文写作,不过能帮助你减少一
点畏难情绪,多写一点论文。

\section{关于实证型论文写作的实用建议}
写期刊论文就如同写一部浪漫戏剧的电影剧本:你得了解格式。听起来很奇怪,不过你真的应该庆幸有一本《美国心理学会出版手册》(Publication manual of the American Psychological Association)。一旦你知道了什么应该放在哪里,什么绝不应该放在哪里,你就会觉得写论文还是挺容易的。如果你还没有最新版的 《美国心理学会出版手册》(APA, 2010),你应该去买一本。

{\kaishu 1.大纲和准备工作}

在众多不良写作习惯中,“不列大纲”被我排在很靠前的位置——紧随其后的是“戴着粗糙的羊毛手套打字”,仅次于“让我的宠物狗帮我记录”。列大纲是写作的一部分,并不是“真正写作”的前奏。那些总是抱怨文思枯竭的写作者往往是不列提纲的。瞎写一气之后,他们就会抱怨写作如何如何难。这毫不奇怪——如果你不知道自己要写些什么,那么你肯定写不出来。多产的作者都喜欢列提纲。“清晰的思路孕育清晰的作品”,津泽(Zinsser, 2001: 9) 这样说。在和科学界进行交流之前,先把你的思路理清楚。

列大纲能够让你对论文大概有些思路。你打算写多长?你打算花多大的篇幅来介绍已有的研究?你打算把这篇论文写成一篇简单的报告还是一篇常规的学术论文?这些通常都取决于你自己,不过我建议你尽量简洁。学术期刊多年来被大量论文充斥着,近年来心理学的趋势是推崇简短的文章。有不少高质量的期刊只刊登短小的论文(例如《心理科学》),还有一些最近开辟了短小报告的专栏。短即是好。想象一下你在阅读学术期刊时的心情。你是希望快点看到结尾呢,还是希望作者再洋洋洒洒地写上14页?不要把所有的东西都塞进一篇文章里。你在职业生涯中会写很多文章,所以你可以把
一个遗漏的观点发展成为一篇新的论文。

内在的读者——一个假想的会阅读你论文的人——将帮助你来做决定。你应该如何详细地解释视觉注意力的竞争理论?你应该对某种统计方法加以解释,还是假设所有读者都已经理解?研究领域的其他专业人士——包括和你有同样研究兴趣的教授和研究生——是你最大的读者群体。你应为他们而写作。还有小部分另外的读者群体,包括本科生、记者、相关领域的工作人员和其他读者(例如博客博主或幽默作家)。对很多读者而言,英语是他们的第二甚至第三外语;当你被那些生僻而时髦的词语诱惑的时候,要考虑一下他们。为了更好地定位你的读者,你可以列一个你想要投稿的期刊名单。《实验心理学杂志:普通心理科学》通常拥有广泛的读者;有的期刊,例如《视觉认知和自我认定》,读者面就窄很多。当你为专业人士写作时,你可以假定你的读者知道相关领域的理论、发现和方法。你可以用通顺而专业的语气写作。你的目标是要让读者觉得你是一个值得交谈的正常人——别太严肃,也别太随便。


{\kaishu 2.标题和摘要}

大多数浏览你论文的读者只会看看标题和摘要,所以要认真对待。标题必须兼顾概括性和具体性:告诉读者你的论文是关于什么的,但是又不能太具体而显得过于功利和无趣。如果你想写一个时髦又幽默的标题,你要考虑到十年以后的情况如何。将来的读者能不能理解这种幽默?在数字化时代,读者通过电子资料库的搜索引擎来搜索标题或摘要从而找到你的论文。因此,在摘要里一定要包含你希望定位你的论文的所有关键词。比如,在我所有关于自我认知的文章中,我都会在摘要部分反复使用同义词,包括“自我关注”“自我关心”“自我意识”等。几乎所有人都是最后才标题和摘要的,你只要随大流就可以了。

{\kaishu 3.导论}

导论部分包含研究中的所有细节。在一篇论文中,导论部分的被阅读率是最高的。鉴于此,这一部分也是最让写作者们惧怕的。很多人告诫初学者,导论是没有固定格式的(例如:Kendall, Silk \& Chu, 2000)这是一派胡言——导论当然是有格式的。好的作者都遵从好的格式,很轻松就能辨别出来。

在导论的开始部分,先总体概括你的论文 ,一般长度为1$\sim$2页。 在这一部分,你应该描述你研究这个问题的缘由及理论依据。这一部分的目的在于论证本文存在的意义并吸引读者,同时帮助读者对全文建立一个总体概念。

总体概述全文后,用一个小标题来提炼导论的第二部分。这个小标题或许能够呼应你论文的题目。导论的第二部分是核心部分:这一部分你将描述相关理论,总结已有研究并更为具体地阐述你所做研究的目的是为了解决什么问题。利用好小标题和副标题来突出重点。如果有两种理论,那就为两者各取一个副标题。第二部分要紧紧围绕第一部分中你提出的研究议题。

第三部分的小标题应该是“目前的实验(The Present Experiments)”或“目前的研究(The Present Research)”。你已经在第一部分针对提出的问题给予了概括性的描述,在第二部分概述了相关理论及研究成果。现在,读者们理解了你所提问题的背景和重要性,那么在第三部分,你应该介绍你的实验,并解释为什么这个实验能够解决你提出的问题——这一部分可能有1$\sim$4页,根据详细程度不同而有所区别。这一部分的结尾通常是接下来研究方法部分的开头(方法或研究1)。

上述格式向读者介绍了你论文研究的问题(第一部分),总结了与问题相关的理论及研究(第二部分),还清晰地阐明了你的研究是怎样解决你的问题的(第三部分)。这样的格式能够引领读者的思路,也能帮助作者时刻围绕中心议题展开论述。当然会有例外——篇幅较短的报告,或许仅仅另起一段就足够了,不需要用小标题——但是上述格式适用于绝大多数的论文。

导论部分应该介绍您的研究,而不是不遗余力地把与问题相关的所有的研究从头讲一遍。简短的报告可能需要2$\sim$3页的导论;想要向期刊投稿的好大喜功的作者的论文可能有12$\sim$20页的导论。在写作一般的研究论文时,导论部分最好不要超过10页。

{\kaishu 4.研究方法}

研究方法部分可能看起来并不光鲜亮丽,但是它反映了你是如何认真对待研究的(Reis, 2000)。好的研究方法部分能够让其他的研究人员复制你的研究。与导论部分相似,方法部分也有固有的格式。这一部分由若干个小的部分组成。首先,研究对象或研究对象与设计,这一部分讲述样本的规模和特征,如果是实验研究,还包括实验设计。如果你的研究还涉及仪器——比如心理学仪器、非常规软件、反馈控制板、声控开关——你还需要用一个专门的部分叫做“仪器介绍”。测量方法部分适用于研究中包含集合测量、测试和评估工具的情况,例如神经认知测试、兴趣总结以及态度的自我报告机制和个体差异。

在完成上述各个部分之后,你将进入过程部分——研究方法部分的核心。在这一个部分,描述你做了什么和说了什么。读者往往对过程部分特别关注,所以不要让他们感觉到你在刻意隐瞒什么。尽可能详细地介绍自变量和因变量的情况。你的目的是将你的过程与已经发表的文章中的过程联系起来。如果你的实验使用了一种巳有的操作方法,那么即使这种方法已经广为人知,也请列明之前使用过类似方法的实验。如果你使用了一种新的操作方法,那么需要介绍以前用过类似方法的研究或者能够证明你的方法可行的已有研究。如果你的自变量包含分类(例如:社会焦虑程度高与低),那么需要介绍这种分类的依据(临界点、规则、惯例)并列举用过类似分类方法的已有研究。将你的过程与过去已有的研究联系起来能够降低读者对你的方法科学性的质疑。

读者们希望了解你是如何测量因变量的。如果你的因变量是经过慎重筛选后确定的,你应该列明以前使用过同样测量法的论文。如果是专业测试,那么要提供详细的测试手册,同时介绍最近的研究中有哪些用过同样的测试。如果你的因变量是特设的,比如你所写的自我报告项目,则要详细列明每个项目并介绍用过类似项目的论文。如果是自我报告测量法,则要标明数值——例如,7度可以是1$\sim$7, 0$\sim$6, -3$\sim$3——每一个数值都对应了相关标签(比如:1=一点也不,7=非常)。如果你的因变量测量涉及生理学或行为学的相关理论,那么要简要介绍一下过去哪些研究能证明你的测量方法是有效的。

如果你的论文有一系列的研究,你可以通过说明后续研究方法与第一个实验相同的办法来节省点篇幅。如果三个实验都用了同一种仪器,你就不必重复介绍三遍了。在介绍后面的实验时,只需说明它们与之前使用的仪器相同。

{\kaishu 5.结论}

结论部分用于描述你的分析。初学者往往觉得有必要报告有关数据的每一种可能的分析,或许因为论文答辩组成员希望看到这样的分析。但是,期刊论文应该是简明扼要的:只报告与你讨论的问题相关的结果。差劲的结论部分只是一长串的句子和统计;优秀的结论部分则讲述了一个故事(Salovey, 2000)。首先,在结论部分的开头,探讨一下你的研究的可靠性。这一部分可以分析自我报告测量法的内部逻辑性、预测评判间一致性、分析实验操作检查或者解释简化与处理数据的方法。

其次,按照逻辑顺序描述你的分析。这并没有一个放之四海而皆准的形式——根据你的方法和假设而有所区别——但是应该尽量把你的主要发现放在聚光灯下。萨洛维(Salovey, 2000)曾经建议把最有趣和重要的发现放在最前面。在描述结论的时候,不要盲目地一个实验接一个实验地接连陈述。介绍每个实验的时候,重新强调你的假设和报告数据,然后讨论这些实验的意义何在。但是初学者会反对:“针对结论部分的讨论应该放到下一部分——问题讨论部分!”这里存在一个由本科阶段学习带来的误区。结论部分不应该只是数据的展示。不要只汇报一个单项的趋势变化,然后说变化很明显。描述你的假设,汇报数据,然后说明这些发现的意义。哪个组的数据高于另一个组?结果是否与你的假设相符。

再次,多用图表或表格来增强结论部分的条理性。我常常写这样的论文评语:“作者应该列一张统计数据表。”如果是实验性研究,设计一张包括方法、标准差和单位尺寸的表格。更进一步,你可以加上95\%的置信区间——评论者会赞赏你的开放性,读者也能够对你的数据加以评论。如果是相关性研究,设计一张包括方法、样本量、置信区间、内部相关性评估和相关矩阵的表格。有了这些信息,读者能够自己创造和测试你提供的数据的结构方程模型(Kline, 2005)。没有法律规定你不能同时使用图表和表格:图表是为那些想要了解宏观数据样式的读者准备的,而表格则是针对那些想要了解具体细节的读者而设计的。

{\kaishu 6.讨论}

如果你的文章包含多项研究,那么每一个结论部分后面都应该紧跟一个讨论部分。这些讨论部分比一般性讨论部分涉及的面窄。它们总结了每项研究的结论并讨论这些研究如何探讨了论文的核心问题。讨论部分同样应该讨论实验的局限性,比如在研究过程中获得的意想不到的结果或者问题。如果你的讨论部分只是对结论部分的总结,你可以另外设计一个“结论与讨论”部分来作为补充。

{\kaishu 7.一般性讨论}

在一般性讨论部分,你可以将自己的研究放在其他理论或既有研究的聚光灯下重新审视。这一部分的开头应该对你的问题和发现作简单回顾:一般一到两段就够了。好的一般性讨论部分多种多样——你的问题、方法、研究领域都决定了你应该讨论些什么——但是这一部分应该很简短。想象一下你如何阅读一般性讨论。你是快速阅读、跳过,还是抱怨作者没完没了地讨论研究的每一个细节问题?应该尽量让这一部分短于导论部分。如果你愿意,可以在一般性讨论部分的最后用
一段的篇幅总结全文。

本科阶段研究方法课的老师会告诉你,应该在一般性讨论的最后谈谈你的研究的局限性;论文评议组成员或许也希望看到这个部分。讨论局限性在教学阶段或许是有益的,但是到了向专业期刊投稿的阶段就往往没有太大意义。有些局限性是普遍存在的:的确,如果能够有更大量和有效的样本就更好了;能够包含更多的测量方法也不错;的确,可以想象,在未来如果有研究能够运用更多的测量方法分析更多的样本,结论就会有所不同。别把你的读者当傻子——每个人都知道在类似的研究中存在的这些局限性。有些局限性在某一领域的研究中是普遍存在的。认知心理学家知道自己用了人为设计的计算机实验;社会心理学家知道自己运用了简单方便的本科学生作为研究对象。专家们都知道你的研究存在这些普遍的局限,所以不用再浪费时间来写这些显而易见的东西。相反,你可以用一些篇幅来讲讲你的研究特有的局限性。但不要太过于夸大这些局限性——提出来,然后巧妙地解释为什么它们并没有乍一看上去那么严重。

{\kaishu 8.参考文献}

参考文献部分用于收集整理对你的论文观点产生影响的资源信息。将你的作品放入科学领域,你的参考文献告诉了读者你如何看待你的研究。参考文献要有选择性——你不需要把读过的所有与研究有关的文章都列入参考文献,你也不应该把没有读过的书或文章列入参考文献。阅读论文的专家一眼就能看出你引的是二手资料。参考文献部分虽然不及导论部分光鲜亮丽,也没有结论部分强健有力,却值得你认真对待。作为一名论文审稿人,我见过很多草率的参考文献。懒惰的作者总是在挑战美国心理学会的写作格式,也常常忘记为行文中的引文作注释。“有什么大不了的?”有些人会说,“只是参考文献而已”。你的同仁能够看出你对参考文献的草率;应该让富有批判精神的匿名审稿人看到你最好的作品。

老练的作者利用参考文献来提高论文被自己希望的审稿人阅读的几率。编辑们在考虑你的论文的审稿人选的时候,常常直接翻到你的参考文献部分,看看你都引用了哪些人的作品。我不确定这个小技巧是否有效,但是试试也没有什么坏处。还有,别忘了在你的新作中引用你自己之前的文章。自我转引被很多作者认为是厚颜无耻的自我吹嘘。我就遇到过很多人对自我转引犹豫不决,他们大多是初学者。引用你过去的作品能将你最近的论文和你的研究脉络联系起来。如果有人对你最近的文章很感兴趣,那么他/她也许会有兴趣阅读你的其他作品。自我转引能够帮助他们找到这些文章。


\section{提交论文}
当你准备把论文投出去时,它应该是条清理晰、几近完美的。如果你总是想:“我先把它发送出去吧,等修改的时候再好好整理”,那我劝你还是打住,马上着手修改为好。只有受虐狂才会把潦草的草稿发给期刊编辑。原稿总是能够吸引眼球和获得审稿人的尊重,并且能够向编辑们展示你是一个严肃的值得信任的专业人士,相信你能够很好地根据修改意见进行修改。在你提交稿件以前,一定要花些时间阅读期刊网站上公布的投稿须知。请仔细阅读,因为每一份期刊的要求都有所差别。大多数期刊都接受电子稿件,稿件一般通过电子邮件或者在线提交软件提交。

不论你通过何种方式提交你的论文,你都需要写一封介绍信给编辑。有些人选择标准、简洁的信件;有些人则喜欢添油加醋地强调文章的优点和重要性。我曾经问过一些编辑重要期刊的朋友偏好何种介绍信。他们无一例外地喜欢简洁的介绍信。它包含一些程式化的内容:论文标题、作者的电子邮箱地址,以及一些常规的内容(这份稿件没有投往其他期刊,稿件资料的收集方式均合理合法,等等)。有一个副主编曾经提到他从来不看介绍信,因为在线提交系统使他很难阅读。另外一位说她更希望被文章打动,而不是被介绍信打动。

在介绍信里,你可以建议几位你认为合适的审稿人,以及几位不合适的人选。我从几位编辑朋友那里获知,他们往往会尊重“不合适人选”的提名,而对“合适人选”有所保留。也许期刊的某位副主编是非常适合你的论文的审稿人选,如果你乐意,你可以建议编辑把你的稿件发给这位副主编。(虽然我尝试过很多次,但是最终我的稿子从来没有被送到我建议的人那里去。)


\section{读懂审稿意见并重新提交论文}
我在随意翻阅一些较早版本的《儿童发展》时,无意中发现了一篇20世纪70年代早期的评论文章。作者描述了同行间互相审稿的流程,提到通常的反馈周期是6周。想想看,30年前,作者把像砖头一样厚的稿子寄给编辑,编辑再把稿子寄给审稿人。审稿人把他们的意见用打字机打出来,再寄还给编辑。编辑们打一份执行意见信(action letter),留存一份副本,再把执行意见信和审稿人的审稿意见寄给作者。今天,作者、编辑和审稿人通常用电子方式交流,用先进的在线系统投稿、向审稿人和编辑寄送通知单等,避免了由于邮寄信函而造成的延误。当你在等候审稿结果的时候,应该好好感谢高科技带来的便利。

当编辑的执行意见信寄到的时候,他往往会在信中总结审稿人的主要意见,并告知稿件是否被采用。结果可能有三种:采用、要求修改、拒绝。

\begin{itemize}
\item 采用的情况比较容易理解。编辑通常会说你的稿件被采用了,然后会要求你填写一些表格;有时候编辑会让你在稿子发表前做一些小的改动。原稿直接被采用的情况并不多见。即使他们很喜欢你的论文,他们也常常会要求你作些删减或添加内容。有的编辑偶尔会无条件地接受一些论文——所以说要十分注重第一稿的质量。
\item 在希望犹存的情况下,编辑会要求你修改论文。这一类回信差别很大,有的令人无比振奋,感觉距离稿件被采用只有一步之遥了;有的则列出一大堆修改意见,让人沮丧不已。有的门开得比较大,只要求做一些简单的修改,例如重写某个部分或添加某些信息。有的门只留了一道缝,要求要做大的修改,例如重新收集数据或是重新考虑研究的概念基础。有时候,编辑们会告诉你,他们将把做出重大修改的论文作为新稿件加以处理。
\item 在大门彻底关闭的情况下,编辑不希望再看到你的论文。有时候,退稿信会鼓励你把稿子投到别处;有时候,编辑会给你寄来一台碎纸机,让你彻底毁了这篇论文。如果门关上了,你就别再重新提交论文来挑战编辑的底线了。
\end{itemize}

即使是老练的研究者也常常搞不清楚编辑是否愿意给论文的修改稿一次新的机会。“拒绝”这个词并不一定意味着你不能重新提交论文。很多编辑都会对未被采用的稿件使用“拒绝”这个词。他们拒绝了你的第一稿,却有可能会接受修改稿。我猜想有些不喜欢说“不”的编辑会用一些让人泄气的话来拒绝作者——“如果您增加三个实验并重写导论和一般性讨论部分,我们会很愿意重新考虑您的修改稿。”当不确定的时候,把审稿信给朋友参谋一下,或者给编辑去封简短的邮件确认一下。

如果大门还开着,你要考虑清楚自己是否愿意修改。编辑可能希望有新的数据、新的分析,或是重新组织的某些部分。这个项目是否值得付出更多呢?默认的首选应该是修改并重新提交。你要记住,所有期刊的退稿率都非常高。如果你收到修改论文并重新提交的邀请,你已经在为降低退稿率做贡献了。如果这份期刊很有权威性,你应该努力修改,例如增加一个实验。如果这篇论文并不那么重要,你或许可以试试其他期刊,而不是费时间重新收集数据。

决定要修改并重新提交论文之后,你应该做个计划。仔细研究编辑的来信和意见,并提炼出修改要点。(不要用“可修改的要点”这样的词——这样的表述太模棱两可了,就像“可饮用的”或“可做的”一样,似乎可做可不做。)修改要点就是要修改的部分。仔细阅读编辑的来信和意见,把所有提示需要修改的意见都标记出来。可能是文字上的修改——增加、删减或是重写——或是修改分析部分。也可能是比较大的改动,比如增加或删除某个实验。很多审稿意见天马行空,很长的一篇只有寥寥几个修改要点。在你标出修改要点后,就尽快修改。在第三章里,我把修改稿件放在目标列表中比较重要的位置。因为它们离发表更近,所以不要磨磨蹭蹭的。有的编辑会给个修改期限,如60天或90天。

当你重新提交稿件时,你需要写一封再次投稿的介绍信来说明你是如何处理批评与意见的。至于应该写一封简短的信来标出大的改动之处,还是应该列一份详细的改动清单,依我私下与编辑们交流的经验来看,他们更喜欢翔实仔细的信函。大多数的编辑抱怨作者的信写得太简略(“我们修改了很多;我们希望您能满意”),作者们要么拒绝修改,要么就只提那些作了修改的部分,却从不解释为什么有些地方未作修改。所以,在信里详细地列出你做了哪些修改,哪里没有修改,这有助于编辑接受你的修改稿。

二稿介绍信应该具体和有建设性;你应该坦诚地、透彻地讨论所有修改要点。那些成功发表大量作品的作者都是写介绍信的高手。这些信很好地介绍了你所做的修改,并向编辑证明你很好地处理了反馈意见,你是一位严谨的科研工作者。简短、模糊的信让人感觉作者要隐瞒什么;长而翔实的信显得作者态度积极和真诚。信也要写得礼貌和专业——你的信不是为了显示对一位偷懒的审稿人的不满,也不是为了向一位好挑刺的审稿人展现你的骄傲,更不是为了夸耀你高超的统计水平。这些是很有诱惑力,但还是应该以科学大义为重。

我收集了大量成功的二稿介绍信,写信的人都是我的同事,他们都发表了大量的论文,也是很多期刊的编辑。以下列举一些要点。

(1)开头部分应该感谢编辑给予的建议和再次提交论文的机会。虽然你会觉得文章被退改不如被直接采用来得令人高兴,但这起码比被直接拒绝要好得多。

(2)给每一个修改要点起一个小标题。很多作者根据审稿人的序列来组织这封信。通常的做法是对应审稿人1、审稿人2的评论来拟定小标题,以此类推。每个部分都应该覆盖每一位审稿人的每一个意见,并用数字标识。用数字标识简洁明了,并且便于查找已经出现过的评论。例如,也许两位审稿人都提到应该增加样本的一些细节。虽然你已经讨论过审稿人1的意见,但到了审稿人2的时候,还是需要重复一下。只需要很快地重复一下这个意见,然后指明与上述第几条重复即可。

(3)每一个修改要点的阐述应该包含三部分。首先,简单总结意见或是批评的内容。其次,解释你针对这一评论做了哪些修改;如果可能,请列出这一要点具体在你论文的第几页。最后,阐述你的修改怎样回应了评审意见。

(4)编辑们并不指望你对每一条意见都一一修改,但是他们想知道你不做修改的理由。我见过最极端的二稿介绍信,作者固执地拒绝做一些无关紧要的修改,例如把几张小的表格合并成一张大的表格或是删掉10\%的文字。好好选择你需要修改的部分。如果你不接受修改意见,要在你的介绍信里详细地说明你为什么不愿意修改。

(5)注意保持专业性。别显得卑躬屈膝或是刻意谄媚。编辑们并不认为审稿人是天才,所以他们也不希望你把审稿人的意见形容成:杰出的、前所未有的、出色的、深刻的,等等。请你把自己放在编辑的位置上思考一下。一封溜须拍马的二稿介绍信到底是会打动你,还是会让你觉得“这个人真虚伪”呢?

一封好的二稿介绍信会让你看起来是一位认真的科研人员——其实你本来就是。那些认真对待批评意见的人写的论文值得被发表。有时候我会花比修改论文更长的时间来写这封信。我的一篇论文的二稿介绍信(Silvia \& Gendolla, 2001)有3200字,差不多和这本书的第五章一样长。我发表的很多论文都没有3200字。


\section{“如果他们拒绝了我的稿件怎么办?”}
很多作者很害怕收到负面的反馈和遭到拒绝。传统的成就动机理论显示有两个最主要的影响表现的动机:成功需求和避免失败的需求(Atkinson, 1964)。情境因素能够夸大这些动机,而写论文看起来能够唤起作者对避免失败的需求。很多作者——尤其是学界初学者——对“被拒”总是耿耿于怀。他们担心编辑们会怎么说;他们想象某一位审稿人在读他们论文的时候皱眉的样子;他们非常害怕收件箱里的退稿信。

避免失败的本能会让人们反复地问:“如果他们拒绝了我的稿件怎么办?”他们当然会拒绝你的论文。你写论文的时候就应该假设会被拒。决策理论指出,在不确定的前提下,基本概率是预估结果的最合理的依据。如果一份期刊拒绝大概80\%的稿件,那么稿件被接受的基本概率就是20\%。在缺乏其他信息的情况下,理性的判断是,你的论文有20\%的概率会被接受。因为没有期刊的拒绝率低于50\%,所以我假设我投的稿件都会被拒绝。这是唯一合理的结论,而且我被拒的次数也验证了我对理性分析的坚待。

“真是暗无天日啊,”你也许会说,“如果你知道你的稿子会被拒,你怎么还能打起精神来写呢?”首先,我们不应该寻找写作的动力,而是应该坚持执行写作计划,不论刮风下雨 其次,初学者往往觉得只有他们才会收到退稿信。其实那些已经发表了很多论文的作者同样会收到很多退稿信。心理学界最多产的作者在一年内收到的退稿信可能比有些写作者十年里收到的还要多。我甚至觉得被拒的基本概率反倒让人觉得安心。我对将发生什么并不确定,所以当我收到退稿信的时候我并不觉得太糟糕,而且在我完成论文以前,我也不会放纵自己沉溺于自己的文章即将变成铅字的幻想之中。

如果你假设自己的文章会被拒,你就能写出更好的文章,原因是你对避免失败的需求被屏蔽了。为了避免失败而写作的作者所写的文章读来小心翼翼、空洞而充满犹疑。他们总是设法使自己的文章看起来不坏,而不是看起来更好。读者可以感受到这种恐惧。相反,为了成功而写作的作者,他们的文章读起来充满信心和控制感。这些作者把重心放在作品的长处上,强调研究的重要性,传达着一种颇有说服力的自信。

审稿人是否会讨厌你的文章?是的,有的时候他们的确会讨厌你的文章。以下是我最近收到的一封退稿信的节选。在审稿意见总结部分,编辑写道:

{\kaishu 两位审稿人都认为您的论文未达到发表的水平。一位审稿人认为您的论文意义不大,对相对立的理论有所误读,结论与研究的证据未能很好匹配,而且写作也不甚精确。另一位审稿人认为论文未能推动建立完整而准确的模型,论证不够有力,部分重要的研究和观点缺失,且作出了一些错误的理论假设和批评。}

而且这还是通过编辑转述的——其中一位审稿人真是挑剔。不过这也没关系。我提取了审稿意见中的一些修改要点,修改了论文,然后投给了另一份期刊。考虑到基本概率,也许还是会被拒。

有时候,拒绝的决定是不公正、刻薄甚至毫无道理的。有时候你能够看出编辑或审稿人并没有仔细阅读你的论文。请你克制住向编辑抱怨的冲动。我听说有些人向编辑投诉,怒气冲冲地指责审稿人又懒又不称职。也许因为编辑往往和审稿人的私交很好,这些信大多石沉大海。有人建议你应该写封投诉信发泄一下,但是不要寄出。这似乎更不合理——为什么要浪费你的写作时间来做这些毫无意义的事情呢?把你的时间用于修改论文吧。世界是不公平的(p<0.001),所以你只需吸取审稿意见中有用的建议,修改你的论文,然后投到别的期刊去。

为了写得更多,你应该重新考虑一下你对待被拒和发表的认知模型。被拒就好像是为发表文章而交纳的销售税:你发表的论文越多,你收到的退稿信也会越多。如果你按照本书的建议来做,你将很快成为你们系里收到拒信最多的人。



\section{“如果他们要我修改所有的部分怎么办?”}
期刊是科学界公开的记录。你的论文将被印在无酸纸上,被永远存放在图书馆的书架上。如果人们能够把自己的研究与其他人的研究联系起来,在研究中阐述自己的观点,合理地分析数据并客观公正地说明自己和其他人已经取得的成果,那么科学进步的步伐就会更快。期刊不是心理学家宣传个人观点的论坛——在简报或学术会议上你可以那样做。学界对公开发表的论文的要求很高,并运用同行评审的方式来控制质量。你会被要求修改你的论文;有时候这些修改涉及的范围很广。如果这让你觉得不舒服,那你会不情愿地听到一个事实:最终得以发表的论文质量普遍高于初稿。能够发表的论文,其观点更集中,较少自相矛盾,更严密。互相审核制对于作者来说是令人厌烦的,但是这一制度是达成心理科学发展目标的核心所在。

\section{合作撰写期刊论文}
有时候,我们需要很多人合作来完成一项研究,但其中大多数人不会参与论文写作。我问过很多人如何与其他作者一起合著论文,几乎所有的人都说是其中一位作者撰写了大部分内容。合著者们一同列提纲,但是由其中一位作者完成写作。论文写完以后,所有的作者一起阅读、讨论、做必要的修改。对这一模式的改进办法是把各个部分分派给不同的作者。通常的做法是让一个人来写方法与结论部分,另一个人写剩余部分。不过,我也发现有人做到了真正意义上的“合著”。有一对合著者在电脑前放了两把椅子,讨论写些什么,然后把键盘传来传去。另一位说他和一位同事把两台电脑搬到一个房间里,然后一起完成了他们的课题基金申请报告。这样做使他们能够解决申请报告中纠结的问题,也能够随时向对方提出问题。可见,通过合作来完成论文写作也是可行的。

你要小心选择合著的人,不要在还未详细讨论由谁来写的情况下就投入一项与人合作的研究。如果你的搭档是一位突击写作者,请对他承诺会很快写完或是对研究表现出的激情持谨慎的态度。热情不代表投入。如果你无法信任你的搭档,那么你应该自己写初稿,并确保自己是第一作者。有时候,你辛苦写完了初稿,你的搭档却永远无法完成修改的工作。你应该在给他们初稿的同时给一个最后期限。例如:“我希望这篇论文能够在两周内提交,所以麻烦你在这之前回复我。”期限一过就提交论文。我的一个朋友给一位拖拖拉拉的合著者写了封邮件,邮件主题是“不带你玩儿了”。这招很管用。

对于研究生们来说,拖拉的合著者是个大麻烦,特别是如果一起合著的人是系里的导师。很多学生抱怨导师拖延了他们的论文——有的导师给学生的论文写意见要拖上好几个月甚至好几年。对学生来说,催促导师是有难度的,所以得想些办法。试试让其他人来催你的导师。为什么不向系里的其他老师抱怨?如,系主任或研究生项目负责人。如果这也没用,把本书的这一章节复印一份,匿名放到你导师的邮箱里。这一举动虽有些鲁莽,但希望能够把导师的注意力吸引到你的论文上来。最后,给导师一个期限,超出期限之后你自行提交论文。如果你的导师不愿意读学生的论文并提出意见,说明他缺乏对研究生教学和科学发展的投入。你可以告诉他,“我真的需要在4周内提交这篇论文”,然后在2$\sim$3周后提醒他。


\section{写评论文章}
在写了那么多实验论文之后,或许是时候考虑写点评论文章了。评论文章的读者众多:寻找新观点的研究人员、在全新领域学习的学生、备课的教师、关注心理学最新动态的政策制订者等。论文写作其实不难,只要你掌握了美国心理学会的要求就容易上手,但是评论文章不一样。写作动机层面还是一样——坚持执行你的写作计划,但是具体组织安排层面就很不同。研究者可以出于不同的目的写作各种不同类型的评论文章,结构、方法也差别很大(Cooper, 2003),而且没有统一的格式。

正因为评论文章千差万别,你必须做好计划。首先要想清楚评论文章的篇幅。有的期刊倾向于发表短小精悍的评论,例如《当代心理学研究方向》;另一些,例如《心理学探究》《心理学公报》《心理学研究》,都接受篇幅较长、较全面的文章。你想写多长?其次,你要考虑你的读者群是谁。除了综合性的评论期刊,心理学领域还有很多评论是写给特定读者的,例如《实验心理学研究》和《人格与社会心理学研究》等。你希望你的读者面广一些,还是希望你的读者是一小部分专业研究人员?

当你对篇幅和读者有所考虑以后,你需要列一个提纲,写明你的核心观点。评论文章必须提出自己原创的观点,而不是简单地复述已有的研究。最糟糕的评论文章是把对其他文章的描述弄成一个大杂烩。读一篇没完没了的评论——这篇文章发现了这个,那个实验证明了那个,另一项研究说明了这个——就好像看着衣物在洗衣机里不停翻滚,但是最终洗衣机里好歹还会有洗干净的衣服出来。为了提出你原创的观点,可以参考创造性方面的专家提出的关于“解决问题”和“发现问题”的区别(Sawyer, 2006)。一篇“解决问题”的评论描述一个问题(例如一个有争议或模棱两可的研究领域),然后提出解决问题的方法(例如一种新的理论、模型或解释)。一篇“发现问题”的评论提出一个新的概念或提出一个值得关注的新话题。真正好的评论应该包含解决问题和发现问题两方面。例如,解决具有争议的两种理论,通常为未来的研究指明方向。你想解决的问题是什么?你的结论里又有哪些新的观点?

评论文章最常见的缺点就是没有原创的观点。很多作者把研究改头换面解释一番,却没有结论;另外一些作者讨论了互相对立的理论却没有解决方案。有两个原因导致了上述问题:首先,如果作者本人没有新的观点,他当然无法提出新的观点。有时候就是这样。在阅读了大量的文献之后,你或许会发现你并没有什么要补充的。如果是这样的话,你就不要执意去写一篇评论文章,仅仅证明你花了那么多时间来阅读了文献。其次,有些作者不列提纲。他们在一堆文章旁边坐下来,开始描述每篇文章写了什么,然后加上一小段“评论总结”,就完事了。一个复杂的项目需要一份强有力的提纲——如果没有提纲,你的观点就会被淹没在浩如烟海的已有研究中。那些不喜欢列提纲的人不应该写评论文章,而应该到本地的动物收容所领养一条狗,因为狗不会因为他们这样荒谬而自以为是的习惯而嫌弃他们,狗会一如既往地爱着他们。

如果你有好的观点,别藏着掖着。你的观点应该写在文章的开头几段里。评论文章的第一部分,你应该大致介绍一下文章的核心观点,分几大部分,然后透露一下你打算讨论的原创观点。按照时间顺序来写评论——理论一、理论二,然后分析,这看起来很有吸引力,可是千万别这样写。评论文章包含太多的信息,所以你需要在文章的开头就给读者一个清晰的思路。与出色的推理小说不同,好的评论文章在第一页就揭开了谜底。

写作评论文章看起来有一定难度,确实不容易。这也是突击写作者很少写评论文章的原因:有太多东西要读,要消化,要写。但是善于反思、有规划的作者就没什么好害怕的。如果你有一个时间表,那写评论文章也绝非难事:你有清晰的目标、不可回避的时间安排、好的习惯,所以完成评论文章只是时间问题。当你决定要写一篇的时候,花一点写作时间来收集好的建议。鲍迈斯特和赖瑞(Baumeister \& Leary, 1997)写了一篇非常棒的写作评论文章的指南;你还可以参考一下本(Bem, 1995) 、库伯(Cooper, 2003)和海森堡(Eisenbery, 2000)的建议。

\section{小结}
当人们在为第一篇论文苦苦挣扎的时候,很多作者哀叹道:“为什么他们一点也不在乎我的研究?”如果“他们”指的是广义的世界的话,我向你保证他们真的一点也不在乎你的研究;但是如果“他们”指的是同一领域的研究人员,那你应该想到他们其实是有一些兴趣的。记住你是在为与你有着共同研究兴趣的专业人士写作具有技术含量的文章。或许你的文章在找到归宿之前被拒绝了一两次,但是真正好的文章总会找到归宿。为了写得一手好论文,你必须充分掌握文体,提交干净整洁的初稿,还要善于写出漂亮的二稿介绍信。你会发现期刊的世界并不可怕,只不过速度真的很慢。
\chapter{统一性}
要写好就得实践,此乃至理名言,其言名在于其理真。要写好的唯一办法就是强迫自己保证每天都写。

你若是为报社工作,要求每天写两三篇文章,六个月后你的写作能力就会大为改观。这并不是说你就此成为作家,你的文笔也许还会杂乱无章、充斥陈词滥调,但你在练习如何将语言付诸笔端、增强自信心、发现最常见的问题。

一切写作最终都是在解决问题。这个问题也许是到何处去寻觅事实,或是如何组织材料;也许是方法、角度,或是语气、风格。无论如何,你都得面对问题,解决问题。有时你会苦于找不到解决方案而绝望。你会想,“就是活到九十岁,我也摆脱不了这个困境。”我自己就常这么想。一旦解决了问题,我就像外科医生一样去除了第500个阑尾。这些我都经历过。

统一性是好的写作的保障。所以首先要明了各个部分的统一性问题。统一性不仅能避免读者晕头转向、误人歧途,还能满足读者潜意识中渴望秩序的需求,保证一切都在掌控之中。因此,在诸多变量中做出选择,坚持你的选择始终如一。

选择之一是人称代词的统一性。你是以亲历者的第一人称,还是以旁观者的第三人称写作?或是以第二人称写作?这第二人称可是体育记者的宠儿,是海明威惯用的技法。(“你知道这次一定是巨人之间最刺激的冲撞,之前你绝对没从直播间见过,不过你不再是乳臭未干的臭小子了,一定扛得住!”)

另一个选择是时态的统一性。大多数人习惯于用过去时写作。(“有一天我去了波士顿。”)而另一些人用现在时写作得心应手。(“我现在就坐在扬基有限公司的餐车里,火车缓缓进入波士顿站。”)但如果时态换来换去,就会使人手足无措。这并不是说你只能用一种时态;时态选择的最重要目的是能够使作者处理好时间的种种变化,从过去到假设中的将来。(“从波士顿车站打电话给母亲,我这才意识到假如我告诉她我将到来,那她就会等我了。”)但是,面对读者你必须选定一种主干时态,在这之间可以瞻前顾后。

还有一种选择是语气的统一性。你可能想用悠闲随意的语气面对读者,就像《纽约客》所竭力打磨的那样。也许你想用一本正经的方式向读者描述一件大事,或陈述一系列事实。两者都行。其实,任何一种语气都行,但是不要两三种混搭。

对于还没有学会掌控的作者来讲,这种要命的混搭司空见惯。游记是最明显的例子。“我和妻子安娜一直想去香港,”一位作者这样开始,往事历历在目,“去年春天的一天我们偶然看见一幅航空公司的招贴画,于是我说,‘咱们去旅游!’孩子们都长大了。”这位作者这样写,然后继续兴高采烈地详细描述他和妻子如何转停夏威夷,如何在香港机场兑换钱出笑话,如何找到旅店。好!他亲身带我们旅游,我们也切身体会到他和妻子的经历。

但突然他转向旅行手册。“香港可为好奇的观光客提供众多令人神往的体验,”他这样写。“你可以从九龙乘华丽的渡轮,观看无数舢板在拥挤的海港疾驶,而感到惊叹不已;或者花一天时间逛一逛传奇的澳门街巷,体验其作为走私密谋老窝的斑斓历史。你还可以乘老式缆车爬上……”然后作者又写回到自己和妻子找中餐馆吃饭,一切又不错。大家都喜欢那儿的饭菜,随后作者又讲述了一段自己的历险经历。

之后作者突然在写导游手册:“进入香港,首先要有有效护照,但无须签证。一定要接种甲肝疫苗,按照医嘱看是否需要接种伤寒疫苗。香港的气候适宜,只是七月和八月……”作者本人和妻子安娜都不见了,很快——大家也都不见了。

并不是说疾驶的舢板和注射肝炎疫苗不能写进去。烦扰读者的是作者并没有确定文章的写作样式,也没有明确如何面对读者。他以不同的面貌,冲着读者就来,想到什么材料就用什么材料。其结果不是掌控好材料,而是被材料所掌控。假如他花一点儿时间确立好统一性,一切就不会这样了。

因此,写作前问自己几个基本问题。例如:“我要以什么身份面对读者?”(记者?信息提供者?普通男人或女人?)“我要用什么人称和时态?”“我要用什么文体?”(客观报道?个人但正式?个人而随意?)“我对材料采取什么态度?”(介入式?疏离式?裁判式?反讽式?娱乐式?)“我要写多少?”“我要强调的唯一要点是什么?”

最后两个问题特别重要。多数非虚构作家都有定性情结。他们觉得自己有义务使自己的文章具有最终的定性结论,这关乎题材、荣誉以及写作的神圣性。这种反应的确令人称道,但并不存在什么最终结论。今天你所认为定性之事晚上就会变为不定,而作家孜孜以求每一个事实细节,到头来却发现自己只是在追逐彩虹,永远坐不下来写作。没人能“针对”什么笼统的事写出一本书或一篇文章。托尔斯泰并不能针对战争与和平写出一本书,梅尔维尔也不能针对捕鲸写出一本书。他们所做的,只是针对时间和地点,以及针对那些时间地点中的个别人物,缩减自己的选材,描述出统一的故事——一个人追逐一条鲸鱼。每一个写作计划在开始前都必须缩减。

因此,往小里想。首先确定题材中的哪一角是你想啃掉的,然后全力以赴,仅止于此。这也是精力与士气的问题。庞大的写作计划会耗尽你的热情。热情是你保持写作得以进行和保持吸引读者的动力。你的激情一旦开始退潮,读者会第一个知晓。

至于你想强调什么要点,每一部非虚构作品都应该留给读者一个他们自己从前没有过的发人深省的想法。不需要两个或五个——只需要一个。那么首先确定你想留在读者脑海里的唯一要点是什么。这不但会指明你该走的路,你所希望达到的目的地,还会影响你所采取的语气和观点。有些要点最好是用严肃认真的风格,有些含而不露,另一些则幽默诙谐。

一旦确定了各部分的统一性,你就能融任何材料于框架之中。假如那位去香港的游客选择完全用聊天的语气描述他和妻子安娜的经历,他就会找到一种自然的方法融会贯通地向大家讲述九龙渡轮以及当地的天气情况。他个人的秉性和目的也将完整无缺,文章也会浑然一体。

但经常发生这样的事,写作之前自己已经确定好了一切,结果却发现所确定的并不对。素材开始将你引向未曾预见的方向,而你觉得用另一种语气写更舒服。这很正常——写作行为本身会激发一连串你从未预想过的念头或回忆。假如你感觉对,就不要逆流而上。相信素材,它会引领你进人你并无意踏足的地带,那里的气氛对头。顺势调整你的风格,继续前行直达你的目的地。不要做预设计划的囚徒。写作并不恭维蓝图。

假如拘泥于原计划,文章的后部分就会与前部分严重脱节。但至少你知道哪一部分忠实于自己的本意。这时就可以进行修补。回到开头,重写那个部分,这样你的基调和风格就会从头到尾统一起来。

这并没什么丢人的。剪刀和糨糊——或者电脑上相同的功能——是值得尊敬的作家的工具。切记:无论你组合得好坏,每一个部件都必须统一到你最终所搭建的大厦之中,不然整座大厦就会坍塌。
\chapter{开头与结尾}
文章中最重要的句子是第一句。如果第一句不能吸引读者继续读到第二句,那么你的文章就死定了。而如果第二句不能吸引读者接着读到第三句,那也同样是死定了。句子如此排列,每一句都向前拖着读者,直到读者上钩,作者就是这样建构至关重要的文章单位:“开头”。

开头应该多长?一段还是两段?四段还是五段?没有现成的答案。有一些开头只用几句鱼饵就巧妙地钩住读者;另一些则慢悠悠地走上几页,形成缓慢但稳定的引力。每一篇文章都提出不同的问题,唯一有效的检验是:开头是否好使?你的开头不一定是最佳开头,但如果它完成了该做的工作,就要心存感激,继续写。

有时候开头的长度依你为其写作的读者而定。文学评论的读者期望作者以某种发散的方式开始,他们会跟进这些作家,欣赏和玩味他们将在文章的何处露面,以及这些作家如何悠闲地转悠,最终才到达目的地。但我要告诫你,不要指望读者会跟进多久。读者想很快知道文章里有什么是他们感兴趣的。

因此文章的开头首先必须立即抓住读者,迫使他继续阅读。它必须诱惑读者,给读者以新鲜感、新奇感、悖论、幽默、惊奇,或者与众不同的想法、有趣的事实、某个问题等。什么都行,只要它能激发读者的好奇心,拽住他的袖子。

其次,开头必须言之有物。它必须提供坚实的细节,告知读者为何写这篇东西以及读者为何一定要读它。但不要耽搁在原因上。多哄诱读者一会儿,保持他的好奇心。

继续建造。每一段都应该增益前一段。多考虑增加坚实的细节,少考虑取悦读者。但要特别注意每一段的最后一句话——那是迈向下一段的关键跳板。尽力给这句话多一点点幽默或惊奇,就像在单人喜剧的惯例中时而出现的“妙语”。让读者微笑,这样你就至少赢得读者再读一段。

让我们看几个开头,它们的节奏不同,但所保持的紧迫感却类似。我先以自己的两篇专栏文章为例。这两篇最初登在 《生活》和《看》杂志上。根据读者的评议,这两份杂志的消费者主要的阅读场所是理发店、美发沙龙、机场,以及医生的诊室(“有一天我在理发,看见了你的文章”)。我提此事是想提醒大家,大多数期刊阅读是在吹风机下而不是在台灯下进行的,因此作家没有多少时间可以用来闲扯。

第一个例子是一篇题为《阻止福特鸡肉香肠》的文章开头:

我常想知道热狗里面有什么。现在我知道了,但却希望并不知道。

只有两个短句,但读者不继续读第二段都难:

我的麻烦开始于农业部颁布热狗成分之时——这些成分就是指热狗中法律上允许的所有成分——农业部是应家禽业要求而颁布热狗成分的,以放松其条件使鸡肉也可以包括在内。换句话说,福特鸡肉香肠能在法兰克福香肠的领地找到幸福吗?

以上用一句话解释了该专栏文章所基于的事件。然后是一段妙语来恢复轻松的语调。

农业部就此分发了问卷,得到的1066份回答多数带有敌意。按照这些回答,这个想法本身就不可思议。公众的情绪被一位妇女恰到好处地激发起来,她大声疾呼:“我才不吃家禽肉呢,绝不。”

又一个事实,一个微笑。每当你有运气得到一个这样有趣的引语时,想办法用上。这篇文章接着细化了农业部所说的可以做热狗的成分——一连串的材料,包括“牛、羊、猪或者山羊的可食性肌
肉,可取自膈膜、心脏或者食管部分……(但不包括)唇部、口鼻部或者耳部肌肉”。

从这里继续向前——期间不由自主地稍稍探讨一下食管肌肉问题——然后进人这场家禽业利益与法兰克福熏猪牛肉香肠业利益之间的争议,接着引入要点,那就是美国人会吃任何哪怕是稍微有点儿像热狗的东西。最终的暗示是,推而广之,美国人不知道,或者不在乎自己吃的食物里究竟是什么。该文的风格保持轻松,带一点儿幽默。读者开始受其离奇的开头吸引,但结果发现内容要比预期的严肃。

下面是一个慢节奏开头的例子,它吸引读者的主要是好奇心而非幽默感,文章叫《感谢上帝,球迷们》:

按常规,谁都不想看第二眼——甚至连一眼都不想看——来自棒球投手伯利·格兰姆斯出生地威斯康星州澄湖那片滑溜溜的榆树皮,但是那片树皮现在陈列在纽约州库珀斯城国家棒球博物馆和荣誉堂里。标签的说明显示,这就是格兰姆斯在比赛中嚼的那种树皮,意在“增加唾液以便抹在球上投唾沫球。棒球湿的时候会以迷惑人的方式沿弧线飞向本垒板”。这也许是美国今日可提供的最无趣的事实之一了。

但棒球迷们不能以常理来衡量。我们沉迷于比赛的细节,凭着自己对曾看过的球手的比赛记忆,在余生里对此唠叨个没完。因此没有一条细节是无关紧要的,每一条细节都将我们重新与球手相连。我这个岁数正好能记得当时的伯利·格兰姆斯,还有他那湿乎乎的球迷惑性地飞向本垒板,而当我发现他的树皮时,便全神贯注地研究起来,就好像遇见了罗赛塔石碑“原来他是这么干的”,我边想边仔细观察那片诡异的植物残片。“滑溜溜的榆树!真见鬼。”

这只是我儿时闲逛那座博物馆时所遭遇的几百件事之一。恐怕没有其他博物馆对我们的过去会产生如此特别的心路历程……

读者现在被稳稳当当地钩住了,作者任务中最难的部分到此结束。

引这个开头的一个原因是提醒大家,文章的救星常常并不在于作者的风格,而是在于某些作者能够发现的奇异的事实。我去到库珀斯城,在博物馆里花了一整个下午记笔记。四处都有怀旧的搅扰,我崇敬地盯着卢·格里克的衣帽箱和博比·汤姆森的赢球棒。我坐在取自波罗球场的大看 台座椅上,用未钉鞋钉的鞋底刨着来自埃贝茨球场的本垒板,尽心尽职地抄写所有可能有用的标签和说明。

“这些是泰德完成环绕各垒跑时踏本垒板穿的鞋,”一条标签如是说,标明那双鞋为泰德·威廉姆斯在他最后一次棒球赛中打那次出名的本垒时所穿。那双鞋的状况比沃尔特·约翰逊穿的那双要好得多——后者的鞋边都烂开了。但文字说明恰好提供了棒球迷想要的某种合理性解释。“当我在场上投球时,脚必须舒服,”了不起的沃尔特说。

博物馆五点关门,我回到汽车旅馆,对脑子记的和查询的内容都很放心。但直觉告诉我第二天上午还要回去转一圈,而就在这一次我才注意到伯利·格兰姆斯滑溜溜的榆树皮,使我灵机一动将其作为理想的开头。它至今还有效。
。
这个故事的启示是,你所收集的素材应该总是比你将要用到的多。每一篇文章的坚实度需要你从充裕的细节中选取少量最适合你的部分得以保证——所收集的材料越多越好。但在某一个时间点上,你必须停止调研,开始写作。

另一个启示是扩大你的材料范围,而不是只研读显而易见的材料,访谈显而易见的人。看看各种标记、告示牌,以及沿美国路边涂写的形形色色的垃圾广告。读读包装上的标签、玩具说明书、药品说明,还有墙上的涂鸦。读读每月从电力公司、电话公司以及银行飞来的那一张张自负自大的账单。读读菜单、目录以及二等邮件。在不起眼的报纸缝隙间寻觅,比如星期日房地产一栏——通过人们想要的房屋露台,就可以分辨出这个社会的秉性。我们每天的景观充斥着荒诞的奇文轶事。要注意这些。它们不但具有重要的社会意义,而且足够新奇,可避免写出与众趋同的文章开头。

说到与众趋同的文章开头,有许多种我情愿永远见不到。一种是作未来考古学家状:“当某个未来考古学家偶然发现了我们的文明遗迹时,他对这个自动唱机又会做出何种结论呢?”甚至连这家伙还未出场,我就已经厌倦他了。我也同样厌倦来自火星的客人:“假如有来自火星的生物降落在我们的星球上,他会惊奇地看见一群群衣着单薄的地球人躺在沙滩上烤着肌肤。”我厌倦了“不久前的一天,一个鼻子像纽扣状的小男孩正在新泽西州帕拉默斯郊外的空地上与他的小狗泰里一同散步,这时他看见有什么怪怪的像气球的东西升出地面。”而且我非常厌倦那种“有共性”的开头:“约瑟夫·斯大林、道格拉斯·麦克阿瑟、路德维格·维特根斯坦、舍伍德·安德森、博尔赫斯、黑泽明都有何共性?他们都喜爱西部片。”咱们还是让未来的考古学家、火星来客以及纽扣状鼻子男孩退下吧。尽量给你的开头以新鲜的视角或细节。

请思考下面的开头,琼·迪迪翁撰写的这篇名为《洛杉矶38区罗曼大街7000号》的文章:

罗曼大街7000号坐落在雷蒙德·钱德勒和达希尔·哈米特的那些崇拜者所熟悉的洛杉矶区域:在好莱坞的下方,日落大街南面,由“模范影业公司”、仓房、两户一幢的平房所构成的中产阶级破旧街区。由于派拉蒙、哥伦比亚、德西露、赛缪尔·高德温等影业公司在附近,住在这周围的许多人多少与电影业有联系。比如,他们都曾洗印过影迷照片,或者认识琼·哈洛的美甲师。罗曼大街7000号看起来像是一个褪了色的电影外景地、一幢彩色建筑物,上面的现代艺术装饰斑驳陆离,窗户现在不是被木板封住就是被六角形网眼镀锌网格封上,而在入口处的夹竹桃中,一块胶皮脚垫上写着:欢迎。

实际上这里并不欢迎谁,罗曼7000号属于霍华德·休斯,大门锁着。休斯家族的“传播中心”坐落在这哈米特一钱德勒领地晦暗不明的阳光下,这情形不禁使人怀疑人生的确是一部电影脚本。休斯家族帝国在我们那个时代是世界上唯一一个工业综合体——多年来它拥有机械制造、外国石油钻探工具分部、酿造厂、两家航空公司、大量不动产持股、一家主要影业公司、一家电子与导弹企业——而所有这些都由一个人经营,他的工作方法极像电影《夜长梦多》中的人物。

碰巧,我就住离罗曼7000号不远,我特意时不时驾车路过那里,我猜想那种心情同亚瑟王学者们拜访康沃尔郡海岸一样。我对霍华德·休斯的民间传说感兴趣……

我们希望这篇文章能引导人们略见休斯是如何经营的,并对司芬克斯之谜给予一些提示——而真正拽我们进入文章的则是不断递增的充满怜悯之情与已逝辉煌的细节。认识琼·哈洛的美甲师与摄影辉煌的联系太微不足道,并不欢迎谁的欢迎垫子已成为影业黄金时代的稀奇遗物。在那个时代好莱坞的窗户并没有被镀锌网格所封,那只公鸡由迈耶、德米尔、扎纳克等巨子掌控,人们可以确切地看见他们行使巨大的权利。我们要想知道更多的内容,那就得继续读。

另一个办法就是讲故事。这个办法很简单,显而易见而且简便易行,而我们却常常忘记使用这一现成之法。叙述是吸引注意力的最古老、最有力的方法,人人都想听故事。想方设法以叙述的形式传达信息。下面的开头是埃德蒙·威尔逊对发现死海古卷的记述,那些古卷是现代所发现的最令人震惊的古代遗物之一。威尔逊并没有花时间搭建平台。这可不是那种“从早餐到床铺”的叙述模式,没经验的作者才用此类模式,他们叙述一次钓鱼的行程从天亮前闹钟响起开始。威尔逊直人正题——啪!我们的注意力一下就被抓住了:

1947年早春的某一天,一个叫穆罕默德狼人的贝都因男孩在死海西岸的悬崖边放山羊。正当他向上爬去追走散的羊时,发现了一个以前从未见过的山洞,便随意向里面扔了一块石头。里面传来奇怪的破碎声。男孩吓了一跳,逃走了。但他后来同另一个男孩又回到那里,一起探索了那个洞穴。里面的一些坛子的残片之中有几个高高的陶土坛子。当他们取下碗形的盖子的时候,坛子里发出很难闻的气味,那气味来自坛子内部的深色长方形块状物。他们将块状物弄出洞穴,看见这些东西被厚厚的亚麻布包裹着,外面还涂了黑黑一层似乎是沥青或蜡的东西。两个人摊开里面的东西,发现是长篇手稿,以平行栏目的形式撰写在缝在一起的薄卷上。虽然这些手稿已经退色,有些地方已经破碎,但大体上相当清晰。他们发现其文字并非阿拉伯语。他们对手稿疑惑不解,将其保留下来,迁居到哪儿都随身带着。

这些贝都因孩子属于一个非法走私的帮派,他们将山羊和其他物品从外约旦走私到巴勒斯坦。他们向南转了一大圈,为的是绕过由海关人员荷枪实弹把守的约旦桥,并将物品漂过河去。他们现在正在去伯利恒的路上,去那里的黑市卖东西……

然而关于如何写开头并没有什么固定的规则。在别让读者跑掉这一宽泛的规则之内,所有作者都必须以最自然的写作方式和最适合自己的方式处理自己的题材。有时你可以用一句话讲出整个故事。下面是七部令人难忘的非虚构书籍的开场白:

世界伊始,上帝创造了天和地。——《圣经》

在罗马历699年之夏,即基督出世之前的公元55年,高卢总督盖尤斯·尤利乌斯·恺撒将他的目光转向了不列颠。——温斯顿·S·丘吉尔,《英语国家历史》

将此字谜拼起来,你会发现牛奶、奶酪与鸡蛋、肉、鱼、豆与谷类、绿叶菜、水果与根茎菜,这些构成了我们每日所需的基本食物。——厄玛·S·龙鲍尔,《烹饪之乐》

对于马努斯岛的土著人,世界就是一个大盘子,周边向上卷,底部是平坦的泻湖村子,那里高脚屋就像长脚鸟一样站立,平静安宁,不受潮涨潮落的影响。——玛格丽特·米德,《在新几内亚长大》

这个问题在美国女性心里埋藏搁置多年,无人问津。——贝蒂·弗里丹,《女性的奥秘》

在五或十分钟之内,不超过这个时间,其他三个人打电话给她,问她是否听说那里出事儿了。——汤姆·沃尔夫,《太空英雄》

你所知道的比你认为的要多。——本杰明·斯波克,《育儿经》

以上是针对如何开头的一些建议。现在我要讲一讲如何结尾。知道何时结束一篇文章的重要性是大多数作者始料不及的。你应该像选择第一句话一样充分考虑如何选择最后一句话。好吧,也可以说几乎要一样充分考虑。

这也许难以置信。假如读者开头就跟上你了,随着你转过死角,越过颠簸地带,当结尾就在眼前,他们当然不会离你而去。当然,他们会离去,因为眼前的结尾结果只是幻景。就像牧师的布道辞一样,本来是建构成一系列完美的结局,但却永远不结束,一篇文章在该停之处不停那就讨人厌了,因而是失败之作。

我们大多数人仍然是年少之时作文老师灌输给我们的教条的囚徒:每一篇故事都必须有一个开头、中间和结尾。我们至今仍能想象出那种提纲,以罗马字母为主干标记(I,Ⅱ以及Ⅲ),画出我们将忠实地踏上的路线图;然后是次主干标记(Ⅱa以及Ⅱb),标示出我们将短暂探寻的次要路径。但我们总是许诺要回到第Ⅲ部分,总结我们的旅程。

这种方式适用于对自己的基础没把握的小学生和中学生。它迫使学生认识到每一篇作品都应该有一个逻辑严谨的设计。这个教条值得让任何年龄的人都知道,甚至连专业作家跑题的频率也常常比他们自己所愿意承认的严重得多。但是如果你打算写出好的非虚构作品,你就必须摆脱第Ⅲ部分死死的控制。

当你看见屏幕上出现始于“总而言之,人们注意到……”这类句子,或者“那么我们从中能够收获何种结论呢?” 这类问题,你就知道已经是到了第Ⅲ部分。这些信号预示你将以压缩的形式重复你已经详细说过的内容。这时读者的兴致开始动摇,你也已建立起来的张力开始松弛。然而你还是想要忠实于波特小姐,你的老师,她让你发誓对神圣的提纲效忠。于是你提醒读者什么是总而言之可以被注意到的。你又重新搜罗一遍你已经举证过的结论。

但读者能听见你费力转动的声响。他们会注意到你的所作所为,以及这些作为使你感到多么无聊。他们感到愤愤不平的颤动。你为何不多想想打算如何结束这篇东西?你在结尾的总结难道只是因为你觉得读者太笨,不明白你的要点吗?你还在不停地转动。但读者另有选择。他们撤了。

以上是记住文章最后一句重要性的负面理由。不清楚这句应置于何处会毁了一篇好端端的文章;本来这篇文章通篇结构严整,只是最后阶段出了问题。写好结尾的正面理由是,最后一个好句子——或最后一个好段落——本身应给人以快乐。它给予读者向上的力量,待文章结束时,它仍令人回味。

完美的结尾应该稍微给读者一点儿惊奇,而且要恰到好处。读者不希望文章结束得太快,或者太突然,或者是重复说过的。好的结尾读者一看便知。就像好的开头,它好使。它就像舞台喜剧中落幕前最后一句台词。我们正在一场剧的中间(我们认为如此),突然一位演员说了些滑稽、夸张或者警句之类的话,舞台灯光随之熄灭。我们惊讶地发现这场戏结束了,随后为其奇妙的结束方式感到愉悦。使我们愉悦的是剧作家完美的掌控。

对于非虚构作者,将其归纳为规则的最简单方法是:当你准备好停止之时,停止。如果你已经陈述完所有事实而且强调了你要强调的要点,赶紧找最近的出口。

经常只需要几句话就可以结束一切。理想的状况是,这些句子应该概括文章的中心思想,最后的结束句应该以其得体性和意料之外的效果震动人心。这里举例说明门肯是如何结束他对卡尔文·柯立芝总统的评价的。柯立芝总统对国民“客户”的吸引力是,其“政府几乎不统治国民,因 而杰斐逊的理想最终得以实现,杰斐逊的信徒们兴高采烈”:

我们最受苦之时并非白宫像宿舍一样平静的时候,而是有那么一个无能的保罗在房顶叫喊时。除去哈定这个无足轻重的总统外,柯立芝博士的前任是个救世主,后两任也是如此。觉醒了的美国人不得不在这几位和柯立芝之间选出一位,他们还有片刻可犹豫的吗?柯立芝当政期间无惊天动地之举,但也没有头疼之事。他没什么主意,因而也不讨人厌。

这五个短句很快就送读者上了路,而且读者走时还带着令人发醒的思绪。柯立芝没注意因而也不讨人厌这个想法,不禁给读者留下一种享受回味的余地。这的确好使。

我在写作中经常做的是将事件带回到一个循环中——也就是在结尾敲响回音,同开头曾响起的音符呼应。这样可以满足我的对称感,也愉悦读者,达到共鸣,完成我们共同踏上的征程。

但一般来讲,最有效的是引语。回头翻看自己的笔记,找出某些给人以结束感的、或者滑稽的、或者增加某种意想不到的最后细节的语句。有时它会在访谈期间蹦出来——这时我常想“结尾就是它了!”——或者在写作过程中出现。60年代中期,伍迪·艾伦正在确立自己作为美国专职神经质艺人的名声,他在夜总会演独角戏。当时我为杂志撰写第一篇长文,记述了他的到来。那篇文章的结尾是这样的:

“如果大家看完演出,欣赏我这个人,”艾伦说,“而不是仅仅喜欢我的笑话;如果他们看完演出,无论我说什么,他们还想听我说,那么我就成功了。”按照其回头客来判断,他的确成功了。伍迪·艾伦就是那位知己先生,而且他似乎注定要多年把持这一特许演出权。

但他的确也有自己的问题,在美国无人分享,无人欣赏。“我一直被一个事实所困扰,”他说,“那就是我母亲真的特别像格劳乔·马克斯\footnote{格劳乔·马克斯(Groucho Marx,1890-1977):美国喜剧演员,电影、电视明星,以机敏闻名}。”

这里有一句话来自那么遥远的左派领地,没人料想到它的出现。它所承载的惊奇是巨大的。这样的结尾难道会不完美吗?假如有什么使你感到惊讶,它也会使你的读者感到惊讶——和愉悦,特别是当你结束故事和送他们上路之时。
\chapter{零零碎碎}
这一章包括零零碎碎的小小忠告,涉及方方面面——像大家所说的那样,我将其收集在一把伞下。

动词

要使用主动语态动词,除非没有能绕过被动语态动词的捷径。主动动词与被动动词风格之间的区别,从清晰度和力度上来讲,对于作者就是生与死的区别。

“乔看见他了”语气强。“他被乔看见了”语气弱。第一句简短而精确,它对于谁做了什么不留任何疑问。第二句必然冗长并且还有一种枯燥乏味的特性:有什么事被某人向另一个人做了。而且该句还有歧义。他被乔看见多少次?一次?每天?每周一次?包含被动结构的风格会消耗读者的精力。没人会真正明白有什么被谁向谁实施了。

我用“实施了”这个词,因为它是那种被动语态作者喜欢用的词。他们更喜欢用带拉丁语词根的长词,而非简短的盎格鲁一撒克逊词——这就增加了麻烦,使句子更粘连。简短比冗长好。在林肯第二次就职演讲词中的701个单词中(这本身已经是用词经济的奇迹了),其中505个单词是单音节词,122个是双音节词。

动词是你所有工具中最重要的。动词推进句子,给句子以动力。主动动词推得紧,被动动词时时拖。主动动词还能使人预见某种活动,因为此类动词要有代词(“他”),或名词(“那个男孩”),或某个人(“斯科特太太”)将其付诸行动。许多动词在其意象或声音中还带有含义的暗示:闪耀,耀眼,旋转,欺骗,分散,昂首阔步,拨弄,纵容,烦恼。恐怕没有其他语言具有如此巨大的动词供给、如此鲜亮的色彩。不要选一个乏味或勉强够用的动词。用主动动词激活句子,避免用那种需要尾部带介词才能完成任务的动词。不要说“set up a business”,可以说“start or launch a business”(开始一项事业)。\footnote{这里的动词都是“开始做”的意思,前者为动词短语,更口语化,后者更简明、标准}不要说“公司总裁下台了”。他辞职了吗?他退休了吗?他遭解雇了吗?要准确。要用准确的动词。

如果你想看主动动词如何给写就的词以活力,不要只回到欧内斯特·海明威、詹姆斯·瑟伯或亨利·大卫·梭罗那里。我推荐英王詹姆士一世钦定《圣经》和威廉·莎士比亚。

副词

多数副词是不需要的。假如你选用一个意思确切的动词,然后再加一个带有同样意思的副词,你就会搞乱句子,惹恼读者。不要告诉人们收音机嘟嘟叫地响,“嘟嘟叫”已经包含响的意思。不要写有人紧紧地咬牙,咬牙本来就是紧紧的,别无他法,不必再用“紧紧地”。在不经意的写作中,强力动词反复被多余的副词所削弱。形容词和其他词性的词也是如此:“毫不费力地容易”,“有一点儿艰苦”,“完全使人目瞪口呆”。“目瞪口呆”一词的美感就在于它暗指某种完全彻底的震惊,我无法想象有人会部分地目瞪口呆。假如有什么事容易到毫不费力,直接用“毫不费力”好了。那么什么叫“有一点艰苦”?也许是和尚享用的铺满地毯的单间吧。不要用副词,除非这些词的确起作用。求你不要以“获胜的运动员咧嘴大笑”之类的表达来报道新闻。

在讨论此类问题之时,让我们停止使用“决定性”地以及其含义模糊的远亲们。每天我在报上都会看到某些局势决定性地变好,另一些则决定性地恶化,但是我从来都不知道好转的程度有多么决定性,或谁做的决定,就如同我从来都不知道某个“特别公平”的结果会有多么特别,或者是否相信一个“具有争议性的真实”的事实。“他是纽约大都会队有争议性的最好投手,”得意洋洋的体育记如此写道,渴望达到帕纳塞斯艺术圣山的高度,而瑞德·史密斯却从没用“争议性”之类的词就达到了。这名投手是该队的最佳投手吗?——这可以通过论证加以证明。如果能,请省略“争议性”这个词。或者他也许是最佳投手?——这样此观点就具有争议性了。恕我坦言,我也不知道。这实际是个难以定夺之事。

形容词

多数形容词也是不需要的。像副词一样,它们四处喷洒在句子中,这些作家也不停下来想想那些概念早已在名词中了。此类散文到处充满险峻的峭壁和花哨的蜘蛛网,或者表达某些颜色已经不言自喻之物的颜色的形容词,如黄色水仙、褐色泥土。如果你想对水仙做一番评价,选一个诸如“艳丽”之类的词。如果你所处的乡间的泥土是红色,尽管用红土来形容。这些形容词才能起到名词本身起不到的作用。

多数作者几乎是无意识地在其散文之壤播撒形容词,使其更茂盛和漂亮;句子变得越来越长,他们充斥其中的有诸如“威严的榆树”、“活泼的小猫”,“老道的侦探”,“沉睡的泻湖”等等。这是习惯性地用形容词,你需要摒弃这个习惯。并非每棵橡树都是节节疤疤的。只为装饰而存在的形容词对于作者是自我放纵,对于读者则是负担。

同样,规则很简单:让形容词起到需要起的作用。“他望了望灰蒙蒙的天和黑压压的云,决定驶回港口。”深暗色的天和云是做出此决定的原因。假如告诉读者房子毫无光彩或者女孩儿美丽很重要,那就直接用“毫无光彩”和“美丽”。这些形容词会具有恰到好处的力量,因为你已学会节俭地使用形容词。

小修饰语

剪掉修饰你感觉如何、怎么想以及你看见什么之类的小词儿:一点儿、某种、颇为、相当、很、太、非常、在某种意义上,还有几十种更多的此类词语。这类词语会冲淡你的风格和说服力。

不要说你有点儿困惑、有种累的感觉、有点儿郁闷、有些儿烦恼,就直说困惑、累、郁闷、烦恼。不要用小小的胆怯去围绕你的散文。好的写作精练而自信。

不要说你不太满意,因为旅馆相当贵。说你不满意,因为旅馆贵。不要告诉大家你相当幸运。那有多幸运?不要将一件事描述为颇为壮观或很了不起。“壮观”和“了不起”之类的词是不受度量的。“很”、“非常”表示强调是个有用之词,但它也常常是个赘词。没有必要称某人很有条理。他要么有条理,要么就是没条理。

更大的问题是权威性问题。每一个小修饰语都会削减读者的部分信任。读者要作者相信自己并且相信自己所说的。不要减弱这个信念。不要有点胆量。要有胆量。

标点符号

这里只是有关标点的一些粗略想法,决没有将其作为入门读物的意思。假如你不知如何句读——其实许多大学生也不知——找本语法书吧。

句号:有关句号没有太多可说的,只是多数作家没有及时用到位。如果你发觉自己无望地陷人长句之中,那也许是因为你试图在合理的情况下让该句做更多的事,也许想要表达两个不相似的想法。解决的最快办法是将一句断为两个甚至三个短句。在上帝眼里,句子接受的最短长度并无规定。好作家当中,短句为主,但别跟我提诺曼·梅勒——他是个奇才。假如你要写长句子,也得首先做个奇才。或者至少要保证,从头到尾,在句法和标点上,所写的句子都要可控,这样读者才知道在弯弯曲曲的路径上的每一步他所处的位置。

感叹号:不要用感叹号,除非你必须要造出某种效果来。感叹号造了一种煽情的气氛,例如描写初入社交界的少女评论只让她个人激动的事件时所表现出来的扣人心弦的激动:“爸爸说我一定是喝了太多的香槟了!”“但说实话,我当时能跳一晚上舞!”这些句子中的感叹号敲击着我们的头脑,使我们感到事情是多么好玩、多么美妙,但我们大家因此而遭的罪比这些句子所给予的好处要多。因此要与之相反,造好句子,使词序起到你要强调的作用。同时也要抵制用感叹词来通知读者你在开玩笑或用讽刺语。“我从未想到水枪也可以上膛!”读者对你提醒这是滑稽的一刻会感到恼怒。同时他们也被剥夺了自己发觉其好玩的乐趣。幽默通过低调陈述可以取得最佳效果,而感叹词却没有任何微妙之处。

分号:这个标点符号有一种陈腐的19世纪气息。我们将它与小心翼翼的平衡句,以及康拉德、萨克雷、哈代深思熟虑的“从一方面看来”和“从另一方面看来”之类的词组联系起来。因此分号对于现代虚构作者来说应该慎用。然而我注意到在本书我所引的段落中分号相当经常地出现,而且我自己也常用它——一般是用来针对句子的前部分增加相关的想法。虽然如此,分号带给读者的,假如不是停止,至少也是个停顿。因此要谨慎使用分号,切记它会将你想奋力达到的21世纪初的动力放慢到维多利亚时代的节奏。要依靠句号和破折号。

破折号:不知为何,这个极为宝贵的工具被广泛认为是不太适用的——就像一桌温文尔雅的好英语晚餐上的土包子。但他却有完全的会员资格,并且能使你摆脱许多尴尬境地。破折号有两个用法。其一是在句子第二部分详述或说明你在第一部分陈述的某个想法。“我们决定继续走——还有100英里路,我们可以及时赶到吃晚饭。”破折号凭其形状本身就可以推进这句话,解释他们为何要继续走。其二涉及两个破折号,将长句中解释性的想法分开。“她告诉我上车——整个夏天她都在要我剪头发——于是我们静静地驶向城里。”本来需要一个单独句子来处理的解释性细节,现在顺便被利索地解决了。

冒号:冒号看起来变得比分号更陈旧,其许多功能已经被破折号所取代。在类似于罗列一连串项 目之前起到纯粹的暂停作用方面,冒号仍然很有效。“介绍手册上说,船将在以下港口停泊:奥兰、阿尔及尔、那不勒斯、布林迪西、比雷埃夫斯、伊斯坦布尔和贝鲁特。”类似这样的功能,哪一个标点符号也抵不过冒号。

语气变化词

学会尽快警醒读者前一句之后发生的任何语气变化。至少有几十个词可以完成此项任务:但、不过、然而、仍然、还、反而、因而、所以、同时、现在、之后、当今、随后,以及更多。如果你改变方向时用“但”开始,那么毫无疑问,读者在理解一句话时会多么容易。但与此相反,如果他们必须等到最后才意识到你改变了语气,那将会多么难。

我们很多人都被教导说每一句话都不该以“但”开始。假如这就是你所学的,那抛弃它——在开始处没有比它更强的词了。它宣告同之前的句子形成完全的反差,这样读者就事先准备好了这一变化。如果你想缓解一下太多以“但”开始的句子,就转为用“然而”。然而,这是一个弱性词,需要审慎地放置。不要以“然而”开始句子——它会像一片湿洗碗布挂在那儿。也不要以“然而”结束句子——到那时它已经失去了其“然而性”。将其置于合理早的部位,就像我刚才一样将其置于三句话之前。这样其突兀性就变成了优点。

“不过”的作用与“但”几乎相同,但其意思更靠近“仍然”。在英语中,这两个词都可以置于句首——“不过他还是决定去”或者“不过他仍然决定去”——来替代概述读者已被告知有关内容的一个长长的词组:“尽管所有危险都已向他指出这一事实,他还是决定去。”查清句中的所有位置,看此类短词可否同时表达与冗长、乏味的从句一样的意思。“我反而乘了火车。”“我还是不得不羡慕他。”“就这样我学会了抽烟。”“因此见到他是容易的。”“同时我与约翰谈过了。”这些轴心词节省了多少膨胀的词语啊!(这里的感叹号表明我真是这个意思。)

缩约词

如果你使用像“I'll”(我将)、“won’t”(不会)、“can’t”(不能)这样的缩约词来自然而然地适应你所写的,你的文风相对于你的性格会变得更温和、真挚。“我将很高兴见到他们,如果他们不生气的话”比起“我将会很高兴见到他们,如果他们不会生气的话”就不那么生硬。(大声朗读后一句,你将听见它有多做作。)并没有什么规定反对这种非正式用语——相信自己的耳朵和直觉。我只建议避免使用那种容易混淆的形式——读者在搞清楚是哪个意思之前很可能已经深入到句子里。经常是其含义并非读者所想到的意思。还有,不要自己造缩约词,这会降低你的文风。坚持用你在词典中能查到的。

概念性名词

在差的写作中,表达某种概念的名词得到广泛使用,而不是诉说某人做了什么的动词。这里有三个典型的僵死句子:

常见的反响是不信任的嘲笑。

茫然的冷嘲热讽并不是对于这个旧体制的唯一反应。

当前校园的敌意是变化的征兆。

这些句子的怪异之处在于句子里没有人。句子里也没有实意动词——只有“是”或“不是”。读者想象不出有任何人进行活动;所有意思都非人格化地包含在某种概念模糊的名词里:“反响”、“冷嘲热讽”、“反应”、“敌意”。变一变这些冷冰冰的句子,叫人动起来:

多数人只是以不信任的态度嘲笑。

一些人对旧体制报以冷嘲热讽;另一些人……

很容易注意到变化——你会看到所有学生有多么气愤。

我修改过的句子也没有跳跃出什么活力,部分原因是我想使劲揉成形的材料是不成形的面团。但至少这些句子里有了真人和真动词。不要陷入拎了一满袋抽象名词的窘境。那样你会沉人湖底,永无出头之日。

蔓生的名词串儿

这是一种新美语疾患,它将两三个名词串在一起,而实际上用一个名词——或者更好一点儿,一个动词——就够了。现在没人走向破产;我们只是有资金领域问题。天不再下雨了,而只是有降雨活动或者雷暴可能性的天气。求求你啦,让雨下来吧。

现今有多达四五个概念性名词附加在一起的情况,就像一个分子链。这里有一个我最近发现的绝好样本:“促进交际技能发展干预”。看不见一个人,也没有任何实意动词。我想这是一个帮助学生提高写作能力的项目。

夸大其词

“客厅看起来就像那里爆炸了一颗原子弹,”写作新手这样描述一场失控了的晚会之后星期天早晨的情景。好吧,我们都知道他是在夸张,想造出点儿滑稽效果,但我们也都知道原子弹根本就没在那儿爆炸,什么炸弹也没有,也许水弹除外吧。“我感觉就好像十架747喷气式飞机飞过我的头顶,”他写道,“我真想跳出窗户自尽。”这些戏谑之语会戏谑出格——这位作者早已出了界线——读者也早已感到昏昏欲睡,无可抵御。这就像被一个不停地背诵打油诗的人缠住了一样。不要夸大其词。你不会真想跳出窗户。生活中真正可怕滑稽的情形多得很。让幽默悄悄潜人,我们几乎听不见它的到来。

可信度

可信度对于作者同对于总统一样脆弱。不要夸大一件事,不要言过其实。如果读者抓住你在竭力扮假为真,哪怕只有一句虚假的陈述,你所写的一切就会遭到怀疑。这个风险太大,不值得一试。

口述

在美国有许多写作是通过口述完成的。行政主管、公司高管、经理人、教育家以及其他官员们所想的是高效利用时间。他们认为将什么“写”下来的最快办法是向秘书口述,也不用再看一眼。这实际是得不偿失——他们节省了几个小时,却毁坏了自己的人格。口述的句子很容易夸大其词、拖泥带水、废话连篇。业务繁忙,避免不了口述的公司高管们至少应该抽出时间编辑一下自己口述的内容,减词加词,保证最后落实到文字的内容能够真实反映他们的特点;特别是那些要给客户的文件,客户们会依照其文风判断其人格和公司的情况。

写作不是竞赛

每一位作者的出发点不同,去往的目的地也不同。然而,很多作者被同一种想法困扰得不知所措,他们认为自己是在与其他也在写作之人或许写得更好之人竞争。这经常发生在写作课上。缺乏经验的学生会不寒而栗地发现,自己同班的同学在校报上都已署过大名了。但为校报写文章并不算什么太高的资历。我经常发现为校报写文章的兔子们往往被勤勉地移向掌握写作技能这一目标的乌龟们所赶超。同样的恐惧也捆绑了自由撰稿人,他们看见其他作者的文章出现在杂志上,而自己的作品却不停地遭到退稿。忘掉竞争,按照自己的步调走。你唯一的竞赛是同你自己比。

潜意识心理

你的潜意识心理活动对写作的贡献比你想的要多。你经常会花一整天的时间试图在词语的灌木丛中杀出一条路,你感到似乎自己羁绊在那灌木丛中没救了。而第二天早上当你重新投入进去,往往会有好办法出现。在你睡眠之时,你那作者的大脑并没休息。一个作者会一直工作。对你周围的动向保持警觉。经过几天、几个月、甚至几年,通过你潜意识心理活动的渗透,正当你的有意识心理活动劳作于写作之时,急需之刻,你所看见、所听见的大部分都会回来。

最快捷的解决办法

令人称奇的是,句子中一个难解的问题经常可以用简单删除的办法加以解决。不幸的是,这个办法对于卡住的作者来说,通常是最后想到的。首先他们会想方设法安置这个麻烦的词语——将其挪到句子的其他部分,竭力重新调整,增加新词来澄清思路,或者润滑任何卡住的部分。这些努力只会使情况更糟,作者最后只得下结论说这个问题无解决办法——这可不是什么欣慰的想法。当你发觉自己处于如此绝境,再看看那个麻烦成分,问自己,“难道我真的需要它吗?”很可能你并不需要。其实它一直在做无用功——这就是它为何一直带给你那么多痛苦的原因。去掉它,看那受苦受难的句子重放生机,自由呼吸。这是最快的治愈办法,也常常是最好的。

段落

保持段落简短。写作具有视觉性——它首先抓住的是眼球,然后才有机会抓住大脑。短段落在你所写的四周留有空气,使其看起来更有吸引力,而那种冗长的段落会打击读者的兴趣,甚至连开头都不愿意读。

报纸段落应该只有两三句长;报纸的版式横向排得比较窄,因而其行数很快就会增加。你也许会想如此频繁换行有损于表达你的观点。显然,《纽约客》对此类恐惧难以自拔——担心读者读上几英里长,毫无喘息之机。别担忧,利会大于弊。

但也不要失控。连续的短段落与一个大长段落同样讨人厌。我想到所有那些侏儒段落——无动词的奇迹——由当代记者们所写,为的是使文章快而易。实际上他们这样切断自然的思路反倒使读者阅读起来更加困难。试比较以下同一篇文章的两种排列——注意看第一眼时和读起来时彼此的区别:


白宫2号律师星期二提前下了班,驱车到一个眺望波托马克河的偏僻公园,结束了自己的生命。

他手里有一把左轮手枪,身体低垂地靠在内战时期的大炮上,没留下遗言,没有解释。

只有朋友、家人和同事的震惊和悲痛。

还有直到星期二之前读起来还像是任何人的幻想的人生故事。

白宫2号律师星期二提前下了班,驱车到一个眺望波托马克河的偏僻公园,结束了自己的生命。他手里有一把左轮手枪,身体低垂地靠在内战时期的大炮上,没留下遗言,没有解释——只有朋友、家人和同事的震惊和悲痛。他还留下了直到星期二之前读起来还像是任何人的幻想的人生故事。

美联社的文稿(左),其段落轻松、简短,而且第三、四句无动词,具有轰动效果,且带有优越感。“啊哈!看,我为你把它弄得多简明!”那位记者向我们说。我的文稿(右)给予记者写出好英语并把三个句子整合为一个逻辑单元的尊严。


划分段落在非虚构文章和书籍写作中是一个微妙而重要的因素——它是一幅地图,不停地告诉读者你是如何组织想法的。研究一下好的非虚构作家,看他们是怎么做的。你会发现几乎所有人都以段落为单位思考,而不是以句子为单位。每一段都有其自身内容和结构的整一性。

性别歧视语

令作者最烦恼的一个新问题是如何处理性别歧视性语言,特别是代词“他一她”。女权运动有助于揭示性别歧视是多么广泛地潜藏于我们的语言之中,不仅是在挑衅性的“他”中,而且在成百上千的带有令人反感的意思或在价值判断上有弦外之音的词语中。这些词包括以恩宠自居的“小妞儿”,暗示二等地位的“女诗人”、二等角色的“家庭主妇”,某种头脑空空的“丫头们”,贬损女性工作能力的“女律师”,故意好色、很少用于男性的“离婚女子”、“女生”、“金发碧眼女郎”。男子遭抢劫也就抢了,女子遭抢劫就会是一个身材匀称的女乘务员或者时髦的黑发女郎。

更有损人格的——也更微妙的——是所有将女性看为男性家庭财产的用法,而不是将她们看作有自我身份,在家庭中扮演同等角色的人:“早期定居者带着妻子和孩子向西推进。”该把这些定居者改成开发边疆的家庭,或者开发边疆的夫妻带着自己的儿子和女儿向西走,或者男男女女们定居在了西部。现今很少有什么角色不是向男女都开放的。不要用暗示只有男人才能成为定居者、农民、警察或者消防员的词语结构。

另一个棘手问题是由女权主义者对一些带“男人”(man)的词不满所提出的,如:“主席”(chairman)和“发言人”(spokesman)\footnote{此处后缀的情形特指在英语语言环境下,请读者参考阅读——编者注}。她们反对的观点是女人也能同男人一样主持好会议,也同样擅长发言。这样就出现了一些新词,如“chairperson”和“spokeswoman”。这些来自60年代的临时词语使我们意识到性别歧视问题,既在词语中也在态度上。但最终这些仍是临时性的词语,有时对这项事业的损害反倒多于帮助。一个解决办法是找另一个词:“chair”代替“chairman”,“company representative”(公司代表)代替“spokesman”。你也可以将名词转成动词:“代表公司,琼斯女士说……”当某种职业既有男性也有女性称谓的形式时,找一个通用的替代词:男演员和女演员都可以称为表演者。

但这还是留下了代词等待解决。“他”就是折磨人的词。“每一个雇员都应该决定他认为什么对他自己以及他的家属是最好的。”我们对这类无数的句子又该怎么办呢?一个办法是将其转为复数:“所有雇员都应该决定他们认为什么对他们自己和他们的家属是最好的。”但这只是在小剂量下才有效。将每一个“他”都转成“他们”,这样的风格很快就会变得淡而无味。

另一个通常的解决办法是用“或”:“每一个雇员都应该决定他或她认为什么对他或她自己是最好的。”但同样,这也得慎用。作者经常会发现在一篇文章中有几种情况,他或她可以用“他或她”,只要自然就好。“自然”的意思,指的是作者已经注意到他(或者她)有这个问题,并尽力在合理的限度内解决此问题。但还是要直面问题:英语语言卡在了阳性通用名词上
(“Man shall not live by bread alone”,人不能光靠面包生存)。要把每一个“他”转成“他或她”,每一个“他的”转成“他的或她的”,就会堵塞语言。

在《写作法宝》的早期版本中,我用“他”来指“读者”、“作者”、“评论者”、“幽默作家”等。我感到假如每次我提到这些人都用“他或她”,这本书读起来就费劲了。(我完全杜绝“他/她”这种形式,斜杠在规范英语中毫无地位。)不过几年来,许多女士写信给我唠叨这个问题。她们说身为作者的她们对总是想象到一个男子在写作、阅读表示反感。她们是对的。我忍受着唠叨。多数唠叨者敦促我用复数形式。我不喜欢复数;复数形式削弱了写作的力度,因为复数不如单数确切,不那么容易唤起视觉想象。我想要每一位作者都想象出有一位读者在竭力阅读他或她所写的。然而我还是发现有三四百处,在这些位置我可以去掉“他”,办法主要是转成复数,而且无大碍;天也没塌下来。在本版中仍有男性代词,但我感到这里的用法是唯一不麻烦的解决办法。

最好的解决办法就是简单地去掉“他”以及其所带的男性特有的内涵,而使用其他代词或者变换句子的其他成分。“我们”可以方便地替代“他”。“我们的”常常可以替换“他的”。(A)“首先他注意到有什么事情发生在他的孩子身上,而后他为此抱怨他的邻里。”(B)“首先我们注意到有什么事情发生在我们的孩子身上,而后我们为此抱怨邻里。”泛指名词可以代替确指名词。(A)“医生经常忽略妻子和孩子。”(B)“医生经常忽略家人。”通过这些小小的变化,可以消除无数错误。

在我的修补中有帮助的另一个代词是“你”。在谈论“作者”做什么以及“他”都遇到些什么麻烦时,我发现在很多地方我都可以直呼作者(“你会经常发现……”)。这并不适用于所有类型的写作,但对于写指导手册或自助手册的作者来说却是个天赐的办法。以本杰明·斯波克博士的口吻对发烧小孩的母亲说话,或者以朱莉娅·蔡尔德的口吻对忘了菜谱的厨师说话,是读者所能听见的最放心的声音。永远都要寻找使自己能够贴近读者的办法。

修改

修改是写好作品的基础,它决定了这场游戏的成败。这个想法让人难以接受。我们在第一稿中都有一种感情投人,我们不能相信自己的作品天生不是完美的。但其不完美的可能性是百分之百。多数作者并非开始就能说出自己想说的,或者尽可能说得好。新孵出的句子几乎总是有错。它不清楚。它没逻辑性。它啰嗦。它太笨重。它装腔作势。它乏味。它充斥赘语。它充满陈词滥调。它缺乏节奏。它可以有几种不同的读法。它不是从前一句接着往下走。它不……要点是,清晰的写作是反复修补的结果。

许多人认为专业作家不需要修改,词语会自动到位。正相反,严谨的作家会不断地修改。我从不认为修改是一个多余的负担,我感谢每一次改进自己作品的机会。写作就像一块好表——它应该走得顺畅,不带任何多余的零件。学生并不苟同我对修改的热衷。他们认为那是惩罚:额外的作业或额外的内场练习。请——如果你是这样的学生——把它当做一件礼物。假如你不理解写作是一个进化的过程,而不是一个完成的产品,你就写不好。没有人指望你一次就写对,甚或两次就写对。

那么我说的“修改”是什么意思呢?我并不是指写一份稿件,然后再写一个不同的文稿,然后再写第三个。多数修改包括重塑、缩紧、精炼你第一次写的原材料。许多部分包括保证给予读者一种他能够轻易从头到尾读下去的叙述的流畅感。不断使自己处在读者的位置上。有什么他应该在句子前面被告知,而你却将其放在句尾了呢?当他开始读A句到B句时,你已经改变了B句的题材、时态、语调、重点,他知道这些吗?

下面看一个典型的段落,想象它是作者的第一稿。其中并没有什么错,它清晰、合语法。但其中到处是粗糙的棱角:作者未能不断地让作者了解时间、地点、语气上的变化,或者变换和激活自己的写作风格。我所做的是,在每一句后面用方括号加上编辑在读第一稿时可能想到的一些建议。之后你会看见我修改过的段落,其中吸收了那些具有纠正性的建议。

在过去的年代,邻居们会互相帮助,他记得。[将“他记得”前置以便建立一种回顾的基调。]事情似乎不再是那样了,然而。[用“然而”表示对比的词必须置于句首。用“但是”开始。同时明确美国这个地理位置。]他想知道其原因是否是因为在现代世界人人都太忙。[以上所有句子长短都一样,而且昏昏沉沉的,节奏也相同;把这句变成问句?]他想到如今人们有这么多事情要做,他们没有时间再保持老式的友谊。[这句基本上重复上一句;去掉或者加具体细节。]在美国的前一个时代,情况可不是这样的。[读者仍在现在状态;调转一下句子说明读者现在是在过去。“美国”一词如果在前面已经插入,这里就不需要了。]他也知道在其他国家情形是大为不同的,他回想起自己在西班牙和意大利乡村生活的岁月.[读者仍在美国。用一个否定连词将读者过渡到欧洲。句子太松散。断为两句?]几乎对他来讲,随着人们变富,房子彼此建造得越来越远,他们似乎将自己从生活中最基本的需求里孤立出来。[讽刺延迟得太长。早一些植入讽刺。突出有关财富的悖论。]然后有另一个想法使他烦恼。[这是本段的真正要点,提醒读者这很重要。避免弱化结构“有……”]正当他近期得病最需要朋友之际,他的朋友们却都离弃了他。[重新调整以“最需要”结尾;最后一个词是回响在读者耳边的词并且给句子以力量。留“得病”到下一句,那是另一个想法。]就好像他们发现他做了羞愧的事而感到内疚一样。[引人“得病”在此作为羞耻的原因。省略“羞愧”,这里已经暗示了。]他回想起在什么地方读过有关世界原始地区的社会,在那里病人都被远而避之,虽然他从未听说过在美国有这样的习俗。[句子开始太慢,而且一直呆滞、乏味。断句为更短的单位。迅速射出讽刺要点。]

他记得邻居们过去总是互相帮助。但那似乎在美国不再发生。其原因是大家都太忙吗?人们的时间真是都被电视、汽车以及健身运动占用而无暇顾及友谊了吗?在以前的年代,情况绝非如此。世界其他地区家庭的生活方式也不如此。甚至在西班牙和意大利最穷的乡村,他回忆道,人们都会带上一条面包串门儿。他的脑中忽然闪念出一个颇具讽刺意味的想法:随 着人们变得富裕起来,他们反倒将自己与生活的丰富性割裂开来。但真正使他烦恼的是一件更令人吃惊的事。他的朋友 离弃他之际正是他最需要他们之时。他这一病倒几乎就好像他做了什么羞愧之事。他知道其他社会有对重病之人避而远之的习俗。但那种习俗只是存在于原始文化群落中。或许真是如此吗?

我的修改并非是所能做的最好的,或是唯一的。这些主要是工匠的技艺问题:改变意群的长短、紧缩语句的流动、突出要点。在抑扬顿挫、语言的细节和新鲜度方面,还大有改进的余地。总体结构同样重要。从头到尾朗读你的文章,切记在前一句中你将读者置于何处了。你可能会发现你写了像下面这样的两句话:

这部剧的悲剧主人公是奥赛罗。又矮小又恶毒,埃古的嫉妒与猜疑在不断膨胀。

埃古这句话本身没有问题。但是作为连接上一句的句子,那它就大错特错了。在读者耳畔回响的名字是奥赛罗,读者会自然而然地认为,是奥赛罗又矮小又恶毒。

当你朗读自己所写时,脑子里想着这些连接处,你会很不安地听见在很多处你使读者迷途、困惑,或者没能告诉读者他所需要的,或者两次告诉他同一件事:这些都是每一份初稿不可避免的松散的结果。你必须要做的是有整体安排——从头到尾要紧凑,行文要简洁、温暖。

学会这项清理过程。我不喜欢写作的艰苦,我喜欢写作的果实。但我热衷于修改。我特别喜欢删节:按删除键,看到不必要的词或词组或句子消失在屏幕上。我喜欢替换一个乏味的词,用一个更精确、多彩的词。我喜欢加强一句与另一句之间的过渡。我喜欢替换单调的句子,增加更愉快的韵律,或更典雅、更有乐感的句子。精练每一个细微之处,我感到自己在靠近我想到达之处,而当我最终到达那里,我知道,是修改而不是写作赢得了这场游戏。

用电脑写作

电脑对于修改和重组是上帝赐予的礼物,或者说技术的礼物。它将词语径直放在你眼前,供你随时考虑——和再考虑;你可以玩转句子,直到弄对为止。段落和页码会不停地自动调整,无论你裁减、变化多少都没关系,然后打印机会将一切整齐打印出来,期间你可以去喝杯啤酒。对作家来讲,几乎没有什么甜美的音乐之声比听见自己的文章全部改好后又被重新打出更美妙了——但不是被作家们用手打的。

本书已经不再需要像较早的版本那样解释如何操作这件称为文字处理器的奇妙机器,它已经进入大家的生活,也无须解释如何将其神奇的功能用于写作、修改以及组织。现在这都是常识了。我只想提醒大家(假如你还不信的话)它所节省的时间和精力是大量的。比起从前用打字机,有了电脑,我更愿意坐下来写,特别是在面对复杂的材料组织工作时。我可 以更早完成任务,也不那么累。这些对作者都是至关重要的益处:时间、产出、精力、享受和掌控。

相信自己的材料

我从事写作这个行当越长,就越意识到没有什么比事实更有意思。人们的所为——还有所说——一直令我惊讶,包括其神奇性、怪癖、戏剧性、幽默,甚至痛苦。有谁能控制所有这些令人震惊之事的发生呢?我发觉自己越来越频繁地对作者和学生说,“相信自己的材料。”但这个告诫似乎难以遵循。

最近我花了一点儿时间,给一个美国小城市报纸当写作指导。我注意到许多记者陷人一种习惯,他们用写特写的风格竭力使新闻更受欢迎。他们的开头包括以下一系列句子:

喔……!

难以置信。

埃德·巴恩斯迷迷糊糊不知道自己在看什么东西。

或者也许只是春倦症。奇怪,四月会对一个小伙子有那么大的作用。

不像是他离家前没有检查车子。

但之后还是一样,他没有记得去告诉琳达。

这很奇怪,因为他总是记得告诉琳达。自从他们早在初中时走到一起,一切就如此了。

那果真是20年前的事吗?

而现在还有小斯库特要担忧。

想想看,那条狗已经露出怀疑了。

这些文章经常从第一版开始,我会读到“接第九版”,但是不清楚写的是什么。然后我会尽职地翻到第9版,发现自己在读一篇有趣的故事,充满确切的细节。我会对记者说,“故事不错啊,但我得读到第9版。你为什么不把它放在开头?”记者的回答是,“哦,开头我在描写色彩。”其预想是,事实与色彩是两个分开的成分。它们不是分开的;色彩对于事实是有机统一的。你的任务是呈现色彩斑斓的事实。

1988年我写了一本有关棒球的书,叫《春季训练》。书中结合了我终身的职业与终身的癖好——这是能够发生在作家身上最好的事情之一;人们如果写自己喜欢的,他们能写得更好,也更享受写作。我选择春季训练作为棒球这个大题目的一角,因为此时是该更新之时了,对球员和球迷都是如此。这项体育运动被还原到其原初最纯粹的状况:在室外进行、在阳光下、在草坪上,没有风琴音乐,年轻的球员们彼此靠近得都能碰着,他们的工资和不满都被仁慈地搁置一边儿六个月。最重要 的是,这是传授和学习的时间。我选择记述匹兹堡海盗队,因为他们在佛罗里达的布雷登顿旧时的球场训练,而且是一个刚开始重建的年轻俱乐部,经理人是吉姆·莱兰,他致力于传授技艺。

我并不想使这项运动浪漫化。我不喜欢棒球电影中击球手打了本垒时的慢动作镜头,来告知我这是一个多么意义重大的时刻。我了解本垒那桩子事,特别是如果他们在第九局的后半局靠两个在外击中赢得比赛。我坚决不让自己的写作走向慢动作镜头——不提醒读者其重要性——或者将棒球称为生命、死亡、中年、失去的青春,或者更纯真的美国的象征。我的前提是棒球是一项工作,一项光荣的工作,我想知道这项工作是如何教和学的。所以我去了吉姆·莱兰和他的教练那儿,我对他们说, “你是教师。我也是教师。告诉我:你怎么教击球?你怎么教把球投给击球员?你怎么教接球?你怎么教跑垒?你怎么在如此残酷的长日程安排中保持这些年轻人的士气?”所有人都慷慨地回答了我,仔细告诉我他们是如何做他们想做的。球员和其他有我想要的信息的男男女女也是如此:有棒球裁判员、物色新秀者、买票员、当地球迷。

有一天我爬进本垒板后的看台去找星探。春季训练是棒球人才的终极秀,整个球场到处都是那些终身评估棒球天才的练达之人。我发现一位六十多岁饱经风霜的人旁边有一个空座,他在用秒表计时并记录着什么。当一局结束的时候,我问他在卡什么时间。他说他叫尼克·卡姆齐克,加州天使队北选才队长,在卡跑垒道上选手的时间。我问他在找什么样的信息。

“嗯,右撇子击球员需要4.3秒抵达第一垒,”他说,“左撇子需要4.1或4.2秒。自然会有些变化一你得考虑人为因素。”

“这些数字说明什么?” 我问。

“嗯,当然双杀平均需要4.3秒,”他说。他说起来就像常识。我从未想过双杀需要多长时间。

“因此这就意味着……”

“如果你看见球员在4.3秒之内到达第一垒,你就会对他感兴趣。”

作为事实,这不言自喻。不需要再加一句指出4.3秒对于一个击球、两个投球还有三个内场球是多么有限的时间。给出4.3秒,读者自己就会感到惊叹。他们也喜欢允许他们自己思考。读者在写作的活动中起很主要的作用,因此必须给他们起作用的空间。尽量不用诸如“令人吃惊地”、“可以预见地”和“理所当然”之类的词语,这些会在读者遇见事实之前就给该事实定了价。相信自己的材料。

随自己的兴趣走

没有什么题材不允许你写。学生常常绕开写靠近自己内心最近的题材一滑板、拉拉队、摇滚乐、轿车——因为他们以为老师会认为这些题材“愚蠢”。对于认真对待人生的人,生活中没有什么方面是愚蠢的。如果你遵循自己的所爱,你就会写好,就会吸引读者。

我读过优美的书籍谈论钓鱼与扑克牌游戏、保龄球与马术表演、登山与大海龟,还有许多其他我觉得自己不会感兴趣的题材。写自己的爱好:烹饪、园艺、照相、编织、古玩、慢跑、航海、戴水肺潜水、热带鸟、热带鱼。写自己的工作:教学、护理、做生意、开店。写自己在大学喜欢而且总想回去研究的领域:历史、传记、艺木、考古。如果你写的时候真诚地与其相连,没有什么题材会是太专业或太古怪的。






\chapter{参考文献}
Allport, G. W. (1961). Pattern and growth in personality. New York, NY: Holt, Rinehart \& Winston.

American Psychological Association. (2010). Publication manual of the American Psychological Association (6th ed.). Washington, DC: Author.

Atkinson, J. W. (1964). An introduction to motivation. New York, NY: Van Nostrand.

Baker, S. (1969). The practical stylist (2nd ed.). New York, NY: Thomas Y. Crowell.

Bandura, A. (1997). Self-efficacy: The exercise of control. New York, NY: Freeman.

Batson, C. D., Schoenrade, P., \& Ventis, W. L. (1993). Religion and the individual. New York, NY: Oxford University Press.

Baumeister, R. F., \& Leary, M. R. (1997). Writing narrative literature reviews. Review of General Psychology, 1, 311-320.

Bem, D. J. (1995). Writing a review article for Psychological Bulletin. Psychological Bulletin, 118, 172-177.

Boice, R. (1990). Professors as writers: A self-help guide to productive writing. Stillwater, OK: New Forums Press.

Brehm, J. W. (1966). A theory of psychological reactance. New York, NY: Academic Press.

Carver, C. S., \& Scheier, M. F. (1998). On the self-regulation of behavior. New Yark, NY: Cambridge University Press.

Cooper, H. (2003). Editorial. Psychological Bulletin, 129, 3-9.

Duval, T. S., \& Silvia, P. J. (2001). Self-awareness and causal attribution: A dual systems theory. Boston, MA: Kluwer Academic.

Duval, T. S., \& Wicklund, R. A. (1972). A theory of objective self-awareness. New York, NY: Academic Press.

Eisenberg, N. (2000). Writing a literature review. In R. J. Sternberg (Ed.), Guide to publishing in psychology journals (pp. 17-34). Cambridge, England: Cambridge University Press.

Ericsson, K. A., Krampe, R. T., \& Tesch-Romer, C. (1993). The role of deliberate practice in the acquisition of expert performance. Psychological Review, 100, 363-406.

Gordon, K. E. (2003). The new well-tempered sentence: A punctuation handbook for the innocent, the eager, and the doomed. Boston, MA: Mariner.

Grawe, S. (2005). Live/work. Dwell, 5(5), 76-80.

Hale, C. (1999). Sin and syntax: How to craft wickedly effective prose. New York, NY: Broadway.

Heider, F. (1958). The psychology of interpersonal relations. New York, NY: Wiley.

Hull, C. L. (1943). Principles of behavior. New York, NY: Appleton-Century-Crofts.

Jellison, J.M. (1993). Overcoming resistance: A practical guide to producing change in the workplace. New York, NY: Simon \& Schuster.

Kellogg, R. T. (1994). The psychology of writing. New York, NY: Oxford University Press.

Kendall, P. C., Silk, J. S., \& Chu, B. C. (2000). Introducing your research report: Writing the introduction. In R. J. Sternberg (Ed.), Guide to publishing in psychology journals (pp. 41-57). Cambridge, England: Cambridge University Press.

Keyes, R. (2003). The writer's book of hope. New York, NY: Holt.

King, S. (2000). On writing: A memoir of the craft. New York, NY: Scribner.

Kline, R. B. (2005). Principles and practice of structural equation modeling (2nd ed.). New York, NY: Guilford Press.

Korotitsch, W. J., \& Nelson-Gray, R. O. (1999). An overview of self-monitoring research in assessment and treatment. Psychological Assessment, 11, 415-425.

Lewin, K. (1935). A dynamic theory of personality. New York, NY: McGraw-Hill.

Parrott, A. C. (1999). Does cigarette smoking cause stress? American Psychologist, 54, 817-820.

Pope-Hennessy, J. (1971). Anthony Trollope. London, England: Phoenix Press.

Reis, H. T. (2000). Writing effectively about design. ln R. J. Sternberg (Ed.), Guide to publishing in psychology journals (pp. 81-97). Cambridge, England: Cambridge University Press.

Salovey, P. (2000). Results that get results: Telling a good story. In R. J. Sternberg (Ed.), Guide to publishing in psychology journals (pp. 121-132). Cambridge, England: Cambridge University Press.

Saroyan, W. (1952). A bicycle rider in Beverly Hills. New York, NY: Scribner.

Sawyer, R. K. (2006). Explaining creativity: The science of human innovation. New York, NY: Oxford University Press.

Silvia, P. J. (2006). Exploring the psychology of interest. New York, NY: Oxford University Press.

Silvia, P. ]., \& Gendolla, G. H. E. (2001). On introspection and self-perception: Does self-focused attention enable accurate self-knowledge? Review of General Psychology, 5, 241-269.

Skinner, B. F. (1987). Upon further reflection. Englewood Cliffs, NJ: Prentice Hall.

Smith, K. (2001) Junk English. New York: Blast Books.

Sternberg, R. ]. (Ed.). (2000). Guide to publishing in psychology journals. Cambridge, England: Cambridge University Press.

Strunk, W., Jr., \& White, E. B. (2000). The elements of style (4th ed.). New York, NY: Longman.

Stumpf, B. (2000). The ice palace that melted away: How good design enhances our lives. Minneapolis: University of Minnesota Press.

Trollope, A. (1999). An autobiography. New York, NY: Oxford University Press. (Original work published 1883)

Wrightsman, L. S. (1999).Judicial decision making: Is psychology relevant? Boston, MA: Kluwer Academic.

Wrightsman, L. S., \& Fulero, S. M. (2004). Forensic psychology (2nd ed.). Belmont, CA: Wadsworth.

Zinsser, W. (1988). Writing to learn. New York, NY: Quill.

Zinsser, W. (2001). On writing well (25th anniversary ed.). New York, NY: Quill.


\end{document}
