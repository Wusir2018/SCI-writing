\chapter{撰写期刊论文}
心理学期刊就好像20世纪80年代高中生电影里目中无人的运动健将或高傲的富家小姐一样——他们拒绝所有人,只有那些长得好看和坚持不懈的人才能获得他们的垂青。撰写期刊论文对自信心真是一种全方位的打击:成功概率很低;受到批评和拒绝的可能性很大;即便成功了,换来的结果也常常并不丰硕。做研究很有趣,但是写研究报告就毫无乐趣可言。尽管如此,我们还是必须撰写期刊论文,因为科学界通过期刊进行沟通。学术研讨会只是会会老朋友和互相切磋一下最近都在干些什么的地方,研讨会上的发言既没有同行审阅也不存档。 只有发表,才是研究过程自然的终结点。

文件柜里藏满了未能写出的文章。我认识很多科研人员,他们有整橱的数据;有些人还存着1980年以来未能发表的数据,他们希望“有一天能够发表”。他们当然这样希望。因为心理学界对在期刊上发表文章尤其热衷,所以学界提供了很好的资源,帮助初学者学习如何发表论文如:APA, 2010; Sternberg, 2000)。然而,大多数的资源都没能提供关于如何提高撰写论文动力的解决方案。本章将给出一些实用的个人建议。我会提供一些关于如何写作强有力的论文以及如何在面对不可避免的批评或失败的情况下坚持写作的建议。本章的这些建议不会使你爱上论文写作,不过能帮助你减少一
点畏难情绪,多写一点论文。

\section{关于实证型论文写作的实用建议}
写期刊论文就如同写一部浪漫戏剧的电影剧本:你得了解格式。听起来很奇怪,不过你真的应该庆幸有一本《美国心理学会出版手册》(Publication manual of the American Psychological Association)。一旦你知道了什么应该放在哪里,什么绝不应该放在哪里,你就会觉得写论文还是挺容易的。如果你还没有最新版的 《美国心理学会出版手册》(APA, 2010),你应该去买一本。

{\kaishu 1.大纲和准备工作}

在众多不良写作习惯中,“不列大纲”被我排在很靠前的位置——紧随其后的是“戴着粗糙的羊毛手套打字”,仅次于“让我的宠物狗帮我记录”。列大纲是写作的一部分,并不是“真正写作”的前奏。那些总是抱怨文思枯竭的写作者往往是不列提纲的。瞎写一气之后,他们就会抱怨写作如何如何难。这毫不奇怪——如果你不知道自己要写些什么,那么你肯定写不出来。多产的作者都喜欢列提纲。“清晰的思路孕育清晰的作品”,津泽(Zinsser, 2001: 9) 这样说。在和科学界进行交流之前,先把你的思路理清楚。

列大纲能够让你对论文大概有些思路。你打算写多长?你打算花多大的篇幅来介绍已有的研究?你打算把这篇论文写成一篇简单的报告还是一篇常规的学术论文?这些通常都取决于你自己,不过我建议你尽量简洁。学术期刊多年来被大量论文充斥着,近年来心理学的趋势是推崇简短的文章。有不少高质量的期刊只刊登短小的论文(例如《心理科学》),还有一些最近开辟了短小报告的专栏。短即是好。想象一下你在阅读学术期刊时的心情。你是希望快点看到结尾呢,还是希望作者再洋洋洒洒地写上14页?不要把所有的东西都塞进一篇文章里。你在职业生涯中会写很多文章,所以你可以把
一个遗漏的观点发展成为一篇新的论文。

内在的读者——一个假想的会阅读你论文的人——将帮助你来做决定。你应该如何详细地解释视觉注意力的竞争理论?你应该对某种统计方法加以解释,还是假设所有读者都已经理解?研究领域的其他专业人士——包括和你有同样研究兴趣的教授和研究生——是你最大的读者群体。你应为他们而写作。还有小部分另外的读者群体,包括本科生、记者、相关领域的工作人员和其他读者(例如博客博主或幽默作家)。对很多读者而言,英语是他们的第二甚至第三外语;当你被那些生僻而时髦的词语诱惑的时候,要考虑一下他们。为了更好地定位你的读者,你可以列一个你想要投稿的期刊名单。《实验心理学杂志:普通心理科学》通常拥有广泛的读者;有的期刊,例如《视觉认知和自我认定》,读者面就窄很多。当你为专业人士写作时,你可以假定你的读者知道相关领域的理论、发现和方法。你可以用通顺而专业的语气写作。你的目标是要让读者觉得你是一个值得交谈的正常人——别太严肃,也别太随便。


{\kaishu 2.标题和摘要}

大多数浏览你论文的读者只会看看标题和摘要,所以要认真对待。标题必须兼顾概括性和具体性:告诉读者你的论文是关于什么的,但是又不能太具体而显得过于功利和无趣。如果你想写一个时髦又幽默的标题,你要考虑到十年以后的情况如何。将来的读者能不能理解这种幽默?在数字化时代,读者通过电子资料库的搜索引擎来搜索标题或摘要从而找到你的论文。因此,在摘要里一定要包含你希望定位你的论文的所有关键词。比如,在我所有关于自我认知的文章中,我都会在摘要部分反复使用同义词,包括“自我关注”“自我关心”“自我意识”等。几乎所有人都是最后才标题和摘要的,你只要随大流就可以了。

{\kaishu 3.导论}

导论部分包含研究中的所有细节。在一篇论文中,导论部分的被阅读率是最高的。鉴于此,这一部分也是最让写作者们惧怕的。很多人告诫初学者,导论是没有固定格式的(例如:Kendall, Silk \& Chu, 2000)这是一派胡言——导论当然是有格式的。好的作者都遵从好的格式,很轻松就能辨别出来。

在导论的开始部分,先总体概括你的论文 ,一般长度为1$\sim$2页。 在这一部分,你应该描述你研究这个问题的缘由及理论依据。这一部分的目的在于论证本文存在的意义并吸引读者,同时帮助读者对全文建立一个总体概念。

总体概述全文后,用一个小标题来提炼导论的第二部分。这个小标题或许能够呼应你论文的题目。导论的第二部分是核心部分:这一部分你将描述相关理论,总结已有研究并更为具体地阐述你所做研究的目的是为了解决什么问题。利用好小标题和副标题来突出重点。如果有两种理论,那就为两者各取一个副标题。第二部分要紧紧围绕第一部分中你提出的研究议题。

第三部分的小标题应该是“目前的实验(The Present Experiments)”或“目前的研究(The Present Research)”。你已经在第一部分针对提出的问题给予了概括性的描述,在第二部分概述了相关理论及研究成果。现在,读者们理解了你所提问题的背景和重要性,那么在第三部分,你应该介绍你的实验,并解释为什么这个实验能够解决你提出的问题——这一部分可能有1$\sim$4页,根据详细程度不同而有所区别。这一部分的结尾通常是接下来研究方法部分的开头(方法或研究1)。

上述格式向读者介绍了你论文研究的问题(第一部分),总结了与问题相关的理论及研究(第二部分),还清晰地阐明了你的研究是怎样解决你的问题的(第三部分)。这样的格式能够引领读者的思路,也能帮助作者时刻围绕中心议题展开论述。当然会有例外——篇幅较短的报告,或许仅仅另起一段就足够了,不需要用小标题——但是上述格式适用于绝大多数的论文。

导论部分应该介绍您的研究,而不是不遗余力地把与问题相关的所有的研究从头讲一遍。简短的报告可能需要2$\sim$3页的导论;想要向期刊投稿的好大喜功的作者的论文可能有12$\sim$20页的导论。在写作一般的研究论文时,导论部分最好不要超过10页。

{\kaishu 4.研究方法}

研究方法部分可能看起来并不光鲜亮丽,但是它反映了你是如何认真对待研究的(Reis, 2000)。好的研究方法部分能够让其他的研究人员复制你的研究。与导论部分相似,方法部分也有固有的格式。这一部分由若干个小的部分组成。首先,研究对象或研究对象与设计,这一部分讲述样本的规模和特征,如果是实验研究,还包括实验设计。如果你的研究还涉及仪器——比如心理学仪器、非常规软件、反馈控制板、声控开关——你还需要用一个专门的部分叫做“仪器介绍”。测量方法部分适用于研究中包含集合测量、测试和评估工具的情况,例如神经认知测试、兴趣总结以及态度的自我报告机制和个体差异。

在完成上述各个部分之后,你将进入过程部分——研究方法部分的核心。在这一个部分,描述你做了什么和说了什么。读者往往对过程部分特别关注,所以不要让他们感觉到你在刻意隐瞒什么。尽可能详细地介绍自变量和因变量的情况。你的目的是将你的过程与已经发表的文章中的过程联系起来。如果你的实验使用了一种巳有的操作方法,那么即使这种方法已经广为人知,也请列明之前使用过类似方法的实验。如果你使用了一种新的操作方法,那么需要介绍以前用过类似方法的研究或者能够证明你的方法可行的已有研究。如果你的自变量包含分类(例如:社会焦虑程度高与低),那么需要介绍这种分类的依据(临界点、规则、惯例)并列举用过类似分类方法的已有研究。将你的过程与过去已有的研究联系起来能够降低读者对你的方法科学性的质疑。

读者们希望了解你是如何测量因变量的。如果你的因变量是经过慎重筛选后确定的,你应该列明以前使用过同样测量法的论文。如果是专业测试,那么要提供详细的测试手册,同时介绍最近的研究中有哪些用过同样的测试。如果你的因变量是特设的,比如你所写的自我报告项目,则要详细列明每个项目并介绍用过类似项目的论文。如果是自我报告测量法,则要标明数值——例如,7度可以是1$\sim$7, 0$\sim$6, -3$\sim$3——每一个数值都对应了相关标签(比如:1=一点也不,7=非常)。如果你的因变量测量涉及生理学或行为学的相关理论,那么要简要介绍一下过去哪些研究能证明你的测量方法是有效的。

如果你的论文有一系列的研究,你可以通过说明后续研究方法与第一个实验相同的办法来节省点篇幅。如果三个实验都用了同一种仪器,你就不必重复介绍三遍了。在介绍后面的实验时,只需说明它们与之前使用的仪器相同。

{\kaishu 5.结论}

结论部分用于描述你的分析。初学者往往觉得有必要报告有关数据的每一种可能的分析,或许因为论文答辩组成员希望看到这样的分析。但是,期刊论文应该是简明扼要的:只报告与你讨论的问题相关的结果。差劲的结论部分只是一长串的句子和统计;优秀的结论部分则讲述了一个故事(Salovey, 2000)。首先,在结论部分的开头,探讨一下你的研究的可靠性。这一部分可以分析自我报告测量法的内部逻辑性、预测评判间一致性、分析实验操作检查或者解释简化与处理数据的方法。

其次,按照逻辑顺序描述你的分析。这并没有一个放之四海而皆准的形式——根据你的方法和假设而有所区别——但是应该尽量把你的主要发现放在聚光灯下。萨洛维(Salovey, 2000)曾经建议把最有趣和重要的发现放在最前面。在描述结论的时候,不要盲目地一个实验接一个实验地接连陈述。介绍每个实验的时候,重新强调你的假设和报告数据,然后讨论这些实验的意义何在。但是初学者会反对:“针对结论部分的讨论应该放到下一部分——问题讨论部分!”这里存在一个由本科阶段学习带来的误区。结论部分不应该只是数据的展示。不要只汇报一个单项的趋势变化,然后说变化很明显。描述你的假设,汇报数据,然后说明这些发现的意义。哪个组的数据高于另一个组?结果是否与你的假设相符。

再次,多用图表或表格来增强结论部分的条理性。我常常写这样的论文评语:“作者应该列一张统计数据表。”如果是实验性研究,设计一张包括方法、标准差和单位尺寸的表格。更进一步,你可以加上95\%的置信区间——评论者会赞赏你的开放性,读者也能够对你的数据加以评论。如果是相关性研究,设计一张包括方法、样本量、置信区间、内部相关性评估和相关矩阵的表格。有了这些信息,读者能够自己创造和测试你提供的数据的结构方程模型(Kline, 2005)。没有法律规定你不能同时使用图表和表格:图表是为那些想要了解宏观数据样式的读者准备的,而表格则是针对那些想要了解具体细节的读者而设计的。

{\kaishu 6.讨论}

如果你的文章包含多项研究,那么每一个结论部分后面都应该紧跟一个讨论部分。这些讨论部分比一般性讨论部分涉及的面窄。它们总结了每项研究的结论并讨论这些研究如何探讨了论文的核心问题。讨论部分同样应该讨论实验的局限性,比如在研究过程中获得的意想不到的结果或者问题。如果你的讨论部分只是对结论部分的总结,你可以另外设计一个“结论与讨论”部分来作为补充。

{\kaishu 7.一般性讨论}

在一般性讨论部分,你可以将自己的研究放在其他理论或既有研究的聚光灯下重新审视。这一部分的开头应该对你的问题和发现作简单回顾:一般一到两段就够了。好的一般性讨论部分多种多样——你的问题、方法、研究领域都决定了你应该讨论些什么——但是这一部分应该很简短。想象一下你如何阅读一般性讨论。你是快速阅读、跳过,还是抱怨作者没完没了地讨论研究的每一个细节问题?应该尽量让这一部分短于导论部分。如果你愿意,可以在一般性讨论部分的最后用
一段的篇幅总结全文。

本科阶段研究方法课的老师会告诉你,应该在一般性讨论的最后谈谈你的研究的局限性;论文评议组成员或许也希望看到这个部分。讨论局限性在教学阶段或许是有益的,但是到了向专业期刊投稿的阶段就往往没有太大意义。有些局限性是普遍存在的:的确,如果能够有更大量和有效的样本就更好了;能够包含更多的测量方法也不错;的确,可以想象,在未来如果有研究能够运用更多的测量方法分析更多的样本,结论就会有所不同。别把你的读者当傻子——每个人都知道在类似的研究中存在的这些局限性。有些局限性在某一领域的研究中是普遍存在的。认知心理学家知道自己用了人为设计的计算机实验;社会心理学家知道自己运用了简单方便的本科学生作为研究对象。专家们都知道你的研究存在这些普遍的局限,所以不用再浪费时间来写这些显而易见的东西。相反,你可以用一些篇幅来讲讲你的研究特有的局限性。但不要太过于夸大这些局限性——提出来,然后巧妙地解释为什么它们并没有乍一看上去那么严重。

{\kaishu 8.参考文献}

参考文献部分用于收集整理对你的论文观点产生影响的资源信息。将你的作品放入科学领域,你的参考文献告诉了读者你如何看待你的研究。参考文献要有选择性——你不需要把读过的所有与研究有关的文章都列入参考文献,你也不应该把没有读过的书或文章列入参考文献。阅读论文的专家一眼就能看出你引的是二手资料。参考文献部分虽然不及导论部分光鲜亮丽,也没有结论部分强健有力,却值得你认真对待。作为一名论文审稿人,我见过很多草率的参考文献。懒惰的作者总是在挑战美国心理学会的写作格式,也常常忘记为行文中的引文作注释。“有什么大不了的?”有些人会说,“只是参考文献而已”。你的同仁能够看出你对参考文献的草率;应该让富有批判精神的匿名审稿人看到你最好的作品。

老练的作者利用参考文献来提高论文被自己希望的审稿人阅读的几率。编辑们在考虑你的论文的审稿人选的时候,常常直接翻到你的参考文献部分,看看你都引用了哪些人的作品。我不确定这个小技巧是否有效,但是试试也没有什么坏处。还有,别忘了在你的新作中引用你自己之前的文章。自我转引被很多作者认为是厚颜无耻的自我吹嘘。我就遇到过很多人对自我转引犹豫不决,他们大多是初学者。引用你过去的作品能将你最近的论文和你的研究脉络联系起来。如果有人对你最近的文章很感兴趣,那么他/她也许会有兴趣阅读你的其他作品。自我转引能够帮助他们找到这些文章。


\section{提交论文}
当你准备把论文投出去时,它应该是条清理晰、几近完美的。如果你总是想:“我先把它发送出去吧,等修改的时候再好好整理”,那我劝你还是打住,马上着手修改为好。只有受虐狂才会把潦草的草稿发给期刊编辑。原稿总是能够吸引眼球和获得审稿人的尊重,并且能够向编辑们展示你是一个严肃的值得信任的专业人士,相信你能够很好地根据修改意见进行修改。在你提交稿件以前,一定要花些时间阅读期刊网站上公布的投稿须知。请仔细阅读,因为每一份期刊的要求都有所差别。大多数期刊都接受电子稿件,稿件一般通过电子邮件或者在线提交软件提交。

不论你通过何种方式提交你的论文,你都需要写一封介绍信给编辑。有些人选择标准、简洁的信件;有些人则喜欢添油加醋地强调文章的优点和重要性。我曾经问过一些编辑重要期刊的朋友偏好何种介绍信。他们无一例外地喜欢简洁的介绍信。它包含一些程式化的内容:论文标题、作者的电子邮箱地址,以及一些常规的内容(这份稿件没有投往其他期刊,稿件资料的收集方式均合理合法,等等)。有一个副主编曾经提到他从来不看介绍信,因为在线提交系统使他很难阅读。另外一位说她更希望被文章打动,而不是被介绍信打动。

在介绍信里,你可以建议几位你认为合适的审稿人,以及几位不合适的人选。我从几位编辑朋友那里获知,他们往往会尊重“不合适人选”的提名,而对“合适人选”有所保留。也许期刊的某位副主编是非常适合你的论文的审稿人选,如果你乐意,你可以建议编辑把你的稿件发给这位副主编。(虽然我尝试过很多次,但是最终我的稿子从来没有被送到我建议的人那里去。)


\section{读懂审稿意见并重新提交论文}
我在随意翻阅一些较早版本的《儿童发展》时,无意中发现了一篇20世纪70年代早期的评论文章。作者描述了同行间互相审稿的流程,提到通常的反馈周期是6周。想想看,30年前,作者把像砖头一样厚的稿子寄给编辑,编辑再把稿子寄给审稿人。审稿人把他们的意见用打字机打出来,再寄还给编辑。编辑们打一份执行意见信(action letter),留存一份副本,再把执行意见信和审稿人的审稿意见寄给作者。今天,作者、编辑和审稿人通常用电子方式交流,用先进的在线系统投稿、向审稿人和编辑寄送通知单等,避免了由于邮寄信函而造成的延误。当你在等候审稿结果的时候,应该好好感谢高科技带来的便利。

当编辑的执行意见信寄到的时候,他往往会在信中总结审稿人的主要意见,并告知稿件是否被采用。结果可能有三种:采用、要求修改、拒绝。

\begin{itemize}
\item 采用的情况比较容易理解。编辑通常会说你的稿件被采用了,然后会要求你填写一些表格;有时候编辑会让你在稿子发表前做一些小的改动。原稿直接被采用的情况并不多见。即使他们很喜欢你的论文,他们也常常会要求你作些删减或添加内容。有的编辑偶尔会无条件地接受一些论文——所以说要十分注重第一稿的质量。
\item 在希望犹存的情况下,编辑会要求你修改论文。这一类回信差别很大,有的令人无比振奋,感觉距离稿件被采用只有一步之遥了;有的则列出一大堆修改意见,让人沮丧不已。有的门开得比较大,只要求做一些简单的修改,例如重写某个部分或添加某些信息。有的门只留了一道缝,要求要做大的修改,例如重新收集数据或是重新考虑研究的概念基础。有时候,编辑们会告诉你,他们将把做出重大修改的论文作为新稿件加以处理。
\item 在大门彻底关闭的情况下,编辑不希望再看到你的论文。有时候,退稿信会鼓励你把稿子投到别处;有时候,编辑会给你寄来一台碎纸机,让你彻底毁了这篇论文。如果门关上了,你就别再重新提交论文来挑战编辑的底线了。
\end{itemize}

即使是老练的研究者也常常搞不清楚编辑是否愿意给论文的修改稿一次新的机会。“拒绝”这个词并不一定意味着你不能重新提交论文。很多编辑都会对未被采用的稿件使用“拒绝”这个词。他们拒绝了你的第一稿,却有可能会接受修改稿。我猜想有些不喜欢说“不”的编辑会用一些让人泄气的话来拒绝作者——“如果您增加三个实验并重写导论和一般性讨论部分,我们会很愿意重新考虑您的修改稿。”当不确定的时候,把审稿信给朋友参谋一下,或者给编辑去封简短的邮件确认一下。

如果大门还开着,你要考虑清楚自己是否愿意修改。编辑可能希望有新的数据、新的分析,或是重新组织的某些部分。这个项目是否值得付出更多呢?默认的首选应该是修改并重新提交。你要记住,所有期刊的退稿率都非常高。如果你收到修改论文并重新提交的邀请,你已经在为降低退稿率做贡献了。如果这份期刊很有权威性,你应该努力修改,例如增加一个实验。如果这篇论文并不那么重要,你或许可以试试其他期刊,而不是费时间重新收集数据。

决定要修改并重新提交论文之后,你应该做个计划。仔细研究编辑的来信和意见,并提炼出修改要点。(不要用“可修改的要点”这样的词——这样的表述太模棱两可了,就像“可饮用的”或“可做的”一样,似乎可做可不做。)修改要点就是要修改的部分。仔细阅读编辑的来信和意见,把所有提示需要修改的意见都标记出来。可能是文字上的修改——增加、删减或是重写——或是修改分析部分。也可能是比较大的改动,比如增加或删除某个实验。很多审稿意见天马行空,很长的一篇只有寥寥几个修改要点。在你标出修改要点后,就尽快修改。在第三章里,我把修改稿件放在目标列表中比较重要的位置。因为它们离发表更近,所以不要磨磨蹭蹭的。有的编辑会给个修改期限,如60天或90天。

当你重新提交稿件时,你需要写一封再次投稿的介绍信来说明你是如何处理批评与意见的。至于应该写一封简短的信来标出大的改动之处,还是应该列一份详细的改动清单,依我私下与编辑们交流的经验来看,他们更喜欢翔实仔细的信函。大多数的编辑抱怨作者的信写得太简略(“我们修改了很多;我们希望您能满意”),作者们要么拒绝修改,要么就只提那些作了修改的部分,却从不解释为什么有些地方未作修改。所以,在信里详细地列出你做了哪些修改,哪里没有修改,这有助于编辑接受你的修改稿。

二稿介绍信应该具体和有建设性;你应该坦诚地、透彻地讨论所有修改要点。那些成功发表大量作品的作者都是写介绍信的高手。这些信很好地介绍了你所做的修改,并向编辑证明你很好地处理了反馈意见,你是一位严谨的科研工作者。简短、模糊的信让人感觉作者要隐瞒什么;长而翔实的信显得作者态度积极和真诚。信也要写得礼貌和专业——你的信不是为了显示对一位偷懒的审稿人的不满,也不是为了向一位好挑刺的审稿人展现你的骄傲,更不是为了夸耀你高超的统计水平。这些是很有诱惑力,但还是应该以科学大义为重。

我收集了大量成功的二稿介绍信,写信的人都是我的同事,他们都发表了大量的论文,也是很多期刊的编辑。以下列举一些要点。

(1)开头部分应该感谢编辑给予的建议和再次提交论文的机会。虽然你会觉得文章被退改不如被直接采用来得令人高兴,但这起码比被直接拒绝要好得多。

(2)给每一个修改要点起一个小标题。很多作者根据审稿人的序列来组织这封信。通常的做法是对应审稿人1、审稿人2的评论来拟定小标题,以此类推。每个部分都应该覆盖每一位审稿人的每一个意见,并用数字标识。用数字标识简洁明了,并且便于查找已经出现过的评论。例如,也许两位审稿人都提到应该增加样本的一些细节。虽然你已经讨论过审稿人1的意见,但到了审稿人2的时候,还是需要重复一下。只需要很快地重复一下这个意见,然后指明与上述第几条重复即可。

(3)每一个修改要点的阐述应该包含三部分。首先,简单总结意见或是批评的内容。其次,解释你针对这一评论做了哪些修改;如果可能,请列出这一要点具体在你论文的第几页。最后,阐述你的修改怎样回应了评审意见。

(4)编辑们并不指望你对每一条意见都一一修改,但是他们想知道你不做修改的理由。我见过最极端的二稿介绍信,作者固执地拒绝做一些无关紧要的修改,例如把几张小的表格合并成一张大的表格或是删掉10\%的文字。好好选择你需要修改的部分。如果你不接受修改意见,要在你的介绍信里详细地说明你为什么不愿意修改。

(5)注意保持专业性。别显得卑躬屈膝或是刻意谄媚。编辑们并不认为审稿人是天才,所以他们也不希望你把审稿人的意见形容成:杰出的、前所未有的、出色的、深刻的,等等。请你把自己放在编辑的位置上思考一下。一封溜须拍马的二稿介绍信到底是会打动你,还是会让你觉得“这个人真虚伪”呢?

一封好的二稿介绍信会让你看起来是一位认真的科研人员——其实你本来就是。那些认真对待批评意见的人写的论文值得被发表。有时候我会花比修改论文更长的时间来写这封信。我的一篇论文的二稿介绍信(Silvia \& Gendolla, 2001)有3200字,差不多和这本书的第五章一样长。我发表的很多论文都没有3200字。


\section{“如果他们拒绝了我的稿件怎么办?”}
很多作者很害怕收到负面的反馈和遭到拒绝。传统的成就动机理论显示有两个最主要的影响表现的动机:成功需求和避免失败的需求(Atkinson, 1964)。情境因素能够夸大这些动机,而写论文看起来能够唤起作者对避免失败的需求。很多作者——尤其是学界初学者——对“被拒”总是耿耿于怀。他们担心编辑们会怎么说;他们想象某一位审稿人在读他们论文的时候皱眉的样子;他们非常害怕收件箱里的退稿信。

避免失败的本能会让人们反复地问:“如果他们拒绝了我的稿件怎么办?”他们当然会拒绝你的论文。你写论文的时候就应该假设会被拒。决策理论指出,在不确定的前提下,基本概率是预估结果的最合理的依据。如果一份期刊拒绝大概80\%的稿件,那么稿件被接受的基本概率就是20\%。在缺乏其他信息的情况下,理性的判断是,你的论文有20\%的概率会被接受。因为没有期刊的拒绝率低于50\%,所以我假设我投的稿件都会被拒绝。这是唯一合理的结论,而且我被拒的次数也验证了我对理性分析的坚待。

“真是暗无天日啊,”你也许会说,“如果你知道你的稿子会被拒,你怎么还能打起精神来写呢?”首先,我们不应该寻找写作的动力,而是应该坚持执行写作计划,不论刮风下雨 其次,初学者往往觉得只有他们才会收到退稿信。其实那些已经发表了很多论文的作者同样会收到很多退稿信。心理学界最多产的作者在一年内收到的退稿信可能比有些写作者十年里收到的还要多。我甚至觉得被拒的基本概率反倒让人觉得安心。我对将发生什么并不确定,所以当我收到退稿信的时候我并不觉得太糟糕,而且在我完成论文以前,我也不会放纵自己沉溺于自己的文章即将变成铅字的幻想之中。

如果你假设自己的文章会被拒,你就能写出更好的文章,原因是你对避免失败的需求被屏蔽了。为了避免失败而写作的作者所写的文章读来小心翼翼、空洞而充满犹疑。他们总是设法使自己的文章看起来不坏,而不是看起来更好。读者可以感受到这种恐惧。相反,为了成功而写作的作者,他们的文章读起来充满信心和控制感。这些作者把重心放在作品的长处上,强调研究的重要性,传达着一种颇有说服力的自信。

审稿人是否会讨厌你的文章?是的,有的时候他们的确会讨厌你的文章。以下是我最近收到的一封退稿信的节选。在审稿意见总结部分,编辑写道:

{\kaishu 两位审稿人都认为您的论文未达到发表的水平。一位审稿人认为您的论文意义不大,对相对立的理论有所误读,结论与研究的证据未能很好匹配,而且写作也不甚精确。另一位审稿人认为论文未能推动建立完整而准确的模型,论证不够有力,部分重要的研究和观点缺失,且作出了一些错误的理论假设和批评。}

而且这还是通过编辑转述的——其中一位审稿人真是挑剔。不过这也没关系。我提取了审稿意见中的一些修改要点,修改了论文,然后投给了另一份期刊。考虑到基本概率,也许还是会被拒。

有时候,拒绝的决定是不公正、刻薄甚至毫无道理的。有时候你能够看出编辑或审稿人并没有仔细阅读你的论文。请你克制住向编辑抱怨的冲动。我听说有些人向编辑投诉,怒气冲冲地指责审稿人又懒又不称职。也许因为编辑往往和审稿人的私交很好,这些信大多石沉大海。有人建议你应该写封投诉信发泄一下,但是不要寄出。这似乎更不合理——为什么要浪费你的写作时间来做这些毫无意义的事情呢?把你的时间用于修改论文吧。世界是不公平的(p<0.001),所以你只需吸取审稿意见中有用的建议,修改你的论文,然后投到别的期刊去。

为了写得更多,你应该重新考虑一下你对待被拒和发表的认知模型。被拒就好像是为发表文章而交纳的销售税:你发表的论文越多,你收到的退稿信也会越多。如果你按照本书的建议来做,你将很快成为你们系里收到拒信最多的人。



\section{“如果他们要我修改所有的部分怎么办?”}
期刊是科学界公开的记录。你的论文将被印在无酸纸上,被永远存放在图书馆的书架上。如果人们能够把自己的研究与其他人的研究联系起来,在研究中阐述自己的观点,合理地分析数据并客观公正地说明自己和其他人已经取得的成果,那么科学进步的步伐就会更快。期刊不是心理学家宣传个人观点的论坛——在简报或学术会议上你可以那样做。学界对公开发表的论文的要求很高,并运用同行评审的方式来控制质量。你会被要求修改你的论文;有时候这些修改涉及的范围很广。如果这让你觉得不舒服,那你会不情愿地听到一个事实:最终得以发表的论文质量普遍高于初稿。能够发表的论文,其观点更集中,较少自相矛盾,更严密。互相审核制对于作者来说是令人厌烦的,但是这一制度是达成心理科学发展目标的核心所在。

\section{合作撰写期刊论文}
有时候,我们需要很多人合作来完成一项研究,但其中大多数人不会参与论文写作。我问过很多人如何与其他作者一起合著论文,几乎所有的人都说是其中一位作者撰写了大部分内容。合著者们一同列提纲,但是由其中一位作者完成写作。论文写完以后,所有的作者一起阅读、讨论、做必要的修改。对这一模式的改进办法是把各个部分分派给不同的作者。通常的做法是让一个人来写方法与结论部分,另一个人写剩余部分。不过,我也发现有人做到了真正意义上的“合著”。有一对合著者在电脑前放了两把椅子,讨论写些什么,然后把键盘传来传去。另一位说他和一位同事把两台电脑搬到一个房间里,然后一起完成了他们的课题基金申请报告。这样做使他们能够解决申请报告中纠结的问题,也能够随时向对方提出问题。可见,通过合作来完成论文写作也是可行的。

你要小心选择合著的人,不要在还未详细讨论由谁来写的情况下就投入一项与人合作的研究。如果你的搭档是一位突击写作者,请对他承诺会很快写完或是对研究表现出的激情持谨慎的态度。热情不代表投入。如果你无法信任你的搭档,那么你应该自己写初稿,并确保自己是第一作者。有时候,你辛苦写完了初稿,你的搭档却永远无法完成修改的工作。你应该在给他们初稿的同时给一个最后期限。例如:“我希望这篇论文能够在两周内提交,所以麻烦你在这之前回复我。”期限一过就提交论文。我的一个朋友给一位拖拖拉拉的合著者写了封邮件,邮件主题是“不带你玩儿了”。这招很管用。

对于研究生们来说,拖拉的合著者是个大麻烦,特别是如果一起合著的人是系里的导师。很多学生抱怨导师拖延了他们的论文——有的导师给学生的论文写意见要拖上好几个月甚至好几年。对学生来说,催促导师是有难度的,所以得想些办法。试试让其他人来催你的导师。为什么不向系里的其他老师抱怨?如,系主任或研究生项目负责人。如果这也没用,把本书的这一章节复印一份,匿名放到你导师的邮箱里。这一举动虽有些鲁莽,但希望能够把导师的注意力吸引到你的论文上来。最后,给导师一个期限,超出期限之后你自行提交论文。如果你的导师不愿意读学生的论文并提出意见,说明他缺乏对研究生教学和科学发展的投入。你可以告诉他,“我真的需要在4周内提交这篇论文”,然后在2$\sim$3周后提醒他。


\section{写评论文章}
在写了那么多实验论文之后,或许是时候考虑写点评论文章了。评论文章的读者众多:寻找新观点的研究人员、在全新领域学习的学生、备课的教师、关注心理学最新动态的政策制订者等。论文写作其实不难,只要你掌握了美国心理学会的要求就容易上手,但是评论文章不一样。写作动机层面还是一样——坚持执行你的写作计划,但是具体组织安排层面就很不同。研究者可以出于不同的目的写作各种不同类型的评论文章,结构、方法也差别很大(Cooper, 2003),而且没有统一的格式。

正因为评论文章千差万别,你必须做好计划。首先要想清楚评论文章的篇幅。有的期刊倾向于发表短小精悍的评论,例如《当代心理学研究方向》;另一些,例如《心理学探究》《心理学公报》《心理学研究》,都接受篇幅较长、较全面的文章。你想写多长?其次,你要考虑你的读者群是谁。除了综合性的评论期刊,心理学领域还有很多评论是写给特定读者的,例如《实验心理学研究》和《人格与社会心理学研究》等。你希望你的读者面广一些,还是希望你的读者是一小部分专业研究人员?

当你对篇幅和读者有所考虑以后,你需要列一个提纲,写明你的核心观点。评论文章必须提出自己原创的观点,而不是简单地复述已有的研究。最糟糕的评论文章是把对其他文章的描述弄成一个大杂烩。读一篇没完没了的评论——这篇文章发现了这个,那个实验证明了那个,另一项研究说明了这个——就好像看着衣物在洗衣机里不停翻滚,但是最终洗衣机里好歹还会有洗干净的衣服出来。为了提出你原创的观点,可以参考创造性方面的专家提出的关于“解决问题”和“发现问题”的区别(Sawyer, 2006)。一篇“解决问题”的评论描述一个问题(例如一个有争议或模棱两可的研究领域),然后提出解决问题的方法(例如一种新的理论、模型或解释)。一篇“发现问题”的评论提出一个新的概念或提出一个值得关注的新话题。真正好的评论应该包含解决问题和发现问题两方面。例如,解决具有争议的两种理论,通常为未来的研究指明方向。你想解决的问题是什么?你的结论里又有哪些新的观点?

评论文章最常见的缺点就是没有原创的观点。很多作者把研究改头换面解释一番,却没有结论;另外一些作者讨论了互相对立的理论却没有解决方案。有两个原因导致了上述问题:首先,如果作者本人没有新的观点,他当然无法提出新的观点。有时候就是这样。在阅读了大量的文献之后,你或许会发现你并没有什么要补充的。如果是这样的话,你就不要执意去写一篇评论文章,仅仅证明你花了那么多时间来阅读了文献。其次,有些作者不列提纲。他们在一堆文章旁边坐下来,开始描述每篇文章写了什么,然后加上一小段“评论总结”,就完事了。一个复杂的项目需要一份强有力的提纲——如果没有提纲,你的观点就会被淹没在浩如烟海的已有研究中。那些不喜欢列提纲的人不应该写评论文章,而应该到本地的动物收容所领养一条狗,因为狗不会因为他们这样荒谬而自以为是的习惯而嫌弃他们,狗会一如既往地爱着他们。

如果你有好的观点,别藏着掖着。你的观点应该写在文章的开头几段里。评论文章的第一部分,你应该大致介绍一下文章的核心观点,分几大部分,然后透露一下你打算讨论的原创观点。按照时间顺序来写评论——理论一、理论二,然后分析,这看起来很有吸引力,可是千万别这样写。评论文章包含太多的信息,所以你需要在文章的开头就给读者一个清晰的思路。与出色的推理小说不同,好的评论文章在第一页就揭开了谜底。

写作评论文章看起来有一定难度,确实不容易。这也是突击写作者很少写评论文章的原因:有太多东西要读,要消化,要写。但是善于反思、有规划的作者就没什么好害怕的。如果你有一个时间表,那写评论文章也绝非难事:你有清晰的目标、不可回避的时间安排、好的习惯,所以完成评论文章只是时间问题。当你决定要写一篇的时候,花一点写作时间来收集好的建议。鲍迈斯特和赖瑞(Baumeister \& Leary, 1997)写了一篇非常棒的写作评论文章的指南;你还可以参考一下本(Bem, 1995) 、库伯(Cooper, 2003)和海森堡(Eisenbery, 2000)的建议。

\section{小结}
当人们在为第一篇论文苦苦挣扎的时候,很多作者哀叹道:“为什么他们一点也不在乎我的研究?”如果“他们”指的是广义的世界的话,我向你保证他们真的一点也不在乎你的研究;但是如果“他们”指的是同一领域的研究人员,那你应该想到他们其实是有一些兴趣的。记住你是在为与你有着共同研究兴趣的专业人士写作具有技术含量的文章。或许你的文章在找到归宿之前被拒绝了一两次,但是真正好的文章总会找到归宿。为了写得一手好论文,你必须充分掌握文体,提交干净整洁的初稿,还要善于写出漂亮的二稿介绍信。你会发现期刊的世界并不可怕,只不过速度真的很慢。