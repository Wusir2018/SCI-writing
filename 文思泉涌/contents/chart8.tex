\chapter{还有好东西要写}
本书介绍了一个实用的体系,目的是帮助你成为一位高产的学术写作者。第二章反驳了妨碍人们写得更多的一些借口,并介绍了整个体系的核心:按照写作计划按时写作。为了帮助你执行写作计划,第三章和第四章描述了如何设定好的目标和优先项,如何监控写作和如何组建“失写互助组”。第五章告诉你如何写得更好,第六、第七章提供了一些关于期刊论文写作和专著写作的有用的建议。令人感到讽刺的是,这本介绍如何写得更多的书本身篇幅却很短,但是我要说的已经都在这里了。这个体系其实很简单。

\section{计划之乐}
执行写作计划的过程是充满乐趣的。你每周可以写更多页的文章,这样你就能完成更多的期刊论文、更多的课题报告、更多的章节和更多的专著。按计划写作,你就能将“找时间写作”的不确定性和痛苦转化为对“什么时候能完成”的期待。项目都能够在截止期限前完成。你在开学第一周花在写作上的时间和暑假里一样长:你的写作计划让你的写作成果呈现出令人赏心悦目的正态分布。写作会变得寻常、固定和有规律,而不是沉重、无常和被迫。

执行写作计划还带来了令人意想不到的快乐:手工业者的骄傲。外在的对写作的奖励很有限,而且难以预料——往往是在一堆拒信中能够找出一份用稿通知。内在的奖赏对于突击写作者来说更是稀有。突击写作者总是带着罪恶和紧张的情绪,从来不觉得写作的过程是有意义的。正因为长时间地持续恶性循环,写作对于突击写作者而言总是和精疲力竭的阴霾相伴,这更加剧了他们对写作的厌恶。当你执行写作计划的时候,行为主义者或许会说,你掌控了你的自我强化计划(Skinner, 1987)。你知道什么时候你将因为实现了目标而得到奖赏。我的目标是每个工作日早上写作。有时候我写了很多;有时候却收获甚少,疲惫不堪。但是即使是在那些糟糕的日子里,我还是为我能坐下来写作感到高兴:我很自豪地在我的SPSS表格里打上“1”,然后我给自己一个小小的奖励(比如来一杯上等的咖啡)。我从不想写作——想出门去吃个面包圈的冲动有时候很强烈,但是不管怎样我写了。在坚持写了这么久以后,正是每天的小小胜利,而不是最终发表的成就感,激励着我坚持写作。



\section{少点空想,多些实干}
你不需要特殊的天赋、基因或是动力来帮助你写得更多。你也不需要“想写”才写——人们通常很难喜欢这项毫无乐趣又没有截止期限的工作——所以不必等到你喜欢上写作才写作。高效的写作有赖于习惯的力量,而习惯养成的关键在于重复。制订一个写作计划,然后坐下来写作。最初的几个星期,你或许会怨天尤人,还会恨得牙痒痒,但至少你是在计划写作的时间里诅咒,而非无所事事的时候。熬过了几周之后,你就会对写作安排习以为常,而你再也不用承受除此之外的时间里必须写作的压力了。一旦你的写作计划僵化成一种牢固的习惯,你就会对“想要写东西”的论述感到不可思议。习惯的力量会促使你坐下来写作。

令人感到讽刺的是,写得更多并不会让你喜欢上写作或者更想写作。写作很难,永远都很难;写作令人头疼,永远都令人头疼。大多数日子里,我都不想坐在那张硬邦邦的椅子上,打开电脑,继续一篇写了一半的论文。不过,上课也令人沮丧;奋战在冗长的会议中就更别提了。那么,人们是怎么对待这些任务的呢?他们只是“出现”而已。做一个写作计划,然后按时“出现”。少些空想,多点实干。威廉·津泽(William Zinsser, 2001: 285) 说:“想好你要做什么,然后做好决定,然后开始做。”





\section{写作不是比赛}
你想写多少,就写多少。虽然本书介绍了如何写得更多的方法,但是千万不要觉得写得越多便越好。本书更确切的题目应该是《如何在正常的工作周内带着较少的焦虑和负疚情绪更高效 地写作》,但如果这样的话,相信没人会买这本书了。如果你想写得更多,那么写作计划能够帮助你。你每周会花更多的时间写作,也会更高效。逐渐地,你就会把那些堆积如山的数据清理干净,你写作的时候也会更为得心应手。如果你不想写得更多,写作计划能够帮助你消除写作带来的罪恶感和不确定性。你就不用总是牺牲周末的时间来进行无谓的突击写作了。如果你一生中只打算写很少的作品,那你的写作时间可以用来思考。用这些时间来阅读和思考你的职业发展。

文章发表得更多并不意味着你就是一位更为成功的人、心理学家或科学家。心理学界许多高产的作者不停地重复着同一个观点:从论文到评论文章,从评论文章到专著的部分章节,从部分章节再到报刊文章和专业指南的某个章节。高产的写作者会有更多的公开作品,但并不一定就比别人有更多好的观点。写作不是比赛。不要仅仅为了发表而发表。不要计算你发表了多少文章,出版了多少专著。你应该为拥有这样的文章而骄傲——它们可以在某些地方发表,但不能在所有地方发表——它们潜伏在你书橱的文件夹里。如果你发现自己正在扳着指头计算你学术大厦上的砖头,你或许应该花一段写作时间来思考你的动力和目标究竟是什么。




\section{享受生活}
一份写作时间表能够平衡你的生活——不是伪科学里所谓内心无欲无求之后获得的奇妙平衡感,而是将工作与娱乐区分开来的平衡。突击写作者傻傻地寻找着整块的时间,但是他们找到的是晚上和周末。于是,他们牺牲了正常的生活。学术写作真的比与家人共处、爱抚小狗、品尝上好的咖啡更为重要吗?得不到爱抚的狗该有多么悲伤;一杯无人品尝的咖啡永远只是流失的咖啡因。你应该像保护你的写作时间一样保护你的正常生活。晚上和周末的时间应该用来与家人朋友团聚,制作独木舟参加阿尔瓦·阿尔托(Alvar Aalto)老式家具拍卖,尽管你未必需要它们,看最新的电视剧,粉刷百叶窗或者教你的猫怎么上厕所。随便你如何打发你的业余时间,就是别写东西——你在工作日有时间干这个。



\section{小结}
这本书告一段落了。谢谢你的阅读。我很享受写这本书的时光,但是对我来说,是时候写点别的了。对你来说,也是时候写点别的了。让我们往前看。威廉·萨拉扬(William Saroyan, 1952: 2)这样写道:“每当我想着还有好东西要写的时候,我都欣喜无比,因为好东西永远写不完,而我知道我会成就其中的一部分。”