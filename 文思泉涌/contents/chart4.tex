\chapter{组建属于你的“失写互助组”}
抱怨是学者与生俱来的权利。抱怨的级别逐步加深。读本科的时候,学生们抱怨教授、课本和周五上午九点的课。读研究生的时候,学生们的抱怨开始涉及一定的专业领域——他们被枯燥的统计课折磨得不行,抱怨导师的专横,还有无处不在的写了一半的论文。当然,教员和教授们更是把抱怨上升为一种“高雅”艺术,尤其是说起晋升教务长的机会或是预留停车位的特权时。

有时候,这些抱怨与写作有关。教授和研究生们都很喜欢抱怨写作:要写完论文有多难,他们可能没办法在截止日期前完成课题经费申请报告,或是他们在春假时没有完成预期的写作任务。对写作的抱怨通常都是糟糕的,尤其是如果内容还与前述笫二章提及的那些借口有关。当人们坐在一起讨论,如果有更多的时间写作或者能有一台新的电脑,他们就肯定能完成多少任务的时候,他们其实还是在那些无用的、浪费时间的、突击式写作的坏习惯里转圈。 不过我们能不能想办法把学术界喜欢抱怨的“优良传统”改良一下,让它发挥一些好的作用呢?我们能否利用自身喜欢集体哀号的原始冲动来帮助我们写得更多呢?

这一章将介绍如何组建你自己的“失写互助组”,这个互助组是一群想要写得更快更好的人聚集在一起的互助俱乐部。它运用激励原理、目标设定和社会互助的方式来帮助人们保持良好的写作习惯。如果你接受了第二章和第三章提到的建议,那么你应该已经制订了写作计划表、写作任务列表,并挑选了一部分优先写作项。互助组能够巩固这些好的写作习惯,防止你退回到突击式写作的黑暗时代。

\section{“失写互助组”模型}
和所有其他学校的心理学系一样,北卡罗来纳大学格林斯伯勒分校(University of North Carolina, Greensboro)的心理学系也有一群希望能够更高效地写作的教授。几年前,我的一个朋友切丽·罗根(Cheryl Logan)提议组建一个每周会面一次的作者俱乐部。我们觉得围绕一项目标设定的研究来组建这个互助组挺有意思,组建互助组的目的就是提一些实用的建议来帮助大家保持最佳的写作动力(Bandura, 1997)。我建议把这个互助组命名为“特罗洛普社区”,以此来纪念维多利亚时期的小说家安东尼·特罗洛普。特罗洛普一生写
了63本书;大多数都是系列作品。心理学家们能够从他身上学到很多东西。他一生大部分的创作,包括最经典的《巴塞特郡纪事》(包含6部小说),都是他在邮局全职工作期间写成的(Pope Hennessy, 1971)。为了完成写作,他每天早晨从5:30写到吃早饭。正如他在自传中所说的那样,“每天写三个小时,任何人都足以完成所有应该完成的写作”(Trollope, 1883/1999: 271)。

特罗洛普是位很棒的作家,但是“特罗洛普社区”却是个很傻的名字。切丽建议取名为“失写”——一种丧失写作能力的疾病——也很好地反映了我们当中的大多数人对写作的感受。我们召集了一些同事,“失写互助组”就这样诞生了。北卡罗来纳大学成立“失写互助组”的目的是为了给大家一个讨论各自正在进行的写作任务的机会,听听别人对于写作中存在的问题的意见和分析,帮助大家设定更为合理的目标。我们没有正式进行过项目总结,所以没有一手的材料来分析我们的“失写”模型。不过,我们多年来坚持定期见面,并且相信这样做是非常有效的。我们也有理由为我们的“失写”模型而沾沾自喜,因为有人效仿了我们——其他大学的朋友们听说了我们的模式后,也建立了他们的“失写互助组”。一个成功的“失写互助组”必须具备以下五个要素。

{\kaishu 1.要素一:设定具体的、短期的目标,并监控整个互助组成员的进展}

动机研究表明,设定最贴近的目标有利于强化动机(Bandura, 1997)。当人们设定具体的、短期的目标时,他们能够看到达成目标的路径和考量他们达成目标的速度。当互助组成员会面的时候,每个人都应该准备好下一次开会以前他们各自想要完成的目标。这些目标可以参考第三章,它们必须是具体的。比如“考虑一下我的论文”这样的目标应该遭到组内其他成员的唾弃;而像“写个大纲”“写问题讨论部分”“至少写1000个字”和“打电话给国家精神卫生研究所的工作人员讨论我的课题报告”这样的目标应该受到赞赏和鼓励。“争取写作”不是写作——不能允许互助组成员设定诸如“争取完成大纲”或者“争取写100个字”这样的目标。

如第三章所述,我们必须监控我们的目标进展。我们每次开会的时候都会带一个目标文件夹,然后每个人都会口述下次开会前他/她打算做什么。我们把每个人具体的目标写下来,夹在文件夹里。下次开会前,我们会检视过去一周的目标并汇报我们是否完成了目标。这样做的目的是防止成员浑水摸鱼或是记错他们一周前所说的目标。

一个“失写互助组”应该每周或每两周活动一次。超过两周,目标就会变得太宽泛和模糊。我们互助组的核心成员每周活动。有部分成员只能每两周抽出时间,所以他们设定的目标会比每周见面的成员宽泛一些。

{\kaishu 2.要素二:只关注写作 ,而非其他}

教授们有很多职责。“失写互助组”的活动很容易演变为对行政人员、教学或是研究生的批斗大会。要避免这样的情况发生。我们的很多次活动都很简短:从回顾上一周的目标开始,检查哪些完成了,哪些没完成,然后设定新的目标。每个人可能就花几分钟。如果有多余的时间,通常会讨论一下每个成员面临的挑战,例如与出版人员沟通商定一份合同,或是鼓励一个不喜欢写作的研究生第一次准备一篇课题申请报告。

当“失写互助组”顺利运转起来后,可以考虑在活动的时候一起读一些关于写作的书。可以在回顾和设定目标以后对这些书展开讨论。本书自然是一个选择;如果没有这本书,你也可以找到很多值得阅读和讨论的书。如果互助组成员对写作风格很头疼,可以读《写得更好》(On Writing Well, Zinsser, 2001)和《垃圾英语》(Junk English, Smith, 2001)。如果组员觉得写作动机很难保持,那么就应该读《作者的希望之书》(The Writer's Book of Hope, Keyes, 2003)、《教授也是写作者》(Professors as Writers, Boice, 1990)和斯蒂芬·金(Stephen King)的《论写作》(On Writing, 2000)。

{\kaishu 3.要素三:“胡萝卜加大棒”}

“失写互助组”应该用非正式的社会性奖励来强化好的写作习惯。如果组员提交了一份课程申请报告或是完成了一篇期刊论文,这实在是一件值得庆祝的大事。如果互助组成员都迷恋咖啡因,你们可以用请客喝咖啡的方式来奖励好的写作习惯。小小的“胡萝卜”是“失写互助组”走向成功的一大秘诀。不过,互助组也不能永远无条件地提供支持。如果有人总是完不成他/她的写作目标,互助组就应该及时介入。互助组不是纵容各种借口的温床,也不是保护坏的写作习惯的堡垒。很少会有组员彻底放弃——因为这些人根本就不会参加什么互助组——不过我们应该做好准备,以应对有些组员会一而再、再而三地完不成计划的情况。好的办法是在这个时候问问这个人的写作计划是什么。这个问题通常能够揭示这个人为什么一直不能完成计划。然后,我们应该敦促这个顽固的组员制订符合实际的计划,并施加压力,迫使他下个星期尽力去完成。每个星期都重复上述过程,直到他打破瓶颈并开始写作为止。如果这些方法都无法奏效,可以考虑用心理学上历史悠久的激励行为的办法——电击法来解决问题。

{\kaishu 4.要素四:针对教师和学生分别组建“失写互助组”}

北卡罗来纳大学“失写互助组”只接受教师加入,不邀请研究生参加。这听上去有点不公平,不过我们有充分的理由支持教师和学生分开。教师和学生有着不同的写作优先项(参见第三章),他们也面临不同的困难和挑战。研究生在面对一群老师的时候往往会觉得受到了威吓,并无端地认为他们的写作目标(例如完成硕士论文)比教授们的目标低级。当只有教授出席的时候,教授们才能够自然地讨论在指导学生写作过程中遇到的难题以及因为学生参与而导致延期的任务。当只有学生的时候,学生们才能够比较直率地讨论某门课程作业的困难和有导师参与的课题的写作问题。

如果你是一名研究生,你也许有很多志同道合的朋友。你们面对的是同样的写作挑战——在同一时间段内必须完成的报告与论文,所以“失写互助组”是一种非常好的互相帮助的形式。组建一个只有学生参与的写作小组,不要让你们的导师知道——因为他/她可能也想加入。

{\kaishu 5.要素五(可选):喝咖啡}

由于北卡罗来纳大学“失写互助组”的成员都喝咖啡成瘾,所以我们选择在系办公楼旁边的咖啡馆碰头。虽说咖啡是这个小组活动必不可少的要素,但也许对其他小组而言并不那么重要。茶——或者甚至是白开水——都能发挥同样的效力。

\section{小结}
看到这里,心理学家已经清楚地理解为什么“失写互助组”能够帮助人们写得更多了。社会心理学家意识到,这个互助组是社会压力的有力来源。突击写作者会感受到来自有计划的写作者的压力,这会迫使他们制订计划并好好执行。行为心理学家意识到,这个互助组在对想要培养的行为给予强化和对不当的行为给予惩罚。临床心理学家认识到,这个互助组能够为那些无力改变他们低效方式的人提供深入的分析与建议。认知心理学家指出,分析成功与失败能够帮助人们评价他们的行为策略。发展心理学家发现,他们能够暂时从实验室孩子们的尖叫中逃出来,在咖啡馆里获得片刻的安宁。和你系里的朋友一起组建一个“失写互助组”吧——这将使写作充满乐趣。







