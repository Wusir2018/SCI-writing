\chapter{撰写专著}
著名的心理学家都是因为他们的专著而名垂青史的。没有人阅读戈登·奥尔波特(Gordon Allport)和克拉克·赫尔(Clark Hull)的论文,但是人们都在阅读《人格的模式与成长》(Allport, 1961)和《行为的原理》(Hull, 1943)。这一章是关于如何写书的。如果你想写一本书,你几乎找不到什么有用的建议。心理学界对期刊论文的痴迷使得市面上有很多书、章节和文章介绍如何发表论文(例如:Sternberg, 2000),鼓励专著写作的资料却很少。因此,本章将比其他章节更个人化,它介绍了我从自己写作经历中得出的个人经验(Duval \&Silvia, 2001; Silvia, 2006),同时也介绍了我从前辈那里学到的经验。

你或许想直接跳过本章。你会想:“我绝不会写一本书的,写一篇像样的论文已经够难了。”也许吧。写书和写其他任何东西一样:你坐下来,然后开始打字。写书会比写文章花更长的时间,但是你只需坚持执行你的写作计划,就一样能完成。在写一份课题申请报告的时候,雪莱·杜瓦尔(T. Shelley Duval, 他是《客观自我意识理论》的作者)说:“天啊,在这份报告上花的时间,我都可以再写一本书了。”(Duval \& Wicklund, 1972)(说得没错——我花在写这本书上的时间要比我花在写最近的一份课题申请报告的时间短。)要知道,写一本书比写一篇论文的智力回报可要大得多。专著比期刊论文、合集或是合集里的某个章节都要重要,专著可以用来讨论一些大的问题并得出具有争议性的结论。

\section{为什么写作专著}
与优秀的专著作者会面成了鼓励我写作专著的动力——我觉得这挺有意思的。在就读本科期间,我与雪莱·杜瓦尔一起工作过。我还清晰地记得拜读了他的作品后再与他面对面地交谈书中的内容让人感觉很奇怪。我在堪萨斯大学读研究生期间遇到过很多学者,他们出版的专著水平颇高(例如:Batson, Schoenrade \& Ventis, 1993; Brehm, 1966)。仅拉里·赖茨曼(Larry Wrightsman)一人就写了差不多20本(例如:Wrightsman, 1999; Wrightsman \& Fulcro, 2004)。弗里茨·海德(Fritz Heider, 1958)的《人际关系心理学》至今仍然让这所大学的心理学系增光不少。

学者写书的目的千差万别。有些作者告诉我,他们只是对某个话题充满好奇。写作是研究某个复杂问题的好方法,它能够深化你对问题的理解(Zinsser, 1988)。你的书写完以后,你会生出很多很有价值的研究与思考。也有人告诉我,一本专著是若干期刊论文的升华。当人们想要对某个研究系列作个总结的时候,他们会写一本专著,同时激励其他学者继续探索剩余的问题。对有些作者而言,写本专著是他们表达正在从事的复杂研究的唯一方式。在心理学发展的进程中,有很多作者写了很多专著,因为他们的每一项研究课题都足以写成一本书。也有些人只是简单地认为写书很有趣。

或许你考虑编写一本教材。教学是心理学科学使命的核心——一本好的教材能够把晦涩难懂的学术语言翻译成日常生活中的白话。心理学界永远需要新的高质量的教材。很多作者被写作教材带来的权威感吸引。一部分教材让作者发家致富了,但是绝大多数都差强人意。很多教材表现平平:书是出版了,但是鲜有教师选用,于是出版商拒绝再版。即使是出色的教科书——那些具有综合性、挑战性和前瞻性的教材——也往往难逃此尴尬下场。因为人们甚至从未看见过或听说过这些教科书,于是大大低估了此类书籍的数量。一本书如果没有二版的机会,不再印刷,那么市面上很快就找不到了。如果你不是被钱吸引,那么你需要找到一个很好的理由来投身教材写作,比如说对坐在椅子上打字情有独钟。



\section{怎样通过两个简单的步骤加上一个艰难的步骤来完成你的专著}
{\kaishu 1.第一步:寻找合著者}

写书就好像重新粉刷浴室——有个伴会更有趣。如果这是你的处女作,那么考虑寻找一位合著者。也许有些朋友和你有同样的研究兴趣。为什么不问问他们是否有兴趣加入?有个合著者有着显而易见的好处。两个作者写起来速度更快;这样你就可以用你的写作时间来完成其他的任务,比如论文或是课题报告。还有,具有不同研究兴趣的合著者能够大大提高你的写作水平,从而成就一本更为丰富的专著。同时,有个合著者还有一些隐性的好处。专著的作者们往往会面临一些艰难的抉择:书的结构、组织和章节衔接。如果你是唯一的作者,就没有人能够帮助你做决定。你的合著者将是唯一一位了解所有问题背景的“其他人”。如果你无法找到一位合著者或者你计划写的书最好是独立完成,那么你可以找一位“导师”。你是否有一位朋友或者同事能够对你的奇思妙想给些中肯意见呢?

选一个能写的合著者。显然,如果一位高产的作者和一位低产的作者决定共同写一本书,那将是一场怎样的灾难啊。别被热情冲昏头脑。你心仪的合著者以前写过书吗?他/她发表过论文吗?你认为他/她是一位高产的作者吗?别把你的书和你们的友情都给毁了。高产的作者在一旁抱怨:“他是怎么了?为什么他就不能坐下来写点东西呢?”低产的作者也在抱怨:“她有病吧?她整天都追在我后面催。”这样的组合有时候也能完成一本书,前提是双方都充分理解劳动分工。高产者可以负责写作,而另一位则负责列提纲,针对某些章节的草稿提出批判性的意见以及修改书的某些部分。如果低产的那位还有些特长,那他/她还算是位非常不错的“不执笔”的合著者。

{\kaishu 2.第二步:做好计划}

很多写作者,甚至是很有天分的写作者,也很奇怪地就是不愿意列提纲。事先声明:没有计划的话,要写成一本书是不可能的。书太庞大了。写书的第一步——可能要花上好几个月完成——就是列出一份清楚的章节目录。要列出一份清楚的章节目录,首先需要通过头脑风暴来搞清楚你想写的书是关于什么的。当你进行头脑风暴的时候,你会看到你的想法的层级结构——此时书的章节脉络就逐渐显现。有的人喜欢写很多短小的章节,另一些人写的章节则比较长,总数比较少。仅作参考,一本典型的专著一般包含8$\sim$14个章节,而一本教科书则通常有12$\sim$20个章节。

你的章节目录在写作的过程中会不断发展。当你逐步进入状态后,你可能会有新的想法,你也会重新考虑过去的一些想法。你或许会新增一个章节,把两个比预想的要短的章节合并,或是把一个章节一分为二。这都没问题,但是千万不要还没有做好章节目录就动手写作。我在动手写这本书以前,差不多花了两个月的时间仔细揣摩这本书的章节目录。

与章节目录相配套的是,为每个章节写一个大纲。你应该能够通过几段文字来大致描述这一章节要写些什么。你有两个理由必须列章节大纲。首先,写一本书并不容易,只有傻子才会在还没有想好每章写些什么的时候就动手写书。你可能并不需要知道每一章你具体需要写些什么,但是你必须明白每一个章节的目标是什么,以及这一章节对实现整本书的写作目标有哪些作用。其次,为了获得一份出版合约,你必须向出版商详细地介绍每个章节。那些审阅你的写作计划书的人会非常仔细地检查你的章节目录,以判断你花了多少心思来考虑你的这本专著。

就好像和你一起粉刷房子的搭档能够帮你清洗工具一样,你的写作搭档也能够帮助你列提纲。列提纲的阶段——第一步就包含货真价实的写作——你和你的合著者可能会对写些什么有不同的意见。这都无所谓——这些分歧说明了写作过程中与合著者之间固有存在的某种权衡关系。当独立写作时,你不需要努力妥协,但是在某些艰难的时刻,你必须和你自己的心魔作斗争。如果你有一个合著者,你们会在书的内容、组织和重点等许多方面产生分歧。妥协退让或许会让你很不爽,但是一位好的合著者能把你从牛角尖里拽出来。两个头脑比一个要好得多。

{\kaishu 3.第三步:动手写}

读到这里,即使是最迟钝的读者也看明白了,这本书其实就说了一个简单的道理:如果你想写得更多,你就必须做个时间表,然后坚决执行。写书的时候也是如此。别等待夏天到来,别等待长假。虽然死不悔改的突击写作者也能够在休假的时候写成几个章节,但他们用12个月也写不完一本书。当突击写作者忙于种种日常事务,例如教学、研究和杂务的时候,写书的计划就渐行渐远了。我自己也曾经迟钝过,我用某种悲壮的方式明白了这个简单的道理。我在汉堡大学做博士后的时候开始写《兴趣心理学探索》。那时,我有一个安静的办公室,有美味的德式咖啡,事务清闲,我用6个月的时间突击撰写了这本书的大部分内容。我没有按照计划写作,当我开始终身教职工作的时候,这本书就被耽搁下来了。

我们每个人都曾受到类似的诱惑,就是专挑有意思的章节写,而完全忽略那些生涩的部分。按照这种思路写作的作者可能写了几百页,却还没有完成一个完整的章节。当你写完那些容易的章节时,你就泄气了;所以一定要按照章节的顺序来写作。很多作者建议从第二章开始,循序渐进,最后写第一章和序言。这个建议不错,因为有时候写着写着可能会写偏。很多作者都说他们从来没有一本书是和原计划一模一样的:最终的版本更好,但是都和原来预计的差别很大。 你没办法介绍一本你还没有读过的书,因此在宣称你将要写些什么前,先得看看你实际上写了些什么。

写一本包含大量的阅读、研究和文献内容的书,我得到的最好的建议之一就是按照书的章节而不是按照话题来整理所有资料,作者们都很容易按照书的章节来思考问题——“这篇文章适合放在第四章”,他们这样说道,“我会在第八章的结尾引用这段话”。如果你对自己的专著的认知是以章节为序的,那么你也应该按照章节来整理你的资料。那些再版多次的书的作者也提到这种方式对下一版的资料收集整理颇有益处。

和写文章一样,你也必须监控自己写书的进程。在执行一个时间跨度颇长的写作计划时,人们往往容易迷失方向。 写书的时候,我常常用一张图表来跟踪我写了多少。表7.1展示了我写关于兴趣的那本书时的进程(Silvia, 2006)。这张图表中有一列标明了章节的序号,一列标明了章节的名称。针对每个章节,我都记录了写作的页数和字数。(大多数作者都以页数来衡量文章的长短,而编辑和出版商则用字数来衡量。)图表中的公式会自动计算页数和字数的总和。这张图表还标明了初稿和修改稿是否完成。你也可以根据实际情况增加更多的行和列。比如你的书有两位作者,你可以设一列记录谁负责写这一章。如果每个章节有截止期限,例如很多教科书的章节就有截止日期,那么你也可以用一列专门标注这一日期。

\begin{table}[htbp]
  \centering
  \caption{写作进度表}
  \begin{tabular}{p{0.07\textwidth}p{0.15\textwidth}|p{0.07\textwidth}p{0.15\textwidth}|p{0.07\textwidth}p{0.15\textwidth}}
    \hline
    章 & 页码 & 字数 & 初稿 & 修改稿 & 章标题 \\
    \hline
    1 & 10 & 2770 & Done & Done & Introduction \\
    2 & 23 & 5830 & Done & Done & Interest as an Emotion \\
    3 & 41 & 10952 & Done & Done & What Is Interesting? \\
    4 & 24 & 6596 & Done & Done & Interest and Learning \\
    5 & 32 & 8583 & Done & Done & Interest, Personality, and Individual Differences \\
    7 & 29 & 7838 & Done & Done & How Do Interests Develop? Bridging Emotion and Personality \\
    8 & 33 & 8892 & Done & Done & Interests and Vocations \\
    9 & 21 & 5609 & Done & Done & Comparing Models of Interest \\
    10 & 11 & 3003 & Done & Done & Conclusion: Looking Back, Looking Ahead \\
    References & 63 & 14269 & Done & Done & References \\
    总数 & 310 & 80643 &  &  &  \\
    \hline
  \end{tabular}
\end{table}

\begin{remark}
这是我写一本关于兴趣的书时使用的进度图表(Silvia, 2006)。
\end{remark}


\section{怎样找到一位出版商}
如果你去阅读本书最后列出的“那些关于写作的好书”,你会发现很多作者都描述了要找到出版商有多困难。比如《写作者的希望之书》(The Writer's Book of Hope)的作者拉夫·凯斯(Ralph Keyes, 2003)告诉我们,这本畅销书曾受到许多出版商的冷遇。心理学家或许是幸运的,因为学术著作出版和商业出版之间有质的区别。在真实的世界——那个你读研究生之前生活着的世界里——无数的写作者争相吸引出版商的注意力,而对出版商而言,每一本书都是一场赌博。在学术世界里,写书的人不那么多了。正因为相对较少的人写书,学术著作出版商希望能够与作者建立深厚的感情。学术出版物也比商业出版物风险小。学术著作有固定的市场群体——大学的图书馆、大学课程——和经得起时间考验的吸引其特定读者群的方式。有些学术著作出版机构是非营利的组织。如果你正在写一本好书,出版商会希望与你进行进一步沟通。

写完几章之后,你就应该和图书编辑取得初次联系。外星人钟爱的初次接触的方式——把人们从床上绑架走,再用各种探头挠他们的痒——可能不适用于你的处女作。你应该在会议上与编辑们交谈。在与会嘉宾中你很容易辨别出谁是图书编辑:他们都比教授或研究生穿着得体,他们一般都会站在一张堆满图书的大桌子旁。你也许会说:“我觉得他们在那里是卖书的。”当然,宣传并销售他们的书是编辑们参加会议的两大理由。除此以外,他们还会与潜在的作者建立联系,或跟踪那些正在写作的作者的最新进展。他们希望人们走上前去,和他们讨论新书的想法。你只需要走到一张桌子前,问问是否能和他们讨论一下你正在写的一本书。你会发现他们对你的著作的兴趣对你来说真是一股清新之风,因为那些互助组的同事或许早已对你的想法感到厌
烦了。

我采访的写作者们对应该在写了多少字以后尝试联系出版商这个问题意见不一。有些人很早就开始寻求签订出版合同;有些人则是写完了整本书才开始寻求出版。在做这个决定以前好好考虑一下。在你签署一份合同以前,你只是和自己就写一本书达成了约定。与自己违约是件不光彩的事,但是你既不会有财务损失,也不会惹怒他人——只是你和你的羞耻感间的问题。但是如果你跟出版社签了约,那么你的书就“正式而世俗”地存在了。如果你违约,就会显得很不专业,编辑会不高兴;如果你拿了预付款,你还会欠编辑一笔钱。所以,在你确信你与自己的约定不会被违背以前,不要随便签署一份出版合同。杜瓦尔(Duval)和我在还没有动笔写以前就接到了合同;我写了两章以后就开始为我的关于兴趣的书寻找归宿;而你们正在阅读的这本“巨著”,我是等写完了全部初稿后才联系美国心理学会的。

如果对你的书感兴趣,编辑会让你准备一份书稿写作意向书。你可以在任何出版社的网站上找到意向书写作指南。常规的意向书要求作者描述书籍的写作目的、目标读者和主要竞争对手。你需要提供一个含有具体内容的表格,详细描述各个章节,同时也需要提供一些章节的成稿来表示你的诚意。你或许会被要求提供几位审稿人来审阅你的意向书,而出版商也希望更多地了解你。出版商都清楚,出书这件事,说比做难。如果你以前没有出版过专著,编辑会坚持要求你提供几个章节的样稿。

与期刊论文不同,书稿写作意向书可以寄给几个不同的出版商。为了节约大家的时间,请不要把意向书寄给你不希望它出版你的作品的出版商。有很多具有良好声誉的出版商出版心理学著作。在考虑出版商的时候,挑选那些长期以来一直活跃在你研究领域的出版商。有的出版商可能会有一个关于某个领域的系列图书出版计划。出版商也可能会把你的意向书发给其他同行审阅,通常这些人都是他们以前的客户。有时候出版商会把他们的反馈寄给你;有时候他们自己留着。不管怎样,如果你的书看起来不错,他们就会和你谈谈合同的事。

一本书的出版合同可是件大事——它和你在光碟出租店签的合同是两码事,你必须仔细阅读。以下是一份图书合同的几大核心内容:

(1)合同会约定一个交稿日——这个日子就是你让书稿离开你那脏兮兮、沾了咖啡渍的手套,交给出版商的日子。有时候出版商会设定一系列的交稿时间。比如某些教科书,常常是在某一个月之前交某几章。好好考虑交稿时间——设定为合同签订之日起两年是比较常见的。如果你有监控写作的习惯,那么你应该知道每天你大概写多少字以及你投入写作的频率。实践出真知——根据你的统计数据来确定交稿日。

(2)作者和出版商都会十分在意版税问题。常见的做法是分别针对平装版和精装版商定一个费率,并随着销售数量的增加而增加。合同通常也会约定上述比率的例外情况,例如作者从外文翻译版中所获的版税。有一部分图书——例如余量和赠阅的图书,作者和出版商都不能从中获益。

(3)出版商往往会拿预付款来诱惑作者,此时作者们总是失去理智地上钩。请记住:这笔钱是从你将来应得的版税里预支的,并不是一个“签约奖金”。如果你不需要预付款,你可以拒绝它。如果你希望早点获得版税的一部分,那么你可以就预付款问题与编辑进行交流。如果你打算请人帮你校对清样或是帮你画一些图表,预付款能够派上用场。

(4)合同会列明谁负责申请授权(允许影印来自他人的资料)、画图表和完成索引。通常作者会负责完成上述工作,不过教科书的出版商常常宁愿自己来完成。

(5)合同会约定将来如何处理图书的修订。通常出版商有权提出修订要求。如果你不打算修改这本书,那么合同里会说明出版商有权委托另一作者进行修改。这一条款并没有听起来那么糟糕。如果你不幸去世或者退休,出版商可以继续出售和推广你的著作。合同还会框定将来修订版版税的相应变化——增加或减少。

(6)合同会说明谁拥有此书的版权。对于学术著作,出版商一般拥有该书的版权。出版商还必须说明如果书籍脱销了怎么办。合同通常会约定,作者有权在著作脱销6个月以上的情况下要求出版商再次印刷。如果出版商拒绝,那么作者有权收回该书的出版权。你一定要争取在著作脱销的情况下获得你应有的权利,这样你能够有机会对著作进行修改,然后另外找一家出版商重新出版。

(7)出版商有可能会要求增加一条“优先拒绝权”条款,意思是即便该出版商决定不出版你的下一本书,他们也有权优先获得出版意向书。



\section{处理—些细节问题}
书写完以后,你必须尽快从写作的快感中脱身,投入能给你带来更大喜悦的工作中——准备将书稿投递给出版商。你的编辑会给你寄来一份指南,告诉你应该做哪些准备。在这一阶段,你需要搜罗所有授权,绘制高质量的电子图表,处理在写作过程中一直懒于处理的行文和参考文献中的空格问题。出版商会给你寄一份详细的作者调查表,询问一些关于你和你的作品的问题。这份资料将来会用于图书归类、营销和推广活动,所以你需要花点精力来好好考虑。也许他们还会让你推荐几位艺术家或学者来设计推荐广告。你还要启动你的激光打印机,因为大多数出版商通常除了电子版本以外,还希望收到几份纸质的稿件。

现在你的专著进入了待出版环节,你需要准备好迎接大量的编辑和校对工作——你想象中的成堆的喜悦化成了成堆的工作。大多数的书籍出版周期都很紧张,所以你不能拖后腿。还记得你的优势吗?你可以付一笔钱让别人帮你校对清样。你自己当然也要校对,但是你需要另一位读者用全新的视角来阅读。一位从事编辑工作的好朋友帮我阅读了第一本书的清样;一位在我供职的大学写作中心工作的研究生阅读了我第二本书的清样。如果你需要准备索引,你应该在收到清样的时候弄好它。单调的索引制作过程考验着你的决心,但是同时它也铸就了作者的品质。



\section{小结}
写书是一种干净的家庭娱乐,只是既没有家庭也没有乐趣(如果你像我一样总是把咖啡泼出来,那就连干净二字也无从谈起了)。写书没有什么神秘之处,只是机械地执行你的写作计划。人们出于各种目的而读书:想学习一个全新的领域,想就某一项研究开阔思路,或是想表达你想要说的。如果你有话想说,请写一本书。如果你有很多话想说,请写两本书。当你开始写书的时候,给我写一封电子邮件,告诉我进展如何。我想知道我写的这些小窍门对你有没有用,也想知道你对那些有抱负的专著作者有何建议。