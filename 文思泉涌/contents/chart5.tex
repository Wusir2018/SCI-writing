\chapter{简洁、单刀直入的文风}
学术期刊散发着垃圾写作的酸腐味——我总是把期刊放在离书桌最远的书架上,以免被熏到。但是如果你有机会与这些文章的作者面对面交流,你会发现他们对自己的研究津津乐道,他们的口头表达是那样清晰、生动和有趣。哪里出了问题?尽管这本书的主题是写得更多,而不是写得更好,但你还是应该花点时间来掌握一些要领,让你的作品更有力度。如果人们坚持执行写作计划,大概一个星期以后他们就能写得更多,要写得更好则需要更长的时间来学习——所以我们应该从现在就开始。本章将介绍几个能够帮助你提高写作水平的小技巧。

\section{诊断问题}
有三个原因会导致写作水平偏低。首先,学者们喜欢显摆聪明。有一则德国格言说:“水面颜色深的湖必是深湖”所以,学者们常常弃好词而不用。比如,他们不用“聪明的”(smart),却用“老练的”(sophisticated)或“渊博的”(erudite)。我应该这样说,“水体的透明度的最小值往往与水的深度之间有着极高的相关性(p<0.05)”。其次,学者们从未学过好好写作。他们在研究生院的导师可能不是一个好榜样,而他们引以为榜样的期刊文章也散发着酸腐的垃圾味。最后,大多数学者都没有花足够的时间来成为一名优秀的写作者。和其他所有的技能一样,写作技能也需要长时间的勤学苦练(Ericsson, Krampe \& Tesch-Romer, 1993)。人们必须学习好的写作规则,然后花大量的时间来练习。

为了解决第一个问题,你必须修正对学术写作的态度。也许有些读者会因为你写的东西晦涩难懂而觉得你很聪明,但是你并不需要这样毫无眼光的读者。大多数的科学家都会被好的观点和有趣的发现吸引,所以不要把你的意见藏在一大堆垃圾英语背后。为了解决第二个问题,仔细地阅读这一章,然后买一些其他关于写作的书来读。本书“推荐阅读:那些关于写作的好书”部分附有关于文风和语法的参考书目,或许对你有所帮助。为了解决第三个问题,你需要阅读上述书籍,并在写作时多加练习。要不了多久,你就会逐渐形成独具一格的文字风格,而不再是那个让人不知所云的无名氏了。

\section{选择“好词”}
写作的核心在于用好词。为了写得好,你必须选择“好词(good words)”。英语中包含很多词,其中有很多简短、有表现力而且常见的词——要选用这些词。避免用那些听起来很深奥的时髦词组,更不要用那些学术腔太重的词。除了帮你提高写作水平,选择“好词”也是对读者的一种尊重,因为对他们来说,英语可能是他们的第二、第三或者第四语言。外国学者常常在阅读文章的时候备有双语词典,如果你选的词不在词典里,他们就理解不了。他们会怪自己误解了你的意思,其实你才是应该被责备的人,因为你没有顾及他们的实际情况。

“那专业词汇呢?”你也许会问,“我怎么可以不用‘刺激呈现的异步性(stimulus onset asynchrony)’来写一篇关于刺激呈现的异步性的文章?”科学在需要的时候会创造出新的词——这些专业词汇是有用的。如果用常用的词来解释,这些专业词汇是很容易被理解的。我们应该保留科学词汇,剔除那些从商学、市场学、政治学和军事学那里借用来的词汇。我们不需要诸如“物质激励(incentivize)”或“瞄准(target)”之类的动词,也只有玻璃清洗工才需要“透明(transparent)”之类的形容词。为了连贯,请重复使用同样的专业词汇。不断变换的心理学专业概念会把你的读者搞晕:

{\kaishu 前:神经过敏症指标高的人比体验厌恶情感状态倾向指标低的人反应慢。

Before: People high in neuroticism responded slower than people low in the tendency to experience aversive affective states.

后:神经过敏症指标高的人比神经过敏症指标低的人反应慢。

After: People high in neuroticism responded slower than people low in neuroticism.}

而且有些专业词汇真的是糟糕透顶,所以不要不假思索地引用你在专业期刊上看到的词汇。发展心理学家对“道(path)”和“路(way)”都不满意,他们喜欢用“路径(pathway)”;在某些吹嘘的场合,“路径”变成了“轨道(trajectories)”。认知心理学家喜欢“澄清(clarify)”什么是“消除歧义(disambiguate)”。临床心理学家喜欢说病人“呈现(present)”出某种症状,感觉对方好像是一个情绪低落的管家,托着一个盛满“负面情绪”或“低质量睡眠”的大盘子。临床医生也不再用“医嘱(manual)”和“遵医嘱(follow manual)”,他们发明了“医疗干预(manualized interventions)”一词。情感心理学家生怕他们的读者不理解“评价(appraisal)”这个词的意义,他们喜欢说“认知评价(cognitive appraisal)”“主体评价(subjective appraisal)”有些人还是怕别人不理解,索性说“主体认知评价(subjective cognitive appraisal)”。做跨学科研究的心理学家提出了“生物——社会”模型(biosocial models)、“社会——心理”模型(psychosocial models),最近又有了“生物——心理——社会精神”模型(biopsychosocialspiritual models),用以取代原有的狭隘的“生物——心理——社会”模型(biopsychosocial models)。

心理学家特别青眯“坏词(bad words)”,不过他们不说“坏(bad)”,而是代之以“deficient(有缺点的)”或是“suboptimal(非最理想的)”。心理学家喜欢用“现有文献(existing literature)”, 难道我还需要阅读和引用“不存在的文献(nonexistent literature)”吗?任何在读文献的心理学家都应该知道我们这些专业期刊就是这样“可怕而真实”。“现存文献(extant literature)”是同一宗罪的升级版。那些写某两样东西之间“失联(disconnect)”的心理学家恐怕是与他们的字典“失联”了,因为他们本可以在字典里找到无数更好的词,如“差异(difference)”“区别(distinction)”“分离(seperation)”和“差距(gap)”。有些人在写某一关于某一问题的论文时,把“人”(a person)称为“个体”(individual)而把“人们”(people)称为“个体们”(individuals)。这些人忘记了“个体”是一个模糊的概念:试想,“我们观察一个个体\_\_\_\_ 。”画横线部分应该补充一个名词(例如:兔子)还是一个动词(例如:走路)?你在和朋友说话的时候从不说“个体”或“个体们”,那为什么在和广大的科学界同仁交流的时候要装腔作势地这么说呢?选个对的词,比如人(person)或人们(people)就可以了。而“persons”则更为可恨,这样的词大概只有小镇警长在寻找失踪人口时才会用到吧。

说到“人们(people)”,我在描述我的研究参与者时已经不再用“参与者(participants)”一词了。我的朋友们有研究鸟类、婴儿、老鼠或是学区的,他们的“参与者”和我的完全不同。我研究的是成年人,所以“人(person)”或“人们(people)”是对我的研究对象的准确界定。如果这个决定让你不安,别害怕——美国心理学会(APA)不像时尚界,这里没有“警察”。“参与者”是一个模糊的词,心理学家应该选择更确切的词。比方说,有些研究人员是从儿童、教师和家长那里收集资料来研究儿童心理学的,他们的参与者就应该是上述三个类别的人。不要觉得有什么不妥,就应该称他们为儿童、教师和家长。如果你研究老年人或是年轻人的认知过程,为什么不在你的研究方法和结论部分称他们为老年人或是年轻人?现在就把你的研究方法部分重新写一遍,把所有的“参与者”都改成更为确切的词,这样你会觉得舒服很多。

一些缩略词和首字母简称也是不妥的。我见过有些人把很常见的词,比如“焦虑”(anxiety)或“沮丧”(depression)缩写为ANX和DEP,把简单的词组,比如“引发焦虑”(anxious arousal)和“缺乏快乐感而沮丧”(anhedonic depression)简写成ANXAR和ANDEP, 然后兴奋地讨论ANX、ANDEP、DEP和ANXAR的区别。缩略词和首字母简称只在比它们代表的原词(或词组)更易于理解的情况下才有用。SES(社会经济地位)和ANOVA(方差分析)是好的缩写,ANX和DEP则不是。有些作者认为他们用简称代替常用词能够避免行文冗长。比如写一本关于如何写得更多的书,他们会想着把“写得更多”缩写为WAL。其实读者会觉得读这样的缩写比读原词更乏味。为了避免拐弯抹角,你应该少用缩略词。

删去诸如“非常(very)”“相当(quite)”“基本上(basically)”“实际上(actually)”“事实上(virtually)”“极端地(extremely)”“明显地(remarkably)”“完全地(completely)”“根本(at all)”之类的词。基本上,这些相当无用的词事实上毫无意义,它们事实上就像水草,实际上会完全地阻塞你的句子。在《垃圾英语》一书中,肯·史密斯(Ken Smith, 2001: 98)把这些词称为“寄生强化剂”(parasitic intensifiers)。

{\kaishu 原本含义确切的词现在好像需要附加这些强化剂才能恢复原有的能量。现如今,表达观点或是反对的立场时非要加上“有价值的”或“坚决的”之类的词,否则就会显得不温不火。这些强化剂削弱了原意。}

如果你真心领会了斯特伦克和怀特(Strunk \& White, 2000: 23)关于“省略无用的词”的建议,但是又不太确定哪些是无用的词,那么上述这些“寄生强化剂”基本上就是应该完全删除的。


\section{写出有力度的句子}
现在你对遣词有所体会了——“我上篇文章有没有用‘个体’?”——那么是时候想想如何造句了。“每当你造句的时候,”谢里丹·贝克(Sheridan Baker, 1969: 27)这样写道,“应该就像呼吸一样自然,或许也同样缺少变化”。不断重复类似句型的作者,就好像在不断地使用复读机。英语有三种不同类型的句式:简单句、复合句和复杂句(Baker, 1969; Hale, 1999)。简单句只有一个主——谓结构。学者们对清晰的简单句不屑一顾。这实在可惜。复合句有两个从句,每个句子能够独立存在。有时候用关联词来连接非独立从句,有时候分号能起到同样的作用。与简单句和复合句不同,复杂句包括独立从句和非独立从句。复杂句如果用得好,能够使你的文章干净利落、张弛有度。

在自我感觉良好的时候,我相信并列句是为心理学家发明的。我们总是在写关系、对照和比较:外向型指标高的人和外向型指标低的人、控制组和实验组、时间段一发生的事情和时间段二发生的事情。好的写作者合理地运用并列句来描述关系;拙劣的写作者则尽量不使用并列句,因为他们觉得并列句很啰嗦,相反,他们通过变换词语和句式发明了很多别扭的句子。

{\kaishu 前:双任务组的人在一组“嘟嘟”声的引领下阅读一列词汇清单;另一个组的另外一些参与者仅阅读词汇而不会听到任何声音(“控制组”)。

后:双任务组的人在一组“嘟嘟”声的引领下阅读一列词汇清单;控制组的人仅阅读词汇清单。}

有些并列句式用“标准——变量”的格式:先描述共性,而后解释差异。

{\kaishu 更好:每个人都阅读一列词汇清单,双任务组的人在一组“嘟嘟”声的引领下阅读,而控制组的人仅阅读。

更好:每个人看一组20幅画。控制组的人只是看一看这些画,评估组的人则要标出对每幅画的喜爱程度。}

很多人根本不知道如何用分号——这个符号是并列句的最佳搭档,却常被忽略。就像人们不喜欢球迷或俱乐部的年册一样,很多作者从高中就养成的对分号的不信任绝对是一种偏见。我们要解决这个问题——我们要分号。分号必须连接独立从句;句子的任一部分必须能够独立存在。与句号不同,分号表示两句句子间的紧密关系。与逗号及随之出现的“和”也不同,分号表示两个部分间的平衡。分号是理想的连接两个并列句的工具:

{\kaishu 前:在时间段1, 人们阅读词汇。在时间段2,他们尝试记住尽可能多的词。

后:在时间段1, 人们阅读词汇;在时间段2,他们尝试记住尽可能多的词。

前:阅读组的人阅读词汇,听力组的人听这些词的朗读录音。

后:阅读组的人阅读词汇;听力组的人听这些词的朗读录音。}

除了学会使用分号之外,我们还需要认识一位新朋友——破折号。好的写作者对破折号很上瘾。破折号——严格意义上,它们应被称为“长破折号”——让句子显得干脆。破折号主要有两大用途(Gordon, 2003)。首先,单用一个破折号能够关联一个从句或词组直到句子的最后部分。你在本章已经读到很多这样的例子:

{\kaishu 1.学术期刊散发着垃圾写作的酸腐味——我总是把期刊放在离书桌最远的书架上,以免被熏到。

2.我们要解决这个问题——我们需要分号。

3.除了学会使用分号外,我们还需要认识一个新朋友——破折号。}

其次,用两个破折号能够连接插入语。比如你前面读到的:

{\kaishu 1.现在你对遣词有所体会了——“我上篇文章有没有用‘个体’——那么是时候想想如何造句了。

2.破折号——严格意义上,它们应被称为“长破折号”——让句子显得干脆。}

请你在下次描述参与者和研究设计部分的时候,尝试运用破折号:

{\kaishu 尚可:42个成年人参与了这个实验。其中有12位女士和30位男士。

更好:42个成年人——12位女士和30位男士——参与了这个实验。}

长破折号有一个鲜为人知的兄弟:短破折号。短破折号连接两个概念。这是另一种更简洁的“中间”的表达方式。很少有人能准确地运用短破折号;他们用连字符,且常常带来令人尴尬的后果。发展心理学家对家长—孩子行为感兴趣,但是这并不是说家长们有时候表现得像个孩子——他们用的“家长—孩子”是“家长与孩子间的行为”的缩写。好的写作者知道“教师—家长会议”(短破折号)与“教师-家长会议”(连字符)之间的区别。我们学校的一位研究人员贴了一张有关“婴
儿-家长互动研究”的海报(“先别管少女妈妈了——婴儿妈妈都出来啦!”),现在是时候好好感谢那些默默地为你修改短破折号错误的出版界英雄啦。

你可以尝试用同位语来使句意更为确定。因为句中这些词组表示某种关系,所以你能够省略那些用来连接句子各个部分的连接词。

{\kaishu 前:反事实思维,被定义为关于并未发生过的事情的思维,表现了认知与情感间的交叉。

后:反事实思维,即关于并未发生过的事情的思维,表现了认知与情感间的交叉。

更好:反事实思维——关于并未发生过的事情的思维——表现了认知与情感间的交叉。

前:面部表情研究是认知与情感研究领域的一个热门话题,通过对其研究解决了情感结构上由来已久的矛盾。

后:面部表情研究,认知与情感研究领域的热门话题,解决了情感结构上由来已久的矛盾。}

最后,你能够通过检查学术写作中的两个通病来辨识哪些句子是弱句子。第一个通病,“这样(such that)”病毒。它折磨着那些惧怕简单句的作者。为了避免出现简单句,他们用“这样”来连接一个松松垮垮的主句和他们真正想写的从句。再也不要用“这样”了。用Word自带的“查找”功能把你的文章里所有的“这样”都找出来。如果你找到了,有三种处理办法:把“这样”之前的部分删除,用冒号或破折号替换“这样”,或者重写一句更好的。

{\kaishu 前:我们设立了这样两种情况,即第一种情形下人们被要求尽量准确,而第二种情形下人们被要求尽量快速。

后:第一种情形下人们被要求尽量准确;第二种情形下人们被要求尽量快速。(把句子第一部分删除,并运用分号连接两个对等的句子。)

后:我们设立了两种情况:第一种情形下人们被要求尽量准确,第二种情形下人们被要求尽量快速。(用冒号代替“这样……即”。)

前:人们被分成几个小组,这样的分配过程是随机的。

后:人们被随机分成几个小组。(重写一句更好的。)}

第二个通病,不成立的并列复合句。它折磨着那些错误地认为逗号都应该表示行文中的停顿的人。我们的杂志正在和不断蔓延的不成立的并列复合句做斗争。以下是几个例子:

{\kaishu Positive moods enhance creative problem solving, and broaden thinking.(积极的情绪启发了创造性的问题解决办法,并开阔了思维。)

Experiment 1 demonstrated strong effects of planning on motivation, and clarified competing predictions about how planning works.(实验一表明了计划对动机的重要作用,并澄清了有关计划如何产生效果的猜测。)}

发现问题了吗?知道为什么上述两句是错误的吗?并列复合句要求两个从句都是独立的。在有语病的并列复合句中,第二句不能独立存在,因为它们没有主语 “什么”开阔了思维? “什么”明确了预测?要修改它们很容易。你可以为第二句加上主语(“and they broaden thinking”“and it clarified competing predictions”),或可以在第一句中省略逗号(“Positive moods enhance creative problem solving and broaden thinking.”)


\section{避免被动式、拗口、繁琐的词组}
所有关于写作的书都要求人们用主动语态写作。人们主动地思考,主动地说话,所以主动地写作能够成功捕捉到日常生活中思考和言语的火花。被动语态由于句子的主语被隐藏而使读者感到模糊和不确定。喜欢用被动语态的作者往往希望显示自己很聪明;他们喜欢被动语态的冷冰冰的语气和常常与专业写作联系在一起的特质。被动态写作也很容易纠正。阅读你的文章,把所有的“被(to be)”都圈出来。想想你是否能想到更好的动词?几乎所有的动词都表示“being”,所以你通常能用更生动的动词来替代“to be”。至少把三分之一的“to be”改掉。时刻提高警惕,多加练习,你就会降低被动语态的使用频率了。为了让那些虚弱的句子重新焕发活力,在表达否定含义时使用动词而少用“不(not)”。人们在阅读的时候常常会忽略“not”而误解你的意思。这个把戏能使你的句子简洁,且能更生动地表达你的意思。

{\kaishu 前:人们常常看不到你写的“不”字而误解你的意思。(People often do not see not when reading and thus do not understand your sentence.)

后:人们常常漏看你写的“不”字而误解你的意思。(People often miss not when reading and thus misunderstand your sentence.)}

有些心理学家喜欢用高高在上的被动语气。翻开任何一本期刊,你都会发现心理学家喜欢用“ive”式的词:他们的结论“表现出重要性(are indicative of significance)”, 理论“是历史情景的反映(is reflective of its historical context)”, 数据“对假设是支持的(are supportive of the hypothesis)”。这些都是非常光鲜亮丽
和趾高气扬的被动语气写作:作者选择了奇怪的被动语气,而放弃常用的主动语气。为什么不直接说“结果显示”“理论彰显”“数据支持”呢?把所有“to be + \_\_\_\_ ivc”词组都删去,改成下面的形式:

to be indicative of = to indicate (显示)

to be reflective of = to reflect (反映)

to be supportive of = to support (支持)

to be implicative of = to imply (意味着)

我记得我还读到过“is confirmative of”这样的词组——希望我记错了。

只有提高警惕,才能防止写出来的句子啰嗦冗长。我最近读了一篇文章称,“态度在自然状态下是情绪化的(attitudes are emotional in nature)”。如果态度在自然状态下是情绪化的,那么它在受限制的状态下又是怎
样的?他们会像被圈养的熊猫那样繁殖么?那些写出“在自然状态下”这样的词组的心理学家恐怕是看了太多遍《走出非洲》这部电影了。除非你打算向《国家地理》投稿,否则就不要再用“在自然状态下(in nature)”这样的词了。形容词都是用来描写自然界中的某个事物的,所以所有的形容词都包含“在自然状态下”的意思。接下来,我就不用解释为什么“以\_\_\_\_方式(in a \_\_\_\_ mamer)”也不妥当了。多用副词——“人们快速地做出反应(people responed rapidly)”, 而不是“人们以快速的方式做出反应(people responded in a rapid manner)”——你能因此避免一场灾难。

即使主动的句式也可能是不顺畅或是不生动的。心理学家们常常用这样的短语作为句子开头:“研究表明……(Research shows that...)”“最近的研究发现……(Recent studies indicate that...)”“许多新的发现说明……(Many new findings suggest that...)”或是“大量的研究令人信服地证明……(A monstrous amount of research conclusively proves that...)”这些词组在表辞达意上功用平平,其实文章最后的参考文献就足以说明研究证实了你的观点。你偶尔会需要用到它们——这本书中,我在个人观点与经验所得相左的情况下使用它们——但要尽量少用。

在句首使用一些疙里疙瘩的短语,会削弱句式的含义,例如“然而…(However...)”“比如……(For instance ...)”和“举例来说……(For example)”。请把“然而”移到两个句子的中间:

{\kaishu Before: However, recent findings challenge dual-process theories of persuasion.

After: Recent findings, however, challenge dual-process theories of persuasion.(然而,最近的发现向说服的双重理论提出了挑战。)}

把“举例来说(for example...)”和“比如(for instance)”也挪到合适的位置,不过(在非正式写作中)把“但是(but)”和“然而(yet)”仍然放在句首。请一定牢记,如果“然而(however)”没有正确的标点符号相配合,就会把好好的一个复合句变成断句。

{\kaishu Before: High self-efficacy enhances motivation for challenging tasks; however it reduces motivation if people perceive the task as easy.

After: High self-efficacy enhances motivation for challenging tasks; however, it reduces motivation if people perceive the task as easy.(高水平的自我效能能够加强接受挑战的动机;然而,它也会在人们觉得任务过于简单的时候削弱动机。)}

多写主动句,但是也别刻意回避被动句。和所有的科学写作一样,心理学的文章也会涉及许多客观的媒介,例如概念、理论、构想和关系。我们常常有相对较弱的主体,比如“过去的研究”“认知不一理论”或是“对于焦虑性障碍的认知路径”。如果读者很难锁定一个主语及其行动——一个预测性理论、一个与另一个概念相连的概念、一种影响了现代的研究的传统——主动句就失去了它的效力。一种解决办法是用“研究者”来替代不具人格的主体。以“认知不一理论”为例,我们可以说:“研究认知不一理论的学者们……”此法深受某些认为应该尽量避免将不具人格的主体拟人化来作主语的学者的青睐,但我认为这种看法是一种误导。而且,我怀疑这样做是否有效。诸如“研究者(researcher)”“感兴趣的人(people interested in)”之类的主语不仅语义含糊,而且抽象、指代不明,令人难以捉摸。这一方法还可能会产生误导:有时候我们的确指的是认知不一理论,而非研究这一理论的人。



\section{先写,然后修改}
写文章和改文章是写作的两个不同步骤——不要同时进行。写文章的目的是为了把一堆难懂的、令人诧异的文字堆砌在纸上;而改文章的目的则是清理这些文字,使它们读起来通顺而表达确切。有些作者——还是那些苦苦挣扎的作者——想要写一份没有瑕疵的初稿。这种追求是错误的。这样写作实在太纠结了:这些作者写一句话;思考五分钟;删除;重写;修改一些词;然后抓狂,再写下一句。完美主义让人束手束脚。此外,一句一句地写还会使段落显得缺乏连贯性。写作的基本单位是段落,而不是句子。

熟记这些规则,但不要让这些规则束缚你。边写边改就如同在早晨喝不含咖啡因的咖啡:好心办坏事。初稿应该读起来好像由一个非母语的翻译者匆匆从冰岛语翻译而来。写作,一部分是创作,一部分是批判;一部分是自我,一部分是超我:首先让自我尽情发挥,然后让超我来评判和修正。写就杂乱而令人费解的初稿带来的快感,与大刀阔斧地删减那些陈词滥调的快感,应该说大体相当。


\section{小结}
本章让你对自己的写作形成了一些自我认识。很多人对自己糟糕的写作水平并没有客观的认识——或者借用津泽(Zinsser, 2001: 19)的话:“很少有人意识到他们写得有多糟糕。”有力而清晰的写作风格能够使你的作品从一大堆做作、平庸、粗制滥造和不痛不痒的论文和课题报告中脱颖而出。人们赞赏优秀的作品。评论者知道清晰的作品背后有着清晰的头脑;杂志编辑知道明白的解释背后有明白的思想。读一读本书后列举的那些书单,在你写文章和改文章的时候运用这些好的写作原则,不过,千万别再写“个体”或是“这样……即”这样的文字了。