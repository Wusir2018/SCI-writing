\chapter{导言}
《文思泉涌:如何克服学术写作拖延症》是一本关于如何成为一名深思熟虑且训练有素的写作者的书;它不是教你如何粗制滥造,为了积累成果而发表大量“学术垃圾",也不是教你如何故弄玄虚,把一篇干净利落的论文拉扯得又臭又长。大多数心理学家都希望能够写出更多的作品,他们也希望写作的过程能够不那么令人压力山大、负疚不已或忐忑不安。这本书是为他们而写的。我选择了一个实用的、行为主导的角度来讨论写作。我们不会讨论所谓不安全感、回避心理、防御性和人类内在的心理阻隔等阻碍写作进程的因素;我们也不会讨论培养新的技能,因为你已经掌握了所有能让自己更多产的基本技能,虽然你可能还需要进一步的实践;我们更不会说什么释放你“内在的”之类的话:你尽可用一根拴绳把你的“内在写作者”拴起来,最好再给它戴上一个口罩。

相反,我们会把注意力集中在你“外在”的写作者角色上。高效写作的要求其实很简单:制订时间表,设定目标,时时跟踪你的工作,奖励自己和养成良好的习惯。多产的写作者并没有什么特殊技能,他们只是花了更多的时间在写作上,同时,他们的效率更高而已 (Keyes, 2003)。改变你的写作习惯并不会使写作过程变得更有趣,但是会使它变得容易和轻松。

\section{写作的确是件难事}
你做研究的时候觉得很欢喜。做研究的过程有一种奇异的快感。提出一种观点并设法验证自己的观点令人感到满足。数据收集也是有趣的,尤其是当别人帮你完成的时候。甚至数据分析也挺可爱,看看研究是否站得住脚的确挺让人兴奋的。但是,写研究报告就毫无乐趣可言:写作是辛苦的、复杂的和无趣的。威廉·津泽(William Zinsscr, 2001: 12)说:“如果你觉得写作很难,那是因为写作的确很难。”你必须把复杂的理论、研究方法和数据分析都集中在一篇短小的科研论文中,这并不容易,尤其是当你意识到将来那些不知名的审稿人将对你的作品严加拷问,就像拍打一条落满灰尘的旧地毯。

正是因为收集数据比写作来得容易,许多教授都有堆积如山的研究数据。他们想着“总有一天”会发表这些数据,或者更准确地说,是“总有一年”,因为他们总是在为写作而纠结——教授们热切盼望着长周末、春假、法定假日和暑假。但是,每当长周末过后的周二,人们又总是嘟囔着抱怨自己只写了那么一点点。在规模稍大的系里,每个暑假过后的第一周,到处都可以听到吵吵嚷嚷的叹息和自责声。这种可悲的循环周而复始,人们又开始期待下一次长假。心理学家往往发现,在那些所谓周末、晚间或假期的“空余时段”,写作时间总是被其他更为重要的事情侵占,比如朋友聚会、家庭聚餐、炖锅扁豆汤或是给自家的狗织一顶圣诞帽。

与此同时,我们赶上了好时光,对作品的要求达到了前所未有的高度。越来越多的心理学家向越来越多的杂志寄送越来越多的稿件;越来越多的研究人员在互相争夺日益缩减的研究基金。院长和其他院系领导比以前更看重论文发表的数量。过去和颜悦色的教务长们对教员能够申请到研究基金总是颇感意外,备受鼓舞;而现在,他们拉长着脸,甚至希望新晋员工都能够申请到更多的研究经费。有些院系甚至把教员能否申请到经费与他们的晋升挂钩。在研究型大学,如果无法写出更多的论文,就无法升职或获得终身教职。甚至,在一些小型的教学型高校,对于论文发表的要求也日益提高。所以,这年头,要想在科研领域混口饭吃,真的不容易!


\section{我们现在学习写作的方式}
写作是一项技能,而不是什么天赋或特殊才能。所有的高级技能一样,写作技能必须通过系统的指导和实践来培养。人们必须学习相关的规则和策略,并努力实践 (Ericsson, Krampe \& Tesch-Romer, 1993)。心理学家早已发现,有意识的练习有助于技能培养,但是这一理论似乎还没被用于培养写作技能。我们来比较一下写作技能的教育和其他高级技能的教育。教学很难,所以我们有专门的研究生院教学生如何“教学”。学生们通常要先学习一门“教育心理学”,然后通过当助教来练习如何“教学”。很多研究生在研究生阶段的每个学期都做助教,而后才能成为一名合格的教师。统计和研究方法也很难,所以我们要求学生在高年级阶段不断学习这些内容,通常都由在方法论和统计方面富有经验的专家来讲授。通过多个学期的学习,学生们终于成为老练的研究方法高手。

那么,心理学是怎样训练学生学习写作的呢?最常见的模式是指望学生们能够通过向他们的导师学习来掌握写作的技能。问题是,很多导师自己就在“水深火热”中挣扎——他们常常抱怨根本没有时间写作,常常眼巴巴地等待春假或是暑假的来临 这简直就是盲人骑瞎马。但是,这并不是他们的错:正如很多学生所说,大多数教授也都是在“摸爬滚打”中学习写作的。有些系的确开设了写作课,但是这些课程往往忽视了写作动因方面的问题,转而关注教授如何写课题申请报告或是其他各类报告。

研究生毕业以后,就再也没有导师会对学生们才完成了一半的论文给予指导和鼓励,学生们必须自力更生了。我认为这是令人担忧的,我们并没有给下一代学术写作者充分的教育,却期待他们做得更好。


\section{本书的解决之道}
学术写作可以是一部鸡飞狗跳的闹剧。教授们为写了一半的论文忧心忡忡,抱怨又收到了残酷的退稿信,在经费申请最后期限的前一秒才匆匆忙忙提交了申请报告,幻想着平静的夏日午后可以心无旁骛地奋笔疾书,然后埋怨开学日期的临近严重影响了自己的产能。心理学本身就够戏剧化了,我们真的不需要再添加什么戏剧效果。上述所有都是坏习惯。学术写作本应是循规蹈矩、枯燥而平凡的。为了保障本书以一种平凡的视角来看待写作,本书将不会讨论“写作的灵魂”、各种宗派的“写作灵感”或是“写作的精髓”。只有诗人才喜欢讨论“写作的灵魂”。你应该像个普通人一样写作,而不是像个诗人,甚至不应该像个心理学家。同样,本书也不会探讨任何“防御性”或是“回避心理”,关于这些理论,你大可以到书店的自助角自学了解。 《文思泉涌——如何克服学术写作拖延症》视写作为一系列具体的行为,就像:(1)首先在椅子/板凳/高脚凳/长软椅/马桶/草地上坐下来;(2)然后敲打键盘,写出一段文字。你绝对能够通过简单的办法来培养这些行为。让别人尽情拖延、做白日梦和抱怨去吧,你所要做的就是:坐下来,写。

当你阅读本书时,你要记住,写作不是比赛或者游戏。你想写多少就写多少,长短无所谓。千万不要觉得你有责任写更多,也不要为了发表而发表,写一大堆毫无意义的东西。不要误以为那些发表了大量文章的心理学家就有更多的研究成果。心理学家发表文章的目的多种多样,其中最为重要的是用于学术交流。文章的发表是一项科学活动必要的、自然的终结点。科学家们通过文字互相交流,那些印成铅字的文章构成了心理学的基石,它们阐述了人类是怎样的存在以及人类行为背后的原因。我相信大多数心理学家都在写作这件事情上倍感受挫,他们希望能够写得更多,也希望写作变得容易些。这本书献给他们。


\section{各章预览}
这本薄薄的小册子就如何写得更多给出了实用的、个性化的观点。第二章中,我们彻底检查了人们为写不出东西而找的拙劣借口。我们逐一分析这些借口,发现它们对于写作的效率毫无影响。这章将介绍如何用制订写作计划的方式来分配写作任务。第三章介绍了激励你执行写作计划的各种办法。你将学到如何制订好的目标,通过确立优先性原则来同时处理多项任务,以及如何管理你的写作进程。为了帮助培养你的新习惯,你可以和朋友一起建立写作小组。第四章是关于如何组建既有趣又有益的“失写互助组”(agraphia group)——一种有助于培养良好写作习惯的互助小组。第五章教你怎么写得更好。写得好的论文或是开题报告总能在众多平庸之作中脱颖而出,所以你应该努力写得更好。

第六章、第七章主要介绍了写作的原理。第六章剖析了实用的心理学论文写作技巧。我们可能并不喜欢阅读论文,但是我们必须写作论文。多产的写作者告诉过我他们是怎么写论文的,主流期刊的编辑告诉过我他们希望看到怎样的论文。第六章探讨了关于论文发表的几个入门级问题,例如如何给编辑写投稿信,如何与别人合作写作。第七章讲述了怎么写学术著作。心理学界为有抱负的学者们提供的资源实在有限,基于此,我就如何写学术著作及如何与出版商合作提出了一些个人见解。第八章对全书作了总结,还写了很多鼓励的话。


