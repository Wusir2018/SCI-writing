\documentclass[lang=cn,newtx,14pt,scheme=chinese]{elegantbook}

\title{科学研究与论文写作}

\author{钟澄}
\institute{wusir2018@163.com}

\extrainfo{注意:本书为重排版本,仅供个人学习使用!}

\setcounter{tocdepth}{3}

% \logo{logo-blue.png}
\cover{cover.png}

% 本文档命令
\usepackage{pdfsync}
\usepackage{array}
\usepackage{pfnote}  %每页脚注重新计数
\newcommand{\ccr}[1]{\makecell{{\color{#1}\rule{1cm}{1cm}}}}


% 修改标题页的橙色带
\definecolor{customcolor}{RGB}{32,178,170}
\colorlet{coverlinecolor}{customcolor}
\usepackage{cprotect}

% \addbibresource[location=local]{reference.bib} % 参考文献,不要删除

\begin{document}

\maketitle
\frontmatter


\chapter*{前言}
在过去十余年指导学生的过程中,我深刻地体会到学生在刚进入科研领域时,由于缺乏科学的研究思维,会出现许多共性问题。比如,许多新学生在论文阅读的过程中由于缺乏批判性思维,盲目地信任论文上的结论;在交流选题时,面对新鲜事物,不少新学生会有畏惧的心理状态;在实验的过程中,由于缺乏抓住研究事物本质的能力,新学生通常会在研究的过程中事倍功半;在写作的过程中,新学生通常无法准确地把握论文写作的正确逻辑,同时存在许多表达、格式等方面的问题。因此,为了给新学生及刚接触研究工作的科研工作者提供一些思维方式上的指导,让他们少走弯路,我结合过去十余年的教学科研及培养学生的经历,撰写了这本书。本书行文风格力求通俗易懂,以降低阅读难度。同时,我结合了培养学生过程中所遇到的实际问题来阐述观点,有助于读者更好地明白科学思维的作用与重要性。

本书的写作得到了天津大学的领导、同事、学生、家人和相关部门的支持,同时科学出版社为本书的出版做了大量的工作,在此表示
衷心感谢。

尽管本书的写作目的是为了给广大学生及刚接触研究工作的科研工作者带来便利,然而受限于个人的思想深度及知识水平,不可避免地存在一些纰漏及不足,希望读者不吝批评指正!欢迎来信交流:czhongcn@163.com。

钟 澄

2019年12月10日
\tableofcontents
\mainmatter

\chapter{导言}
《文思泉涌:如何克服学术写作拖延症》是一本关于如何成为一名深思熟虑且训练有素的写作者的书;它不是教你如何粗制滥造,为了积累成果而发表大量“学术垃圾",也不是教你如何故弄玄虚,把一篇干净利落的论文拉扯得又臭又长。大多数心理学家都希望能够写出更多的作品,他们也希望写作的过程能够不那么令人压力山大、负疚不已或忐忑不安。这本书是为他们而写的。我选择了一个实用的、行为主导的角度来讨论写作。我们不会讨论所谓不安全感、回避心理、防御性和人类内在的心理阻隔等阻碍写作进程的因素;我们也不会讨论培养新的技能,因为你已经掌握了所有能让自己更多产的基本技能,虽然你可能还需要进一步的实践;我们更不会说什么释放你“内在的”之类的话:你尽可用一根拴绳把你的“内在写作者”拴起来,最好再给它戴上一个口罩。

相反,我们会把注意力集中在你“外在”的写作者角色上。高效写作的要求其实很简单:制订时间表,设定目标,时时跟踪你的工作,奖励自己和养成良好的习惯。多产的写作者并没有什么特殊技能,他们只是花了更多的时间在写作上,同时,他们的效率更高而已 (Keyes, 2003)。改变你的写作习惯并不会使写作过程变得更有趣,但是会使它变得容易和轻松。

\section{写作的确是件难事}
你做研究的时候觉得很欢喜。做研究的过程有一种奇异的快感。提出一种观点并设法验证自己的观点令人感到满足。数据收集也是有趣的,尤其是当别人帮你完成的时候。甚至数据分析也挺可爱,看看研究是否站得住脚的确挺让人兴奋的。但是,写研究报告就毫无乐趣可言:写作是辛苦的、复杂的和无趣的。威廉·津泽(William Zinsscr, 2001: 12)说:“如果你觉得写作很难,那是因为写作的确很难。”你必须把复杂的理论、研究方法和数据分析都集中在一篇短小的科研论文中,这并不容易,尤其是当你意识到将来那些不知名的审稿人将对你的作品严加拷问,就像拍打一条落满灰尘的旧地毯。

正是因为收集数据比写作来得容易,许多教授都有堆积如山的研究数据。他们想着“总有一天”会发表这些数据,或者更准确地说,是“总有一年”,因为他们总是在为写作而纠结——教授们热切盼望着长周末、春假、法定假日和暑假。但是,每当长周末过后的周二,人们又总是嘟囔着抱怨自己只写了那么一点点。在规模稍大的系里,每个暑假过后的第一周,到处都可以听到吵吵嚷嚷的叹息和自责声。这种可悲的循环周而复始,人们又开始期待下一次长假。心理学家往往发现,在那些所谓周末、晚间或假期的“空余时段”,写作时间总是被其他更为重要的事情侵占,比如朋友聚会、家庭聚餐、炖锅扁豆汤或是给自家的狗织一顶圣诞帽。

与此同时,我们赶上了好时光,对作品的要求达到了前所未有的高度。越来越多的心理学家向越来越多的杂志寄送越来越多的稿件;越来越多的研究人员在互相争夺日益缩减的研究基金。院长和其他院系领导比以前更看重论文发表的数量。过去和颜悦色的教务长们对教员能够申请到研究基金总是颇感意外,备受鼓舞;而现在,他们拉长着脸,甚至希望新晋员工都能够申请到更多的研究经费。有些院系甚至把教员能否申请到经费与他们的晋升挂钩。在研究型大学,如果无法写出更多的论文,就无法升职或获得终身教职。甚至,在一些小型的教学型高校,对于论文发表的要求也日益提高。所以,这年头,要想在科研领域混口饭吃,真的不容易!


\section{我们现在学习写作的方式}
写作是一项技能,而不是什么天赋或特殊才能。所有的高级技能一样,写作技能必须通过系统的指导和实践来培养。人们必须学习相关的规则和策略,并努力实践 (Ericsson, Krampe \& Tesch-Romer, 1993)。心理学家早已发现,有意识的练习有助于技能培养,但是这一理论似乎还没被用于培养写作技能。我们来比较一下写作技能的教育和其他高级技能的教育。教学很难,所以我们有专门的研究生院教学生如何“教学”。学生们通常要先学习一门“教育心理学”,然后通过当助教来练习如何“教学”。很多研究生在研究生阶段的每个学期都做助教,而后才能成为一名合格的教师。统计和研究方法也很难,所以我们要求学生在高年级阶段不断学习这些内容,通常都由在方法论和统计方面富有经验的专家来讲授。通过多个学期的学习,学生们终于成为老练的研究方法高手。

那么,心理学是怎样训练学生学习写作的呢?最常见的模式是指望学生们能够通过向他们的导师学习来掌握写作的技能。问题是,很多导师自己就在“水深火热”中挣扎——他们常常抱怨根本没有时间写作,常常眼巴巴地等待春假或是暑假的来临 这简直就是盲人骑瞎马。但是,这并不是他们的错:正如很多学生所说,大多数教授也都是在“摸爬滚打”中学习写作的。有些系的确开设了写作课,但是这些课程往往忽视了写作动因方面的问题,转而关注教授如何写课题申请报告或是其他各类报告。

研究生毕业以后,就再也没有导师会对学生们才完成了一半的论文给予指导和鼓励,学生们必须自力更生了。我认为这是令人担忧的,我们并没有给下一代学术写作者充分的教育,却期待他们做得更好。


\section{本书的解决之道}
学术写作可以是一部鸡飞狗跳的闹剧。教授们为写了一半的论文忧心忡忡,抱怨又收到了残酷的退稿信,在经费申请最后期限的前一秒才匆匆忙忙提交了申请报告,幻想着平静的夏日午后可以心无旁骛地奋笔疾书,然后埋怨开学日期的临近严重影响了自己的产能。心理学本身就够戏剧化了,我们真的不需要再添加什么戏剧效果。上述所有都是坏习惯。学术写作本应是循规蹈矩、枯燥而平凡的。为了保障本书以一种平凡的视角来看待写作,本书将不会讨论“写作的灵魂”、各种宗派的“写作灵感”或是“写作的精髓”。只有诗人才喜欢讨论“写作的灵魂”。你应该像个普通人一样写作,而不是像个诗人,甚至不应该像个心理学家。同样,本书也不会探讨任何“防御性”或是“回避心理”,关于这些理论,你大可以到书店的自助角自学了解。 《文思泉涌——如何克服学术写作拖延症》视写作为一系列具体的行为,就像:(1)首先在椅子/板凳/高脚凳/长软椅/马桶/草地上坐下来;(2)然后敲打键盘,写出一段文字。你绝对能够通过简单的办法来培养这些行为。让别人尽情拖延、做白日梦和抱怨去吧,你所要做的就是:坐下来,写。

当你阅读本书时,你要记住,写作不是比赛或者游戏。你想写多少就写多少,长短无所谓。千万不要觉得你有责任写更多,也不要为了发表而发表,写一大堆毫无意义的东西。不要误以为那些发表了大量文章的心理学家就有更多的研究成果。心理学家发表文章的目的多种多样,其中最为重要的是用于学术交流。文章的发表是一项科学活动必要的、自然的终结点。科学家们通过文字互相交流,那些印成铅字的文章构成了心理学的基石,它们阐述了人类是怎样的存在以及人类行为背后的原因。我相信大多数心理学家都在写作这件事情上倍感受挫,他们希望能够写得更多,也希望写作变得容易些。这本书献给他们。


\section{各章预览}
这本薄薄的小册子就如何写得更多给出了实用的、个性化的观点。第二章中,我们彻底检查了人们为写不出东西而找的拙劣借口。我们逐一分析这些借口,发现它们对于写作的效率毫无影响。这章将介绍如何用制订写作计划的方式来分配写作任务。第三章介绍了激励你执行写作计划的各种办法。你将学到如何制订好的目标,通过确立优先性原则来同时处理多项任务,以及如何管理你的写作进程。为了帮助培养你的新习惯,你可以和朋友一起建立写作小组。第四章是关于如何组建既有趣又有益的“失写互助组”(agraphia group)——一种有助于培养良好写作习惯的互助小组。第五章教你怎么写得更好。写得好的论文或是开题报告总能在众多平庸之作中脱颖而出,所以你应该努力写得更好。

第六章、第七章主要介绍了写作的原理。第六章剖析了实用的心理学论文写作技巧。我们可能并不喜欢阅读论文,但是我们必须写作论文。多产的写作者告诉过我他们是怎么写论文的,主流期刊的编辑告诉过我他们希望看到怎样的论文。第六章探讨了关于论文发表的几个入门级问题,例如如何给编辑写投稿信,如何与别人合作写作。第七章讲述了怎么写学术著作。心理学界为有抱负的学者们提供的资源实在有限,基于此,我就如何写学术著作及如何与出版商合作提出了一些个人见解。第八章对全书作了总结,还写了很多鼓励的话。



\chapter{妨碍写作的借口都“貌似有理”}
写作是一项严峻的挑战,就好像修理污水管或是经营一家殡仪馆。虽然我从未给死尸穿过衣服,但是我敢确定,给尸体作防腐处理要比写一篇关于此项活动的文章来得容易。写作很难,这就是为什么我们当中有那么多人写得那样少。如果你在阅读这本书,那么也许你能体味那种屡屡受挫的感受。每次我和教授或是研究生聊起写作这桩事,他们总是提到很多阻碍因素。他们相信,如果不是这样或那样的因素妨碍了他们,他们本来能够写得更多。我把这些借口称为“貌似有理”的借口。乍一看,这些妨碍写作的借口挺像模像样的,但是只要稍加深究,就会发现它们根本站不住脚。本章将列举几个最为常见的妨碍写作的借口,并教大家如何用最简单的方法来克服它们。

\section{借口一}
“我找不到时间写作”,也可以称为“如果我有更多整块的时间,我就能写得更多了”。

这个借口简直是学术界的“尚方宝剑”。我们都这么说;有些屡屡受挫的作者甚至把它高高举起作为人生指南。但是这个假设根本靠不住,就好像有些人相信人类只使用了脑容量的10\%。正如所有的假设一样,这一借口得以存在是因为它让人感到舒服。人们总是想当然地认为周遭的环境总是和自己作对,而如果时间表上有更多大块的空余时间,就可以有更多的时间用来写作,自然就能写得更多了。系里的同仁们对此表示理解,因为他们自己也觉得找不到时间写作。与同事们一起共同经历灰暗的挫折,从某种程度来说有一种奇怪的窃窃的甜蜜感。

为什么这个借口是假的?关键问题就在“找”字上。当人们赞成这个说法时,我的脑海里总是浮现一幅画面,他们的眼睛在时间表上游走,就好像自然科学家在努力寻找一种名叫“写作时间”的生物,这种生物隐藏得太深了,根本寻不到它们的踪迹。你需要“找时间教书”吗?当然不用——你有一张课表,你按部就班,从不迟到。如果你认为写作时间藏匿在你每周计划的某个角落,你就永远不会写得更多。如果你认为非等到整块的时间,比如春假或是暑假,否则就不能写作的话,那么你也不会写得更多。“找时间”对于写作来说是毁灭性的。以后再也不要这么说了。

相反,你不应该“找时间”,而是“安排时间”来写作。高效的写作者都会制订一个时间表并严格遵守。就这么简单。现在,花几分钟想一下你想要的写作时间表。好好想想你的一周安排:是否每周总有那么一些相对空闲的时间?如果你周二和周四有课,那么周一和周三的早上可能是最好的写作时间。如果你觉得下午或是晚上精神更好,那么就安排在晚些时候。每个人根据自己的其他安排,会有不同的黄金“写作时间”。关键在于规律性,而不在于天数或是小时数。你每周是花一天还是五天来写作并不重要——只要你腾出一段有规律的时间来,把它标在你的周计划表上,然后在这段时间坚持写作。开始的时候,你可以每周安排四个小时左右。等看到你所写的文章字数突飞猛进时,你可以再适当延长写作时间。

每次讨论到写作安排时,人们总是问我:那么你的安排呢?(有些人的口气里带着挑衅的意味,仿佛他们希望我耸耸肩,回答道: “怎么说呢,制订计划这种事情,说起来容易做起来难啊。”)我每周一到周五上午八点到十点用来写作。我起床,煮咖啡,然后坐在我的办公桌前。为了避免干扰,我写作前不查邮件,不洗澡,也不换衣服。简言之,我起床,然后写作。开始和结束的时间可能会前后调整,不过我每个工作日大概写两个小时。我不是一个喜欢早起的人,不过早上写作有很多优势,我能够在被处理邮件、学生约谈和与同事会面等事务淹没之前,抓紧宝贵时间写点东西。

大多数人都有一个既浪费时间又毫无成效的坏习惯,被称作“突击写作“ (binge writing) (Kellogg, 1994)。想写,拖延,为拖延感到万分内疚和焦躁,“突击写作者”(binge writer)最终选择某个周六什么也不做,只写东西。这样他们的负疚感得以缓解,整个“突击写作”的周期又开始循环。“突击写作者”花在为没有写作而感到内疚和不安上的时间,要远远超过制订计划的人花在写作上的时间。当你执行计划时,你就没有时间担心没有写作,抱怨找不到时间写作和沉溺于夏天你能够写多少东西的幻想上了。相反,你在固定的时间写作,然后彻底忘了它。我们有很多比写作更值得关注的事情。比如我总是担心我是不是喝了太多的咖啡或者我的狗是不是又到后院那个恶臭的水塘里喝水,但是我从不担心我要找时间写这本书:我知道我会在明天早上八点钟继续。

当“突击写作者”因他们的坏习惯而遭到质疑时,他们常常自我辩护:此乃天性使然,“我不是那种喜欢制订计划并能严格执行的人”。这根本就是废话。人们拿天性来说事儿,是因为他们不想改变(Jellison, 1993)。那些宣称自己不是“计划达人”的人在其他方面却成了计划专家:他们总是在固定的时间教书,在固定的时间上床,在固定的时间看自己喜欢的电视节目等。我就遇到过自称没有能力坚持每天写作的人,却无论刮风下雨都能够坚持在固定时间出门慢跑。千万不要还没有开始就选择放弃——制订计划是高效写作的唯一秘诀。如果你不打算制订计划,那么请你小心合上这本书,把它弄得像新的一样,然后当作礼物送给那些想成为一个好的写作者的朋友。

你必须坚决地捍卫你的写作时间。记住,你要安排时间写作,而不是找时间写作。你自己决定了这段时间是用来写作的。在这段时间内,你不能与同事、学生或是导师见面与约谈;也不能批改作业或者备课;更不能查阅电子邮件、看报纸或是查看天气预报。关掉你的网络和电话,关上门。(我以前会在办公室的门上挂一个“请勿打扰”的牌子,但可恶的是,这块牌子常被人误解为“他关上了门,但是他想让我知道他在办公室,所以我应该敲门而入” 。)

我得提醒你们,其他人未必会理解你对写作时间的忠诚。人们会希望在这段时间和你开个会,他们并非存心捣乱,只是无法理解你为什么没有时间。他们会怨你不知变通,认为你顽固不化,甚至揣测是不是有其他什么不便道明的原因使你不愿见他们。对我来说,最常见的问题是研究生们希望我能够早上九点见他们——这个时间对他们来说最方便,但是不巧,这正好在我的写作时间内。同样的,有时候我也不得不在写作时间里参加一些会议,因为这是唯一一个对所有与会者来说都方便的时间。

怎样应对这些无心打扰到你的人呢?对他们说“不”——这个词也许不能让你远离毒品(南希·里根\footnote{南希·里根,美国总统里根的夫人,曾在美国发起“对毒品说不”运动。——译者注}除外),但能够帮助你捍卫你的写作时间。你有两个很好的理由说“不”。首先,只有“失败的写作者”才会把你的拒绝当作是对他的挑战。我遇到的所有认真的写作者都非常尊重我对写作时间的坚持。他们也许会有一点不高兴,因为我无法在他们希望的时间安排会面,但是他们都会表示理解,因为制订计划是写作的唯一出路(这些人也会在他们的写作时间里拒绝我的会面要求)。为此而恼怒和满腹牢骚的人都不是好的写作者,所以别受他们拖累。其次,人们不会想要占用你的上课时间、你的家庭聚会时间或是你的睡觉时间,却想要占用你的写作时间,因为他们认为你的写作时间无关紧要。作为一个学者,你是职业的写作者,就像你是一名职业的教师一样。把你的写作时间当成你的上课时间,对那些无心打扰的人说“不”,并解释为什么你不能(注意:是“不能”而不是“不愿意”)打破你的写作计划。如果你不喜欢说“不”,那就撒谎。如果你也不喜欢撒谎,那就用你在读研究生的时候学过的理论:归咎于一个“常见又固定的职责”或是一个“世俗的拖累”。

在既定的写作时间里奋笔疾书,但是也不要教条化地圃于写作时间。如果你在写作时间结束后仍然文思泉涌或者在其他时间里坚持写作,岂不是更好?我把这称为“意外之作”。一旦你养成了习惯,坐下来写东西就会变得容易了。不过你要提高警惕,不能用“意外写作”来替代正常的写作时间。不管你在放春假的时候写了多少——你都应该遵守你的写作计划,严格执行。如果你发现自己胡说——“我周末已经写得够多了,周一我就轻松一下吧”,那这本书能够帮到你:合上它,用你非惯用的那只手的拇指和食指夹住它,在自己面前狂甩五分钟,以便提醒你怎样才是正确的做法。

或许你对“计划写作”是否有效还有保留意见。“这真的就是秘诀吗?”你也许会问,“难道就没有其他的法子能够多写点?”没有了——做好计划并严格执行是唯一的办法。职业写作者拉夫·凯斯(Ralph Keyes,2003: 49)在对成功写作者们的写作习惯做了大量研究后发现,“使一位写作者得以高产的秘诀就是坐在书桌前日复一日地写作”。如果你每周安排四个小时写作,你会对你能够完成的字数感到惊讶,确切地说是震惊,惊得哑口无言,呆若木鸡。你会发现自已完成了之前无法想象的事,例如提早完成了开题报告。你会收到改稿通知,并在一周内完成。你不敢再和系里的同事交流对写作的恐惧,因为你害怕他们会说“你已经不再是我们的战友了”,而且他们的话千真万确。


\section{借口二}
“我需要先作一些数据分析”,或者“我要先看几篇文章”。

这个借口最为阴险,危害也最大。首先,这个借口看起来合情合理。你也许会说,“我总不能既不看参考文献又不作数据分析,就直接写文章吧”。没错,但是我见过很多人把这个借口当作颂歌每日吟唱。同事们一开始都很尊重他们,相信他们要么是完美主义者,要么就是数据狂人。但是他们写得不多,也从不作所谓的数据分析。“突击写作者”往往也是“突击阅读者”和“突击数据分析员”,阻碍他们写作的坏习惯同样会妨碍他们做其他与写作相关的准备工作(Kellogg, 1994)——阅读、列提纲、提炼观点、分析数据,等等。像其他所有“貌似有理”的借口一样,这个借口也经不起深究。

要破解它很容易:在你安排的写作时间内,做一切你需要做的事情。要做一些数据分析吗?在你的计划时间里做。要看一些参考文献吗?在你的计划时间里做。要校验清样吗?也在你的计划时间里完成。要读一本如何写作的书呢?你知道应该什么时候读。写作的含义远不只是坐在电脑前打字,任何与完成写作任务相关的活动均可称为写作。例如,为了完成一篇论文,我常常会连续花几段写作时间来作数据分析。有时候我会花上一整段写作时间来做一些鸡毛蒜皮的事,比如研究某一杂志的投稿要求、画图表或校样稿。

这也是为什么只有制订计划才能够保障写得更多的又一原因。专业的写作包含很多部分:广泛的文献阅读、仔细的分析、文字严谨的研究方法陈述。我们无法“找时间”来完成所有相关文献的检索与阅读,就如同我们无法“找时间”写下阅读这些文章的笔记。所以安排好你的写作时间来完成这些工作吧。这样你就不会为找不到时间来读文献和作分析而感到惶恐了,因为你知道什么时候你会完成它们。


\section{借口三}
“我需要一台新电脑才能更好地写作”(同理,这里的“电脑”可以替换为“新的激光打印机”“新椅子”“新书桌")。

在所有的借口中,这条是最让人抓狂的。我不确定人们是否真的相信它——与其他的借口不同,这一条其实真的只能勉强算是一个“借口”。我自己的实例可能就能破解这个借口。当我还在读研究生时,我刚开始严肃地写作,我从一个同学的男朋友那里买了一台很旧的电脑。这台电脑即使在1996年也可以算是一件古董:没有鼠标,没有Windows,只有一个键盘和DOS系统下的WordPerfect 5.0软件。这台电脑寿终正寝以后,我的一些文件也因为无法复制而随之陪葬了。我又买了一台手提电脑,常常席地埋头苦写。我现在正在用2001年购买的一台速度奇慢、老得掉牙的东芝笔记本电脑写这本书——在如今电脑更新换代如此迅速的时代,我的这台笔记本电脑老得都可以去领养老金了。

大概有八年的时间,我都坐在一把折叠椅上辛勤笔耕。这把折叠椅退休以后,替代它的是一把稍微时髦,但是同样坚硬的老式埃姆斯椅。它是一把再简单不过的椅子:没有装饰和靠垫,也不能调节高度和角度。为了满足大家的好奇心,附上一张我写这本书的地方的图片。如图2.1,只有一张大而简单的书桌(没有抽屉,没有键盘架,没有豪华的文件收纳体系,等等)、一台激光打印机和一个咖啡杯垫。在我买下这张“蓝点”(注:著名家具品牌)书桌以前,我用的是一张10美元的胶合板折叠书桌,为了显示我的品位,上面铺了一块4美元的桌布。我就坐在那把折叠椅上,在那张折叠书桌旁,写完了我那本关于兴趣的专著的大部分内容(Silvia, 2006)和近20篇学术论文。

\begin{figure}[!htb]
\label{fig2-1}
\centering
\includegraphics[width=0.9\textwidth]{fig2-1.png}
\caption{我写作本书的地方}
\end{figure}


低效的写作者总喜欢悲悲切切地抱怨没有属于自己的哪怕一小块空间用来写作。我对这个老掉牙的借口不屑一顾。我从来没有专属的家庭办公室或是私人写作空间。在狭小的公寓或是房子里,我在客厅、卧室、客卧、主卧甚至浴室里写作,我只需要一张小桌子。我在家里的客卧里完成了这本书的写作。即使现在,我已经写了那么多书和论文,也买了自己的房子,我在家还是没有独立的书房来写作。我也不需要——总有一间浴室是空着的。

我身边的突击写作者曾多次提到打印机问题是阻碍他们写作的原因之一,他们提到这一问题的频率之高让我颇为吃惊。 “如果我家里有一台激光打印机的话……”他们用充满渴望的口气这样抱怨。他们没有意识到自己不能像印钞票一样打印写作稿——打印机只能用来打印你已经写好的稿子。我非常喜爱我的激光打印机,每个认真的写作者都应该买一台,不过打印机真的不是必需的。当雪莱·杜瓦尔(T. Shelley Duval)和我在合写一本关于自我意识的著作时(Duval \& Silvia, 2001),我只有一台石器时代的喷墨打印机,他什么都没有。要用喷墨打印机打印一本书需要很长的时间,到头来我们的部分草稿还是青绿色和红褐色的,因为打印机的黑墨水用完了。

当人们抱怨他们家里没有高速互联网时,我真心祝贺他们有这样正确的判断。如果你们仔细看图2.1,就会发现我的电脑上根本没接网线。我太太在家里的书房里安装了高速网络,我没有。这玩意儿只会让我分散注意力。写作时间就是用来写作的,不是用来查邮件、看新闻,或是浏览最新期刊的。有时候我会觉得下载一些文章可能对写作有点帮助,不过我可以在办公室里下载。最好的自控就是让客观环境不需要自控。

威廉·萨拉扬(William Saroyan, 1952: 42)这样写道,“写作,你只需要一张纸和一支笔。”设备永远不能帮你写作;只有制订写作计划并努力执行才能帮助你成为一名高效的写作者。如果你不相信我所说的,那么就看看比尔·斯顿夫(Bill Stumpf)最新的采访。作为家具设计业的传奇人物,斯顿夫为业界领军者赫尔曼·米勒(Herman Miller)公司设计办公家具。斯顿夫因为参与了艾龙椅(Acron)的设计而闻名,这真的是有史以来最棒的办公椅了。但是作为一名写作者(Stumpf, 2000),他深知家具能做的只有这些了。 “我不确定家具和写作能力这两者之间是否存在必然联系,”他说,“我想赫尔曼·米勒一定不喜欢听到我这样说
(Grawe, 2005: 77) 。”


\section{借口四}
“我只是在等待对的时机”,或者“我在灵感来临的时候才能写出好的文章”。

这最后一个借口是最可笑和最无厘头的。我无数次从那些不知何故拒绝制订写作计划的人那里听到这样的理由。 “好的作品都是在我有灵感的时候一气呵成的,”他们说,“在我没有心情的时候逼着我写也无济于事。我必须感到我想写了才行。”毫无建树的写作者这样说真是可笑。这就好像烟鬼们总是辩解吸烟能够使他们感到放松,而实际上吸入尼古丁压根儿就只会导致精神紧张(Parrott, 1999)。当那些备受折磨的人为不制订计划而辩解时,他们是在支持使他们深受折磨的原因本身。如果你相信你应该只在有灵感的时候写文章,那就请你问自己几个简单的问题:这种策略效果如何?你对自己文章的数量感到满意吗?你是否常常为找时间写作或是为仅仅完成了一半的论文而感到紧张?你是否牺牲了晚上或是周末的时间用来写作?

要驳倒这个借口也不难:研究表明等待灵感是徒劳的。博伊斯(Boice, 1990: 79$\sim$ 81)为那些等待灵感的突击写作者做了一个意义深远的研究。他召集了一批为写作而头痛的大学教授,随机地给他们分配了不同的写作策略。第一组(限制写作组)的教授们被禁止在任何非紧急情况下写作;第二组(顺其自然组)的教授们有50段写作时间,但是仅在他们感到有灵感的时候写;第三组(附加干预组)的教授们有50段写作时间,并且在这些时间内必须写作(如果他们没写,就得向一个他们不喜欢的组织交罚款)。统计项是每天写作的数量和每天提出的创造性观点的数量。图2.2的数据显示了博伊斯的结论。首先,附加干预组的产量最高:他们的产量是顺其自然组的3.5倍,是限制写作组的16倍。那些只在有灵感的时候写作的人比那些被告知没事别写的人多写了一点点——灵感的作用实在是被高估了。其次,那些被逼着写作的人提出了更多创造性的想法,他们平均提出想法的间隔是1天;顺其自然组是2天,而限制写作组是5天。由此可见,写作本身孕育了继续写下去的基础。

\begin{figure}[!htb]
\label{fig2-2}
\centering
\includegraphics[width=0.9\textwidth]{fig2-2.png}
\caption{不同写作策略的效果}
\end{figure}

有些类型的写作实在太不可爱了,以至于没有一个正常人会喜爱它们。什么样的人会对写开题报告感到热情高涨呢?谁会早上醒来,就对写作“具体目标”和“协议/合约安排”感到欢欣鼓舞呢?写课题申请报告就好像报税,而且你还没法请个会计来帮你做。如果你对阅读美国卫生与公共服务部提供的科研基金SF424申请指南有着难以抑制的热情,那么你根本不需要这本书。如果你和其他人一样,那么要完成课题申请报告,你需要的不仅仅是“灵感”。

那些等待灵感的人应该从云端降落,回归到正常的普通大众当中。古希腊人为诗歌、音乐和悲剧都各分配了一个神,但是从未听说他们为按照美国心理学会(APA)格式写作期刊论文安排了什么神灵。作为学者,我们不是在进行文学创作,也没有书迷等在酒店门口,手捧《人格与社会心理学公报》,要求我们在上面签名。我们写的是专业性的、学术性的文章。有些类型的学术写作可能稍微轻松——例如教科书,或是比如这本书——但是即使这样,也要求我们把有用的信息归纳齐整,然后传递给读者。我们的写作很重要,因为它是实用的、清晰的和启发思考的。

拉夫·凯斯(Ralph Keyes, 2003)告诉我们,最杰出的小说家和诗人——在我们看来最应该等待灵感的人——实际上也并非只在有灵感的时候才写作。高产的安东尼·特罗洛普(Anthony Trollope, 1883$\sim$ 1999:121)这样写道:

\begin{lstlisting}
有些人认为他们为灵感而生,所以他们应该允许自己等待灵感来唤醒他们。每当我听到这样的布道时,总是很难掩饰我内心的不屑。在我看来,没有什么比这种说法更荒诞了,就是一个鞋匠说他要等待灵感,或是牛油烛小商贩在等待神的旨意来融化牛油也没有这么荒诞。有一次有人告诉我最可靠的有助写作的方法是在椅子上涂一点鞋线蜡。我宁愿相信鞋线蜡也不相信灵感。
\end{lstlisting}

那么这些杰出的写作者都怎样写作呢?猜猜看。成功的专业写作者,无论他们写的是小说、非小说、诗歌还是戏剧,他们的高产都有赖于有规律地写作(通常是坚持每天写作)。他们都不相信必须等有了感觉才能写作。正如凯斯(Keyes, 2003: 49) 所说,“认真的写作者笔耕不辍,不论有无灵感。随着时间推移,他们发现规律性显然是比灵感更靠谱的朋友”。有人或许会说,那就订个计划并且严格执行吧。

\section{小结}
本章对一些常见的妨碍写作的借口进行了冷静而批判性的审视。我们都喜欢躲在这些温暖的外衣下,但是披着这温暖的外衣还想码字实在是太难了。如果你还对这些借口恋恋不舍,那就把这章多读几遍,直到你被彻底洗脑而坚信只有制订计划才有未来。如果你不相信这一点,那么这本书恐怕就帮不了你了,因为无论你喜不喜欢写作,能够写得更多的秘诀只有保持规律性地写作。制订好了写作计划,你就可以继续读下一章了。它将介绍一些简单而有效的激励工具,让你能够坚持写作并且提高你的写作效率。
\chapter{赘语}
与赘语作斗争就像同杂草作斗争一样,作者总是稍稍滞后。五花八门的新赘语一晚上就冒出来,到中午就成了美国话语的一部分。看看尼克松总统的助手约翰·迪恩在水门事件电视听证会那天的创举吧。第二天美国人人都说“就在当前这个时间点上”,来代替“现在”。

看看披挂在动词后面但并无必要的介词吧。我们不再“领导”委员会。我们“领导起”委员会。我们不再“面对”问题。当我们能“空闲得出来”几分钟,我们“针对问题面对”它。也许你会说,那都是细枝末节,不值得烦扰。但这的确值得烦扰。写作的改进同我们能去除的赘语数量成正比。“空闲得出来”中的“得出来”就不应该有。仔细审查付诸笔端的每一个词,你会发现毫无目的的词语数量惊人。

举形容词“私人/个人的”为例,如:“我的一个私人朋友”、“他的个人感受”,或者“她的私人医生”等。这些只是成百上千可去除词语的典型代表。用“私人朋友”来区分“生意朋友”,结果使语言和友谊都贬值了。某人的感受就是那个人的个人感受——那就是“他的”的含义。至于私人医生,指的也就是被叫进突然病倒的女演员化妆室的男医生或女医生而已,这样该演员就不必由剧院指派的那种公事公办的医生来诊治了。有朝一日我倒愿意看见那人变成“她的医生”。医生就是医生,朋友就是朋友。其余皆赘语。

赘语即佶屈聱牙的词语,它排挤掉简明的同义词。甚至在约翰·迪恩之前,业界人士已经停止说“现在”。他们说“在当下”(“我们所有的操作员在当下都在帮助顾客”),或者说“在目前的这段时间里”,或者说“不久很快”(意思是“一会儿”)。其实这个意思都可以用“现在/马上”来表示目前的时间(“现在我可以见他了”),或者用“现今”来表示历史上的现在(“现今物价高涨”),或者简单用动词进行时(“在下雨”),而没有必要说“在目前的这段时间里我们正在经历降雨”。


“经历/感受”这个词是最糟糕的赘语之一。甚至连牙科医生都会问你是否感受到疼痛。而假如椅子上坐的是他自己的孩子,他就会说,“疼吗?”简而言之,他就会是他自己。他通过在职业角色中用浮夸的词语,使自己不但听起来更重要,而且还钝挫了事实中痛苦的一面。此类语言同样用于航空乘务员演示当机舱内用尽空气时氧气罩会如何落下。“假如飞行器一旦最终遭遇不太可能发生的紧急情况,”乘务员如此开始——其用语本身就够夺人氧气的了,大家都已经准备好了任何灾难的发生。

赘语是乏味的委婉语,贫民窟成为社会经济萧条区、垃圾清扫工成为废物处理人员、城市垃圾场成为减量单位。我想起比尔·莫尔丁的卡通,描述有两个流浪汉在货车车厢里。其中一人说,“我开始只是个简单的盲流,可现在是绝对失业了。”赘语是政治正确性的极端表现。我见过一则男孩夏令营广告,组织者的目的是提供“个别关注给那些极少数与众不同的孩子”。

赘语是公司官方语言,用以掩盖其错误。当DEC公司裁减3000个岗位时,其告示并没提及解雇,而是称之为“不情愿之措施”。在空军发射的火箭爆炸失事后,火箭被称为“提早撞击地面”。当通用汽车公司关闭一家工厂时,那是“与产量计划相关的调整”。破产的公司是处于“负现金流状态”。

赘语是五角大楼的语言,称侵略是“加强保护性反应打击”,对其庞大预算需求的解释是为了“反威慑力”。正像乔治·奥威尔在《政治与英语语言》一文中指出的那样——该文写于1964年,但在柬埔寨、越南以及伊拉克战争期间经常被引用——“政治性演讲与写作主要是为不可辩护者辩护……因而政治性语言不得不由委婉语、设问句以及纯粹的云山雾罩式的模糊语构成”。奥威尔警告说,赘语不只是恼人,而且是致命的工具。这在最近几十年间的美国军事冒险中得以证明。在乔治·布什任总统期间,伊拉克的“平民伤亡”成立“间接破坏”。

语言伪装在亚历山大·黑格将军任里根总统国务卿期间达到新高峰。在黑格之前,没人想到会说“在此成熟之机的关键时刻”来表示“现在”。他告知美国人民与恐怖主义作斗争可以用“有意义的制裁性强制手段”,中程核导弹正处于“关键性漩涡之中”。至于公众对此心存的忧虑,他的意思是“交给艾尔”,但他实际说的是:“我们必须将此推到公众关注度更低的分贝。我认为在这一领域的情况没多少学习曲线可得。”

我可以继续从各行各业引出例证——每个行业都有不断增长的行话库,它们扬起尘埃,迷住大众的眼晴。但都列出来会很烦人。在此提出来的目的是引起大家注意:赘语是写作的敌人。因此要警惕并不比短词强的长词:assistance/help(帮助),numerous/many(许多),facilitate/ease(促进),individual/man or woman(个人),remainder/rest(剩余),initial/first(首先),implement/do(实施),sufficient/enough(足够的),attempt/try(试图),referred to as/called(被称为),还有成百上千更多的词。警惕所有含糊的时髦新词:paradigm(范例)与parameter(参数),prioritize(优先考虑)与potentialize(使成为潜力)。这些都是窒息写作的杂草。能写“与什么人交谈”,就不要写“与什么人对话”。不要写“与什么人协调配合”。

同样阴险的还有人们用于解释自己打算如何解释的各种词组:“我也许可以补充”、“应该指出的是”、“使人有兴趣注意的是”。假如你也许可以补充,就补充吧。假如有什么应该指出,就指出吧。假如使人有兴趣注意,就使它有趣好了。当有人说“那会使你感兴趣吗”之时,我们对下面所说的到底是什么不都会感到疑惑不解吗?不要胀大本无须胀大之事,如:可能除外的情形是/除外,由 于这样一个事实/因为,他完全缺乏这样一种能力/他不能,直到那样一个时刻/直到,目的就是/为了⋯⋯

有什么一眼就能认出赘语的办法吗?有一个办法,我在耶鲁的学生发现其行之有效。我会在一篇文章中用括号括上任何一个无用的成分。通常情况下只有一个词被括上:动词后赘的不必要的介词(命令起来),或者同动词意思一样的副词(高兴地笑),或者描述已经很清楚的事实的形容词(高高的摩天楼)。我的括号经常括上削弱句子力度的小小修饰语(有一点儿),或者诸如“在某种意义上”之类的毫无意义的词组。有时候我的括号会括上整个句子——就是那种基本上重复前一句,或者读者无须知道,或读者自己能明白的句子。大多数初稿可以砍掉一半而不损失任何信息或作者的语气。

我括上学生的浮夸词而不划掉这些词的原因,是避免冒犯其视若神明的散文体。我要完好无损地保留他们的句子,以便他们自己分析。我对学生说,“我也可能错,但我认为去掉这个地方并不影响意思。你自己决定。读一下不带括号的部分,看是否通顺。”在学期的前几周,我发还给学生的文章都是括号。整段整段都被括上。但很快学生们就学会在心里给赘语加上括号,到了期末,他们的文章就几乎没有赘语了。现今,当初的许多学生已成为专业作家,他们对我说,“我至今还能看见你的括号——它们会跟我一辈子。”

你也可以培养同样的眼力。从文章里找出赘语,无情地修剪。对所有可以去除的部分都要心存感激。重新审校你写的每一句话。每一个单词都起新作用吗?有没有什么地方虚夸、做作或者赶时髦?你是否只是因为自己觉得写得漂亮,而对那无用的部分不能割舍?

简洁,再简洁。
\chapter{风格}
作者努力写出简洁的句子,但总有那肿胀的怪物埋伏在暗处捣乱。有关写作初期的告诫就谈到此。

“但是,”你也许会说,“如果我去除你认为是赘语的所有部分,如果我将每一句话都剥到只剩骨头,那我还能剩下什么?”这个问题问得合理。简洁推到极致也可能会指向一种不比“狄克喜欢珍妮”和“看斯波特跑”这样的句子更复杂多少的风格。

我将首先从木匠手艺的层面回答这个问题。然后涉及更大的问题,即作者是谁,如何保护作者的身份。

很少有人意识到自己写得多么糟糕。没人向他们指出有多少过剩或含糊的词语溜进自己的写作风格中,以及它们如何阻碍了作者想说的话。假如你给我一篇八页的文章,我叫你删减到四页,你会叫喊说那做不到。然后你回家去做,结果会好得多。之后便是难的部分:删减到三页。

要点是你必须先将自己的写作拆开,然后再搭建起来。你必须知道基本工具是什么,以及其预设的作用是什么。再拿那个木匠手艺比喻为列,首先必须锯好木头,然后钉钉子。之后,你如果有雅兴,再切修棱角、添加别致的顶部。但千万不要忘记你是在练习一项基于一定原则之下的技能。如果钉子不牢,房子就会坍塌。如果动词不牢、句子结构摇晃,句子就会分崩离析。

我得承认,有一些非虚构作家,如汤姆·沃尔夫和诺曼·梅勒,他们建构起了不起的房屋。但这些作家花了许多年来学习技艺,最终才搭建起神奇的塔楼和空中花园,使我们这些从未梦想过如此装饰的人惊叹不已。他们知道自己在做什么。没人一夜间就成为汤姆·沃尔夫,就连他本人也不能。

那么首先要学会钉钉子。如果你所建房屋既结实又好用,就该对具简洁之力感到满意。

但你没有耐心去成就一种“风格”——去修饰简朴的词语,这样读者就会把你认作一种特殊的人。你只会寻觅花哨的比喻、华而不实的形容词,就好像“风格”是什么能在风格店买到的东西,是可以用装饰漆那鲜亮的颜色装饰词语的东西。(装饰漆是装修工用的彩色漆。)风格店并不存在,风格对于写作中的人是有机体,就像头发是他自己身体的一部分,或者假如他秃顶,那么就是他身体缺乏的那部分。试图添加风格就像加假发在秃头上。瞧第一眼时,之前秃顶的人看起来年轻,甚至还帅气,但瞧第二眼时——看假发时人们总会瞧第二眼——那人看起来就不对劲了。这里的问题并非是他看起来没有梳理好,他梳理得很好,我们真得敬佩假发匠人的高超工艺,问题是他看起来不像自己了。

这个问题是那些特意装点自己文体的作家的通病。你丧失的是使你自己独一无二之处。读者会看出你是否在装腔作势。读者要的是,与他们交谈之人听起来是真挚的。因此,写作的一个基本准则是:做你自己。

然而,没有什么准则比这一条更难遵循。这要求作者做到两件事,而按其新陈代谢的本能来说,这些都是难以做到的。他们必须放松,必须有信心。

叫作者放松就像在检查疝气时叫人放松一样;至于信心,看,他多么僵硬地坐着,眼睛直勾勾地盯着等待他造词儿的电脑屏幕。看,他多么频繁地站起来找吃的或喝的。作者会想方设法躲避写作实践。我可以证明我在报社工作期间,作为记者,我每小时去饮水机的次数大大超过身体对水的需求。

如何拯救作者出苦海呢?很不幸,并没有拯救的办法。我只能安慰大家说并非只有你的遭遇如此。有些日子好过一些,另一些日子则难熬到让你绝望,不再想写作。大家都有过这些日子,而且还会有更多这样的日子。

当然,最好还是尽量减少难熬的日子。这就使我回到如何放松这个问题上来。

假如你就是那位坐下来准备写作的作者。你想到你写的文章必须具有一定的长度,不然就不会显得重要。你想到文章印出来有多么庄严。你想到那么多人阅读你的文章。你想到文章必须具有重重的权威性。你想到文章的风格必须炫目。怪不得你会浑身发紧:你在忙于想着自己甚至还未开始的文章写完之后所要承担的巨大责任。而你发誓要对这一任务称职,于是四处找大词,找那些如果你不是想刻意给人突出印象就根本想不到的词,并且一头栽进去。

第一段是一场灾难——整段话成了似乎来自机器的一系列笼统词语的组合。人是不会那样写的。第二段也好不了多少。但是第三段开始有点儿人情味,而到了第四段你开始听起来像自己了。你开始放松了。令人称奇的是,编辑会常常删掉文章的前三四段,甚至前几页,而始于作者听起来开始像自己的部分。前几段的问题不只是缺乏人情味和浮夸华丽,而是根本就没说什么,只是刻意地要写出一个花哨的前言。作为编辑,我总是在寻找说类似以下这样话的句子:“我永远不会忘记那一天……”这时我就想,“啊哈!真人说话啦!”

作者用第一人称写作显然是最自然的。写作是把两个人之间密切的交往付诸笔端,写作保持人情味才会顺畅。因此我敦促人们用第一人称写作,用“我”或“我们”。但大家表示抗拒。

“说我认为什么,或者我感觉什么,那个我是谁啊?”他们问。

“不说你认为什么,那么你又是谁呢?”我回答他们,“只有一个你。没有任何其他人想的和感觉的一模一样。”

“可是没人会在乎我的观点,”他们说,“那样会让我觉得太突出了。”

“如果你对他们讲述有趣的事,他们会在乎的,”我说,“而且要用自然而然的词语讲述。”

然而,要作者用“我”并不容易。他们认为必须先赢得袒露自己情感和思想的权利,不然就会显得自以为是,或者不庄重。此类恐惧也影响了学术界,因而出现了学术性的“某人/有人”的用法(“有人不能苟同于莫尔特比博士对于人类状况的观点”),或者无人称的用法(“希望费尔特教授的专著会理所当然地拥有更广泛的读者”)。我可不想见用“有人/某人”这样的人——很乏味。我要的是对自己的话题有激情的教授来向我诉说该话题为何使他着迷。

我意识到写作中有广大领域不允许用“我”。报纸不要第一人称代词“我”用于新闻报道中;许多杂志也不要它出现在文章中;商业、社会机构不要它用在大量发送到美国家庭的报告中;大学不让把“我”用在学期论文或学位论文中;还有英语教师也不赞成用第一人称代词,除非是作为书面语的“我们”(“我们在麦尔维尔对于白鲸的象征用法中看到……”)。以上那些禁用是合理的。报纸文章就应该由客观报道的新闻构成。在学生没经过一番挣扎学会从其内在优点和外在评论评价一部作品之前,教师不希望学生避重就轻地发表意见——“我觉得哈姆雷特挺愚蠢”。我也理解教师的这些想法。“我”这个词有可能变成某种自我陶醉和自我逃避的工具。

然而,我们的社会已经变得害怕袒露自己的心声。向我们发送宣传品、寻求支持的社会机构听起来惊人地相似,但所有这些机构——医院、学校、图书馆、博物馆、动物园等——当然是由那些有不同梦想和展望的男男女女创建和管理的。这些人都在哪儿?在所有那些非人称的被动语态句子中,如“已经采取行动”和“重点已经确定”,很难瞧见他们的真面目。

即使是在不允许用“我”的时候,也有可能传达一种个性化的我的意思。政治专栏作家詹姆斯·赖斯顿在专栏写作中并没用“我”,但我却很清楚他是什么样的人,对其他很多随笔作家和记者,我也可以这么说。好作家在字里行间是看得见的。如果不允许你用“我”,写作时至少要用“我”思考,或者第一稿用第一人称写,然后再把“我”去掉。这样能预热你的非人称风格。

风格与心理绑在一起,写作的心理机制根深蒂固。我们表达自己的特有方式,或由于“作者心理阻滞”而没能表达好自己的原因,部分埋藏于潜意识心理中。作者心理阻滞的种类就像作家种类一样多,我也没有捋清它们的意图。这是一本薄书,我的名字也不叫西格蒙德·弗洛伊德。

但是我也注意到避免“我”的新理由:美国人在言辞上不愿意冒任何风险。我们的上一辈领袖们直言不讳自己的立场和信仰,现今的领袖们却千方百计地巧用词语逃避这一宿命。看看他们如何在电视采访中拐弯抹角,就是不立场鲜明。我记得有一次福特总统向一行来访的商人保证他的财政政策会奏效。他说:“每月我们所看见的是不断增亮的云彩,此外别无他物。”我对此的理解是那云彩还相当黑。福特的句子含义模糊不清,等于什么也没说,却给他的选民打了镇静剂。

后来的当局者们也并没有起色。国防部长卡斯帕·温伯格在1984年评价波兰危机时说:“有继续严重关注的余地,而且形势仍然严重。严重的形势持续越长,严重关注的余地也就越多。”老布什总统被问及他有关突击步枪问题的立场时说:“有不同的群体认为可以禁止某种枪支。我不在其列。我是在那些深刻关注者之列。”

不过我的全能冠军当属70年代身为四任主要内阁成员的艾略特·理查森。很难确定从他那模棱两可的句子宝库中选哪一句,还是看这句吧:“但是呢,均衡地来讲,我认为少数族裔及妇女维权行动还是取得了一定的成果。”\footnote{原文如下: And yet, on balance, affirmative action has, I think, been a qualified success.}一句13个单词的话中就有5个词含义模糊。在现代公共话语中,我给它最空泛句子一等奖,但与其媲美的还有他分析如何消除生产线工人单调乏味的句子: “这样呢,最后,我形成一个坚定的信念,我在开始提到过:就是这个问题太新,无法最终判断。”


那就是坚定的信念吗?像摇摇晃晃来回摆动的年迈拳击手这样的领袖不能鼓舞信心——也不配鼓舞信心。作家也是如此。推销你自己,你的题材就会发挥自己的吸引力。相信自我身份,相信自己的见解。写作是一种自我行为,你得承认这一点。要全力以赴使自己不断向前。
\chapter{写作常见问题汇总}
以下检查清单(checklist)中所说的这些问题,虽然看似细节,但并不是小题大做、吹毛求疵。我们非常有必要在投稿前,仔细检查并尽可能避免这些问题。这么做的原因主要如下。

(1)论文的发表是需要经过同行评审的。站在审稿人的角度,你可以想象下,如果有两份稿件在你面前,第一眼望过去的时候,映入你眼帘的往往是图片和文字的整体感觉。在这种情况下,整体感觉本质上是由格式和一些细节来传达的。如果其中一份看上去格式很混乱,比如有些地方多输入空格,有些地方没有空格;上下标不注意;不同段落之间行距间距不同;文字中夹杂中文全角字符;许多拼写和语法错误;参考文献格式混乱;图片作得不规范等等。另一份看着特别舒服,没太多问题。这个时候,你会有什么感觉?你会不会感觉那个写作格式混乱的人,他做实验的时候是不是也会比较粗糙、态度不够严谨认真呢?那么既然如此,他设计的实验方案以及获得的数据结果和结论可靠吗?当然,我知道这两者之间不能画等号,科学工作者本身是要基于证据来得出结论,不能带有倾向性,但是大量出现以下检查清单中的问题,会给人带来很不好的心理暗示,因为这反映出作者的态度。毕竟我们是在向某一个期刊投稿,期刊本身就是有明确格式要求的。期刊这么做也是有原因的,是为了能够让审稿人和读者清晰阅读论文,理解论文,并便于论文的传播、交流和分享。而且我在过去的审稿中也确实发现,一般作图和文字格式比较混乱的文章,整体实验设计和结果推导方面也确实相对来说问题会更多。

(2)站在审稿人和读者的角度,太多的格式错误使人无法流畅地阅读,这样就影响了他们清晰理解作者的写作意图。可能作者的工作不错,但由于大量格式错误,分散了读者的注意力,读者没有把握到你实际想阐述的逻辑线索等,进而低估了你论文的贡献。

(3)站在导师的角度,看到一份格式错误太多的稿件,同样很容易干扰导师对于文章核心内容的聚焦,因为之前提到的格式错误问题,会干扰注意力。比如每句话都存在基本格式或语法错误,这样就很难把所有句子都衔接在一起连贯地看它们的观点展现、逻辑表达是否合理。

(4)同样的原因,这样反而很干扰创造力的发挥。检查清单上列的这些要点都属于没太多创造性的工作,我们在这个上面花费的时间和精力越多,越耽误我们投入在创造性工作上的时间和精力。所以我们如果在第一次成稿中就能注意这些问题,那么之后就可以专心地集中在具有更高创造性的事情上了。

此外,以下给出的所有案例都是我实际指导和修改学生论文过程中,所遇到的具体问题。

\section{影响表达的问题}

\vspace{0.5cm}
{\kaishu 一、定语太长,影响理解句子含义}
\vspace{0.5cm}

案例:The sandwich structured flexible Zn-air battery device were assembled with the flexible electrodes of the cotton textile waste Zn plated and the NiFe hydroxide face-to-face separated by the poly (vinyl alcohol) (PVA)-KOH hydrogel polymer electrolyte.

说明:案例中的“cotton textile waste Zn plated”这里定语太长,影响阅读。

\vspace{0.5cm}
{\kaishu 二、句子太长,影响含义表达和读者的阅读感受}
\vspace{0.5cm}

案例:By using energy storage systems (ESSs), the power system can shift part of the peak load to low power consumption period, thus utilizing surplus power during low power consumption period, improving the load rate of the power grid, in order to achieve the purpose of energy saving, which can save resources, reduce pollution, and be more friendlyto our environment.

说明:这个案例整段就一句话构成,句子过长了,导致句子中间停顿太多。一方面非常影响含义的有效表达,另一方面也使读者的阅读感受不佳,可以考虑改造为:By using energy storage systems (ESSs), the power system can shift part of the peak load to low power consumption period. Thus, surplus power during low power consumption period can be utilized to improve the load rate of the power grid, achieving the purpose of energy saving. As a summary, using ESSs in power grid can save resources, reduce pollution, and be more friendly to our environment.

案例:Figure2 shows the SEM image and EDS results of NMCTW. It can be seen that the waste cotton textile is uniformly covered by Ni after the deposition (Figure 2a), and energy dispersive spectrum (EDS) mapping of Ni element further indicates Ni metal exist which are evenly dispersed on the surface of the waste cotton textiles substrate (Figure 2b and c), which is conducive to function as an conductive electrode substrate.

说明:第二句跨度过长,也是存在同样的问题。例如可以改造为:Figure 2 shows the scanning electron microscopy (SEM) image and energy-dispersive spectroscopy (EDS) spectrum of NMCTW. It can be seen that the cotton textile waste is uniformly covered by Ni after the deposition (Figure 2a). The EDS mapping of Ni element further indicates the presence of metallic Ni particles. The Ni particles evenly dispersed on the surface of the cotton textiles waste substrate (Figure 2b and c) act as a flexible conductive electrode substrate.



\section{格式问题}

\vspace{0.5cm}
{\kaishu 一、英文论文中所有符号应为英文字符和半角字符}
\vspace{0.5cm}

这个是经常出现的问题,在以往我检查的稿件中,大部分都出现过文中夹杂使用中文全角字符这个问题,这可能是由于输入法没及时切换产生的问题。

案例:I would like to submit the manuscript entitled “Long-battery-life flexible zinc-air battery...
说明:其中的双引号是中文全角字符,应改为英文半角字符。

案例:…is stirring at 90 ℃ for about…

说明:案例中的摄氏度为宋体格式,正确的应该是英文格式,如Times New Roman格式:… is stirring at 90℃ for about...

\vspace{0.5cm}
{\kaishu 二、检查标点符号是否正确,包括句点、空格等}
\vspace{0.5cm}

案例:… by morphological regulation, which could enhance the performance of WS2 [9,10] Through the construction...

说明:第一句缺少句点。

案例:Reproduced with permission from Ref.[32] , Copyright 1998, Springer Nature.

说明:文中“Ref.[32]”后多输入了一个空格。这也是经常犯的错误。还包括少输入空格,此类格式错误包括语法错误等,可以通过一些软件(如grammarly)很好地解决,也可以在word软件中打开标点符号的标记,便于识别。

案例:The voltage decreases sharply at the end of discharge, dem-onstrating the discharge failure of the battery. The reactions occurring on the Zn electrode during the discharge can be expressed as follows: 

\begin{equation}
 Zn^{2+} + 4OH^- + 4OH EN)_4^{2-}
 Zn(OH)_4^{2-} - ZnO + H_2O + 2OH^-
\end{equation}

说明:方程格式混乱,存在多处错误。修改后的形式为

\begin{equation}
 Zn^{2+} + 4OH^- \rightarrow Zn(OH)_4^{2-}
 Zn(OH)_4^{2-} \rightarrow ZnO + H_2O + 2OH^-
\end{equation}

此外,我制作了一份文档,提供了一些常用的但经常容易输入错误的符号,下载地址:https://pan.baidu.com/s/1KedVanP6z9dHjcz0Ov-seQ。

\vspace{0.5cm}
{\kaishu 三、字体格式不统一}
\vspace{0.5cm}

(一)上下标是否正确

案例:Co3O4@NCNTS

说明:Co3O4应改为$Co_3O_4$。同时掌握批量搜索功能,比如Word中可以采用Ctrl+F调用出批量搜索功能,检查是否还有其他地方的Co3O4没有注意上下标问题。

(二)缩写格式需要统一

案例:The morphology and composition of the as-obtained Zn anodic electrode and the air electrode with nickel iron hydroxide catalyst electrodes were characterized by field emission electron microscope (FESEM, S-4800, HITACHI, Japan) equipped with... Figure 2 shows the SEM image and EDS of NMCTW. It can be seen that the waste cotton textile is uniformly covered by Ni after the deposition (Figure 2a), and energy dispersive spectrum (EDS) mapping of Ni element further indicates Ni metal...

说明:第一次在实验中出现缩写的时候,将扫描电子显微镜缩写为FESEM,但随后Results部分是以SEM来指代。改动方式之一:将之前的FESEM改为SEM。而且此处还存在第二个问题,原文写的是“electron microscope”,而非“scanning electron microscope”,所以简称也无法和SEM对应。

案例:2.3 Assembly and tests of the cotton textile-based flexible zinc-air battery

The sandwich structured flexible Zn-air battery device were assembled with the flexible electrodes of the cotton textile waste Zn plated and...

说明:表述不统一,文中有的地方表达为zinc-air,有的地方则是Zn-air,而且注意Zn/zinc和air之间的连接符号也不统一。

(三)仪器表达方式统一

案例:...field emission scanning electron microscope (FESEM, S4800,HITACHI, Japan) equipped with X-ray energy dispersive spectroscopy system (EDS). The crystalline structures of the as-obtained electrodes were measured by X-ray diffraction (XRD, D/max 2200/PC,Cu Ka radiation). Electrochemical tests were conducted in a three-electrode cell configuration using the electrochemical workstation (CHI760E and PARSTAT 4000).

说明:SEM(扫描电子显微镜)仪器给出了型号、厂家和产地(国家),但XRD(X射线衍射设备)没有厂家和产地等信息,之后的电化学工作站描述也是同样问题,


\section{参考文献问题}

\vspace{0.5cm}
{\kaishu 一、文献的规范引用}
\vspace{0.5cm}

全文引用的每一篇参考文献都要下载到全文,然后仔细阅读引用的这篇文章是不是真的提到过你在正文中所说的相关内容。如果提到过,作者是不是也是引用其他文献的。如果是,那么你还得继续找到被引用的这个更原始的文献,找到最终的源头。否则有可能别人引的文献,根本没提到这个内容,然后一旦跟着引了,就会出现明显错误了。当然还有一种情况就是由于引文习惯不良(比如没有很好掌握文献管理工具的使用)等,导致引文错误,引的论文根本没提到所要表达的内容。

文中表述的每一句话,尤其关于结论、观点类的,都要准备好相关的证据。仔细从源头出发,思考这句话是否正确。不能仅仅因为文献是这么阐述的,所以就不假思索地认为这是对的。毕竟文献中也只是阐述了作者的观点,科学不断在进步,过去的观点有错误或者实验有瑕疵,那也是很有可能的。

案例:“In addition, side $reactions^9$ related to Li electrodes or oxidative products and electrolyte decomposition are usually involved in battery electrochemistry, particularly during the charging process.”

说明:学生初稿中提到锂电极的副反应,然后在副反应这里引了文献9。但是文献9没有提到任何关于锂电极的副反应。

\vspace{0.5cm}
{\kaishu 二、参考文献在正文中的格式错误}
\vspace{0.5cm}

这也是常见的问题,解决方案很简单,如果使用Endnote之类的文献管理工具,一般期刊都会提供相应的参考文献style文件,只要导入相应的style,就可以生成正确的格式了。

案例:...performance of $WS_2$[9][10].

说明:学生初稿中正文的文献引用格式是[9][10],但基本上很少期刊是这种方式的引用格式。常见的比如[9,10]、9,10(上标)等。

案例:... which both deteriorate the catalytic property of the as-obtained electrodes. The transparent NiFe HUFs electrode electrodeposited for 500s have the large number of active sites and the good conductivity, resulting in the high catalytic performance of as-obtained electrode, which have also been reported in our previous studies.

说明:学生原稿件是拟投ACS(美国化学学会)旗下的一个期刊。文献引用应该在标点之后的,而不是在句号之前。此外,该版本中还存在不少语法错误。

\vspace{0.5cm}
{\kaishu 三、参考文献列表中的错误}
\vspace{0.5cm}

(一)期刊术语缩写不正确或者全称不正确

解决方式:以Endnote文献管理工具为例,Endnote中的Tools可以找到Term Lists选项,这个Term Lists 给出了常见期刊的缩写方式,如图\ref{fig5-1}所示。

\begin{figure}[!htb]
\centering
\includegraphics[width=0.9\textwidth]{fig5-1.png}
\caption{在文献管理工具中,正确编辑期刊的全称以及缩写}
\label{fig5-1}
\end{figure}



但是,这个Term Lists通常是不完整的,所以有一些期刊包括新期刊,没有对应的缩写方式。针对期刊全称输入有误的情况,将期刊全称输入正确后,检查Term Lists中是否有对应缩写;对于期刊全称正确,但无对应缩写的,可以自己新建Term Lists 中的Term,这样以后修改论文时,会非常方便,每次可以自动更新参考文献列表,而不需要每次手工去编辑参考文献列表。

案例:
[102] D.Su, S.Dou, G.Wang, Chem Commun (Camb) 50 (2014) 4192.

[103] D.B. Kong, X.Y.Qiu, B. Wang, Z.C.Xiao, X.H. Zhang, R.Y.Guo, Y.Gao, Q.H. Yang, L.J.Zhi, Science China-Materials 61 (2018) 671.

说明:学生初稿中,参考文献[102]和[103]中的期刊名称的缩写不规范。

(二)参考文献中的作者拼写是否正确

如果发现作者名字输入有误这些情况,也是直接在文献管理工具中进行修改。

(三)参考文献标题错误,包括上下标等

解决方式:如果发现 Endnote导入的文献,存在标题错误(如空格问题、上下标问题)或者缩写方式不统一等问题,不要直接在Word中修改参考文献中的错误,这样一来工作量大,二来后期一旦再次刷新文献,之前的改动就无效了,下次还要重新再改一次。正确方式是直接在Endnote中对文献的格式进行修改,比如Endnote中是可以直接对文献的上下标进行改动的(图\ref{fig5-2})。

\begin{figure}[!htb]
\centering
\includegraphics[width=0.9\textwidth]{fig5-2.png}
\caption{在文献管理工具中,正确编辑期刊的各项信息,包括标题和作者信息等}
\label{fig5-2}
\end{figure}



案例:以下就是以后容易出问题的改动方式,是在Endnote生成的Reference list(参考文献列表)中,直接调整上下标。导致的一个显著问题就是当以后参考文献有变化,重新利用Endnote生成参考文献列表后,这些调整都会失效,然后还需要再次操作一下。一方面增加工作量,另一方面也非常容易遗漏。

\begin{figure}[!htb]
\centering
\includegraphics[width=0.9\textwidth]{fig5-3.png}
\label{fig5-3}
\end{figure}

\section{改动某处时,是否将其他牵连到的相关部位也进行了相应的改动}

每当我们修改了一个地方的时候,还需要考虑到这个部位的改动,是否还牵连文中其他部位的改动,包括关键词、图片、表格、正文所有部位、补充材料等。

\begin{figure}[!htb]
\centering
\includegraphics[width=0.9\textwidth]{fig5-4.png}
\label{fig5-4}
\end{figure}

说明:原稿中,图中的“new Zn anode and electrolyte replacement”等表达方式不佳,在导师指导后,正文部分的文字已修改为“replacement of Zn anode and electrolyte”,但是相应的Figure(图片)中的注释没改动。

\section{全文前后是否有矛盾之处}

家例:However,the most current research focuses are the single component (zinc anode, electrolyte and air electrode) of zinc-air batteries,there is little research available on integral battery configuration. Zinc-air battery is prone to the leakage and volatilization...There are still many studies which induce the componentized cell structures after studies of certain single component on zinc-air batteries to illustrate the application prospect of the target research [6].

说明:本文之前说相关研究少(“there is little research available”),之后又说相关研究还是很多的(“There are still many studies”),前后矛盾。特别是在写作大型综述时,由于前后文相隔比较远,此外可能会借鉴不同作者观点,容易导致出现前后表达矛盾的问题。

\section{第六节检查清单(checklist)的使用方式}

根据过去指导学生的反馈来看,关于使用检查清单进行检查的推荐方法之一是每次只检查一种类型的错误。不要试图一次性地通读文章,把各类型错误都找出来。因为试图一次性地找出各种错误类型,最后的结果往往是每种类型的错误都会有遗漏。

我暂时还没有条件去仔细地查证思考背后的原因。我初步猜测是不是因为人类大脑进化至今,其实还并没特别擅长高通量的并行事件处理,所以通过一次性的阅读,试图发现不同的错误类型,最后的结果是每种错误类型的检查都会有所遗漏。比如,我们可以这次只检查图片中的(a),(b)等标记是不是格式一致,至于其他图片错误或者文字拼写错误、格式错误等,我们可以分解到第二步去检查。当然如果能力提升上来后,也可以逐步增加合并检查的事项。所以我们经常看到的检查清单类型,都是一个个打勾的框,检查完一个类型,打一个勾,再检查下一个类型的问题。检查的时候,注意举一反三,思考同类型的错误,是不是会出现在其他地方,包括标题、关键词、摘要补充材料、图片和表格等等。

在我观察这个检查清单效果时,我发现男性学生一般出错率要高于女性学生。讨论这些,是为了帮助我们可以更好地认识自己,不断提升自己工作的方式。这个我猜测会不会是由进化所决定的。男性远古主要是以狩猎方式生存的,他们需要长时间跟踪猎物,在合适的时候出击。因此,大脑的思维方式是要锁定猎物,尽可能排除猎物以外的一切干扰,以降低能量损耗,将能量用于最重要的那件事情上。而不擅长这个技能的男性,可能被淘汰了,没法繁衍至今。所以,这也是为什么现代大部分男性逛商场,一般而言会直奔主题,直接找到要买的那件物品,买完走人,而不是随意闲逛,也不太会被不想买的东西干扰。这种思维方式就导致一次只执行检查清单中的一个任务是比较好的,不要试图同时执行太多的任务。而远古女性接受信息有“面”接受的特征,她们的生存并不是依赖特定的猎物,而是要在野外尽可能地搜寻到用于生存或改善生活的东西。比如,她们会同时留意到地上的作物、树上的果实,同时还能注意到一些可供穿着、可供装饰的东西。所以,她们才可以发现除了猎物以外的生存方式,这要求她们不是聚焦在一个点,而是要非常发散地捕捉到各个可能的“细节”。所以逛商场的时候,她们可能会被许多东西所吸引。

福特汽车公司也是相对较早发现这一类似规律的组织,并将这种方式用于汽车的流水线作业,从而极大提升了效率。也就是说,每个工人只负责一个工位,只负责重复一件相对简单的事情,通过许多工人和工位的流水线组合完成一辆整车的制造,而不是每个人负责一辆车的制造。

这个概念转化到检查清单上来说,相当于不要试图一次性检查出所有错误,而是每一次的检查只检查特定的错误类型。比如,这次检查语法错误中的时态错误,那你就只看时态,直到时态问题对了后,再来看单复数问题(当然,随着能力的提升,可以将几项不同类型的错误合并检查);又如,检查参考文献时,这次就只检查正文中引用的参考文献是否正确,下一次只看参考文献中作者拼写、卷期号和题目输入等是否正确。

同时,也要学会不断寻找甚至自己开发软件,来高效准确地检查出这些错误,比如借助Grammarly等工具。









\end{document}
