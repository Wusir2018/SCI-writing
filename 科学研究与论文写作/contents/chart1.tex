\chapter{科学探索的本质关于人生的意义}

以目前我们的认知而言,人类的生命是有限甚至是非常短暂的。

想象下,如果有一天你躺在病床上,已经走到了生命的尽头,如果这个时候,有机会让你再重新来过,你会选择怎么样的方式度过这一生呢?

相信很多人在某个时刻都会问过自己这样的问题。这个问题的本质关乎人生的意义是什么?或者更直白点,人为什么活着?

每个人对这个问题的回答不尽相同,而对于这个问题的不同回答方式,其实也决定了这个人的生活方式和思维方式:是穷尽一生仅仅追求财富、名誉和社会地位?抑或是不假思索地跟着周围人的节奏走?或者认为人生没有意义,只管纵情于声色犬马之中?当然,这个问题并没有所谓的标准答案。而科学探索本质上也是一种生活方式和思维方式。每个人对科学探索会有不同的理解方式,而对于我而言,科学探索的本质是拓展自我对世界(世界也包括自身)的认知。在拓展过程中,会产生一系列附属产物,例如技术发明、论文、专利、著作、财富、声誉等等,这些附属产物可以更好地帮助我们拓展和提升认知。

打比方的话,科学探索更像是大航海时代中探索新大陆的一个过程。这其中最大的乐趣是发现新大陆本身,而至于发现宝藏、获得商机、累积财富、提高声望和地位、扩张船队等只是航海探索过程中的附属产物,但并不是它最初的目的。

因此,并不是只有在研究生阶段或者在高校、科研院所工作时才会接触科学探索。只要我们以科学的方式进行探索活动,都有可能属于科学探索的范畴。所以,一旦具备科学的思维方式,会影响生活的方方面面。这也是之前说科学探索本质上也是一种生活方式和思维方式的原因。例如,在孩子的教育问题上,不少家长总是抱怨孩子“不听话”,让自己感到很无奈。具备科学思维方式的人,他可能会去思索这是为什么?然后他会先去查看针对这个问题,别人已经做过哪些研究和分析。接着,他就可能会接触到过去未学习过的一些认识,例如关于儿童的大脑发育特点、大脑的神经网络结构、神经连接建立的过程、左脑和右脑发育程度不同等。进而他就会了解到为什么很多时候和儿童用讲道理的方式沟通是“无效”的。在不断探索和学习的过程中,他会对如何更好地和孩子进行沟通方面有更多的认识。但同时,他也会思辨性地思考他人的观点以考察这是否适合自己的实际情况。随着认知的不断深入,他也许会发现已有观点或研究中存在的争议、矛盾、认知误区、空白领域等,进而有可能通过科学的方法,提出和传播自己的观点。这种不断有新发现的过程本身是充满乐趣的。同时,在这一过程中,他有可能会进一步改善亲子关系,拥有幸福充实和彼此相互理解的生活状态,而不是一味地互相争吵、指责和埋怨。

人类大多生来是具有强烈的好奇心和创造欲的,对探索未知事物有天生的渴望。这点从婴幼儿的成长上可以明显看到。比如在沙滩上,有一些水,一滩沙子,孩子们几乎可以不知疲倦地玩上一整天。这是因为水和沙子没有固定的形状,可以随着孩子的创造呈现任意的形状,这其中充满未知性。而正是这样的未知性会驱动孩子乐此不疲地一直玩下去。所以,我们会听到一些科学家说类似这样的话:“我从来没有‘工作’过一天,每一天都是在‘玩’。”

有不少成年人喜欢游戏、追剧、旅游、美食等,也是因为类似的原因。一方面是由于未知事物对我们的吸引。出于好奇心的驱动,我们想看到游戏中的人物升级后是什么样的,我们想看到下一集电视剧中的人物剧情是怎么发展的,我们想去体验我们没去过的地方、没尝过的食物。

同样地,科学探索也是人类受某种未知事物的吸引而进行的探索工作。大千世界,是不是总有一些事物深深地吸引着我们呢?我们为什么深受某些疾病的困扰?让我们变得如此与众不同的意识究竟是什么?假设人类的科学技术不断发展,我们会不会有一天创造出带有意识的人工智能?那既然如此,会不会我们已经经历过这样的时代呢?如何证明我们不是因此而诞生的“产物”呢?阿尔兹海默症(俗称老年痴呆症)让我们的至亲至爱丧失记忆,不认得亲人,变得性情古怪。这也驱使了大量科学家投身于其中,想理解这一疾病的成因,进而为开发药物提供准确指导。人类对宇宙的探索,对生命起源的研究,不断地拓展了人类的认知边界,提升了人类的认知水平。

以人工智能为例,计算机如果要判断一个颜色是不是红色,它可以如何做呢?比如,当我们看一些计算机程序的颜色管理器,会发现每一种颜色有对应的十六进制数(可以换算成二进制)。所以最简单的方式就是用程序语言告诉计算机,哪些范围内的二进制数属于红色。定义完毕后,计算机永远不会出错。当你输入这个范围内的数,计算机一定给你反馈这是红色。但是,哪怕你输入的数比这个范围大了一点点,计算机都会认为这个不是红色。

可是我们看人类呢?人类似乎不是这样认知的。但我们也可以很好地掌握颜色的判断。学龄前儿童就已经可以很好地学会辨识颜色了。我们是怎么教孩子的呢?比如,我们只是给他一个苹果,告诉他这是红色。那么他下次看到夕阳颜色,看到红色颜料,他都会描述出来这是红色。这看似很自然,但其实很不一般。因为如果面对的是计算机,按照之前描述的程序语言的方式,这几种红色背后的数值是不同的,我们相当于只教会了计算机,苹果的颜色对应的二进制数是红色,那么计算机怎么判断和苹果颜色不同的其他二进制数是不是红色呢?生活中类似这样的无法用简单规则让计算机精准学习的东西还有许多,比如语言的学习、表情情绪的理解。而这些都是婴幼儿已经很好地掌握的技能了。

类似这样的问题就引起人类的好奇心了。人类到底是怎么学会这些的?能不能让计算机也像人类那样学习?所以人类学习依赖的神经网络模型终于被人们发现了,并以此创造了所谓的机器学习算法。基于这样的算法,在越来越多的领域,人类已经无法“战胜”计算机了。AlphaGo 的故事大家都知道,而AlphaGo的“弟弟”AlphaGo Zero甚至可以在不学习人类棋谱,通过自己和自己下棋(其实这个的本质就是对神经网络的训练)的方式,下赢已经在围棋上完胜人类的AlphaGo。

当然,拓展人类认知边界或者提升人类的认知水平的形式是多种多样的。以图\ref{fig1-1}为例,圆圈内部代表人类已经探索的领域,圆圈外部代表人类尚未探索的领域。提升人类的认知水平可以是探索你感兴趣的一些未知事物,即圈外的部分。比如,在你做这个研究之前,人类不知道这个事情,而是因为你做了之后,大家才意识到事情的本质。再如,以前人们认为地球是宇宙的中心,后来发现并不是。这个发现对宗教、人类的意识形态等产生了巨大的影响。

\begin{figure}[!htb]
\centering
\includegraphics[width=0.5\textwidth]{fig1-1.png}
\caption{未被探索的领域与已被探索的但存在争议和此后有可能被修正的领域}
\label{fig1-1}
\end{figure}

除此之外,也可能是这个领域已经有人在探索了。但是大家的探索是有分歧的,或者是还未被证实的。如图\ref{fig1-1}中圈内所示标记问号的区域。例如,领域内存在很多种不同类型的观点,这些观点有些可能是互补的,有些可能是矛盾的,有些是还未被验证的假说。但是通过我们的探索,后人有可能知道到底哪一种认知才是更接近真实的。

还有第三类就是这个认知是人类已有的,甚至已经形成共识了,比如被写进教科书的,即图\ref{fig1-1}中圈内标记阴影的部分。但是,通过一些人的科学探索,我们会发现这个认知并不全面,甚至是“错误”的,还存在进一步改写或者提升这种认知的可能。这个在科学史上是非常多见的,比如早期人们认为物质的质量不会发生变化,但随着人们对微观事物(如电子)的高速运动进行探索时,发现质量也是会改变的。所以,以上三类人类认知的拓展,都是相互关联的,在探索未知和有争议的领域的时候,很有可能还会改变我们认为已成定论的领域的认知。

类似地,化学的发展也和人类认知的不断发展有密切关系。这其中涉及的具体事实较难考证,我只能基于一些猜测来展开。举个例子,我觉得化学的萌芽、发展也许和人类追求财富和永生有关,想通过炼金术点石成金,或者炼制出长生不老的仙丹。点石成金按照现在的观点来看,是相当于通过某种过程将石头(主要是碳酸钙)变成金这种元素。随着实践活动和理论的发展,在某个阶段,人们终于意识到化学反应过程中,虽然不同物质之间会发生转化,但是组成物质的基本元素种类不会发生变化。因此,点石成金是不可能的。进而人们形成了元素是不会发生改变的这种观点,就好比元素周期表上的某一个元素不可能变成另外一个元素。但是,随着原子物理学(放射学)的发展,人们发现,元素还是有办法从一种变成另外一种的。这又打破了人们此前认为的元素不会改变的传统观念。

被誉为物理学最美的实验之一——卢瑟福的$\alpha$粒子散射实验的起源也是汤姆生本来想安排学生(卢瑟福)做实验来验证自己提出的原子模型假说,结果卢瑟福的实验反而否定了他老师的假想。他的这一工作极大地推进了后期量子力学的产生和发展,而量子力学的发展又再一次对人类的世界观造成了巨大的改变。

因此,科学探索并不是单纯地去记忆、去调取教科书上的知识,或是去回答试卷上的某道问题,而是有可能改写教科书上的内容,甚至创造新的内容。所以,我从2011年开始第一次给学生讲授“表面分析技术”课程起,就和学生交流到:我上课过程中和你们交流所涉及的技术内容,只是人类在该领域探索和发明创造中非常小的一部分。而且人类为了更好地认知世界,还正在不断探索和开发新的技术。关于这些技术的基本结构、工作原理,等你们工作学习需要的时候,随时可以在教科书、在网上查阅得到。我更想和你们交流的是这些科学家、工程师们到底是如何想到要去做这样一个东西的?他们为什么要这么去做?而不是说只是去把他们已经创造的东西,给详细地记忆和背诵出来。


