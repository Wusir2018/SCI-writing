\chapter{研究}
之前介绍的是我们要研究什么,这里介绍的就是如何研究。本书定位主要是面向基于实验的科学探索研究工作,因此主要围绕实验设计、实验开展、实验调整展开。

\section{实验设计}
初步确定完选题后,就可以通过实验等方法对你感兴趣的领域进行探索了。现代科学发展至今,一般来说,很少有哪个领域的探索工作是和已报道的工作完全不相关的,已有工作多少会有一定的借鉴价值。所以设计实验方案的过程,我同样会采用第二章提到的“搜集信息——归纳——提炼”的方式,以及寻根问底的思维方式。还是以第二章提到的一个实例为例:比如我们想研究锌空气电池中的催化剂领域。基于第二章所说的文献阅读方式,通过集中浏览已有文献的前言和图片框架,会了解到如下问题:

(1)为什么要研究这个问题(研究的意义)?你可能了解到学者们研究这个问题,有的是为了缩小锌空气电池的充电电压和放电电压之间的差距,以提高能量效率(放电输出的能量尽可能可以接近充电输入的能量,以避免能量的浪费)。有的则可能侧重延长催化剂的工作寿命问题,期望获得具有长寿命的锌空气电池。也有同时想解决多个问题的。

(2)明白了第一点后,可以选择一个你感兴趣的切入点。比如针对缩小电池充电和放电电压差,那么我们可以继续追问,充电和放电电压是由什么决定的?同样通过第二章介绍的方式,根据作者文章的前言、研究框架和一些相关教材,可以了解到放电过程和氧还原反应有关,充电过程和氧析出反应有关。此外,也会了解到充放电电压还和电极整体的导电性等因素有关。进一步地,可以追问氧还原、氧析出反应和充放电电压又是什么关系?就会发现当施加某个电压、反应进行时,这两个反应的反应速率和我们采集到的电极上的电流大小(电流大小与单位时间内反应参与和生成的产物量有关)有关。一般而言,这两个反应的反应速率越快,产生同样的电流所需要的电压也就越小,而这就是我们想要达到的目的。所以,问题就进一步聚焦为如何提升反应速率?这个问题还可以进一步分解,以氧还原为例,怎么能让氧还原反应进行而且速率尽可能提高呢?是不是一个是(a)在定的空间(体积)或者质量下,电极材料中参与反应的区域(活性位点)尽可能地多(相当于投入作战的士兵尽可能地多)?另一个是(b)电极材料中每个活性位点上的反应速率尽可能地快,即增强材料的本征催化活性(相当于尽可能增强士兵的单兵作战能力)?继续追问,如何增大活性位点上的反应速率呢?首先是不是要知道这个反应大概是由哪几个步骤进行的?是不是只有知道了大致\footnote{这里有一点需要说明的是,这里加上“大致”两字,是因为一般而言,这些反应的具体步骤是十分复杂的,学术界也还在不断通过先进的实验手段和理论计算等方法进行研究中,而且随体系不同,反应路径可能也会不同,所以它本身也是一个很好的研究课题。初步接触课题者。尤其在未开展实际实验前,要彻底搞明白这个问题应该是有难度的,所以这里用了“大致”二字,以防被误认为这种反应步骤已经是100\%确定,毫无争议的了。也以防由于过分钻牛角尖,导致迟迟不敢动手开展实验,事实上有很多问题,需要动手起来,将实践和深度思考一起结合,会更有效果。即使某处不确定,也不妨碍我们开展实验,我们本身就可以通过自身的实验来让这些不确定的地方变得更清晰。此外,科学研究中没有完全绝对的事情,所以我上面说的逻辑线索,也只是为了举例方便、便于理解,并不代表完全正确。}的反应步骤才知道如何提升速率?一旦有了这步思考,就会知道既然要发生氧还原反应,那首先要有(b1)氧气分子能吸附在电极材料上。吸附之后,在外加电压和溶液作用下,(b2)氧气分子会历经各个过程被还原成$OH^-$,(b3)$OH^-$脱附(离开)电极材料表面进入溶液。电极材料表面还原成初始的状态开始下一批氧气分子的吸附。到达这样的理解深度后,离设计出自己的实验方案已经很接近了。例如,可以针对先前提到的(a)来设计自己的实验方案,譬如采用已知的具有高活性的电极材料(选择单兵作战能力强的士兵),通过电极/气体/溶液界面的整体结构设计,使单位面积/体积或者单位质量内,电极材料表面能尽可能多地接触和吸附上氧气分子(投入尽可能多的士兵),从而有进行后续反应的可能。也可以针对(b)来设计,在已知高活性电极材料的基础上,进一步增强其本征催化活性(进一步提升单兵作战能力)。譬如寻找合适的电极材料能更好地实现(b1)——(b3)步骤。如果要从这个角度切入,可以继续将这个问题进行分解,类似之前交流的方式,了解到(b1)——(b3)步骤背后的基本原理和影响因素是什么。检索和阅读学习相关资料,了解影响因素后,就知道如何进一步提升本征催化活性了,如从改善电极材料的本征导电性入手,从改变电极材料表面电子结构切入等等。接着在已有研究基础上,提炼出自己的研究方案。当然,也可以将以上几个方面同时综合考虑。

所以,整个环节有些类似不断抽丝剥茧的分解和“逆向”的过程,寻找到那个最关键的值得你去研究的问题(选题)。分解问题过程中,同时不断了解已有研究进展到什么程度了。如果已有研究的证据相对确凿,那么我们可以以此为基础继续递进、分解,寻找下一层级的问题,直到找到那个关键的但目前还没被研究清楚的问题,并以此设计出实验方案。如果发现已有报道关于某个领域的研究结论似乎没那么确凿,也值得商榷,那么就可以围绕这个问题进行实验的设计。所以,如图\ref{fig2-1}所示,开展具体研究和选题之间并不是单向的关系,而是相互影响的。随着实验方案的逐步清晰,对于选题的认知也会越来越清晰和聚焦,这个时候会再次涉及第二章中提到的检索和选题工作。可以继续采用第二章介绍的方法,在相关领域进行文献检索,以细化实验方案。比如你设想的方案有可能锌空气电池领域中没有类似工作报道,但在其他领域比如锂空气电池、单纯研究氧还原反应催化剂的领域中已有文献报道了类似的实验思路。那么可以通过检索再次优化知识库,借鉴并进一步细化你的实验方案。具体如何能检索全面,主要依赖于关键词和检索式的合理设计,这在第二章第一节中已有交代,

同样在细化自己实验方案的过程中,浏览和整理文献时,也可以用第二章提到的“图片/文字分块——归纳——重组阅读法”。将你感兴趣的文献集合中涉及实验步骤或者实验方案的内容都整合在一起,方便快速浏览和对比。当把多个实验方案都集中在一起,可以更迅速地发现各个实验方案在实验步骤和实验参数设置之间的相似点和差异点,理解这些异同点,可以更好地帮助你提炼自己的实验方案以及未来实验开展过程中对遇到的问题进行排查。同时,也可以集中将文献中的图片提取汇总。通过浏览这些图片,可以迅速了解和掌握要阐明的相关领域的问题需要涉及哪些实验,例如需要做哪些表征和测试实验等。

\section{实验开展}
有了初步的实验方案后,就可以开展具体实验工作了。实验开展过程中,通常有这么几种情况:1实验反馈结果和方案预期有很大差距,无法获得预期目标。这又可分为三种情况:1.1实验结果重复性良好;1.2实验结果重复性不佳;1.3实验结果重复性规律复杂。比如,有时重复性好,有时重复性不好。

一般而言,1.2和1.3的情况非常常见。之前提到过,我们设计的实验方案往往是由一系列串联的实验步骤所组成的,所以任何一个实验步骤的结果和预期出现偏差,都会导致最终的结果偏差。当出现偏差后,如果按顺序一个一个地检查每个步骤,考虑各种可能性,有时效率会过低且容易偏离主线,忘记当初研究的主旨。读本科期间,我担任过复旦大学勤工助学中心科技开发部经理一职,有了一段相对短暂的编程经历。在编程过程中,经常会产生各种意想不到的代码错误(bug)。那时训练了自己如何从大量的代码中排查错误,寻找代码错误(bug)来源的能力,而这其中的一些解决方案和理念,我发现也特别适合后期的实验过程。结合之前的编程经历以及后期自己在开展实验过程中的经验教训,我为学生们总结提炼了如下方法,以供参考。

\vspace{0.5cm}
{\kaishu 一、“以终为始”指导下的简化法/替换法/排除法/节点测试法}
\vspace{0.5cm}

为了便于理解,我会以类比的方式,展开这里的讨论。譬如当一大段代码输入完毕,最终结果出错的话,不太可能一行行代码地按顺序浏览去寻找出错原因。可以采用一些方法来定位产生错误的原因,比如排除法:将某一个功能模块的代码全部“屏蔽”,只使其输出一个固定参数,然后观察基于这个固定参数、变化其他各种条件的情况下,整体代码的输出是否符合预期?如果符合预期,那么就说明是这个功能模块的代码有问题,极大地缩小了问题范围。如果也不符合预期,那就说明这块代码和其余部分都有可能存在问题。那么可以继续采用排除法,再屏蔽一个新的模块,直到最终的代码可以输出正确结果。此时,就可以缩小问题范围了。在已定位的问题范围中,继续采用排除法不断缩小问题范围,直至找到原因,当然可能会有一个或多个原因。也可以在代码中选取合理的一些节点,将其拆分为多个模块,随后对模块给予相应输入,观察输出是否符合预期(将问题逐步切割简化,进行测试)。这也好比“修电脑”的过程,电脑无法开机后,我们想定位问题出自什么地方,是主板、内存、硬盘还是显卡?或者多者?一般的方法也是比如逐步拆除部件,或者替换上确认没问题的部件,用以缩小问题的范围。

类似地,当实验开展过程中,出现了之前所说的1.2和1.3的问题,也可以仔细考虑如何将复杂的系列实验步骤合理切割成几大块,以快速定位到问题所在。在总体的实验方案中,设置几个节点(关键里程碑事件),根据里程碑事件的达成情况,判断问题是出自哪个环节中。同时,在分解实验步骤时,可以尽可能将一些实验步骤去除或者充分简化或者用一些确认与预期相符的实验步骤替换,从而逐步有计划地缩小排查的问题范围。

\vspace{0.5cm}
{\kaishu 二、极致的分解}
\vspace{0.5cm}

之前介绍的是从整体角度切入,将系列复杂的实验步骤尽可能切割成几大块,以快速定位问题,避免过度拘泥于细节,只见树木不见森林。而这里要交流的就是关于局部,有时候我们认为无可分解的细节或具体事物,其实也可划分,而问题可能就是源自于这些我们认为不会是问题的、已经无法划分的细节或者事物中。在过去的研究中,经常会发现有些问题是一开始意想不到的,或者来自于一些习以为常、容易忽略的地方。这个时候,同样需要本书中反复提及的质疑思维模式,询问每一处真的是完全确凿、毫无置疑的吗?要用自己的实践和思考去验证,而非完全依赖于听来的或者看来的。

例如,在材料的电化学行为研究中,会涉及玻璃碳电极。我们有一种类型的实验会在玻璃碳电极上通过电化学方法沉积目标材料,随后直接在沉积有目标材料的玻璃碳电极上测试一种新体系的电化学信号。做的过程中,发现有时做的电化学结果重复性不错,有时数据重复性不理想,没有明显规律性。我主要就是采用以上两种思维方式来指导学生逐步缩小问题范围,寻找原因。首先,先将问题整体自上而下地分割为两块:①玻璃碳电极上电化学制备目标材料;②沉积有目标材料的玻璃碳电极上测试一种新体系的电化学信号。那么接着我们就要寻找重复性不好的问题是出自于①还是②,还是两者皆有。采用的方法还是基于充分的简化方法,以尽可能减少各种变量对结果的判断。比如针对第一点,我们可以在玻璃碳电极上制备完目标材料后,在最经典的重复性非常好的电化学体系中,测试这个材料的电化学行为,看一致性是否良好。如果一致性良好,就说明第一点没问题。问题范围就局限在第二点了。如果第一点存在问题,那就要继续分解第一点。针对第二点,可以采用商业的或者实验结果高度可重复的对比材料(首先在经典电化学体系中测试对比材料,确保其实验结果重复性良好),测试对比材料在这种新体系中的电化学信号重复性如何。如果重复性良好,那么就说明第一点有问题的概率比较大,需要先考虑第一点是否存在问题。而如果重复性也不好,那就反而发现一个很有趣的现象了,是一个和此前认知不太一样的现象了。不过我们当时并未发生这种情况,因此开始聚焦第一点,将第一点继续分解为1.1玻璃碳(基体)以及1.2基体上电化学制备材料。即:有可能是基体问题,也有可能是基体上制备材料过程中产生的问题。采用替换法,将基体替换为其他基体(比如透明导电玻璃)测试,在电化学制备过程中,严格控制透明导电玻璃和电极间的位置、接触溶液的面积和深度。将制备有目标材料的透明导电玻璃电极在新体系中测试电化学信号(也就是测试之前所说的第二点),发现电化学结果重复性良好。这就说明问题来自于导电基体。那么思考将1.1玻璃碳(基体)问题进一步分解。玻璃碳电极是由如下结构组成的:中间是个玻璃碳材料,四周被聚四氟乙烯包裹,玻璃碳背后引出电极(图\ref{fig3-1})。这里面就涉及我们先前提到的问题,一些我们认为已经细无可分的问题,其实也可继续分解。比如1.1就可继续分级为1.1.1玻璃碳材料本身,1.1.2玻璃碳材料和聚四氟乙烯的结合,1.1.3玻璃碳材料和引出电极的结合。其中1.1.3比较容易排查,测量一下电极和玻璃碳之间的电阻即可,证明连接方面没问题。1.1.1玻璃碳材料本身的问题相对来说更难些,因此先关注 1.1.2,这是个很容易忽略的问题,但如果用这种分解式的思维方式,其实是可以定位到这个问题的。我通过和学生的交流,了解基于玻璃碳电极使用的历史情况,发现学生测试完毕后,为确保彻底将上面的沉积产物溶解,会反复将玻璃碳进行抛光处理。有时觉得没处理干净,还会用稍细的砂纸进行少量打磨。这引起了我的注意,联想到会不会是这种打磨或者抛光处理,影响了玻璃碳材料和聚四氟乙烯的结合(即1.1.2这个问题),导致电化学制备或后续的电化学测试引起的问题呢?例如打磨的时候,有没有可能让两者之间结合部位的狭缝扩大,嵌入了一些杂质颗粒进去,进而干扰了后续的制备和测试过程?然后我就考虑采用一种方法以屏蔽这个问题。在电极抛光处理完毕后,用熔融的石蜡滴加并尽量正好完全覆盖接缝部位(使结合部位不要对后续实验造成影响)。等石蜡冷却后,再进行后续的制备和测试过程。结果发现电化学结果重复性明显好了许多,那么就定位清楚了问题。后来采用电子显微镜等手段对玻璃碳材料和聚四氟乙烯的结合部位进行了观察,会发现随着不断的抛光和偶尔的打磨处理,玻璃碳材料和聚四氟乙烯之间并不是完全紧密良好结合的,确实存在缝隙,而这些缝隙就会暴露出玻璃碳材料的边缘部位,玻璃碳材料的边缘部位同样是导电的,所以也可能沉积生长材料并干扰了后续的电化学测试。

\begin{figure}[!htb]
\centering
\includegraphics[width=0.9\textwidth]{fig3-1.png}
\caption{玻璃碳电极的实物图}
\label{fig3-1}
\end{figure}


另一个例子中,我在早年做实验时,有部分研究内容涉及金属的表面合金化,需要用到惰性气体保护氛围下的热处理炉子。刚开始的实验,也是不如预期。根据先前提到的源头思维方式进行思考:这个实验首先会涉及将金属包埋于粉末中在一定的温度下进行处理,这其中又涉及将金属包埋于粉末中和热处理这两个步骤。其中第一个步骤虽然也有不少影响因素,比如包埋前,金属表面如何处理?金属和粉末接触方式如何?粉末的量又是多少?等等,但这些因素相对可控,可以暂时不用进入那么多细节问题,因为当时也不清楚这些问题对最终结果是否有很大影响,可以看看还有无影响更大的步骤。那么会发现第二步中涉及的温度因素是影响很大的。继续用源头思维方式去追问,我们是如何知道热处理温度的?→根据设备仪表盘的显示。→那么设备仪表盘又是如何获取温度的?→根据热电偶采集的。→热电偶是如何采集的?采集的信息一定准确吗?→随后我就去观察了设备的热电偶,发现是处在如图\ref{fig3-2}所示的这么一个位置。这个时候我就觉得可能会存在一定的问题,因为我们想探测的其实是样品所在区域的温度(加热内胆的内部),但实际测量温度的区域是在另一个位置(加热内胆的外部)。这两个位置可能是存在一定温度差异的,也许这个因素会导致实验结果不符合预期。为快速验证这个问题,我就采购了一个热电偶,将其置于样品加热区域的附近,观察热电偶所采集的温度和设备所显示的温度是否存在差异。测试结果表明,两者确实存在明显差距。之后为彻底解决这个问题,我专门制作了一个可充惰性气体的炉子,将热电偶直接置于加热内胆的内部,从而使样品加热区域接近热电偶采集温度的位置,以获得更接近实际情况的温度。采用新制作的炉子后,这一问题得到解决。

\begin{figure}[!htb]
\centering
\includegraphics[width=0.9\textwidth]{fig3-2.png}
\caption{加热炉的简单示意图(a)调整前;(b)调整后}
\label{fig3-2}
\end{figure}


过去实验过程中,还遇到过大量看似很简单,因此非常容易被忽略的问题。譬如试剂的过期变质问题,我们曾经遇到过一个实验,此后接手的学生总是重复不出以前的实验结果。后来采用排除法,定位到这么一个问题:当时成功重复出这一实验的学生采用的试剂可能已经变质了,实际的有效物质浓度可能会比名义浓度低很多,因此我们将新采购的溶液,配置成浓度含量从初始到低数份不同的溶液,终于在某个低浓度范围条件下,再次重复出了这个实验。又臂如平时一直能稳定重复出的电化学实验,突然某一天做不出了。我当时排查各个步骤后,怀疑是用于配置溶液的超纯水的问题,而学生刚开始不太相信,觉得每天都是用这个水的,大家也都是用这个水做实验,为什么会突然出现问题。我们最早用的纯水机,并没有电阻率的显示,无法显示出水是不是18 兆欧的水质。所以我依据源头思维的方式,就会追问如何证明我们的水中不含杂质,不会对电化学曲线造成影响?之后简单采用替换法,替换了18兆欧的水,这一问题得到了解决。还有比如,一般新学生容易约定俗成地沿用之前几届师兄师姐的一些实验习惯和方式,而有时问题也会隐藏在这些环节中。例如一直沿用下来的参比电极、pH计是不是准确?有没有定期校准过?有时电化学工作站会做一些腐蚀类测试,而腐蚀测试过程中,挥发性的腐蚀物质有可能会腐蚀电极夹,但电极夹通常会被一些塑料套等保护,所以导致虽然被腐蚀了,但是没被察觉到。采用之前提到的极致还原论的思想,都是可以定位到这些问题的。

\vspace{0.5cm}
{\kaishu 三、想法(idea)的快速验证}
\vspace{0.5cm}

除了以上关注细节的思维方式,大刀阔斧的跳跃式思维模式也是经常需要的。在设计实验方案过程中,通常我们一开始并不十分清楚这个方案最终结果会如何。另外,有时我们会针对某些问题,设想出许多不同的想法。遇到此类情况,我们最想知道的是自己设想的想法或者方案,整体方向是否可行?是否有成功可能?如果严格按照当时设想的方案顺序,一步一步地仔细执行,容易陷入大量的细节中,极大推迟我们对于整体方向是否可行的判断,甚至很有可能开展了大量的非决定性的工作,直到最后一步才发现原先的想法根本不可行。

因此,这时就要借助这种对想法进行快速验证的方法,基于“以终为始”的角度去思考:这个方案的“终点”是什么?要达成这个终点,一步步地去倒推,最重要的不可或缺的关键节点(关键里程碑事件)究竟是什么?哪些步骤我们是可以大幅度地跳跃或者省略或者代替的?最后仅保留下来最少的但又对验证想法必不可少的关键步骤。采用这种方式,可以快速识别出一些当时考虑不周的方案设计,并为以后的方案改进、新想法的寻找预留更多的时间。例如,做实验时,可以优先做一些极端条件下的测试快速验证当时自己的想法是否可行;如果有的想法可以先借助商业(或成熟的)材料快速验证的话,那么可以先采用商业材料进行测试。如果发现当时想法可行,那么再来自已合成材料,并改进和优化,达到理想预期。如果发现想法不可行,那么则可以用之前介绍到的一些方法,快速排查,看是哪里考虑不全面或是出了问题。总之,先关注从0到1这个问题(是否可行),随后再来关注从1到 100这个问题(不断优化方案)。



\section{实验习惯}

\vspace{0.5cm}
{\kaishu 实验步骤/样品编号/样品保存习惯}
\vspace{0.5cm}

这里有一个基本原则是尽量让我们所做的每一件事情,未来都可被追溯。

一般我在开展实验的时候,会将实验本划分为左右两个区域。左边区域习惯于按照时间先后,依次记录每一个实验步骤,并尽可能详细充分记录当时的实验条件。右边区域则记录一些根据当时实验情况提炼的一些信息或者备忘的一些内容。详细记录实验步骤的原因一方面是用于后期写作时,阐明当时的实验条件;另一方面是用于分析实验的问题,并启发后期的实验思路。也就是说,实验记录本并不仅仅是用于记录实验结果。我们曾经遇到一个实验,学生本来实验结果重复性良好,但突然就无法重复了。后来我翻看学生记录的实验本,发现当时实验结果重复性良好的时候是在暑期和相对室温高的时候,而实验结果不佳的时间是在寒假天冷的期间。因此怀疑是否是温度问题。为排查是否是这一问题,不仅通过空调提高室内温度,还采用水浴锅等设备使实验用的所有水包括清洗用水都维持20多摄氏度的恒温,之后就解决了这一问题。另一个案例中,学生刚接触课题时,制备出了一种形貌和性能相对不错的材料,但此后想大规模重复该实验,并研究影响因素和机制时,反而再也做不出来刚开始的效果了。也是通过翻阅实验记录本,询问学生实验时涉及的所有条件参数,发现学生在实验过程中,替换过对电极,而这是导致后期无法重复出初期的关键原因。甚至有一些影响因素众多的实验,我会指导学生综合采用视频+录音的方式保存实验过程,便于后期寻找问题,有点类似行车记录仪的功能。

除了良好的保存实验过程的习惯,实验过程中产生的所有样品,包括失败的样品、半成品、原始样品、中间步骤产生的样品等等都可以保留下来。保留的时候注意给每一个保留的样品编号,同时将这个编号和先前提到的实验记录本关联起来,这样可以索引到每一个编号经历过怎么样的实验步骤。这样做对于寻找实验过程中产生的问题,以及启发新思路都有非常大的帮助。曾经有一位学生的课题涉及采用双通阳极氧化铝模板制备目标产物,刚开始实验也是发现重复性不好,有时能做出,有时做不出。后来我们将保留的一些原始氧化铝模板在扫描电子显微镜下进行了表征,并和之前制备成功和制备失败的产物进行了对比,发现采购的氧化铝模板的正面和背面并不是完全一致的,正面光滑平整,背面粗糙。而这个实验,需要粗糙的背面接触集电极才可以获得比较好的结果。之前的实验未考虑到这一点,导致实验重复性不佳。改进后,实验重复性得到明显改善

这里需要注意的是,虽然以上我们给出的案例都是实验结果不符合预期的情况,但是我们要意识到实验不符合预期并不意味着就是失败,就完全没有可思考的地方。如之前所说,一旦开始实验后,其实并没有所谓的失败实验,只要做了就是一种收获。知道这条路不可行,也是一种“成功”。实验结果不符合原先的预期,但实验结果是可重复的或者有一定规律的,往往蕴含着更重大的甚至会颠覆已有认知的发现。比如,我们在第一章说的卢瑟福提出核式原子结构模型的过程。此外,探索失败的过程中,往往会伴随着大量的“意外”事件,这些“意外”事件会使我们跳出一些常规的被理性思考过度约束的限制,出现一些很新颖独特的尝试,而这也会将我们带往一些更新颖的发现。


