\chapter{总结}


\section{研究性论文的基本写作思路和方法}

\vspace{0.5cm}
{\kaishu 一、依托事实/证据(fact/evidence),通过合理的逻辑(1ogic),阐明自己的观点(opinion)和贡献(contribution)}
\vspace{0.5cm}

当我们获得了完善的实验结果,理解了一些现象,得到了一些结论,发现会对这个领域有所贡献后,我们会期望将我们的发现和全世界分享,以供同行和后人参考,推动大家在这个领域取得更多认知上的成长。而以上想法,部分就是通过发表学术论文、专著等来体现的,这也是对我们工作的阶段性总结。

已经有不少优秀的相关书籍,是从学术论文的结构出发,依次讲解前言、实验、结果和讨论、结论等部分如何撰写。因此,本书不再从这个角度出发阐述如何撰写文献,主要是结合过去指导学生的经验教训,从论文写作背后的思维方式角度出发,并给出具体的可以独立完成写作的写作方法和工具。

大部分学术工作,我觉得可以概括为两个为什么:①你为什么要做?即这个研究的意义是什么?是有一些新的发现?澄清了过往的一些争议?还是阐明了领域中的一些空白等等(如图1所述)。②为什么要你做?即已有研究是否涉及过相关的内容了?如果做过,已有研究做到什么程度了呢?你在已有研究的基础上又做了哪些?

论文的贡献可以是填补一个空白(过去没人研究过,但相关研究又是十分有必要的);也可以是解决一个分歧(很多人研究过,但观点不一致,甚至观点互相矛盾):也可以是澄清一种常识性的误区(这种也是非常划时代的工作,代表性的就类似改写教科书的工作,比如人们之前普遍认为塑料是不会导电的,但通过诸多科学家的努力,目前已制备出导电聚合物,使“导电塑料”从不可能变为可能)。总之,科学的本质是提升人类认知的边界,提升人类的认知水平。回到我们的论文,读者通过阅读我们的论文,到底要得到怎么样的收获?而这些主要就是论文的前言部分所要传达的信息。

\vspace{0.5cm}
{\kaishu 二、分清观点(opinion)、事实(fact),给出合理推演它们的逻辑(logic)}
\vspace{0.5cm}

关于研究论文的整体写作,我个人理解是依据实验结果和/或计算等(事实),通过合理的逻辑表达,阐明自己的观点和贡献。在第二章第三节中的选题部分,我重点讨论了关于事实和观点的区分,这同样适用于写作。当我们要阐明我们的观点时,时刻要注意我们表达的这句话、这个观点是否有证据支撑?获取证据的研究手段是否合理?从证据推导到观点的过程是否严谨?这里的证据更多指的是我们的实验结果或者计算结果等。当然有时候,我们想表达某个观点,但还没有或者无法获得相关的结果支撑,或者这样的结果已经被已有文献广泛报道了,我们也可以通过合理充分引用已有文献,并结合合理的推导,来给出我们的观点。但总之,任何一个观点,一般而言,都不是凭空产生的,都需要有这么一个产生的来源,这也是便于读者在我们的基础上继续深入研究。

下面会通过几个实例来直观说明这个问题。

案例1:以下是学生写的前言初稿中的部分内容:Therefore,it is necessary to develop novel electrolyte materials with high water retention capability, excellent carbonation resistance, and stable electrochemical properties for zinc electrodes. Interestingly, the implement of near-neutral gel polymer electrolytes (neutral GPEs) has the advantages of good carbonation resistance, minimized the dendrite growth andsuppressed Zn anode corrosion. Furthermore, neutral GPEs are environmentally friendly and can avoid chemical burning of the skin or fabrics caused by the leakage of highly alkaline electrolytes. Therefore, it is expected to improve the battery life of ZABs including their cycling and storage life by utilizing the neutral GPE. However, to the best of our knowledge, the study of flexible ZABs based on near-neutral electrolyteshas not been reported thus far. Moreover, few efforts have focused on the storage life of ZABs.

这个段落之前是介绍了电解质领域的已有研究现状,随后阐明存在的问题,进而引出需要发展一个什么样类型的电解质。从Interestingly开始,就存在一定问题了,学生大致意思是“近中性电解质具有很多优势包括抑制电解质的碳酸化、最小化锌电极的枝晶生长和抑制腐蚀,可以预期解决之前提到的领域问题,但是无人报道过,所以我们这里要进行这个研究”。关于提到的近中性电解质的优势这里也没有引用过相关的文献,那么所谓的这些优势的依据或证据是什么?还是说这些优势是业内众所周知的常识?如果有证据,那这个证据是已有文献报道过的,还是学生本篇文章要报道的?如果证据是已有文献报道过的,那就说明已有文献其实已经针对这个领域的问题做了相关工作,并研究出了一种具有以上优势的电解质,那么既然如此,本文的贡献在什么地方?而且和后文所说的无人研究过的表述也存在矛盾。如果证据是本文作者所做的,那么相关证据发表了吗?如果已发表,那同样会出现本文贡献是什么的问题。是不是在以上三个优势的基础上,又有了新的工作?但学生并没给出这方面的表述。而如果学生认为这些优势是常识,并不需要证据支撑,那么本文之后同样没有阐述清楚研究意义。并不是说常识就没有研究意义。比如有的研究发现过去的常识性认知是错误的,通过研究发现了一些有别于过去常识性的认知,推翻了过去的一些定论,那这同样是很大的贡献。实际上,在学生所研究的体系中,这种电解质并未被充分研究过,因此以上三种优势在这种情况下是否能完全体现是不确定的,而这正是学生通过系列实验,在本文中所要阐述的内容。所以还未讨论研究结果之前,就给出这样的观点是不合适的。

又比如像下面这个例子(案例2),学生原来用词是strong(强),但SEM(扫描电子显微镜)表征是无法判断出产物和基体是否具有strong的结合,SEM最多只能判断产物和基体之间是不是紧密接触着的。当然这里所谓的紧密接触也是受到 SEM 的观察分辨率限制的。

案例2:In addition, from the cross-section SEM image, $Co_3O_4$ nanosheet arrays show the strong adhesion between the $CC$ and $Co_3O_4$ nanosheets (Fig. Slc), which facilitates electron transport and is expected to exhibit excellent flexibility and high strength, especially under frequent use or large deformation conditions.

下面这个案例3中,论文之前写的是SEM对不同沉积时间制备的产物的形貌表征,接着学生的原文就根据这些形貌表征撰写了如下标记为斜体式样的句子。这些部分属于学生的观点。那么这个观点,在这个环节有证据支撑吗?或者是否是能直接从基本常识推理出的?从前文来看,显然都不是。只是根据形貌上的表征,为什么就能得出“显然沉积时间太短,得到的产物量少,无法获得足够的活性位点”等这些结论呢?这些是要后续做了电催化实验之后才能验证的,而且沉积时间短,得到的产物量少,不一定就意味着活性位点少。例如,短时间沉积如果能获得高度分散但又具有高比面积的催化剂,同样有可能在产物量少的情况下,获得多的活性位点。接着,这个段落还存在类似的问题,也是同样的逻辑不严密、证据不够充分的问题。如后文斜体部分描述的:为什么“transparent(透明)…thin film(薄膜)”就可以得出本句后半部分“具有大的活性位点数量”这一推论呢?

案例3:Obviously, small amount of NiFe hydroxides formed at the short electrodeposition time of 250 s leads to insufficient active sites of NiFe hydroxides. Conversely, the NiFe hydroxides layer grown at prolonged electrodeposition time of 1000s result in too thick NiFe hydroxides films which can inhibit the charge transfer between NiFe hydroxides and NMCTW. These will deteriorate the catalytic property of the as-obtained electrodes. The transparent NiFe hydroxide thin films electrode electrodeposited for 500s may have a larger number of active sites and the good conductivity, resulting in the high catalytic performance of as-obtained electrode, and which have also been reported in our previous studies.

第4个案例,学生表述中给出了三个数据,然后就得出这个电极具有优异的催化性能这一结论(本句还存在语法错误)。这三个数据此前没有任何对比,例如没有和学生实验的其他工艺对比,也没有和已有报道的数据对比,也没有背景方面的交代(比如业界认为大约什么样的数值算是比较优异的)。因此是无法得到这个催化剂性能是否优异的结论的。

案例4: The assembled battery exhibits the discharge and charge potential of approximately 0.99V and 2.05V at 1mA $cm^{-2}$ respectively, and its potential gap between charge and discharge was about 1.06V, implying the NiFe hydroxide thin film excellent electrocatalytic activity.

第5个案例中,学生斜体部分的表述提到“电镜上的亮点是KOH颗粒,KOH颗粒均匀分布在基体中,而且和聚合物基体之间没有缝隙,说明这个凝胶是很好的交联了”。但这其中的推理(因果)关系,我觉得是无法成立的。我的思考过程如下:实际的凝胶是带水的,在进行扫描电子显微镜表征之前,其中的水是会被去除掉的,而在去除水的过程中,KOH(氢氧化钾)析出。首先,KOH颗粒的析出不太可能是“evenly”(均匀)的。其次,KOH析出时候和聚合物基体之间有没有缝隙,还和水是怎么去除的过程相关,并不直接表明凝胶是否具有很好的交联程度。至于交联程度如何,应该需要进一步借助其他分析手段判断。另外,KOH颗粒是固体,而基体是一个多孔疏松的聚合物骨架(想象下其中的水已经被去除)。这两个体系之间有可能是会存在接触缝隙的,只不过扫描电子显微镜无法观察到,所以当用“no gaps”(没有缝隙)之类的表达时,要谨慎。

案例5: The elemental composition of the as-prepared dried PVAKOH GPE is identified by EDX analysis (Figure 3B), which is composedof C, O and K elements. Furthermore, as shown in the elemental mapping images (Figure 3C-F), C, O and K are observed clearly and disperses evenly. Moreover, the discerable bright spots in the FESEM image are KOH particles that are evenly dispersed in the PVA host with no gaps or cracks between the particles and polymer host, indicating the excellent cross-linking in the PVA-KOH GPE.

有时指导学生过程中,发现学生会走向另一个极端,就是生怕自己的观点不够有证据支撑,逻辑不够严谨,干脆尽可能不谈观点,全部都是实验结果的简单描述,或者讨论得不够深入。这当然也是不太理想的做法,我们为什么要做这些实验结果呢?根本目的是为了支撑我们的一系列分论点,而这些分论点又共同支撑起全文的总论点。所以,如果没有基于对实验/计算结果等的充分且深入的讨论,就无法达到这一目的。如何在有证据、逻辑严谨的情况下展开论文的讨论,后文还会给出。

总之,研究结果就相当于一个一个起点,每一个起点通过桥梁(桥梁相当于逻辑表达)的连接会通往一个一个观点,而通过多个分观点的“串联”和/或“并联”,最后我们会获得一个总的观点,也就是全文主旨(一般和全文标题相关)。连接研究结果和观点之间的桥梁一定要可靠存在,而且每一个研究结果和观点都要和文章主旨密切相关(图\ref{fig4-1})。

\begin{figure}[!htb]
\centering
\includegraphics[width=0.9\textwidth]{fig4-1.png}
\caption{通过逻辑表达将研究结果和观点之间建立联系,并最终支撑全文主旨}
\label{fig4-1}
\end{figure}


\vspace{0.5cm}
{\kaishu 三、从知识库到素材库}
\vspace{0.5cm}

每个人都有各自的写作风格,这里给出的也仅仅是我的写作习惯,并不代表是最佳的。对于基于实验结果的论文写作,我一般采用如下的步骤。

(1)在明确论文贡献的基础上,形成初步的论文框架。论文贡献方面就类似图1所讲的,可以是开拓出一个新的体系、阐明一个有争议的问题、解释清楚一个现象背后的原理等。那么针对我们的贡献,我们可以思考下,要阐明这样的贡献,应该以怎么样的形式给出论文的框架,论文的每一部分、每一个图表、每一句话,都应该和我们的贡献有关。

我个人倾向于将准备发表的实验结果依照一定的逻辑顺序进行组织,以此来形成论文框架。其实在开展实验过程中,已经在进行这个工作了。在实验过程中,根据实验结果的反馈不断调整整体实验结果的呈现框架。

刚开始进行论文写作的时候,可能会不太清楚,大概要以怎么样的形式来组织结果的呈现方式(包括结果之间的内在逻辑)。此时就可以利用第二章第二节中所说的方法,先充分搜集和你研究相关的文献,随后汇总这些文献的图片结果。通过集中性地浏览文献图片,就会清楚相关结果应该整理成什么样的呈现顺序,采用什么样的形式表现,以便于读者的理解。比如浏览了大量的相关文献,你会大概有个概念,依次需要哪些实验结果(图和表)表达清楚你的贡献。比如刚开始可能是材料合成制备示意图,之后是合成制备的材料的各类结构/成分表征等,随后是基于制备材料所获得的相关性能,还可能有相关的计算模拟以建立表征和性能间的内在联系,理解现象背后的机制,从而更好地指导未来研究。进一步地,通过集中浏览表征类的图片结果,你就会知道针对你的体系可以做哪些表征类的实验。类似地,你可以了解到需要做哪些性能测试方面的实验,进而将表征结果和性能关联起来,并找出背后的影响机制等。

汇总和浏览同类图片是希望尽可能避免以偏概全。同时,同类图片的集中性浏览,对于你后面的作图工作也有很大帮助,你会清楚自己的实验结果应该如何作图,如何能达到美观且清晰传达自己意图的目的。此外,我们在写作过程中,还经常会涉及和已有文献的结果对比等,所以事先做好信息搜集,后期要做对比之类的工作,就很容易找到相关文献。这里要避免的一个问题是不要仅仅是因为已有文献做了这些实验,那么你也做这些实验。始终要考虑这个实验和你的主题是什么关系,和其他实验结果又是什么关系,是否构成互为支撑的关系,以最终支撑你的主题。

(2)经过了第(1)点后,基本明确了实验结果的呈现方式,即这篇论文依次由哪些实验结果(图和表)构成,那么就可以进行文字部分的写作了。我写作的过程有些类似蜜蜂采蜜的过程:在不同的花中采集蜜汁,随后在体内对蜜汁进行转化和浓缩直至形成蜂蜜。采集花源中的蜜汁相当于步骤(2.1):搜集写作素材的过程,这个搜集过程尽量要来自于广泛的有代表性的论文,以防出现以偏概全的问题,或者后期写作过程中,由于参考素材过少,讨论不够充分和深入的问题。如同蜜蜂选择花源的过程,蜜蜂会挑选盛放期的花,而且会采集许多花;转化和浓缩花蜜蜜汁相当于步骤(2.2):提炼和归纳写作的素材;形成蜂蜜即对应步骤(2.3):根据(2.2)提炼的素材,形成依托于自已实验结果的研究论文。这也如同杜甫所说的“读书破万卷,下笔如有神”。

以上提到的步骤(2.1)和(2.2),在此前的章节已经交流到了大部分的实施步骤。比如步骤(2.1)搜集素材的过程就和第二章节选题中提到的建立知识库是直接相关的。在选题中提到了建立、完善和更新知识库,这个知识库不仅用于指导此前我们所说的选题和研究,同样可用于指导我们的写作,以及未来硕士博士学位论文的写作。当进行写作时,可以在这个知识库中建立新的分组(group),选入写作引言或者写作实验结果时可以参考的文献。如何快速选出相关文献,我在第二章中交流了利用知识管理工具的二次搜索功能,同时结合标题/摘要的快速浏览。

所以,这个知识库的功能是贯穿整个科研生涯的,一库多用,分别适合于选题、研究和论文写作阶段。而并非选题、研究和写作时分别单独新建一次知识库,那样的话,会出现不同知识库中文献大量重复的问题,会极大降低工作效率。

又如步骤(2.2)中提到的提炼和归纳写作素材和第二章节中提到的“图片/文字切块-归纳-重组阅读法”直接相关。当选出相关文献后,根据写作需求,分别从写作引言可参考的文献,以及写作实验/计算结果可参考的文献中提炼相应的内容,并进行归纳。提炼归纳的方法同样在“图片/文字切块-归纳-重组阅读法”中已有交代,

实验性论文一般由标题、摘要、关键词、前言、实验步骤、结果和讨论、参考文献等部分组成。在完成第(1)点,即形成论文框架后,应该对论文的贡献和主旨比较明确了。此时我会倾向先进行前言的写作。先写前言的原因主要是基于如下考虑:前言写作过程中,已会对全文主旨的认知更加深刻,那么后期在写结果和讨论部分时,更有指向性,让写作的每一句话都指向全文的主旨。

关于写作,我常以“树木”模型举例向学生讲解,全文主旨就像树木的根系,写作时,可以先不直接进入细节(比如考虑每一句话具体应该如何写),而是先关注写作提纲,明确写作的这个段落依次需要讨论哪些要点,这些要点互相之间有什么关系,就类似先形成树木的主干树枝等。当然这些讨论的要点是要为全文的主旨服务的,就相当于树木的树干是源自于根系的。清楚提纲后,再来考虑每一句具体的话应该如何表达才能清晰准确地传达自己的意思,就类似形成树木的树叶的过程。而不是一开始就只关注树叶(具体的每句话),忽略了句子和句子之间内在的联系(整体间的内在逻辑)。

借助步骤(2.1)中提到的方式,我们会获得一个可供前言写作的参考文献集合(知识库)。借助步骤(2.2)中提到的方式,我们从该知识库中提取出全面且准确的已有文献的前言(素材库)。通过集中性地浏览前言,并对前言进行归纳分块重组等(素材库的更新和完善),我们会了解到前言的撰写脉络,会了解前言一般会按照这样的脉络展开:背景介绍→引出你所研究的领域→阐明这个领域所要解决或探索的问题→针对这样的问题,已有工作解决或探索到什么程度?取得哪些进展?→还存在哪些不足或空白或不清楚的问题?等等,而这些问题又是十分有研究必要的→最终引出你的具体研究内容。同时,通过阅读同类相关文献,可以了解到前言中的每一个讨论点一般配置多少篇幅。

了解了以什么样的脉络展开后,接着就是每句话具体如何表达了。素材库中我们集合了大量相关文献的前言,并进行了相关归纳。归纳重组素材库的思路在本书中已有详细介绍。当时介绍的主要是图片类的素材归纳,其实文字类的也是一样的原理。例如,快速浏览前言时可以将不同文献中涉及背景介绍的段落或文字归纳在一起,将描述领域问题的段落归纳在一起等等。具体是细分到什么程度,可根据每个人自身的情况。总之通过这种方式,你会搜集到大量的关于同一个意思可以如何表达的语句,而针对同类的意思表达浏览了足够多的素材后,你就可以形成自己的表达语句。这就类似“读书破万卷,下笔如有神”的过程。只不过由于文献数量过多,我将读文献的过程做了一定优化,并不只是按照顺序浏览并做相应笔记的读书方式,而是浏览过程中,就将文献进行归纳(切块)并且重组,这样可以有机会集中阅读和参考同类型的文字素材,以便于记忆和形成自己的表达。当然需要注意的是,如果涉及借鉴已有文献的,一定要充分且合理地引用。

前育完毕后,可以进入实验和结果部分的写作。同样也是有这么一个搜集素材、归纳提炼,最终形成自己文字的过程。经过第(1)点描述的内容后,我们会基本明白依次需要描述和讨论哪些实验结果,那么我们就可以针对这个实验结果进行写作素材的搜集和归纳提炼,以及最终的写作。例如,当前的这个实验结果是涉及表征类的扫描电子显微镜表征,那么你可以把知识库中相关的又具有代表性的文献选出,搜集关于扫描电子显微镜表征描述的段落。等搜集足够多之后你就会知道应该如何描述和讨论基于扫描电子显微镜表征所获得的结果。例如你可能会了解到可以从样品形貌、尺寸、分散程度、均匀性等方面(这几个方面就是我们之前所说的树干树枝)进行展开描述,以及如何描述。同时,由于有大量的文献积累,加上自己的思考,你还可以讨论为什么会形成这样的形貌?和工艺之间是什么关系?从而给出形貌形成背后的机制,以指导后续的科研工作。还可以将这些描述和文中其他实验结果关联起来,以进行深入讨论。还可以与已报道的结果进行对比。在描述和讨论过程中,会涉及已有文献的参考,因此要注意充分且合理地引用文献。以上对于结果的描述和讨论,涵盖了what(是什么——涉及描述),why(为什么是这样的——涉及深入讨论),so what(和全文主旨是什么关系,和全文其他结果是什么关系——涉及深入讨论)。可以采用类似的方法,描述和讨论其他相关结果。同样地,写作的时候,可以尝试我们先前所说的“树木”模型。可以先考虑要描述和讨论清楚这个实验结果,应该依次需要讨论到哪些要点,这些要点如何服务全文主旨,最后再考虑如何用合适的语句将这些要点表达出来。

和知识库类似,这个素材库也是多用的,也是贯穿整个科研生涯的。

\vspace{0.5cm}
{\kaishu 四、通过“横向”和“纵向”的思维方式扩展讨论(discussion)}
\vspace{0.5cm}

以下结合一个具体的指导学生论文过程中的案例做一说明,这位学生的工作是采用电化学方法制备电催化剂。“横向”可以简单理解为和文献或其他标准体系的对比,通过对比扩展和深入讨论,并理解背后的机制。例如,电沉积制备的催化剂,和其他文献报道的相关的催化剂或者商业催化剂相比如何?如果性能上有较大差异(如性能提升了),那么是为什么?通过上述提到的文献阅读和分析方式,我们会思考出一些合理的解释。例如,已有文献也许会讨论本文所研究的催化剂性能提升的缘由,那么我们可以结合思考这个原因适不适合我们研究的体系。如果适合,我们是否在这个因素的基础上,继续强化了这个因素,导致性能提升。还是说出现了新的因素,导致性能提升。那么就可以针对性地去讨论。

同时,基于“源头思维”的方式,理解我们要讨论的这个事物的“本质”到底是什么?不断追根问底。例如要讨论催化活性,那么催化活性到底是什么呢?催化活性取决于两个因素的乘积:一个因素是单位质量的催化剂可贡献的有效活性表面积,就是说比如有一克催化剂,到底能有多少平方米的活性表面积。这个因素和催化剂利用率有关(多少催化剂可以有效接触电解质,并传输电子和离子),也和催化剂粒径尺寸有关。另一个因素是单位活性表面积的催化活性是多少。这个一般是催化剂的本征性能。

通过阅读归纳已有文献中的讨论,以及结合“源头思维”方式,我们就可以进一步细化,给出讨论要点了。比如可以是电化学处理导致催化剂颗粒尺寸细化,提升了催化剂的表面积导致了性能提升。如果是这个原因,我们可以对比所获得的催化剂颗粒尺寸是否相对于可比工艺获得的颗粒尺寸更小?这样就可以不断推动讨论的思考了。类似地,原因还可以是处理后,在催化剂表面引入了一些高活性的结构,使得虽然活性表面积没有很大提升,但是单位活性表面积贡献的活性更大了。或者是采用电化学制备的情况下,获得的催化剂表面特别干净,没有额外残留的吸附物质等封闭了活性表面,进而提升了性能。又或者是电化学制备过程中,催化剂颗粒生长在活性区域,不会有一些催化剂出现在一些电子无法导通的区域。以上基本属于我先前阐述的文献已报道因素的强化,即我们并没有发现一些新的现象或机制,还是在现有的机制框架下进行讨论。但有时在实验过程中,还会发现一些新的已有文献未报道过的现象,例如我们曾经在文章中提出催化剂的形状所引入的尖端电场效应对于催化活性提升的设想。

“纵向”就是说你可以把这处的讨论结合其他部分的实验结果,例如性能和成分结构表征结果等联系起来。

一般而言,不知道如何撰写前言,或者不知道如何描述某个结果,或者讨论不够全面深入,都是由于相关素材搜集和浏览整理得不够全面充分导致的。这其中的原因,一方面是因为不知道如何将相关素材搜集齐全,另一方面可能是信息太多了,比如大量的涉及扫描电子显微镜的结果图,不知道如何选取。这其中都涉及了快速归纳知识库和精准搜索方面的技能。一部分已经在第二章节中的选题讲解过了,包括同义关键词库的构造,设计合适的检索式在各种搜索引擎中进行检索。检索完毕后,可以采用合适的知识管理工具进行二次检索和归纳,从中选出符合我们需要的结果。这同样在第二章节中阐述过,利用文献管理工具(如 Endnote)对文献进行二次搜索、归纳、分组等,以将同类文献归集到一起,从而定位到和你要搜集的素材相关的那些文献,避免涉及的文献过多或者过少导致重要文献遗漏。这种我概括为“自上而下”的知识搜索技能。

另一个重要的精准搜索方面的技能是学会使用支持图片搜索和/或全文搜索的引擎工具。这种我将其概括为“自下而上”的知识搜索技能。以图片搜索举例,比如我们想知道锌空气电池中发生的反应,可以采用如图\ref{fig4-2}所示的搜索方式,首先输入zinc air battery reaction(锌空气电池反应),搜索框会智能联想出一些检索词组,比如zinc air battery electrochemical reaction(锌空气电池电化学反应)似乎更精准,那么我们可以点取这个进行搜索。

\begin{figure}[!htb]
\centering
\includegraphics[width=0.9\textwidth]{fig4-2.png}
\caption{使用支持图片搜索和/或全文搜索的引擎工具进行快速精准搜索}
\label{fig4-2}
\end{figure}


搜索后,可以点击Images,那么就会弹出相关的图片结果了(图\ref{fig4-3})。然后我们可以看到检索记录中,和我们的检索需求非常相关。很多图片给出了详细的锌空气电池中的电化学反应,那么再根据图片,我们就能找到相应的文献了。从这些文献中,我们可以搜集到关于描写和讨论电池电化学反应的段落。当集合了大量的来自不同文献的描述和讨论,最后自己也会知道应该如何描写。同样需要注意的是合理规范的引用问题。类似地,也可以采用支持全文搜索的检索引擎,在已发表文献的全文中搜索你所感兴趣的关键字,以定位高相关度的文献

\begin{figure}[!htb]
\centering
\includegraphics[width=0.9\textwidth]{fig4-3.png}
\caption{图片搜索示例}
\label{fig4-3}
\end{figure}


综合采用以上“自上而下”和“自下而上”的方式,最终可以达到全面且精准的知识检索,用以构造相关的素材库。

\section{观点/综述(perspective/review)类型论文的写作}
观点/综述类型论文的写作也是类似的思路,也涉及完善知识库/素材库、确定框架,并结合素材库将框架扩充丰满,形成最终论文。

下面围绕指导学生具体写作过程中的实际案例来进行阐述,

第一个案例:关于观点类型的论文写作过程。

第一步,确定检索词库。根据我们想写的观点论文的主题、领域等,确定检索词,包括检索词的同义词、相近词等。

例如,我们想写关于万物互联时代对于储能器件和系统的需求,那么我们需要确定万物互联的同义词,比如中文可以称为万物互联,也可以叫物联网等,相应地英文表达的同义词包括Internet of Everything、loE、internet of thing、IOT等。另外,确定关于储能类的同义词,这个时候注意要发散思考,不要局限在我们认为“正确”的词汇上,要换位思考,全面考虑发表论文的作者,如果要表达这个含义,他们的文章标题和摘要会出现什么字眼,比如不能只是单纯地按照中文字面意思仅仅把关键字确定为energy storage(储能),因为很有可能不少作者采用batter*(*是一种通配符,表示同时支持battery,batteries等检索词搜索),power等词汇表达相似意思。

除此之外,还有种关键词、同义词来源是这样的。就是导师会转给学生一些核心文献或者自己平时会浏览到的相关文献,那么可以分析一下相关文献的标题和摘要,看文献所涉及的关键词是否被包含进构造的检索词库中了,如果没有,分析是为什么,举一反三,把遗漏的检索词都放进检索词库中。

第二步,构造检索式。根据第一步获得的同义词组合,利用Scopus 获得合适的文献集。

(1)先利用 Scopus的主题搜索(即关键词同时包含于论文标题、摘要和关键字),构造检索式并测试检索式的搜索情况。如果搜索结果过多,或者相关性太弱,那么进一步调整。第一次的检索式可以构造得尽量全面,后期可以逐步缩小。

将第一组关于万物互联的同义词以or(或)的形式关联:“Internet of Everything” or loE or “internet of thing”or IOT。第二组关于储能器件的同义词同样以or的形式关联:energy or batter* or power。然后这两组词互相以 And(和)形式关联(因为我们关注的文献需要同时包含这两组关键词),采用Scopus 进行搜索。

(2)检索式的调整和最终确定。如果检索结果相关性太弱(很多文章和我们拟讨论的主题无关),那么分析原因,继续调整检索词和检索式。根据检索结果的反馈情况,也可以考虑是否将检索词或者部分检索词限定在标题,以减少不相关的文献。

第一次检索出来的结果有1.8万多。如果觉得检索结果数量过多,可以采用如下方式进一步精简检索结果,留下更相关的文献。考虑到我们要写的观点论文的关键点是万物互联,所以我们会期望要讨论的这批文献最好是和万物互联关系比较大的,也就是说万物互联这组同义检索词最好是出现在标题中的。随后我们将万物互联关键词限定在标题中,再次搜索,检索结果就大幅缩小到7000多了。接着浏览检索结果的相关性如何,是不是和我们拟讨论的主题相关。如果还想进一步缩小检索结果的话,还可以把两组同义词都限定在标题,这样的话检索结果会进一步缩小到2000多。然后会更清晰地发现检索词中包含的energy和power,会引入一些比如关于低功耗的系统设计,或者以某种技术 power LOT,但这种技术和我们本文所要的储能无关,如图\ref{fig4-4}所示。

\begin{figure}[!htb]
\centering
\includegraphics[width=0.9\textwidth]{fig4-4.png}
\caption{部分检索结果分析}
\label{fig4-4}
\end{figure}



注意检索结果中的编号15和22,编号2中涉及的energy balancing… clustering method 和我们拟讨论的主题无关,但是编号 15中涉及的energy harvesting是相关的(万物互联时代用的储能系统也许还期望包括能源收集功能),所以这种情况针对energy这个词很难进一步调整了。考虑到2000结果也不算多,可以都导人Endnote中,利用 Endnote的搜索功能,进行分类了。点击页面中的全部→全选,可以将记录导出成 RIS格式,随后导入Endnote中。

第三步,Endnote文献库整理。将文献集导人Endnote,在Endnote 进行二次搜索和归纳。相关的基本方法在第二章节中已有详细阐述,这里不再展开,下面给出学生的分类情况,仅供参考(图\ref{fig4-5})。

\begin{figure}[!htb]
\centering
\includegraphics[width=0.9\textwidth]{fig4-5.png}
\caption{对Endnote导入的文献进行快速批量分类}
\label{fig4-5}
\end{figure}


Endnote中对这些文献进行分类和归纳的目的是进一步使我们清楚本篇论文可以从哪些点切入讨论,向读者传达什么样的观点,以推动该领域的进步。

第四步,论文的框架确定。在第三步的过程中,会逐步熟悉领域的背景,大致了解相关领域的已有文献涉及了哪些研究。进一步结合源头思维的方式,思考论文的框架,如给出论文的一级标题、二级标题。标题下方可以简单地按照逻辑顺序,概括下我们大概准备讨论哪些点?引言大概准备如何展开?先讲什么,后讲什么。下面是我结合源头思维的方式,对于这个论文框架的思考过程:万物互联究竟是什么?→万物互联和储能器件及系统是什么关系?储能器件及系统如何能更好地推进万物互联的发展?→这个时候,我们会考虑到如果大量物体需要联网的话,是不是每个物体都需要供电单元?这样的单元数量是不是会特别巨大而且非常分散?→那么如果没电的时候,怎么给它们充电呢?一个个去接插座是不是特别麻烦?所以是不是最好支持无线充电或者能源采集功能?这样这个电池可以不需要很长的引线去充电或者可以自己给自己充电。→而且如此多的电池,组成的电池网络是不是也是非常复杂的?是不是需要考虑到电池管理系统如何能更好地兼容这样庞大的电池系统?→宏观方面考虑后,可以进一步细化,考虑具体的每一个电池单元需要怎么样的发展趋势?→既然数量特别巨大,而且非常分散(意味着有的电池出现在一些更换不太方便的地方),我们是不是会希望这个电池的能量密度大?寿命(充放电次数)也尽可能地长,减少更换次数甚至不要更换(电池有搜集环境能量给自身充电的功能)?通过以上思考,其实就会逐步形成论文的框架,并给出关于未来发展趋势方面的我们的思考了,如下面所示:

{\kaishu
IoE 与储能

1前言(Introduction)

IoE的简介及其趋势

IoE系统需要能量转换与存储设备的发展

从两个方面分析应用于 IoE的储能体系——储能系统和储能单元

应用于IoE的储能系统具有的特点:无线连接;可以实现能量转换和储存一体化;需要管理系统

应用于IoE的储能单元考虑的特性:能量密度;寿命;自供能

因此需要发展可无线连接,一体化,或者是有效的管理体系等等顺应 IoE 储能的需求

本文提出了为了顺应IoE发展储能体系的发展趋势,并且给出了方向

2 IoE的储能系统

2.1 IoE与无线连接

2.2 IoE与能源转换与储存

2.3 IoE与能源管理

3 IoE的储能体系单元

3.1 能量密度

3.2 寿命

3.3 自供能

4 对未来储能技术顺应IoE发展的评价

1)建立评价体系

使用不同应用场合的评价体系

2)基于特定的应用建立合理的评价指标[平衡(trade-off)不同应用场景的优缺点]

3)工业化特性和学术研究之间的平衡

4)发展统一的评价方法;评价标准具体化
}

第五步,完成全文。完成全文写作的方式,和第四章第一节中介绍的从“知识库”到“素材库”的方式是一致的,这里不再详细展开。总之,如果不清楚哪方面该如何表达的话,那就充分搜集同类的相关表达,即利用良好的检索习惯,充分且精准地搜集和归纳相关素材,最终提炼出自己的表达,并注意针对已有文献的合理引用。

第二个案例:结合学生的操作过程,给出了关于综述类型的论文写作过程(为了能完整看到思路的修正过程,将一些中途思考不周、后期调整过的中间步骤也给出了)。

第一步,检索。根据大致的主题,测试和调整检索式,根据检索结果的数量和相关度,不断调整检索式,直到可以获得合适的检索结果:①检索结果数量适中,不要数量过于庞大,导致写作的时候无从下手。但也不能过少,导致缺少领域关键文献。②检索结果相关度高,尽可能避免出现大量不相关的结果。

例如,学生本次拟写的综述是关于碳基材料在金属空气电池中的进展和展望。先测试了如下检索式,注意其中AND的用法。检索式中包含了碳、金属、空气、电池(图\ref{fig4-6})。

\begin{figure}[!htb]
\centering
\includegraphics[width=0.9\textwidth]{fig4-6.png}
\caption{检索式的第一次测试}
\label{fig4-6}
\end{figure}



第一次检索出的结果67个,有些少(图\ref{fig4-7})。观察检索结果的标题时,意识到一个问题,如果要搜索金属空气方面的文献,不能只是用 metal-air battery作为检索词。因为很多研究金属空气电池的学者是会具体给出金属空气电池的类型的,比如锌空气电池、铝空气电池等,所以采用这个关键字的话,反而搜不到这些文献了。从检索结果来看不少情况,metal(如metal oxides金属氧化物)是约束在材料体系这里了,并非金属空气电池。进一步思考,不管是锌空气电池、铝空气电池或者是锂空气电池等,它们都会包含air batter*,因此可以在检索词中包含 air batter*以涵盖各种金属空气电池体系。继续将检索式调整如图\ref{fig4-8}所示。


\begin{figure}[!htb]
\centering
\includegraphics[width=0.9\textwidth]{fig4-7.png}
\caption{观察检索结果,思考如何调整检索式}
\label{fig4-7}
\end{figure}


\begin{figure}[!htb]
\centering
\includegraphics[width=0.9\textwidth]{fig4-8.png}
\caption{检索式的调整}
\label{fig4-8}
\end{figure}

经过调整后,检索记录大幅增加了,可以很好地搜出各种金属空气电池了,包括锂空气电池、锌空气电池、铝空气电池等等。如果我们想缩小本次论文的讨论范围的话,我们可以对检索式继续调整,将金属空气电池限定为锌空气电池。通过以上的检索和有限的几次调整后,会发现最终的检索结果比较理想,体现在数量适中、检索出的大部分结果都和我们要讨论的主题很相关。那么就可以将文献集导出,生成 Endnote,进行第二步操作。

第二步,知识库的完善。在Endnote进行二次搜索和归纳。相关的方法已在本书中多处阐述,这里不再详细展开。图\ref{fig4-9}是学生第一次的文献归纳情况。

\begin{figure}[!htb]
\centering
\includegraphics[width=0.9\textwidth]{fig4-9.png}
\caption{Endnote中学生第一次的分类情况}
\label{fig4-9}
\end{figure}


针对这样的归纳情况,我的思考如下:

(1)锌空气电池用的催化剂体系,metal-free(无金属)的研究比较少,所以最好本次论文不只是讨论metal-free的体系,也要包含含有金属的碳基催化剂体系。

(2)从分组数量来看,数量分布不太均衡(其实就是文献分类归纳不太均衡),比如有一个分组分类数量是125篇,分类过粗,这样就不太容易给出本篇论文的架构应该如何安排。可以采用之后第三点所提到的思考方式,进一步细分。

(3)快速分类和归纳文献的过程中,根据题目和摘要,考虑论文作者除了关注材料的化学方面(如什么元素),还关注了什么内容,比如是否涉及物理方面?是否有侧重碳基催化剂中的碳材料形貌(如多孔、一维、二维、三维规则排列等)?如果这样的话,其实就出现第二个可以分类的线索了。又如是否有提到或者侧重材料计算方面的?如果有的话,那么这又是一个线索。还有既然大量文献是涉及碳基材料和金属化合物的组合,那么从这两者间的组合出发是不是也可以归纳和区分?

(4)源头思维方式,思考作者为什么要进行这些研究?然后就应该会了解到锌空气电池中催化剂的作用,碳基材料催化剂在其中的地位,基本工作原理,存在的问题,以及已有文献围绕这些问题的思考和探索。这部分的思考和讨论一般可以出现在论文的第二章节,总领这个领域的基本原理和存在的挑战,以及围绕这些挑战学者们所开展的研究。这个章节的功能有些类似地图所起的作用,一方面给读者交代必要的背景,便于理解我们之后所要开展的讨论;另一方面可以使读者对全文的框架有个清晰的认知。

接着,可以尝试根据第三点提及的内容,在知识库中建立相应的分组(标签)并进行相关文献的分类。例如一篇文献,针对催化剂体系,可能涉及几个不同的角度。

(1)碳基体的类型:比如carbon cloth(碳布)、graphene(石墨烯)、carbon black(碳黑)、carbon nanotubes(碳纳米管)。

(2)掺杂类型:单元(掺杂N、掺杂B或其他),多元(N/S共掺杂、N/B共掺杂、N/P共掺杂、N/S/P共掺杂或其他的排列组合)。

(3)研究角度:研究失效的、研究缺陷的、研究耦合(coupling)效应的、研究通过整体结构设计提升性能的。这里没有举例出研究掺杂的,因为在第(2)点中已经分类过了。而且这里也没有从一些共性的非“特色”的研究人手,例如这些文献基本都会涉及和锌空气电池相关的基本研究(如氧析出、氧还原反应),所以不需要单独基于这些研究角度再去分类和归纳文献。

总之,分类的目的是为了帮助自己在后续过程中形成论文的框架,写作时可以更快索引到相关文献并形成“素材库”,从而启发自己形成一些独特新颖的写作思路。当然,有时候仅仅从标题和摘要无法判断作者的研究角度,此时可以在后期快速阅读全文的过程中,了解作者的研究角度后,再将这个文献添加至相应的分组中。

根据以上(1)$~$(3)点的讨论,可能知识库中的分组会出现这样的形式:

carbon cloth-掺杂N

carbon cloth-掺杂N-失效

carbon cloth-掺杂N-缺陷

carbon cloth-N/S共掺杂-整体结构

graphene-掺杂B-耦合效应

每个连接号(-)之前对应的是某一个切入角度的属性之一。建立和完善分组的过程中,同时将文献添加至合适的分组中即可。例如,某篇文献研究的催化剂是基于carbon cloth,涉及掺杂N,但你没发现第三个角度的属性。那我们就添加至第一行的分组中。如果催化剂基于carbon cloth,涉及掺杂N,同时又研究了催化剂的腐蚀失效之类的。那么添加至第二行的分组中。

由于 Endnote 中的分组是默认按照名称排序的,通过这种方式分类,分组会排列得整齐清晰,在后期索引文献和开展讨论时可以给我们提供很大便利。第二个好处是,仅仅通过浏览分组名称,我们就会大致了解已有文献针对我们感兴趣的领域已经开展了哪些研究,对论文提纲的安排也会有大致的了解。这样的分组过程不会耽误太多时间,一方面我们会结合二次搜索进行快速分组,另一方面,在大多数情况,当我们浏览文献标题或摘要的时候,基本会对文献的相关属性有所了解了,所以不如第一次就尽可能将分组(标签)设置齐全,省得以后来回进行重复的工作。具体使用过程中,可以根据每个人的实际情况,进行变通使用。

经过以上指导后,学生后期的分组情况如图\ref{fig4-10}所示,仅供参考。

\begin{figure}[!htb]
\centering
\includegraphics[width=0.5\textwidth]{fig4-10a.png}
\includegraphics[width=0.5\textwidth]{fig4-10b.png}
\caption{Endnote中学生第二次的分类情况}
\label{fig4-10}
\end{figure}



第三步,初步给出论文提纲。当第二步完善后,并结合源头思维的方式,会对论文的提纲有个初步的概念。可以以论文小标题的形式给出论文提纲,根据情况,可以在每个小标题下方给出讨论点。在后续第四步的写作过程中,还可以一边写作一边继续完善提纲。例如,以下是学生设计的写作提纲,会发现这个提纲是来自于第二步中的分组的启发。

{\kaishu
碳基阴极材料在锌空气电池中的机遇、挑战与进展

1前言(Introduction)

1)根据目前商用空气阴极的应用范畴,从两个方面介绍为什么需要发展碳基材料:一个是贵金属体系,一个是非贵金属体系(包括metal-free体系),都需要碳基材料的帮助

2)根据目前锌空气电池器件的性能特点和发展趋势(高能量密度),需要发展低成本、高效、灵活的碳基阴极材料

2 锌空气电池基本原理与碳基阴极材料的机遇

2.1 锌空气电池工作原理(通过原理引出机遇所在)

2.1.1 电池的构造与空气阴极的组成

2.1.2 空气阴极上的氧电催化过程

2.2 锌空气电池的应用技术指标(结合器件的性能,再次强调碳材料的优势,并提出进一步的要求)

2.3 碳基阴极材料应用于锌空气电池中的挑战(包括碳腐蚀、结构稳定性两方面的挑战)

3 用于锌空气电池的碳基阴极材料研究进展(以目标为导向,即为了顺应锌空气电池的发展,做了以下各个方向的工作)

3.1 形貌设计

3.1.1 多孔结构

3.1.2层状结构

3.1.3 纤维结构

3.2 缺陷工程

3.2.1 ORR/OER 活性调控

3.2.2 稳定性调控

3.3 协同作用构筑

3.3.1 金属-载体协同作用

3.3.2 多组分协同作用

4 结论和展望(conclusions and perspectives)(总结前人的工作并提出建议)
}

第四步,完成全文。具体方法之前已经阐述过,这里不再展开。



