\chapter{写作常见问题汇总}
以下检查清单(checklist)中所说的这些问题,虽然看似细节,但并不是小题大做、吹毛求疵。我们非常有必要在投稿前,仔细检查并尽可能避免这些问题。这么做的原因主要如下。

(1)论文的发表是需要经过同行评审的。站在审稿人的角度,你可以想象下,如果有两份稿件在你面前,第一眼望过去的时候,映入你眼帘的往往是图片和文字的整体感觉。在这种情况下,整体感觉本质上是由格式和一些细节来传达的。如果其中一份看上去格式很混乱,比如有些地方多输入空格,有些地方没有空格;上下标不注意;不同段落之间行距间距不同;文字中夹杂中文全角字符;许多拼写和语法错误;参考文献格式混乱;图片作得不规范等等。另一份看着特别舒服,没太多问题。这个时候,你会有什么感觉?你会不会感觉那个写作格式混乱的人,他做实验的时候是不是也会比较粗糙、态度不够严谨认真呢?那么既然如此,他设计的实验方案以及获得的数据结果和结论可靠吗?当然,我知道这两者之间不能画等号,科学工作者本身是要基于证据来得出结论,不能带有倾向性,但是大量出现以下检查清单中的问题,会给人带来很不好的心理暗示,因为这反映出作者的态度。毕竟我们是在向某一个期刊投稿,期刊本身就是有明确格式要求的。期刊这么做也是有原因的,是为了能够让审稿人和读者清晰阅读论文,理解论文,并便于论文的传播、交流和分享。而且我在过去的审稿中也确实发现,一般作图和文字格式比较混乱的文章,整体实验设计和结果推导方面也确实相对来说问题会更多。

(2)站在审稿人和读者的角度,太多的格式错误使人无法流畅地阅读,这样就影响了他们清晰理解作者的写作意图。可能作者的工作不错,但由于大量格式错误,分散了读者的注意力,读者没有把握到你实际想阐述的逻辑线索等,进而低估了你论文的贡献。

(3)站在导师的角度,看到一份格式错误太多的稿件,同样很容易干扰导师对于文章核心内容的聚焦,因为之前提到的格式错误问题,会干扰注意力。比如每句话都存在基本格式或语法错误,这样就很难把所有句子都衔接在一起连贯地看它们的观点展现、逻辑表达是否合理。

(4)同样的原因,这样反而很干扰创造力的发挥。检查清单上列的这些要点都属于没太多创造性的工作,我们在这个上面花费的时间和精力越多,越耽误我们投入在创造性工作上的时间和精力。所以我们如果在第一次成稿中就能注意这些问题,那么之后就可以专心地集中在具有更高创造性的事情上了。

此外,以下给出的所有案例都是我实际指导和修改学生论文过程中,所遇到的具体问题。

\section{影响表达的问题}

\vspace{0.5cm}
{\kaishu 一、定语太长,影响理解句子含义}
\vspace{0.5cm}

案例:The sandwich structured flexible Zn-air battery device were assembled with the flexible electrodes of the cotton textile waste Zn plated and the NiFe hydroxide face-to-face separated by the poly (vinyl alcohol) (PVA)-KOH hydrogel polymer electrolyte.

说明:案例中的“cotton textile waste Zn plated”这里定语太长,影响阅读。

\vspace{0.5cm}
{\kaishu 二、句子太长,影响含义表达和读者的阅读感受}
\vspace{0.5cm}

案例:By using energy storage systems (ESSs), the power system can shift part of the peak load to low power consumption period, thus utilizing surplus power during low power consumption period, improving the load rate of the power grid, in order to achieve the purpose of energy saving, which can save resources, reduce pollution, and be more friendlyto our environment.

说明:这个案例整段就一句话构成,句子过长了,导致句子中间停顿太多。一方面非常影响含义的有效表达,另一方面也使读者的阅读感受不佳,可以考虑改造为:By using energy storage systems (ESSs), the power system can shift part of the peak load to low power consumption period. Thus, surplus power during low power consumption period can be utilized to improve the load rate of the power grid, achieving the purpose of energy saving. As a summary, using ESSs in power grid can save resources, reduce pollution, and be more friendly to our environment.

案例:Figure2 shows the SEM image and EDS results of NMCTW. It can be seen that the waste cotton textile is uniformly covered by Ni after the deposition (Figure 2a), and energy dispersive spectrum (EDS) mapping of Ni element further indicates Ni metal exist which are evenly dispersed on the surface of the waste cotton textiles substrate (Figure 2b and c), which is conducive to function as an conductive electrode substrate.

说明:第二句跨度过长,也是存在同样的问题。例如可以改造为:Figure 2 shows the scanning electron microscopy (SEM) image and energy-dispersive spectroscopy (EDS) spectrum of NMCTW. It can be seen that the cotton textile waste is uniformly covered by Ni after the deposition (Figure 2a). The EDS mapping of Ni element further indicates the presence of metallic Ni particles. The Ni particles evenly dispersed on the surface of the cotton textiles waste substrate (Figure 2b and c) act as a flexible conductive electrode substrate.



\section{格式问题}

\vspace{0.5cm}
{\kaishu 一、英文论文中所有符号应为英文字符和半角字符}
\vspace{0.5cm}

这个是经常出现的问题,在以往我检查的稿件中,大部分都出现过文中夹杂使用中文全角字符这个问题,这可能是由于输入法没及时切换产生的问题。

案例:I would like to submit the manuscript entitled “Long-battery-life flexible zinc-air battery...
说明:其中的双引号是中文全角字符,应改为英文半角字符。

案例:…is stirring at 90 ℃ for about…

说明:案例中的摄氏度为宋体格式,正确的应该是英文格式,如Times New Roman格式:… is stirring at 90℃ for about...

\vspace{0.5cm}
{\kaishu 二、检查标点符号是否正确,包括句点、空格等}
\vspace{0.5cm}

案例:… by morphological regulation, which could enhance the performance of WS2 [9,10] Through the construction...

说明:第一句缺少句点。

案例:Reproduced with permission from Ref.[32] , Copyright 1998, Springer Nature.

说明:文中“Ref.[32]”后多输入了一个空格。这也是经常犯的错误。还包括少输入空格,此类格式错误包括语法错误等,可以通过一些软件(如grammarly)很好地解决,也可以在word软件中打开标点符号的标记,便于识别。

案例:The voltage decreases sharply at the end of discharge, dem-onstrating the discharge failure of the battery. The reactions occurring on the Zn electrode during the discharge can be expressed as follows: 

\begin{equation}
 Zn^{2+} + 4OH^- + 4OH EN)_4^{2-}
 Zn(OH)_4^{2-} - ZnO + H_2O + 2OH^-
\end{equation}

说明:方程格式混乱,存在多处错误。修改后的形式为

\begin{equation}
 Zn^{2+} + 4OH^- \rightarrow Zn(OH)_4^{2-}
 Zn(OH)_4^{2-} \rightarrow ZnO + H_2O + 2OH^-
\end{equation}

此外,我制作了一份文档,提供了一些常用的但经常容易输入错误的符号,下载地址:https://pan.baidu.com/s/1KedVanP6z9dHjcz0Ov-seQ。

\vspace{0.5cm}
{\kaishu 三、字体格式不统一}
\vspace{0.5cm}

(一)上下标是否正确

案例:Co3O4@NCNTS

说明:Co3O4应改为$Co_3O_4$。同时掌握批量搜索功能,比如Word中可以采用Ctrl+F调用出批量搜索功能,检查是否还有其他地方的Co3O4没有注意上下标问题。

(二)缩写格式需要统一

案例:The morphology and composition of the as-obtained Zn anodic electrode and the air electrode with nickel iron hydroxide catalyst electrodes were characterized by field emission electron microscope (FESEM, S-4800, HITACHI, Japan) equipped with... Figure 2 shows the SEM image and EDS of NMCTW. It can be seen that the waste cotton textile is uniformly covered by Ni after the deposition (Figure 2a), and energy dispersive spectrum (EDS) mapping of Ni element further indicates Ni metal...

说明:第一次在实验中出现缩写的时候,将扫描电子显微镜缩写为FESEM,但随后Results部分是以SEM来指代。改动方式之一:将之前的FESEM改为SEM。而且此处还存在第二个问题,原文写的是“electron microscope”,而非“scanning electron microscope”,所以简称也无法和SEM对应。

案例:2.3 Assembly and tests of the cotton textile-based flexible zinc-air battery

The sandwich structured flexible Zn-air battery device were assembled with the flexible electrodes of the cotton textile waste Zn plated and...

说明:表述不统一,文中有的地方表达为zinc-air,有的地方则是Zn-air,而且注意Zn/zinc和air之间的连接符号也不统一。

(三)仪器表达方式统一

案例:...field emission scanning electron microscope (FESEM, S4800,HITACHI, Japan) equipped with X-ray energy dispersive spectroscopy system (EDS). The crystalline structures of the as-obtained electrodes were measured by X-ray diffraction (XRD, D/max 2200/PC,Cu Ka radiation). Electrochemical tests were conducted in a three-electrode cell configuration using the electrochemical workstation (CHI760E and PARSTAT 4000).

说明:SEM(扫描电子显微镜)仪器给出了型号、厂家和产地(国家),但XRD(X射线衍射设备)没有厂家和产地等信息,之后的电化学工作站描述也是同样问题,


\section{参考文献问题}

\vspace{0.5cm}
{\kaishu 一、文献的规范引用}
\vspace{0.5cm}

全文引用的每一篇参考文献都要下载到全文,然后仔细阅读引用的这篇文章是不是真的提到过你在正文中所说的相关内容。如果提到过,作者是不是也是引用其他文献的。如果是,那么你还得继续找到被引用的这个更原始的文献,找到最终的源头。否则有可能别人引的文献,根本没提到这个内容,然后一旦跟着引了,就会出现明显错误了。当然还有一种情况就是由于引文习惯不良(比如没有很好掌握文献管理工具的使用)等,导致引文错误,引的论文根本没提到所要表达的内容。

文中表述的每一句话,尤其关于结论、观点类的,都要准备好相关的证据。仔细从源头出发,思考这句话是否正确。不能仅仅因为文献是这么阐述的,所以就不假思索地认为这是对的。毕竟文献中也只是阐述了作者的观点,科学不断在进步,过去的观点有错误或者实验有瑕疵,那也是很有可能的。

案例:“In addition, side $reactions^9$ related to Li electrodes or oxidative products and electrolyte decomposition are usually involved in battery electrochemistry, particularly during the charging process.”

说明:学生初稿中提到锂电极的副反应,然后在副反应这里引了文献9。但是文献9没有提到任何关于锂电极的副反应。

\vspace{0.5cm}
{\kaishu 二、参考文献在正文中的格式错误}
\vspace{0.5cm}

这也是常见的问题,解决方案很简单,如果使用Endnote之类的文献管理工具,一般期刊都会提供相应的参考文献style文件,只要导入相应的style,就可以生成正确的格式了。

案例:...performance of $WS_2$[9][10].

说明:学生初稿中正文的文献引用格式是[9][10],但基本上很少期刊是这种方式的引用格式。常见的比如[9,10]、9,10(上标)等。

案例:... which both deteriorate the catalytic property of the as-obtained electrodes. The transparent NiFe HUFs electrode electrodeposited for 500s have the large number of active sites and the good conductivity, resulting in the high catalytic performance of as-obtained electrode, which have also been reported in our previous studies.

说明:学生原稿件是拟投ACS(美国化学学会)旗下的一个期刊。文献引用应该在标点之后的,而不是在句号之前。此外,该版本中还存在不少语法错误。

\vspace{0.5cm}
{\kaishu 三、参考文献列表中的错误}
\vspace{0.5cm}

(一)期刊术语缩写不正确或者全称不正确

解决方式:以Endnote文献管理工具为例,Endnote中的Tools可以找到Term Lists选项,这个Term Lists 给出了常见期刊的缩写方式,如图\ref{fig5-1}所示。

\begin{figure}[!htb]
\centering
\includegraphics[width=0.9\textwidth]{fig5-1.png}
\caption{在文献管理工具中,正确编辑期刊的全称以及缩写}
\label{fig5-1}
\end{figure}



但是,这个Term Lists通常是不完整的,所以有一些期刊包括新期刊,没有对应的缩写方式。针对期刊全称输入有误的情况,将期刊全称输入正确后,检查Term Lists中是否有对应缩写;对于期刊全称正确,但无对应缩写的,可以自己新建Term Lists 中的Term,这样以后修改论文时,会非常方便,每次可以自动更新参考文献列表,而不需要每次手工去编辑参考文献列表。

案例:
[102] D.Su, S.Dou, G.Wang, Chem Commun (Camb) 50 (2014) 4192.

[103] D.B. Kong, X.Y.Qiu, B. Wang, Z.C.Xiao, X.H. Zhang, R.Y.Guo, Y.Gao, Q.H. Yang, L.J.Zhi, Science China-Materials 61 (2018) 671.

说明:学生初稿中,参考文献[102]和[103]中的期刊名称的缩写不规范。

(二)参考文献中的作者拼写是否正确

如果发现作者名字输入有误这些情况,也是直接在文献管理工具中进行修改。

(三)参考文献标题错误,包括上下标等

解决方式:如果发现 Endnote导入的文献,存在标题错误(如空格问题、上下标问题)或者缩写方式不统一等问题,不要直接在Word中修改参考文献中的错误,这样一来工作量大,二来后期一旦再次刷新文献,之前的改动就无效了,下次还要重新再改一次。正确方式是直接在Endnote中对文献的格式进行修改,比如Endnote中是可以直接对文献的上下标进行改动的(图\ref{fig5-2})。

\begin{figure}[!htb]
\centering
\includegraphics[width=0.9\textwidth]{fig5-2.png}
\caption{在文献管理工具中,正确编辑期刊的各项信息,包括标题和作者信息等}
\label{fig5-2}
\end{figure}



案例:以下就是以后容易出问题的改动方式,是在Endnote生成的Reference list(参考文献列表)中,直接调整上下标。导致的一个显著问题就是当以后参考文献有变化,重新利用Endnote生成参考文献列表后,这些调整都会失效,然后还需要再次操作一下。一方面增加工作量,另一方面也非常容易遗漏。

\begin{figure}[!htb]
\centering
\includegraphics[width=0.9\textwidth]{fig5-3.png}
\label{fig5-3}
\end{figure}

\section{改动某处时,是否将其他牵连到的相关部位也进行了相应的改动}

每当我们修改了一个地方的时候,还需要考虑到这个部位的改动,是否还牵连文中其他部位的改动,包括关键词、图片、表格、正文所有部位、补充材料等。

\begin{figure}[!htb]
\centering
\includegraphics[width=0.9\textwidth]{fig5-4.png}
\label{fig5-4}
\end{figure}

说明:原稿中,图中的“new Zn anode and electrolyte replacement”等表达方式不佳,在导师指导后,正文部分的文字已修改为“replacement of Zn anode and electrolyte”,但是相应的Figure(图片)中的注释没改动。

\section{全文前后是否有矛盾之处}

家例:However,the most current research focuses are the single component (zinc anode, electrolyte and air electrode) of zinc-air batteries,there is little research available on integral battery configuration. Zinc-air battery is prone to the leakage and volatilization...There are still many studies which induce the componentized cell structures after studies of certain single component on zinc-air batteries to illustrate the application prospect of the target research [6].

说明:本文之前说相关研究少(“there is little research available”),之后又说相关研究还是很多的(“There are still many studies”),前后矛盾。特别是在写作大型综述时,由于前后文相隔比较远,此外可能会借鉴不同作者观点,容易导致出现前后表达矛盾的问题。

\section{第六节检查清单(checklist)的使用方式}

根据过去指导学生的反馈来看,关于使用检查清单进行检查的推荐方法之一是每次只检查一种类型的错误。不要试图一次性地通读文章,把各类型错误都找出来。因为试图一次性地找出各种错误类型,最后的结果往往是每种类型的错误都会有遗漏。

我暂时还没有条件去仔细地查证思考背后的原因。我初步猜测是不是因为人类大脑进化至今,其实还并没特别擅长高通量的并行事件处理,所以通过一次性的阅读,试图发现不同的错误类型,最后的结果是每种错误类型的检查都会有所遗漏。比如,我们可以这次只检查图片中的(a),(b)等标记是不是格式一致,至于其他图片错误或者文字拼写错误、格式错误等,我们可以分解到第二步去检查。当然如果能力提升上来后,也可以逐步增加合并检查的事项。所以我们经常看到的检查清单类型,都是一个个打勾的框,检查完一个类型,打一个勾,再检查下一个类型的问题。检查的时候,注意举一反三,思考同类型的错误,是不是会出现在其他地方,包括标题、关键词、摘要补充材料、图片和表格等等。

在我观察这个检查清单效果时,我发现男性学生一般出错率要高于女性学生。讨论这些,是为了帮助我们可以更好地认识自己,不断提升自己工作的方式。这个我猜测会不会是由进化所决定的。男性远古主要是以狩猎方式生存的,他们需要长时间跟踪猎物,在合适的时候出击。因此,大脑的思维方式是要锁定猎物,尽可能排除猎物以外的一切干扰,以降低能量损耗,将能量用于最重要的那件事情上。而不擅长这个技能的男性,可能被淘汰了,没法繁衍至今。所以,这也是为什么现代大部分男性逛商场,一般而言会直奔主题,直接找到要买的那件物品,买完走人,而不是随意闲逛,也不太会被不想买的东西干扰。这种思维方式就导致一次只执行检查清单中的一个任务是比较好的,不要试图同时执行太多的任务。而远古女性接受信息有“面”接受的特征,她们的生存并不是依赖特定的猎物,而是要在野外尽可能地搜寻到用于生存或改善生活的东西。比如,她们会同时留意到地上的作物、树上的果实,同时还能注意到一些可供穿着、可供装饰的东西。所以,她们才可以发现除了猎物以外的生存方式,这要求她们不是聚焦在一个点,而是要非常发散地捕捉到各个可能的“细节”。所以逛商场的时候,她们可能会被许多东西所吸引。

福特汽车公司也是相对较早发现这一类似规律的组织,并将这种方式用于汽车的流水线作业,从而极大提升了效率。也就是说,每个工人只负责一个工位,只负责重复一件相对简单的事情,通过许多工人和工位的流水线组合完成一辆整车的制造,而不是每个人负责一辆车的制造。

这个概念转化到检查清单上来说,相当于不要试图一次性检查出所有错误,而是每一次的检查只检查特定的错误类型。比如,这次检查语法错误中的时态错误,那你就只看时态,直到时态问题对了后,再来看单复数问题(当然,随着能力的提升,可以将几项不同类型的错误合并检查);又如,检查参考文献时,这次就只检查正文中引用的参考文献是否正确,下一次只看参考文献中作者拼写、卷期号和题目输入等是否正确。

同时,也要学会不断寻找甚至自己开发软件,来高效准确地检查出这些错误,比如借助Grammarly等工具。





