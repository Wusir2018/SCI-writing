\chapter{选题}

\section{了解该领域的发展情况}
如图\ref{fig2-1}所示,科学探索过程大致是由选题——研究——总结这三个过程构成的。需要注意的是,这三个过程并不是严格的依次递进的发展过程。随着认知的深人,我们会发现三者之间是相互影响的。例如,在研究甚至在总结研究结果的环节中,当我们发现当时的选题存在未考虑清楚或者考虑不周的情况,还需要重新调整或修改选题。

\begin{figure}[!htb]
\centering
\includegraphics[width=0.5\textwidth]{fig2-1.png}
\caption{选题——研究——总结过程是相互影响、互相交织的}
\label{fig2-1}
\end{figure}

当我们进入科研探索时,首先会面临选题的问题,即我们选择什么领域去进行探索呢?一般而言,导师会给出一个大致的研究方向和思路,并会给我们一些关于这个方向的描述或者一些核心文献。所以我们首先需要掌握的技能是如何根据导师的介绍、指导、相关的核心文献,或者自己的研究兴趣,尽可能地将该领域的相关工作充分掌握。

\vspace{0.5cm}
{\kaishu 一、知识库}
\vspace{0.5cm}

我大致将选题工作进一步具体分解为这么几步(下面还将结合具体实例展开)。

(1)同义词库的构造:确定目标领域的关键词及其同义词。可以建立一份文档,专门记录自己是如何确定这些关键词及其同义词来源的,比如可以是根据某篇关键文献提取的,可以是导师给出的,可以是根据同义词、相近词获得的。

(2)检索式的构造:创建尽可能精准的检索式,在代表性的检索引擎中进行搜索。根据第一步所建立的关键词,将关键词以合适的逻辑关系进行组合(如 AND并列、OR或、NOT排除某些关键词),以构造合适的检索式,并根据检索结果的相关性以及对目标领域的覆盖程度(不遗漏相关文献),反复调整检索式。

(3)知识库的建立:采用合适的工具软件,导入和归纳第二步搜集的大量检索结果,以便于后期的频繁搜索和调取。同时知识库中也对这些文献的导入方式做详细记录,方便后期回顾和查看当时的导入方式是否全面合理,如果不合理,可以不断重复第一步和第三步,进行修正。

(4)知识库的更新和完善:在文献阅读和研究过程中,我们会发现一些相关文献,例如文献中引用的文献,或偶然发现的一些相关文献,并没有被导入第三步形成的知识库中。针对这种情况,我们一方面可以将这些文献导入知识库中,同时还可以分析为什么它们当时没被搜索到,是因为检索式当时构造的还不够精准还是遗漏了某些关键词?如果是前者,我们可以重新进行第二步的检索式构造并再次检索,导入当时遗漏的文献。如果是后者,我们可以在第一步中,继续补充相关的关键字,并再次执行其他步骤,完善该知识库。

一般而言,很少有哪个方向,是完全没有前人探索过的。所以,为了真正实现有效的探索工作,我们首先要明确:针对我们所处的领域,如图\ref{fig1-1}所示的研究状况如何?即目前的已有研究已经探索到什么程度了(圈的边界);这个领域存在哪些争议(圈内标记问号的区域)?存在哪些共识(图中标记阴影的区域)?这些共识是由什么样的证据或者逻辑支撑的?这些证据是否真的完备?整个验证过程是否合理?对以上问题的回答,离不开对已有工作充分的检索和整理。而要充分地检索,尽可能不要遗漏关键工作。首先要了解研究者们如果开展了相关的工作,他们在总结和发表自己工作时,其标题、摘要或者全文中会出现哪些关键词?同时为检索全面,还要考虑到不同作者发表工作时,可能他们是表达同一个意思,但使用的词汇不同。因此还要找到这些关键词的同义词,也就是先前提到的第一步。

有了以上的词汇库后,我们第二步就可以基于这些词汇去构造合适的检索式,并在合适的数据库例如文献数据库(Scopus、Web of Science、知网等),专利数据库,博硕士论文数据库中进行检索。这些关键词有的是“并列”的关系,即检索式中同时需要包含不同的关键词;也可能存在“或”的关系,即检索式只需要包含这些关键词中的任何一个;也可能存在“非”的关系,即检索式不包含某个关键词。这个步骤一般很难一次到位,有时会发现构造的检索式检索出大量不相关的结果,或者会遗漏很相关的文献。因此需要根据检索结果,不断返回调整检索式,直到出现的检索结果比较理想,领域内相关工作都被覆盖到了,同时不含有大量的不相关工作。

随着科学技术的发展,被发表的成果也呈现爆炸式的增长。经过第二步的检索,往往会得到成千上万的成果(检索结果)。对这些检索结果进行有效整理、归纳,对于后续的一系列工作包括研究方案的确定、实验的开展、专利申请、研究论文的写作和投稿、毕业论文的写作和毕业答辩等都有重大帮助。我一般是采用Endnote等文献管理软件对以上检索结果进行管理。Endnote本身也包含有更强大的检索功能,如对大小写、空格等敏感。通过借助Endnote的检索功能,可以对第二步得到的检索结果再次进行搜索归纳等工作,以更好地管理我们的知识库。

以上提到的四个步骤会互相影响,往复进行,从而使我们的知识库不断完善。如通过阅读第三步的相关核心检索结果中引用的参考文献,我们也许会发现一些相关的但是并未被先前的检索所包含的工作。这个时候,需要分析原因,再次进行检索和导入工作。例如,如果是因为遗漏了某些检索词导致的,那么我们可以补充第一步的同义词库。如果是因为检索方式考虑不周引起的,例如未考虑到一些符号包括特殊符号、上下标等对检索式的影响,那么我们可以调整第二步的检索方式。例如$Co_3O_4$,有时检索引擎中的文献标题会显示成$Co3O4$这种形式(数字和字母之间有空格或者有其他特殊符号插入,比如表示下标的符号)。那么我们一旦按照“$Co_3O_4$”去检索,就会遗漏这个文献。我们可以将关键字调整为“$Co$”,这样就会包括标题包含$Co_3O_4$的文献了。

此外,如果我们从其他途径(非检索方式)获得了相关的文献,也需要查阅其是否包含在我们的知识库中。如果未被包含,则做类似以上的分析,并再次进行检索,以防遗漏其他相关文献。每一次重新检索时,利用软件的去重导入功能,可以只导入新增的检索结果,而知识库中已有结果不会被再次导入。

下面结合一个具体实例,进一步阐述以上内容。

\vspace{0.5cm}
{\kaishu 二、同义词库的构造}
\vspace{0.5cm}

以我们想了解锌空气电池领域的发展情况为例。首先,我们来分析锌空气电池这个词组,发现它可以分解为两部分,且其中锌空气是修饰电池的。因此设计的检索式可以包含两个部分的检索词:一部分是关于锌空气,另一部分是关于电池。我们可能是第一次听说这种电池,还不清楚对应的英文专有名词是什么,这个时候除了从相关的文献中找到关于锌空气电池的表达,还可以借助一些学术类的翻译网站。比如我们可以在 http://dict.cnki.net/中输入“锌空气电池”,会出现如图\ref{fig2-2}所示的结果。

\begin{figure}[!htb]
\centering
\includegraphics[width=0.9\textwidth]{fig2-2.png}
\caption{了解专有名词的学术表达形式}
\label{fig2-2}
\end{figure}


采用这个网站的优点是因为该网站返回的一些搜索结果是来自于已发表的文献,所以可以看到这些词组包括专有名词的学术表达方式。接着,根据该网站的检索结果,我们可以获得一些关键词,包括锌空气的表达:Zn-air、zinc-air(Zn是zinc的简称);电池的表达:battery。

为了使检索结果全面,我们接着进一步思考这些词汇会不会出现一些同义词,或者有没有其他表达方式。比如会不会有的学者以氧气($O_2$、oxygen)来代替空气,研究了锌氧气电池。当然,第一次由于对这个领域的不了解,很难完全了解涉及的同义、相似或相近表达。不过这也没关系,我们可以先把已经检索得到的文献导入,结合之前所述的方式。例如,在后期浏览这些文献所引用的相关参考文献过程中,注意搜集其他可能的关键词,以扩充该同义词库。

\vspace{0.5cm}
{\kaishu 三、检索式的构造}
\vspace{0.5cm}

由于我们所要检索的文献涉及锌空气电池,因此检索式中需同时包含锌空气和电池这两组关键词,即锌空气和电池以AND的形式出现在检索式中。其中的锌空气/氧气可能有多种表达方式,例如Zn-air、zinc-air,Zn-oxygen、zinc-$O_2$等多种组合,而且zinc 和air 之间的符号有可能是短的连接号(半字线),也可能是长的连接号(一字线)。所以,为了把以上各种情况都考虑进去,进一步地,我们可以把锌空气也拆解为两组关键字的 AND关系,即同时包含锌和空气的表达:锌的关键词是 Zn或 zinc,因此以 Zn or zinc 表示;空气/氧气的关键词涉及air、oxygen 或$O_2$,由于这三者的关系是只要包含任意关键字即可,所以用 oxygen or air or $O_2$表示。而电池这组关键词的表达为battery。考虑有的学者可能会用cell来表达电池,所以我们以battery or cell表示。但同时考虑有的文献,可能会以复数形式表达电池,也就是batteries、cells,所以我们这里可以引人通配符“*”,将电池的关键词表达为:batter *or cell*。总之,检索式中同时包含锌的同义表达和空气/氧气的同义表达,以及电池的同义表达,这样可以全面地覆盖可能涉及锌空气电池的工作。最终构造的检索式如图\ref{fig2-3}所示(采用Scopus数据库为例进行搜索)。



\begin{figure}[!htb]
\centering
\includegraphics[width=0.9\textwidth]{fig2-3.png}
\caption{构造检索式}
\label{fig2-3}
\end{figure}




\vspace{0.5cm}
{\kaishu 四、知识库的建立和完善}
\vspace{0.5cm}

(一)文献导入

将检索得到的记录,全部导入文献管理工具如Endnote中,导入的时候可以选择去除重复文献的方式。同时我们可以在Endnote中新建立一个记录,这个记录的内容包括当时的检索式、检索时间和检索记录数量,如图\ref{fig2-4}所示。这样,后期我们可以了解当时是以怎么样的方式构造这个知识库的,是否需要进一步完善这个检索式,以尽可能避免重复的无效工作。


\begin{figure}[!htb]
\centering
\includegraphics[width=0.9\textwidth]{fig2-4.png}
\caption{检索结果导入文献管理工具并记录检索规则}
\label{fig2-4}
\end{figure}



(二)利用Endnote的分组功能对文献进行分类归纳

导入后,在Endnote中对文献进行分类归纳整理。考虑同一期刊收录的文献研究范围相近,可以先按照期刊排序,随后通过快速浏览标题和摘要,并结合搜索功能,寻找合适的归纳方式。

这其中我们要掌握的技能是学会在Endnote中建立分组,并用良好的分组来整理我们的知识库。这么做的优点是后期使用过程中,可结合清晰的分组以及搜索功能快速索引到我们需要的文献,并且也可以将一类的文献集中浏览对比。需要注意的是,Endnote中的分组概念更类似于一种标签的概念,即文献的分组,相当于给一批文献贴上某一类标签。由此,一篇文献可以有多个标签,即出现在多个分组中。

我习惯的操作方式如下:首先建立一个分组集合(groupset),其中有三个分组(group):“不相关”、“搜索文献”和“未分类文献’(稍后会看到我这么做的原因),如图\ref{fig2-5}左边栏目所示。


\begin{figure}[!htb]
\centering
\includegraphics[width=0.9\textwidth]{fig2-5.png}
\caption{文献管理软件中建立分组}
\label{fig2-5}
\end{figure}


这里的一个重要技巧是要学会使用Endnote的搜索功能,从而一批一批地而不是一篇一篇地分类文献。Endnote具有非常强大的搜索功能。例如,如图\ref{fig2-5}所示,在“未分类文献”分组中,浏览到2篇文献。我们发现它们的标题都含有catalyst(催化剂)这个词,那么就可以在所有记录中,针对这个catalyst做一次搜索,将类似的文献都搜索出,从而降低归纳分类的工作量。如图\ref{fig2-6}所示,我们会检索出399篇文献。随后新建一个catalyst一级分组,将这399篇文献拖入其中。接着在未分类文献中删除已经分组的这批文献,这样未分类文献就下降到666篇\footnote{分组数量显示的 667还包含一个我们自己创建的记录,不属于要分类文献,因此需要分类的文献数量是666篇。}了。

浏览catalyst分组时候,我们可以再次感受到Endnote强大的搜索功能。如图\ref{fig2-6}所示,根据搜索方式的设置,标题中的单词只要含有“catalyst”的都可以被检索到,比如标题包含catalyst的复数catalysts和 electrocatalyst(电催化剂)等单词的文献也可以被检索到。

\begin{figure}[!htb]
\centering
\includegraphics[width=0.9\textwidth]{fig2-6.png}
\caption{利用搜索功能快速归纳文献}
\label{fig2-6}
\end{figure}


下一步,我们可以浏览 catalyst分组中的文献,将不相关的文献去除。采用这种方式生成分组,不相关的文献数量会远远小于相关文献,所以从分组中去除不相关文献的工作量要远小于从未分类文献中一篇篇选出和 catalyst分组相关的文献。

这里需要注意去除分组中不相关文献的方式,有两种去除方式:一种是直接按 delete键删除(或点鼠标右键选删除)。此时,如果这篇文献不隶属于任何分组的话,那么这篇文献会出现在Endnote自带的一个Unfiled分组中,表示这篇文献目前没有分类。那么我们可以最后来看这些没有分类的文献应该如何归纳。另一种是我们可以直接将该分组中不相关的文献拖到其他相关的分组中。如果没有相关分组,那么我们就新建立一个。如果这个文献不是我们本次要关注的,和我们本次研究无关,那么可以将其拖动到先前提到的“不相关”分组中。分组中不相关的文献被拖动到其他分组后,我们再将这些文献从该分组中删除。有时这步操作会引起误操作。比如我们还未将文献成功移动至其他分组中,就将它们从当前组删除了,这个时候这些文献会出现在 Unfiled 分组中,我们可以从中找回。

接着,我们继续对剩余未分类的666篇进行归纳分类,会发现electrolyte(电解质)这个词,那么做如上描述的类似操作。注意检索时,我们同样是在“搜索文献”这个分组中检索关键词的,而不仅仅是未分类文献。这么做的原因是,可能有的文献会同时涉及多个关键词,比如既提到electrolyte,又提到catalyst(但涉及catalyst的文献此时已经不会出现在“未分类文献”分组中了)。采用这种方式,我们在electrolyte分组中不会遗漏可能同时涉及catalyst和electrolyte 的文献。

综上所述,我们要从“未分类文献”分组中寻找和分析适合检索的关键词,但是从“搜索文献”分组中根据相关的关键词检索文献,并进行分类。所以这也是我们在归纳分类之前,先设置三个分组的原因(图\ref{fig2-5})。

类似地,还可以发现anode这个词也经常出现在标题中。相应地,我们也可以构造“anode”这个分组,如图\ref{fig2-7}所示。在这个分组中,我们可以发现一些和锌空气电池不相关的文献,将其移动到“不相关”这个分组中,并从“搜索文献”分组中去除这些不相关的文献,这样下次在“搜索文献”分组中搜索时就不会检索到这些文献了。需要注意的是,有些我们暂时认为不相关的文献,可能会对我们后期的实验有启发,所以后期也可以进行回顾。此外,我们也可以使用同样的搜索技巧去扩充“不相关”分组。


\begin{figure}[!htb]
\centering
\includegraphics[width=0.9\textwidth]{fig2-7.png}
\caption{知识管理库中进一步扩充一级分组并从分组中移除不相关文献}
\label{fig2-7}
\end{figure}



以此类推,我们还可以构造出其他分组。有一些文献属于综述类型,也可以单独分类到一个分组中。综述的标题一般会包含有advances、overview、challenges、perspective、review等词,可以根据这些词语,进行检索和分类。

通过这种搜索和批量分类的方式,最后未分类文献从1000多篇文献快速大幅缩小为205篇,如图\ref{fig2-8}所示。以上实例中描述的步骤,我自己操作下来,大概在10分钟内完成。初学者可能由于对专有名词不太敏感,会相对慢一些,但是掌握了这种方法,至少会大幅快于未经检索就直接分类的方式。

\begin{figure}[!htb]
\centering
\includegraphics[width=0.9\textwidth]{fig2-8.png}
\caption{将知识管理库中的大部分文献采用搜索方式快速分组归纳完毕}
\label{fig2-8}
\end{figure}



最后剩下的未分类文献中,如果很难找到共性的可搜索的关键词,那么我们可以一篇篇地浏览,思考如何分组或者留待以后分组。

通过这种方式,可以迅速对一个陌生领域获得初步的了解,这些分组名称其实大致反映了这个研究领域的脉络。比如这个案例中,通过观察图\ref{fig2-8}中左侧的分组信息,我们可以看到学者们研究了锌空气电池领域的如下对象:空气电极、阳极、催化、催化剂、电解质、纳米材料等。另外,通过阅读综述分组中的文献,也可以迅速深入地了解一个领域。

类似地,我们可以在大领域中的某一个细分领域做类似如上的操作,对其进一步细分。通过这样的操作,基本可以完全掌握这个领域的发展情况,为下一步确定自己的研究方向做准备。

例如,以之前已经分组的catalyst为例:我们也是首先按照期刊排列,随后识别出一些共性关键词,进行搜索和批量归纳。从之前几篇文章我们可以看到(图\ref{fig2-6}),它们都涉及锌空气电池所使用的催化剂材料,但其中所使用的催化剂的类型(成分)是不同的。例如,标题中依次出现 Co3O4、Mn3O4、MOF、NiFe、Co-N-C、Co9S8等。这个时候就可以利用Endnote 的大小写敏感方式进行搜索了,比如分别搜Co、Mn、Ni,这样就可以分别将标题中含有Co、含有Mn、含有Ni的给分类出来了。

如图\ref{fig2-9}所示,以Co为例,在搜索框的Title中输人Co(注意输人时选择的是Contains,表示包含的意思),同时选择Match Case(大小写敏感),这个时候会出现186条检索记录,表示在catalyst分组中有186篇文献的标题中含有Co。我们可以建立分组“catalyst-Co基”,将这些文献移动到其中,并从“catalyst未分类文献”分组中移除刚才已分组的文献,继续寻找合适搜索的关键字,以进行新的归纳。

\begin{figure}[!htb]
\centering
\includegraphics[width=0.9\textwidth]{fig2-9.png}
\caption{对知识管理库中的一级分组进一步细分,形成二级分组}
\label{fig2-9}
\end{figure}


“catalyst-Co基”二级分组中还可以进行类似的操作,形成三级分组,直到整个分组不断地被细化。例如,我们会看到这些文献有的涉及不同成分的Co基催化剂(如Co3O4、Co9S8、NiCo2S4以及多种组分的复合、nitrogen-doped掺杂),有的涉及不同形貌的Co基催化剂(如 core-shell核壳结构、nanowire纳米线)等等。那么根据以上特点还可以进一步依据合理的关键字进行检索和归纳。直到层级越来越细。分组中可能会出现“catalyst-Co基-形貌调控”“catalyst-Co基-缺陷调控”“catalyst-Co基-成分研究”等等(我一般习惯将第三级的分组按照文献的重点研究内容,或者想解决的问题来划分)。

最终,通过浏览这些分组(实际上类似标签),结合搜索功能,并阅读相关文献,就可以非常全面和深刻地理解这个领域的发展状况。

\section{如何阅读和跟踪文献}
知识库形成后,就可以进行文献的阅读了。对这样一个知识库的有效管理和学习是十分重要的,它不仅在研究初期对于了解领域的发展情况,寻找自己的研究方向起了重要作用,对于研究中期的实验开展,实验中遇到问题的解决,以及研究后期实验结果的总结和发表等也有着重要作用。

我们现在处于一个信息爆炸的时代,每天每个细分领域都有大量的文献产生。从图\ref{fig2-6}中的分类可以看到,仅仅是一个细分领域都可能有上千篇文献,如果热门领域可能涉及上万篇,而且每天还不断在增加。例如,我在2015年写一篇综述和一本书的时候,当时这个领域涉及2万多篇文献,最后成文的综述和书中引用了约2000篇文献。而且我们有时候还会关注这个领域的书籍、博硕士论文、专利等,那么涉及的信息就更多了。

当面临海量的文献等信息,应该如何更有效地阅读文献呢?每个人可能都有自己行之有效的方法,我这里仅仅分享自己的经验教训,具体使用的时候,也可以结合自己的思维模式,寻找更适合的方式。

我觉得文献也好,书籍也好,如果读完之后,还是和新的一样,没有任何标记的话,未来将很难有效利用相关信息。我从本科期间开始,也尝试了多种方式,最早期的是将相关的文献打印,分门别类理在文件夹中,阅读的时候会对这些文献做各种标记和注释,并用一些便签纸粘贴在相关文献上,方便后期的索引。这种方式适合非常早期的学术生涯,那个时候要接触的文献普遍还不多,可能1000篇以下甚至更少,用这种方式勉强可以管理知识库。但是此类方式也存在明显弊端,其一是快速检索和翻阅不便,容易反复且重复地翻阅和浏览“无关”信息。想象下,如果一篇论文20页的话,100篇就涉及2000页了。在实际的文献阅读过程中,除了少数一些文献需要从头到尾地仔细阅读和注释,大部分文献,更多的情况下,我们只想阅读我们感兴趣的部分,并不是全文按顺序浏览,或者说问题导向型的阅读。例如,我们刚开始接触某个领域,有时获得了一个实验结果,但不清楚如何全面深入地分析这个实验结果,这个时候,我们只想集中浏览不同的文献是如何分析类似实验结果的;又或者我们开展实验之前,只想集中地浏览关于这种类型的实验,了解目前已有文献是如何设计实验步骤和设置参数的;又或者当我们进入文章写作的时候,只想快速和集中浏览已有文献的前言或者某部分的讨论是如何组织和展开的。就好比图\ref{fig2-10}中每一个方框代表文献中的一页,我们可能只想连贯地阅读某些页数的某些区域(图\ref{fig2-10}中标记的虚线方框)。此时,可想而知如果面临几百成千页的一个纸质文件,很难完成以上操作。而且考虑到文献其实是非常大量的,如果都用摘抄笔记的方式实现以上工作不太现实,且会占用大量时间。第二个弊端是由于打印的文献是纸质的,无法执行搜索。当文献的阅读量大幅扩张后,我们经常会遇到这样的操作:例如想快速寻找到涉及某个关键字的文献;了解某位作者的文献。这些操作在 Endnote 中很容易执行,但是当知识库全部是纸质文献的形式,就非常难实现。

\begin{figure}[!htb]
\centering
\includegraphics[width=0.9\textwidth]{fig2-10.png}
\caption{问题导向型的阅读方式}
\label{fig2-10}
\end{figure}

由于以上显著弊端的存在,随着文献阅读量的增大,我随后采用了Endnote进行知识库的管理。我将所有文献都导人Endnote中,只选取其中需要精读的重点文献采用打印的方式浏览、阅读和做笔记,其余的文献采用我下文介绍的方式进行阅读。同时我设计的文献阅读策略,可以同时达到如下目的:①有助于记忆;②便于后期快速索引;③尽可能避免重复工作和时间精力上的浪费;④单位时间获取尽可能多的信息,适合千篇甚至万篇规模量级的文献的阅读理解;⑤每一次的阅读都尽可能有新的收获;⑥每次阅读时,尽可能保证阅读的连贯性;⑦有利于归纳总结,发现其中的共性问题和差异;⑧对后期的研究开展、论文发表等也能起到作用;⑨有助于提升思维方式。

\vspace{0.5cm}
{\kaishu 一、图片/文字切块-归纳-重组阅读法}
\vspace{0.5cm}

我将我阅读文献的方式命名为“图片/文字切块-归纳-重组阅读法”。

我们阅读的文献一般是由图片/表格(Figures/Tables)+文字组成的。图片一般会包含作者的实验/计算结果、针对实验/计算结果的分析,以及为便于读者理解给出的示意图等。表格一般也是针对实验/计算结果的汇总。文字则一般包括引言、实验描述和围绕之前提到的图片/表格的解释展开、参考文献等。考虑到人脑一般对图片类的信息接受速度和记忆程度都要远远高于纯文字信息,因此我会初步通过快速浏览文献中图片(Figures)的方式代替逐字逐句浏览的方式来阅读文献,以满足之前提到的目的①和④。

确定知识库中哪些文献是我要关注的之后,第一步,我会利用计算机程序等方法,将这些文献中的图片批量地按顺序提取出来到一个文档中(假设命名为《1号文档》)。这个文档后期会涉及来自于许多文献的图片,可能几十篇文献,也可能成百上千篇。另外,这些图片后期还会做归纳分类,打乱顺序,因此提取图片过程中还要做好索引,以便后期也能知道这些图片是来自什么文献的。第二步,通过浏览这些图片,可以快速了解如下信息:本文做的是什么体系?创新点或者贡献在什么地方?研究了哪些内容?采用了哪些研究方法?得到了什么结论?还有哪些值得进一步研究的地方?而这些信息又会有助于我们回答如下问题:如果我们来设计实验的话,大致应该如何设计?如果我们来撰写论文的话,应该如何编排图表?当然并不是说一次阅读就要掌握以上所有信息,而是根据当前所处的研究阶段,选择性地重点关注某些相关信息。

刚开始不熟悉这个领域,有些图片可能第一次见到,不太理解,这没关系。等这个文档中的图片足够多,浏览的次数足够多,你会发现有许多图片都是相似的,都是要说明类似的问题,自然就会慢慢理解。正所谓“读书百遍,其意自现”。我觉得这和人脑基于神经网络的学习方式有关。等我们看了几十上百篇后,可能会发现已有文献存在的一些共性之处,有些图片似乎是同一个类型的。比如文献刚开始可能会出现材料的合成示意图,之后会有材料的各种表征结果(如X射线衍射、扫描电子显微镜、透射电子显微镜分析等),随后会出现材料或者材料装配成器件后的性能表征结果(如电化学曲线用于分析材料的活性、电池测试系统用来测试材料组装成电池后的性能),以及一些模拟计算等。

当达到这样的理解程度后,我们就可以根据目前研究所处的阶段,有侧重地浏览相关图片。例如,如果处于早期的选题阶段,我们可能会更关心反映作者研究思路、创新点或者贡献的图片,用以启发自己的思路。而涉及具体实验步骤以及所得结果相关的图片,此时可能相对来说,可以浏览得快速一些。研究阶段的初期,我们可能会涉及实验方案的制定,想了解需要做哪些实验可以使我们对这个课题的研究有个全面深入的探索。其实这也涉及我们后期的论文框架。这时,我们可以集中浏览和学习涉及作者实验结果的图片,以启发和形成自己的研究框架。同时,当开展某一个具体实验时,例如电池的循环性能测试,你也许想知道自己应该怎么样设置这个实验参数。那么你可以集中只浏览各个文献中涉及电池循环性能测试的图片,以了解你该如何设置实验参数(如放电深度、循环次数/时间、充放电电流/电压)从而便于开展你的实验以及将你的实验结果和已有的文献结果进行对比。类似地,在研究阶段的中期和后期论文写作阶段,我们陆续获得了一些实验结果,想对其深入分析。这时候,我们也会涉及对同类图片的集中浏览,以快速了解已有文献所取得的结果特点以及是如何分析讨论的,从而了解如何分析我们所取得的结果,以及将我们的结果和已有结果进行对比分析讨论。

为了实现之前所说的集中浏览,可以采用如下的第三步操作:再新建一个文档(假设命名为《2号文档》),将同一类型的图片都归纳整合在这个文档中,便于我们迅速对比和浏览同类图片,快速获得感兴趣的信息,从而不被一些我们暂时不想重点浏览的图片信息分散精力和耗费时间。比如将关于示意图或者材料合成类的图片都归纳在一起,那么你就可以集中浏览示意图或者材料合成这种体现文章研究体系的工作了。类似地,你也可以把涉及材料表征的,或者性能测试的,或者第一性原理计算的分别归纳在一起。在这个过程中,可以将先前提到的《1号文档》镜像成2份,比如《文件1.1》和《文件1.2》。其中《文件1.1》是完整的图片集,这里的图片是按照文献中出现的顺序排列的,通过顺序浏览某篇文献的图片,你可以知道作者是按怎么样的逻辑开展和表达工作的。《文件1.2》中的图片则是用于归纳的,比如以剪切的方式将相关图片从《文件1.2》转移到《2号文档》中,这样一来,《文件1.2》的图片会越来越少,剩余的就表示还未分类的图片。

我一般阅读的习惯是先根据自己本次需要关注的重点,浏览《2号文档》中已归纳的部分,随后看《文件1.2》中未归纳的部分是否还有什么值得参考的信息。如果有的话,再将其归纳到《2号文档》中。之前提到,整合这些文件的过程中,需要用一种简单的方式知道每个图片是来自哪篇文献的,以便于寻找到图片出处。我们稍后会提到,这对之后的工作有很大帮助。通常我是用计算机程序来实现这点的。

如图\ref{fig2-11}所示,文献中的文字部分同样可以采用这种方式进行切块、归纳和重组。例如,将一批相关文献的前言整合在一起,可以很好地帮助我们理解这个领域的现状和存在的问题,以及作者针对这个问题的思考和探索,这对于我们寻找自己的实验方案以及后期的写作也会有帮助;将各个文献中的实验描述部分整合在一起,可以帮助我们开展自己的实验,细化实验方案;将文献中针对某一结果的讨论整合在一起,可以帮助我们了解应该如何去全面深入地讨论所获得的实验结果,同时也可以方便后期将我们的实验结果与已有结果进行对比。

\begin{figure}[!htb]
\centering
\includegraphics[width=0.9\textwidth]{fig2-11.png}
\caption{将相近的内容整理、合并、归纳,以便于后期使用}
\label{fig2-11}
\end{figure}


以上就是我概括的“图片/文字切块-归纳-重组阅读法”。此外,上述操作还可以进行各种变通。譬如之前介绍的是一次的切块-归纳-重组,随着研究的深入,可能会发现原先的切块过粗了,那么可以进行二次的切块-归纳-重组。我们之前反复提到,这些文档会伴随着你整个科研生涯并持续起作用,因此需要培养一个好的文档管理习惯,尤其当文档数量增多时,不良的习惯会导致不少问题的出现。比如一个容易出现的问题是文档之间内容有大量重复,这会导致在浏览时经常浏览重复内容,造成时间的浪费。所以在之前的介绍中提到过,有的文档之间是通过剪切的方式形成的,这样确保它们之间不会有重复内容。当然剪切过程中,容易有误操作,比如剪切完毕后,以为自己复制了,但其实没有复制,或者保存过程中电脑死机了等等,这些都要尽量避免。我是采用计算机程序比对+实时保存来避免这些问题。又比如如果所有的文档都是以剪切重组的方式生成的,虽然很方便你集中浏览同类信息,但是如果以后有一天你想浏览单篇文献的内在逻辑就不方便了。所以我先前介绍中,提到过有一份文档是未被打乱顺序的(《文件1.1》),这份文档就可被用来参考单篇文献的内在逻辑,还有文档增多后,也要尤其注意文档的命名方式,要知道这些文档是怎么来的,和其他文档是什么关系。此外,以后写文章或者讨论结果时,你经常会涉及需要知道这个图片、这个结果或者这段文字是出自哪篇文献的情况,以进行合理的引用,或者对比讨论等。那么同样需要一个机制能知道被你切块重组完的信息的源头是什么。我通常是在每个切块信息之前,通过一个特殊符号,引出关于这个信息(图片或者文字)的原始文献来源。而且通过这种方式,在剪切过程中,我也很容易通过比对特殊符号数量来确认操作过程中有没有造成信息的缺失。当然,这样的方式不只适用于图片,同样适用于文字的整理。

精读代表性的文献和书籍、开展实验等,也需要同步进行。根据我个人的经验,在未开展研究前,对很多问题往往没有很直观的认知,甚至有些似是而非的理解,因此一边开展实践(研究),一边带着问题去阅读学习,会更有效果。阅读过程中同样要养成本书反复提到的追根问底的源头思维模式,从源头的基本原理出发,思考作者的观点是否有足够依据,而不是不假思索地全盘接受。

\vspace{0.5cm}
{\kaishu 二、文献跟踪}
\vspace{0.5cm}

根据本章第一节介绍的方法,研究开展初期会通过合理的检索式构造,对相关领域做一次全面的检索,形成一个相对完整的知识库,在此后的研究中,一旦这个检索式产生了新的检索结果,我们可能想第一时间知道。但每天检索过于麻烦,毕竟大多数情况,不是每天都有结果被更新的。那么我们就可以对这个检索式的检索结果设置一个提醒功能,每当这个检索式产生了新的检索结果(说明该领域有新论文发表),就通过邮件或者RSS订阅来提醒我们(现在很多检索引擎如Scopus、Web of Science都有此类功能)。必要的时候,再执行一次检索并将新增记录导人我们的知识库,以不断完善知识库。

\section{确定自己的研究}
通过第一节描述的方法并结合第二节提到的文献阅读方法,我们相当于是在完善研究领域的地图(图\ref{fig1-1}),会逐渐明白:①这个地图的边界在哪里,即哪些领域目前还未被探索;②边界内部存在疑问的地方是什么,即哪些领域虽然被探索了,但是还存在争议、矛盾或者不清楚的地方;③边界内部已经被公认的一些“事实”的依据或者证据等是什么,是否真的完全确凿。

对以上三点问题的回答,其实就是确定自己研究方向的方式。这其中需要有追根问底的源头思维,以及注意区分观点(opinion)和事实(fact)的思维方式。

\vspace{0.5cm}
{\kaishu 一、 追根问底的源头思维}
\vspace{0.5cm}

新手在阅读文献过程中,容易犯的一个错误是将作者的观点不假思索地认为是事实,并以此来构建自己的“地图”,但这样的地图很可能会误导未来的工作。因为之前第一章讨论过,科学探索的过程中,本身就伴随着人类认知的不断提升,这其中也包括对过去认知的改变,即以前普遍被认为是正确的观点,后来被认为是不完善甚至是错误的。当然,我认为科学研究的过程中没有绝对的正确和错误,所谓的正确和错误的描述往往是要伴随大量定语和状语的约束。比如,当我们说质量是不变的时候,还得加上低速宏观物体等约束条件,而且还得定义这个不变是指质量变化小于多少幅度算是不变的。因为我们现在知道即使是低速宏观物体,它的质量也是会随着运动而改变的,只是我们无法感知到而已。但如果通篇都采用这样的表达方式,未免太过累赘,所以涉及此类描述的时候,本书也会简单地用正确和错误等词汇指代。同样地,观点和事实也存在这样的问题。我个人觉得绝对的事实可能是个理想状态,所以也许我们认为的事实可以足够逼近实际情况,但不见得它完全等同于客观真理。我还没太深入地思考过其中的原因,也许和量子物理中提到的观察者效应有关。甚至进一步追究下去,许多语言的表达本身就很难精准达意。人类目前所产生的语言中,其实有许多词汇,大家可能都能感知到它所代表的意思,但要给出精准的定义,却很困难。比如意识这个词也属于这个范畴。人们都明白意识这个事物的存在,但很难精确表达清楚这个概念。这就如同盲人摸象一样,每个人可能都表述了它的一部分,但并不是全貌,每个人的表述都有可能是部分正确的,但互相就会觉得对方是片面甚至是错误的。比如摸着大象耳朵,说大象像扇子的人会觉得摸着大象的腿,说大象像柱子的人是错的。所以为了便于交流,当涉及正确、错误、观点、事实等词汇时,本书还是采用大家约定俗成的对于词汇的理解方式进行展开,但这并不意味着我没有意识到这其中存在的巨大复杂性。我认为阐明这点在科学研究中是非常必要的,以后我们会看到大量提升人类认知的成果就是在人们习以为常的被忽略的正确和错误的边界、观点和事实的边界上产生的。

正是由于以上提到的这些复杂性,我们在构建地图的过程中,要有寻根问底的源头思维方式,要从源头考虑我们目前脑海里得到的这个信息是如何来的?它是基于作者的某个实验结果,还是来自于对这个实验结果的分析?前者很大概率属于事实类范畴,后者一般就属于观点类范畴。事实和观点之间还隔着一座桥梁,一些新手往往会忽略这个桥梁,默认作者的观点是正确的,并以此来构建自己的知识和思维体系。但是,很多情况下,事实和观点之间的推导论证并不一定完全合理,也就说这样的事实并不必然导致这样的结论,事实和观点之间的桥梁可能是座“断桥”(图\ref{fig2-12})。对于一些基于实验的论文来说,产生这种的原因包括实验本身设计的不合理性,或者对于实验结果的解释有误区或者偏差。

\begin{figure}[!htb]
\centering
\includegraphics[width=0.9\textwidth]{fig2-12.png}
\caption{关注事实和观点之间的推导过程}
\label{fig2-12}
\end{figure}


以下案例可以很好地说明这个问题。2016年,Science上发表了一篇有关整联蛋白$\alpha_4\beta_7$抗体的文献\footnote{Byrareddy S N, Arthos J, Cicala C, Villinger F, Ortiz K T, Little D, Sidell N, Kane1M A, Yu J, Jones J W, Santangelo P J, Zurla C, McKinnon L R, Arnold K B, Woody C E,Walter L, Roos C, Noll A, Van Ryk D, Jelicic K, Cimbro R, Gumber S, Reid M D, Adsay V,Amancha P K, Mayne A E, Parslow TG, Fauci A S, Ansari A A. Sustained virologic control in SIV+ macaques after antiretroviral and $\alpha_4\beta_7$ antibody therapy. Science, 2016,354: 197-202.}。其中的研究结果表明,对患SIV(猴免疫缺陷病毒)的恒河猴采用抗逆转录病毒药物疗法(ART)结合$\alpha_4\beta_7$抗体注射,即使在停止治疗后,也能有效地控制血液病毒含量并重建免疫系统。这意味着人类向战胜HIV(人类免疫缺陷病毒)又迈出了一步,在当时引起了广泛关注。然而,在三年后,对当年研究结果的质疑纷至沓来。同样是在Science上,仅2019年一年,就有三篇文献对当时的研究结果表达了不同意见。Di Mascio等\footnote{Di Mascio M, Lifson J D, Srinivasula S, Kim I, DeGrange P, Keele B F, Belli A J,Reimann K A, Wang Y, Proschan M, Lane H C, Fauci A S. Evaluation of an antibody to $\alpha_4\beta_7$ in the control of SIVmac239-nef-stop infection. Science,2019,365:1025-1029.}对当年的实验进行了重复,发现注射了$\alpha_4\beta_7$抗体的实验组和对照组在淋巴结和直肠组织中的病毒含量并无明显区别。此外,他们对当年使用的病毒进行了基因测序,发现当年使用的病毒实际上是SIVmac239-nef-stop,而不是野生型的SIVmac239病毒。Abbink等\footnote{Abbink P, Mercado N B,Nkolola JP, Peterson R L, Tuyishime H, McMahan K, Moseley E T, Borducchi E N, Chandrashekar A, Bondzie E A, Agarwal A, Beli A J, ReimannKA, Keele B F, Geleziunas R, Lewis M G, Barouch D H. Lack of therapeutic efficacy of anantibody to $\alpha_4\beta_7$ in SIVmac251-infected rhesus macaques. Science, 2019, 365:1029-1033.}则在50只感染野生致病型SIVmac251病毒的恒河猴中进行了类似的研究,发现$\alpha_4\beta_7$抗体给药对恒河猴SIVmac251感染模型并没有产生治疗效果。Iwamoto等\footnote{Iwamoto N, Mason R D, Song K, Gorman J, Welles H C, Arthos J, Cicala C, Min S, King H A D, Belli A J, Reimann K A, Foulds K E, Kwong P D, Lifson J D, Keele B F, Roederer M. Blocking $\alpha_4\beta_7$ integrin binding to SIV does not improve virologic control, Science, 2019,365:1033-1036.}也在30只恒河猴中重复了类似的实验,且与2016年的实验使用的是同样的病毒与相同的ART方案,但是也没有发现$\alpha_4\beta_7$抗体注射与ART联合疗法在体内会有持续的病毒控制。

从以上案例可以看到,即使依托于实验所产生的作者观点都可能存在疑问或者需要被考证(可能是实验无法重复、可能是实验本身设计不够合理),更何况还有大量观点其实都未经考证或者没有确凿实验证据支撑。当然,我们构建思维地图的过程中,不可能穷尽和测试每一种事实,也很难将每一个观点的来源依据都彻底了解清楚,不可避免地会基于大量他人观点来构建我们的地图。这里的讨论并不是说我们要避免使用他人观点来构建自己的地图,而是要意识到由于我们地图的构建往往是要大量依托于已有观点,那么这些观点都有之后被修正甚至被纠正的可能性。就好比第一章说到的航海的例子,有些是他人描述的新大陆,我们可以将它们放在我们的地图中,但我们心里要清楚,它们有可能是不存在的或者有可能位置还不够精确,而不是不假思索地100\%地把它当成一个确定性事件。所以,我们在阅读书籍,文献的过程中,需要有独立思考过程,具备反思脑海中观点、观念和想法来源的能力,以检验这些想法是否还能有效支撑我们的地图构建,提升自己基于事实或者证据类事物的思考能力,以及提升自己根据同行对于事实的合理推导,以形成自己地图的能力,切莫人云亦云。

在过去指导学生的过程中,当问及学生这样的结论是怎么得来的?为什么会有这样的想法?一些新学生经常会比较诧异,会觉得文献就是这样讲的,大家都这么觉得,这个还要问吗?不都是这样的吗?经过以上的详细讨论,相信我们会对这样的思维定式、误区或者说盲点有所改观。

这种区分观点和事实的思维方式其实适用于方方面面,包括之后我们要讨论的具体的研究工作开展、论文写作和发表等等。

经过如上的独立思考和深度思考过程,我们就可以结合自己的兴趣包括导师对我们的指导,开始在这个地图中进行愉快的探索活动了。我们可以在地图边界以外区域探索,去做一些别人没做过的事情;我们也可以去研究地图中还存在争议的一切区域;我们甚至可能可以获得一些和前人共识不同的认知。

还是以第一节中提到的演示案例为例,如果我们准备探索锌空气电池中的 catalyst(催化剂)领域,可以研究一些新的之前没有被报道过的催化剂体系,也可以针对已经研究过的催化剂体系中,还存在的不清楚或者有争议的问题进行研究。除了在阅读已有研究的过程中,需要有追根溯源的思维方式,当我们进行研究的时候,也需要问自己为什么要做这个研究?这个研究的意义是什么?比如我们为什么要去探索新的催化剂体系呢?是想解决已有催化剂的什么问题?是要提高性能?降低成本?拓宽它的应用面,可以做传统催化剂不能做的事情?或者是发现一些新现象?还可以进一步追问下去。比如你想做的是提高催化剂的性能,那么可以继续追问你想提高的是催化剂哪些方面的性能?这些性能受限背后的根本原因是什么?理解限制性能的原因,那么我们就有办法针对这个原因,寻找解决思路。一般而言,要解决一个问题,大部分精力往往是花在如何提出和描述出一个正确的问题,找到产生这个问题的原因。如果真的能明白产生问题的原因,往往离解决这个问题也不远了。但有时由于思维惯性等因素,人们遇到问题,会条件反射式地去尝试各种方案解决这个问题,处于一种问题解决型的“应激模式”状态中,而没有花时间去深度思考自己第一反应所产生的针对这个问题的解决方案是否是正确的,究竟为什么会产生这个问题。条件反射式地解决问题的技能当然是必要的,它在人类生存和繁衍中起了重要作用。但是在科学研究过程中,由于思维误区、思维定式、思维盲点等因素,这种类似条件反射的问题解决技能并不总是有效,所以不能100\%地完全采用这种方式进行探索。很多时候,我们越努力,反而越找不到解决方式,是因为我们行驶错了方向。

例如,2018年高考全国卷Ⅱ的作文题给出了下面一段材料:“二战”期间,为了加强对战机的防护,英美军方调查了作战后幸存飞机上弹痕的分布,决定哪里弹痕多就加强哪里。然而统计学家沃德力排众议,指出更应该注意弹痕少的部位,因为这些部位受到重创的战机,很难有机会返航,而这部分数据被忽略了。事实证明,沃德是正确的。

这道作文题所提到的材料就很好地阐明了我们先前提到的观点,即有时候人们第一反应所产生的思路,据此而选择的方向并不一定是正确的。因此需要具备源头思维的能力,比如针对这个案例,我们需要考虑为什么要加强对战机的防护?防护的根本目的是什么?→是为了让战机能够返航以再次作战,并保障飞行员安全或者无法返航的时候,可以确保飞行员安全等。→那么基于这个目的(假设是让战机顺利返航),应该如何加强战机呢?→应该要找到是什么原因导致战机无法返航的?(这也是我们先前交流的,要解决一个问题,往往首先要将大部分精力放在找到这个问题背后的原因)→上面这个问题可继续追问为:当战机什么部位受损时会无法返回?即什么部位受损时,会产生致命的损伤?当到这一步时,解决方案已经比较清楚了,完全不同于“哪里弹痕多就加强哪里”的思路。

\vspace{0.5cm}
{\kaishu 二、发现和成长的观念而非对错观}
\vspace{0.5cm}

同时需要注意的是,对于科学研究,我觉得很难简单地用对错来评价。由于科学研究本身充满未知和不确定性,之前提到有时我们会行驶错方向,或者做出来的实验结果和预期不符,但是这也不见得代表就是失败,也可能那个方向反而会有些更惊人的发现。而能否将不如预期转化为“成功”,就需要具备这种发现和成长的思维观念。一旦具备这种思维方式,在科学领域的探索中,几乎就没有失败之说了,一切都是收获,失败本身也是种收获。知道此路不通,并且为什么走不通,也是种收获,对后人也是一种贡献。也许我们以后会看到这样的文献,这个材料的性能可能不够好,没法在这个领域获得应用,但是这篇文献也许会分析清楚为什么这个材料的性能不好,那么理解了为什么之后,后人可以基于这样的分析进一步改进,提出更好的材料设计方案。

导电塑料的发现也源于日本科学家白川英树的一位学生在做乙炔聚合反应的实验研究时,误将高于正常用量浓度1000倍的催化剂加入反应体系。当发现实验参数错误后,白川英树并没有完全忽略这个实验,而是留意观察并分析了获得的产物,通过之后和其他科学家Alan G. MacDiarmid、Alan Heeger等的合作,最终获得了导电聚合物\footnote{参考来源:https://www.nobelprize.org/prizes/chemistry/2000/macdiarmid/biographical/.}。

日本科学家田中耕一在研发能通过有效吸收激光能量实现大分子无损电离的基质时,在一次实验中,他误将甘油当作丙酮,制成了甘油-钴超细粉末的混合基质。由于认为将昂贵的钴超细粉末丢弃十分浪费,他准备在甘油挥发之后,回收利用钴纳米粉末。为了加速甘油的蒸发,田中耕一打开了激光束来照射甘油-钴超细粉末混合物。为了尽快确认甘油是否去除,田中耕一始终运行着光谱仪并观察着结果。然后,他观察到了一个从未见过的信号峰。田中耕一注意到了这一现象,并重复了多次实验,发现信号峰的位置始终没有改变。基于这一现象他开展了系列实验并推动了基质辅助激光解吸/电离技术的发展\footnote{参考来源:https://www.nobelprize.org/prizes/chemistry/2002/tanaka/facts/.}。

综合以上讨论,确定自己的研究方向可以概括为三个步骤:①提出“好”的问题;②思考背后的原因;③根据自己思考的原因,提出验证的设想。其中第一个步骤又有多种方式,一种是以问题导向的方式不断寻根问底地追问目标领域的问题是什么,产生这一问题背后的原因又是什么,已报道工作解决到什么程度,还需要解决什么问题。当然,我们做任何事情的时候,也不要过度拘泥于一种思维模式。以上更多的是基于左脑的理性思维方式,但有时候,右脑的思维方式也会对寻找到我们的选题有很大帮助,比如图形化的方式、灵感闪现、跨领域交叉等。

在探索过程中,以上三步会不断相互影响和调整。经常会发现实验/研究结果不符合我们最初的想法,这里有几种情况。

(1)实验结果的可重复性非常差,数据结果很离散,无法支撑我们当时提出的设想。这背后的原因又有多种,一种常见的原因是实验操作过程中,未能很好地控制各个实验步骤,导致最终实验结果的无法重复。我们设计的实验方案,往往是由一系列串联的实验步骤所组成的,所以任何一个实验步骤考虑不周,都会导致最终的结果偏差。比如每个步骤的出错概率是10\%,如果整体实验是由10个步骤构成的,那最终出错概率可达65\%了。这点我们在之后的章节中还会深入展开,有很多因素可能是我们当时都习以为常,觉得不会是问题,所以被完全忽略的。另一种可能性是我们的实验设计和操作本身是没有问题的,这个实验结果的无法重复或者说结果离散就是其本身的实验现象。这其中可能蕴含着重大的科学发现,很可能值得我们进一步探究背后的原因。假想我们做电子双缝衍射这个实验,同时想观察很少量的电子分布情况的话,我们就会发现每一次实验,它们的分布都几乎是不同的,但这恰恰就是“正确”的实验结果。而对这一实验结果背后原因的进一步深入讨论,可能会对我们以往的认知有重大改变。

(2)实验结果和预期不符,但可重复性很好,往往这种情况背后同样会蕴含着重大的发现。既然实验结果可以重复,就说明背后是有某种机制在支配的,而这种机制和我们最初所想的是不一样的(当然要排除实验设计错误之类的),那么对这种现象的进一步深入研究很可能会改写我们的传统认知。比如第一章我们提到的$\alpha$粒子散射实验就是这样的。

所以,无论实验是否符合预期,只要我们掌握了科学的思维方式和观念,都会有所收获,而这些收获都会伴随着我们认知的进步、我们的成长。因此,本科——硕士——博士阶段的求学过程其实是个非常激动人心,也非常有成就感的一个过程。

\vspace{0.5cm}
{\kaishu 三、拥抱新事物}
\vspace{0.5cm}

过去指导学生过程中,在交流选题时,面对新鲜事物,不少新学生会有畏惧的心理状态,说出类似这样的话:“老师,这个我没学过”“老师,这个我没听说过”“老师,这个没人教过我”“老师,我从来没做过这个事情”“老师,这个我不知道”“老师,书里没说过可以这样做”“老师,这个太难了”“老师,别人说做这个事情没有意义”。人们对未知事物有畏惧心理,这很正常,这也是进化赋予人类的一种保护模式。经过以上的讨论,我们应该会充分明白,我们未来要去做的科学研究本来就应该是你不知道、你还不了解、你没接触过,或者存在争议的,过去还没人教过你的这些事情。而且不仅是你不知道、你没接触过,甚至可能是其他人也不了解的事物,正是要通过你的探索和深度思考,使这些不清楚的事物逐渐变得清晰,这就是你对于整个世界的贡献,这其实是件非常有成就感的事情。所以,未知的、你不了解的,才是一个“正常”的状态。要相信你的导师会带领你在未知领域进行探索,用发现新事物的欣喜替代你原先的畏惧。在以下章节,就会交流如何进行探索活动。
