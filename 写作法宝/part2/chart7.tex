\chapter{统一性}
要写好就得实践,此乃至理名言,其言名在于其理真。要写好的唯一办法就是强迫自己保证每天都写。

你若是为报社工作,要求每天写两三篇文章,六个月后你的写作能力就会大为改观。这并不是说你就此成为作家,你的文笔也许还会杂乱无章、充斥陈词滥调,但你在练习如何将语言付诸笔端、增强自信心、发现最常见的问题。

一切写作最终都是在解决问题。这个问题也许是到何处去寻觅事实,或是如何组织材料;也许是方法、角度,或是语气、风格。无论如何,你都得面对问题,解决问题。有时你会苦于找不到解决方案而绝望。你会想,“就是活到九十岁,我也摆脱不了这个困境。”我自己就常这么想。一旦解决了问题,我就像外科医生一样去除了第500个阑尾。这些我都经历过。

统一性是好的写作的保障。所以首先要明了各个部分的统一性问题。统一性不仅能避免读者晕头转向、误人歧途,还能满足读者潜意识中渴望秩序的需求,保证一切都在掌控之中。因此,在诸多变量中做出选择,坚持你的选择始终如一。

选择之一是人称代词的统一性。你是以亲历者的第一人称,还是以旁观者的第三人称写作?或是以第二人称写作?这第二人称可是体育记者的宠儿,是海明威惯用的技法。(“你知道这次一定是巨人之间最刺激的冲撞,之前你绝对没从直播间见过,不过你不再是乳臭未干的臭小子了,一定扛得住!”)

另一个选择是时态的统一性。大多数人习惯于用过去时写作。(“有一天我去了波士顿。”)而另一些人用现在时写作得心应手。(“我现在就坐在扬基有限公司的餐车里,火车缓缓进入波士顿站。”)但如果时态换来换去,就会使人手足无措。这并不是说你只能用一种时态;时态选择的最重要目的是能够使作者处理好时间的种种变化,从过去到假设中的将来。(“从波士顿车站打电话给母亲,我这才意识到假如我告诉她我将到来,那她就会等我了。”)但是,面对读者你必须选定一种主干时态,在这之间可以瞻前顾后。

还有一种选择是语气的统一性。你可能想用悠闲随意的语气面对读者,就像《纽约客》所竭力打磨的那样。也许你想用一本正经的方式向读者描述一件大事,或陈述一系列事实。两者都行。其实,任何一种语气都行,但是不要两三种混搭。

对于还没有学会掌控的作者来讲,这种要命的混搭司空见惯。游记是最明显的例子。“我和妻子安娜一直想去香港,”一位作者这样开始,往事历历在目,“去年春天的一天我们偶然看见一幅航空公司的招贴画,于是我说,‘咱们去旅游!’孩子们都长大了。”这位作者这样写,然后继续兴高采烈地详细描述他和妻子如何转停夏威夷,如何在香港机场兑换钱出笑话,如何找到旅店。好!他亲身带我们旅游,我们也切身体会到他和妻子的经历。

但突然他转向旅行手册。“香港可为好奇的观光客提供众多令人神往的体验,”他这样写。“你可以从九龙乘华丽的渡轮,观看无数舢板在拥挤的海港疾驶,而感到惊叹不已;或者花一天时间逛一逛传奇的澳门街巷,体验其作为走私密谋老窝的斑斓历史。你还可以乘老式缆车爬上……”然后作者又写回到自己和妻子找中餐馆吃饭,一切又不错。大家都喜欢那儿的饭菜,随后作者又讲述了一段自己的历险经历。

之后作者突然在写导游手册:“进入香港,首先要有有效护照,但无须签证。一定要接种甲肝疫苗,按照医嘱看是否需要接种伤寒疫苗。香港的气候适宜,只是七月和八月……”作者本人和妻子安娜都不见了,很快——大家也都不见了。

并不是说疾驶的舢板和注射肝炎疫苗不能写进去。烦扰读者的是作者并没有确定文章的写作样式,也没有明确如何面对读者。他以不同的面貌,冲着读者就来,想到什么材料就用什么材料。其结果不是掌控好材料,而是被材料所掌控。假如他花一点儿时间确立好统一性,一切就不会这样了。

因此,写作前问自己几个基本问题。例如:“我要以什么身份面对读者?”(记者?信息提供者?普通男人或女人?)“我要用什么人称和时态?”“我要用什么文体?”(客观报道?个人但正式?个人而随意?)“我对材料采取什么态度?”(介入式?疏离式?裁判式?反讽式?娱乐式?)“我要写多少?”“我要强调的唯一要点是什么?”

最后两个问题特别重要。多数非虚构作家都有定性情结。他们觉得自己有义务使自己的文章具有最终的定性结论,这关乎题材、荣誉以及写作的神圣性。这种反应的确令人称道,但并不存在什么最终结论。今天你所认为定性之事晚上就会变为不定,而作家孜孜以求每一个事实细节,到头来却发现自己只是在追逐彩虹,永远坐不下来写作。没人能“针对”什么笼统的事写出一本书或一篇文章。托尔斯泰并不能针对战争与和平写出一本书,梅尔维尔也不能针对捕鲸写出一本书。他们所做的,只是针对时间和地点,以及针对那些时间地点中的个别人物,缩减自己的选材,描述出统一的故事——一个人追逐一条鲸鱼。每一个写作计划在开始前都必须缩减。

因此,往小里想。首先确定题材中的哪一角是你想啃掉的,然后全力以赴,仅止于此。这也是精力与士气的问题。庞大的写作计划会耗尽你的热情。热情是你保持写作得以进行和保持吸引读者的动力。你的激情一旦开始退潮,读者会第一个知晓。

至于你想强调什么要点,每一部非虚构作品都应该留给读者一个他们自己从前没有过的发人深省的想法。不需要两个或五个——只需要一个。那么首先确定你想留在读者脑海里的唯一要点是什么。这不但会指明你该走的路,你所希望达到的目的地,还会影响你所采取的语气和观点。有些要点最好是用严肃认真的风格,有些含而不露,另一些则幽默诙谐。

一旦确定了各部分的统一性,你就能融任何材料于框架之中。假如那位去香港的游客选择完全用聊天的语气描述他和妻子安娜的经历,他就会找到一种自然的方法融会贯通地向大家讲述九龙渡轮以及当地的天气情况。他个人的秉性和目的也将完整无缺,文章也会浑然一体。

但经常发生这样的事,写作之前自己已经确定好了一切,结果却发现所确定的并不对。素材开始将你引向未曾预见的方向,而你觉得用另一种语气写更舒服。这很正常——写作行为本身会激发一连串你从未预想过的念头或回忆。假如你感觉对,就不要逆流而上。相信素材,它会引领你进人你并无意踏足的地带,那里的气氛对头。顺势调整你的风格,继续前行直达你的目的地。不要做预设计划的囚徒。写作并不恭维蓝图。

假如拘泥于原计划,文章的后部分就会与前部分严重脱节。但至少你知道哪一部分忠实于自己的本意。这时就可以进行修补。回到开头,重写那个部分,这样你的基调和风格就会从头到尾统一起来。

这并没什么丢人的。剪刀和糨糊——或者电脑上相同的功能——是值得尊敬的作家的工具。切记:无论你组合得好坏,每一个部件都必须统一到你最终所搭建的大厦之中,不然整座大厦就会坍塌。