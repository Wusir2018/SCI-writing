\chapter{开头与结尾}
文章中最重要的句子是第一句。如果第一句不能吸引读者继续读到第二句,那么你的文章就死定了。而如果第二句不能吸引读者接着读到第三句,那也同样是死定了。句子如此排列,每一句都向前拖着读者,直到读者上钩,作者就是这样建构至关重要的文章单位:“开头”。

开头应该多长?一段还是两段?四段还是五段?没有现成的答案。有一些开头只用几句鱼饵就巧妙地钩住读者;另一些则慢悠悠地走上几页,形成缓慢但稳定的引力。每一篇文章都提出不同的问题,唯一有效的检验是:开头是否好使?你的开头不一定是最佳开头,但如果它完成了该做的工作,就要心存感激,继续写。

有时候开头的长度依你为其写作的读者而定。文学评论的读者期望作者以某种发散的方式开始,他们会跟进这些作家,欣赏和玩味他们将在文章的何处露面,以及这些作家如何悠闲地转悠,最终才到达目的地。但我要告诫你,不要指望读者会跟进多久。读者想很快知道文章里有什么是他们感兴趣的。

因此文章的开头首先必须立即抓住读者,迫使他继续阅读。它必须诱惑读者,给读者以新鲜感、新奇感、悖论、幽默、惊奇,或者与众不同的想法、有趣的事实、某个问题等。什么都行,只要它能激发读者的好奇心,拽住他的袖子。

其次,开头必须言之有物。它必须提供坚实的细节,告知读者为何写这篇东西以及读者为何一定要读它。但不要耽搁在原因上。多哄诱读者一会儿,保持他的好奇心。

继续建造。每一段都应该增益前一段。多考虑增加坚实的细节,少考虑取悦读者。但要特别注意每一段的最后一句话——那是迈向下一段的关键跳板。尽力给这句话多一点点幽默或惊奇,就像在单人喜剧的惯例中时而出现的“妙语”。让读者微笑,这样你就至少赢得读者再读一段。

让我们看几个开头,它们的节奏不同,但所保持的紧迫感却类似。我先以自己的两篇专栏文章为例。这两篇最初登在 《生活》和《看》杂志上。根据读者的评议,这两份杂志的消费者主要的阅读场所是理发店、美发沙龙、机场,以及医生的诊室(“有一天我在理发,看见了你的文章”)。我提此事是想提醒大家,大多数期刊阅读是在吹风机下而不是在台灯下进行的,因此作家没有多少时间可以用来闲扯。

第一个例子是一篇题为《阻止福特鸡肉香肠》的文章开头:

我常想知道热狗里面有什么。现在我知道了,但却希望并不知道。

只有两个短句,但读者不继续读第二段都难:

我的麻烦开始于农业部颁布热狗成分之时——这些成分就是指热狗中法律上允许的所有成分——农业部是应家禽业要求而颁布热狗成分的,以放松其条件使鸡肉也可以包括在内。换句话说,福特鸡肉香肠能在法兰克福香肠的领地找到幸福吗?

以上用一句话解释了该专栏文章所基于的事件。然后是一段妙语来恢复轻松的语调。

农业部就此分发了问卷,得到的1066份回答多数带有敌意。按照这些回答,这个想法本身就不可思议。公众的情绪被一位妇女恰到好处地激发起来,她大声疾呼:“我才不吃家禽肉呢,绝不。”

又一个事实,一个微笑。每当你有运气得到一个这样有趣的引语时,想办法用上。这篇文章接着细化了农业部所说的可以做热狗的成分——一连串的材料,包括“牛、羊、猪或者山羊的可食性肌
肉,可取自膈膜、心脏或者食管部分……(但不包括)唇部、口鼻部或者耳部肌肉”。

从这里继续向前——期间不由自主地稍稍探讨一下食管肌肉问题——然后进人这场家禽业利益与法兰克福熏猪牛肉香肠业利益之间的争议,接着引入要点,那就是美国人会吃任何哪怕是稍微有点儿像热狗的东西。最终的暗示是,推而广之,美国人不知道,或者不在乎自己吃的食物里究竟是什么。该文的风格保持轻松,带一点儿幽默。读者开始受其离奇的开头吸引,但结果发现内容要比预期的严肃。

下面是一个慢节奏开头的例子,它吸引读者的主要是好奇心而非幽默感,文章叫《感谢上帝,球迷们》:

按常规,谁都不想看第二眼——甚至连一眼都不想看——来自棒球投手伯利·格兰姆斯出生地威斯康星州澄湖那片滑溜溜的榆树皮,但是那片树皮现在陈列在纽约州库珀斯城国家棒球博物馆和荣誉堂里。标签的说明显示,这就是格兰姆斯在比赛中嚼的那种树皮,意在“增加唾液以便抹在球上投唾沫球。棒球湿的时候会以迷惑人的方式沿弧线飞向本垒板”。这也许是美国今日可提供的最无趣的事实之一了。

但棒球迷们不能以常理来衡量。我们沉迷于比赛的细节,凭着自己对曾看过的球手的比赛记忆,在余生里对此唠叨个没完。因此没有一条细节是无关紧要的,每一条细节都将我们重新与球手相连。我这个岁数正好能记得当时的伯利·格兰姆斯,还有他那湿乎乎的球迷惑性地飞向本垒板,而当我发现他的树皮时,便全神贯注地研究起来,就好像遇见了罗赛塔石碑“原来他是这么干的”,我边想边仔细观察那片诡异的植物残片。“滑溜溜的榆树!真见鬼。”

这只是我儿时闲逛那座博物馆时所遭遇的几百件事之一。恐怕没有其他博物馆对我们的过去会产生如此特别的心路历程……

读者现在被稳稳当当地钩住了,作者任务中最难的部分到此结束。

引这个开头的一个原因是提醒大家,文章的救星常常并不在于作者的风格,而是在于某些作者能够发现的奇异的事实。我去到库珀斯城,在博物馆里花了一整个下午记笔记。四处都有怀旧的搅扰,我崇敬地盯着卢·格里克的衣帽箱和博比·汤姆森的赢球棒。我坐在取自波罗球场的大看 台座椅上,用未钉鞋钉的鞋底刨着来自埃贝茨球场的本垒板,尽心尽职地抄写所有可能有用的标签和说明。

“这些是泰德完成环绕各垒跑时踏本垒板穿的鞋,”一条标签如是说,标明那双鞋为泰德·威廉姆斯在他最后一次棒球赛中打那次出名的本垒时所穿。那双鞋的状况比沃尔特·约翰逊穿的那双要好得多——后者的鞋边都烂开了。但文字说明恰好提供了棒球迷想要的某种合理性解释。“当我在场上投球时,脚必须舒服,”了不起的沃尔特说。

博物馆五点关门,我回到汽车旅馆,对脑子记的和查询的内容都很放心。但直觉告诉我第二天上午还要回去转一圈,而就在这一次我才注意到伯利·格兰姆斯滑溜溜的榆树皮,使我灵机一动将其作为理想的开头。它至今还有效。
。
这个故事的启示是,你所收集的素材应该总是比你将要用到的多。每一篇文章的坚实度需要你从充裕的细节中选取少量最适合你的部分得以保证——所收集的材料越多越好。但在某一个时间点上,你必须停止调研,开始写作。

另一个启示是扩大你的材料范围,而不是只研读显而易见的材料,访谈显而易见的人。看看各种标记、告示牌,以及沿美国路边涂写的形形色色的垃圾广告。读读包装上的标签、玩具说明书、药品说明,还有墙上的涂鸦。读读每月从电力公司、电话公司以及银行飞来的那一张张自负自大的账单。读读菜单、目录以及二等邮件。在不起眼的报纸缝隙间寻觅,比如星期日房地产一栏——通过人们想要的房屋露台,就可以分辨出这个社会的秉性。我们每天的景观充斥着荒诞的奇文轶事。要注意这些。它们不但具有重要的社会意义,而且足够新奇,可避免写出与众趋同的文章开头。

说到与众趋同的文章开头,有许多种我情愿永远见不到。一种是作未来考古学家状:“当某个未来考古学家偶然发现了我们的文明遗迹时,他对这个自动唱机又会做出何种结论呢?”甚至连这家伙还未出场,我就已经厌倦他了。我也同样厌倦来自火星的客人:“假如有来自火星的生物降落在我们的星球上,他会惊奇地看见一群群衣着单薄的地球人躺在沙滩上烤着肌肤。”我厌倦了“不久前的一天,一个鼻子像纽扣状的小男孩正在新泽西州帕拉默斯郊外的空地上与他的小狗泰里一同散步,这时他看见有什么怪怪的像气球的东西升出地面。”而且我非常厌倦那种“有共性”的开头:“约瑟夫·斯大林、道格拉斯·麦克阿瑟、路德维格·维特根斯坦、舍伍德·安德森、博尔赫斯、黑泽明都有何共性?他们都喜爱西部片。”咱们还是让未来的考古学家、火星来客以及纽扣状鼻子男孩退下吧。尽量给你的开头以新鲜的视角或细节。

请思考下面的开头,琼·迪迪翁撰写的这篇名为《洛杉矶38区罗曼大街7000号》的文章:

罗曼大街7000号坐落在雷蒙德·钱德勒和达希尔·哈米特的那些崇拜者所熟悉的洛杉矶区域:在好莱坞的下方,日落大街南面,由“模范影业公司”、仓房、两户一幢的平房所构成的中产阶级破旧街区。由于派拉蒙、哥伦比亚、德西露、赛缪尔·高德温等影业公司在附近,住在这周围的许多人多少与电影业有联系。比如,他们都曾洗印过影迷照片,或者认识琼·哈洛的美甲师。罗曼大街7000号看起来像是一个褪了色的电影外景地、一幢彩色建筑物,上面的现代艺术装饰斑驳陆离,窗户现在不是被木板封住就是被六角形网眼镀锌网格封上,而在入口处的夹竹桃中,一块胶皮脚垫上写着:欢迎。

实际上这里并不欢迎谁,罗曼7000号属于霍华德·休斯,大门锁着。休斯家族的“传播中心”坐落在这哈米特一钱德勒领地晦暗不明的阳光下,这情形不禁使人怀疑人生的确是一部电影脚本。休斯家族帝国在我们那个时代是世界上唯一一个工业综合体——多年来它拥有机械制造、外国石油钻探工具分部、酿造厂、两家航空公司、大量不动产持股、一家主要影业公司、一家电子与导弹企业——而所有这些都由一个人经营,他的工作方法极像电影《夜长梦多》中的人物。

碰巧,我就住离罗曼7000号不远,我特意时不时驾车路过那里,我猜想那种心情同亚瑟王学者们拜访康沃尔郡海岸一样。我对霍华德·休斯的民间传说感兴趣……

我们希望这篇文章能引导人们略见休斯是如何经营的,并对司芬克斯之谜给予一些提示——而真正拽我们进入文章的则是不断递增的充满怜悯之情与已逝辉煌的细节。认识琼·哈洛的美甲师与摄影辉煌的联系太微不足道,并不欢迎谁的欢迎垫子已成为影业黄金时代的稀奇遗物。在那个时代好莱坞的窗户并没有被镀锌网格所封,那只公鸡由迈耶、德米尔、扎纳克等巨子掌控,人们可以确切地看见他们行使巨大的权利。我们要想知道更多的内容,那就得继续读。

另一个办法就是讲故事。这个办法很简单,显而易见而且简便易行,而我们却常常忘记使用这一现成之法。叙述是吸引注意力的最古老、最有力的方法,人人都想听故事。想方设法以叙述的形式传达信息。下面的开头是埃德蒙·威尔逊对发现死海古卷的记述,那些古卷是现代所发现的最令人震惊的古代遗物之一。威尔逊并没有花时间搭建平台。这可不是那种“从早餐到床铺”的叙述模式,没经验的作者才用此类模式,他们叙述一次钓鱼的行程从天亮前闹钟响起开始。威尔逊直人正题——啪!我们的注意力一下就被抓住了:

1947年早春的某一天,一个叫穆罕默德狼人的贝都因男孩在死海西岸的悬崖边放山羊。正当他向上爬去追走散的羊时,发现了一个以前从未见过的山洞,便随意向里面扔了一块石头。里面传来奇怪的破碎声。男孩吓了一跳,逃走了。但他后来同另一个男孩又回到那里,一起探索了那个洞穴。里面的一些坛子的残片之中有几个高高的陶土坛子。当他们取下碗形的盖子的时候,坛子里发出很难闻的气味,那气味来自坛子内部的深色长方形块状物。他们将块状物弄出洞穴,看见这些东西被厚厚的亚麻布包裹着,外面还涂了黑黑一层似乎是沥青或蜡的东西。两个人摊开里面的东西,发现是长篇手稿,以平行栏目的形式撰写在缝在一起的薄卷上。虽然这些手稿已经退色,有些地方已经破碎,但大体上相当清晰。他们发现其文字并非阿拉伯语。他们对手稿疑惑不解,将其保留下来,迁居到哪儿都随身带着。

这些贝都因孩子属于一个非法走私的帮派,他们将山羊和其他物品从外约旦走私到巴勒斯坦。他们向南转了一大圈,为的是绕过由海关人员荷枪实弹把守的约旦桥,并将物品漂过河去。他们现在正在去伯利恒的路上,去那里的黑市卖东西……

然而关于如何写开头并没有什么固定的规则。在别让读者跑掉这一宽泛的规则之内,所有作者都必须以最自然的写作方式和最适合自己的方式处理自己的题材。有时你可以用一句话讲出整个故事。下面是七部令人难忘的非虚构书籍的开场白:

世界伊始,上帝创造了天和地。——《圣经》

在罗马历699年之夏,即基督出世之前的公元55年,高卢总督盖尤斯·尤利乌斯·恺撒将他的目光转向了不列颠。——温斯顿·S·丘吉尔,《英语国家历史》

将此字谜拼起来,你会发现牛奶、奶酪与鸡蛋、肉、鱼、豆与谷类、绿叶菜、水果与根茎菜,这些构成了我们每日所需的基本食物。——厄玛·S·龙鲍尔,《烹饪之乐》

对于马努斯岛的土著人,世界就是一个大盘子,周边向上卷,底部是平坦的泻湖村子,那里高脚屋就像长脚鸟一样站立,平静安宁,不受潮涨潮落的影响。——玛格丽特·米德,《在新几内亚长大》

这个问题在美国女性心里埋藏搁置多年,无人问津。——贝蒂·弗里丹,《女性的奥秘》

在五或十分钟之内,不超过这个时间,其他三个人打电话给她,问她是否听说那里出事儿了。——汤姆·沃尔夫,《太空英雄》

你所知道的比你认为的要多。——本杰明·斯波克,《育儿经》

以上是针对如何开头的一些建议。现在我要讲一讲如何结尾。知道何时结束一篇文章的重要性是大多数作者始料不及的。你应该像选择第一句话一样充分考虑如何选择最后一句话。好吧,也可以说几乎要一样充分考虑。

这也许难以置信。假如读者开头就跟上你了,随着你转过死角,越过颠簸地带,当结尾就在眼前,他们当然不会离你而去。当然,他们会离去,因为眼前的结尾结果只是幻景。就像牧师的布道辞一样,本来是建构成一系列完美的结局,但却永远不结束,一篇文章在该停之处不停那就讨人厌了,因而是失败之作。

我们大多数人仍然是年少之时作文老师灌输给我们的教条的囚徒:每一篇故事都必须有一个开头、中间和结尾。我们至今仍能想象出那种提纲,以罗马字母为主干标记(I,Ⅱ以及Ⅲ),画出我们将忠实地踏上的路线图;然后是次主干标记(Ⅱa以及Ⅱb),标示出我们将短暂探寻的次要路径。但我们总是许诺要回到第Ⅲ部分,总结我们的旅程。

这种方式适用于对自己的基础没把握的小学生和中学生。它迫使学生认识到每一篇作品都应该有一个逻辑严谨的设计。这个教条值得让任何年龄的人都知道,甚至连专业作家跑题的频率也常常比他们自己所愿意承认的严重得多。但是如果你打算写出好的非虚构作品,你就必须摆脱第Ⅲ部分死死的控制。

当你看见屏幕上出现始于“总而言之,人们注意到……”这类句子,或者“那么我们从中能够收获何种结论呢?” 这类问题,你就知道已经是到了第Ⅲ部分。这些信号预示你将以压缩的形式重复你已经详细说过的内容。这时读者的兴致开始动摇,你也已建立起来的张力开始松弛。然而你还是想要忠实于波特小姐,你的老师,她让你发誓对神圣的提纲效忠。于是你提醒读者什么是总而言之可以被注意到的。你又重新搜罗一遍你已经举证过的结论。

但读者能听见你费力转动的声响。他们会注意到你的所作所为,以及这些作为使你感到多么无聊。他们感到愤愤不平的颤动。你为何不多想想打算如何结束这篇东西?你在结尾的总结难道只是因为你觉得读者太笨,不明白你的要点吗?你还在不停地转动。但读者另有选择。他们撤了。

以上是记住文章最后一句重要性的负面理由。不清楚这句应置于何处会毁了一篇好端端的文章;本来这篇文章通篇结构严整,只是最后阶段出了问题。写好结尾的正面理由是,最后一个好句子——或最后一个好段落——本身应给人以快乐。它给予读者向上的力量,待文章结束时,它仍令人回味。

完美的结尾应该稍微给读者一点儿惊奇,而且要恰到好处。读者不希望文章结束得太快,或者太突然,或者是重复说过的。好的结尾读者一看便知。就像好的开头,它好使。它就像舞台喜剧中落幕前最后一句台词。我们正在一场剧的中间(我们认为如此),突然一位演员说了些滑稽、夸张或者警句之类的话,舞台灯光随之熄灭。我们惊讶地发现这场戏结束了,随后为其奇妙的结束方式感到愉悦。使我们愉悦的是剧作家完美的掌控。

对于非虚构作者,将其归纳为规则的最简单方法是:当你准备好停止之时,停止。如果你已经陈述完所有事实而且强调了你要强调的要点,赶紧找最近的出口。

经常只需要几句话就可以结束一切。理想的状况是,这些句子应该概括文章的中心思想,最后的结束句应该以其得体性和意料之外的效果震动人心。这里举例说明门肯是如何结束他对卡尔文·柯立芝总统的评价的。柯立芝总统对国民“客户”的吸引力是,其“政府几乎不统治国民,因 而杰斐逊的理想最终得以实现,杰斐逊的信徒们兴高采烈”:

我们最受苦之时并非白宫像宿舍一样平静的时候,而是有那么一个无能的保罗在房顶叫喊时。除去哈定这个无足轻重的总统外,柯立芝博士的前任是个救世主,后两任也是如此。觉醒了的美国人不得不在这几位和柯立芝之间选出一位,他们还有片刻可犹豫的吗?柯立芝当政期间无惊天动地之举,但也没有头疼之事。他没什么主意,因而也不讨人厌。

这五个短句很快就送读者上了路,而且读者走时还带着令人发醒的思绪。柯立芝没注意因而也不讨人厌这个想法,不禁给读者留下一种享受回味的余地。这的确好使。

我在写作中经常做的是将事件带回到一个循环中——也就是在结尾敲响回音,同开头曾响起的音符呼应。这样可以满足我的对称感,也愉悦读者,达到共鸣,完成我们共同踏上的征程。

但一般来讲,最有效的是引语。回头翻看自己的笔记,找出某些给人以结束感的、或者滑稽的、或者增加某种意想不到的最后细节的语句。有时它会在访谈期间蹦出来——这时我常想“结尾就是它了!”——或者在写作过程中出现。60年代中期,伍迪·艾伦正在确立自己作为美国专职神经质艺人的名声,他在夜总会演独角戏。当时我为杂志撰写第一篇长文,记述了他的到来。那篇文章的结尾是这样的:

“如果大家看完演出,欣赏我这个人,”艾伦说,“而不是仅仅喜欢我的笑话;如果他们看完演出,无论我说什么,他们还想听我说,那么我就成功了。”按照其回头客来判断,他的确成功了。伍迪·艾伦就是那位知己先生,而且他似乎注定要多年把持这一特许演出权。

但他的确也有自己的问题,在美国无人分享,无人欣赏。“我一直被一个事实所困扰,”他说,“那就是我母亲真的特别像格劳乔·马克斯\footnote{格劳乔·马克斯(Groucho Marx,1890-1977):美国喜剧演员,电影、电视明星,以机敏闻名}。”

这里有一句话来自那么遥远的左派领地,没人料想到它的出现。它所承载的惊奇是巨大的。这样的结尾难道会不完美吗?假如有什么使你感到惊讶,它也会使你的读者感到惊讶——和愉悦,特别是当你结束故事和送他们上路之时。