\chapter{零零碎碎}
这一章包括零零碎碎的小小忠告,涉及方方面面——像大家所说的那样,我将其收集在一把伞下。

动词

要使用主动语态动词,除非没有能绕过被动语态动词的捷径。主动动词与被动动词风格之间的区别,从清晰度和力度上来讲,对于作者就是生与死的区别。

“乔看见他了”语气强。“他被乔看见了”语气弱。第一句简短而精确,它对于谁做了什么不留任何疑问。第二句必然冗长并且还有一种枯燥乏味的特性:有什么事被某人向另一个人做了。而且该句还有歧义。他被乔看见多少次?一次?每天?每周一次?包含被动结构的风格会消耗读者的精力。没人会真正明白有什么被谁向谁实施了。

我用“实施了”这个词,因为它是那种被动语态作者喜欢用的词。他们更喜欢用带拉丁语词根的长词,而非简短的盎格鲁一撒克逊词——这就增加了麻烦,使句子更粘连。简短比冗长好。在林肯第二次就职演讲词中的701个单词中(这本身已经是用词经济的奇迹了),其中505个单词是单音节词,122个是双音节词。

动词是你所有工具中最重要的。动词推进句子,给句子以动力。主动动词推得紧,被动动词时时拖。主动动词还能使人预见某种活动,因为此类动词要有代词(“他”),或名词(“那个男孩”),或某个人(“斯科特太太”)将其付诸行动。许多动词在其意象或声音中还带有含义的暗示:闪耀,耀眼,旋转,欺骗,分散,昂首阔步,拨弄,纵容,烦恼。恐怕没有其他语言具有如此巨大的动词供给、如此鲜亮的色彩。不要选一个乏味或勉强够用的动词。用主动动词激活句子,避免用那种需要尾部带介词才能完成任务的动词。不要说“set up a business”,可以说“start or launch a business”(开始一项事业)。\footnote{这里的动词都是“开始做”的意思,前者为动词短语,更口语化,后者更简明、标准}不要说“公司总裁下台了”。他辞职了吗?他退休了吗?他遭解雇了吗?要准确。要用准确的动词。

如果你想看主动动词如何给写就的词以活力,不要只回到欧内斯特·海明威、詹姆斯·瑟伯或亨利·大卫·梭罗那里。我推荐英王詹姆士一世钦定《圣经》和威廉·莎士比亚。

副词

多数副词是不需要的。假如你选用一个意思确切的动词,然后再加一个带有同样意思的副词,你就会搞乱句子,惹恼读者。不要告诉人们收音机嘟嘟叫地响,“嘟嘟叫”已经包含响的意思。不要写有人紧紧地咬牙,咬牙本来就是紧紧的,别无他法,不必再用“紧紧地”。在不经意的写作中,强力动词反复被多余的副词所削弱。形容词和其他词性的词也是如此:“毫不费力地容易”,“有一点儿艰苦”,“完全使人目瞪口呆”。“目瞪口呆”一词的美感就在于它暗指某种完全彻底的震惊,我无法想象有人会部分地目瞪口呆。假如有什么事容易到毫不费力,直接用“毫不费力”好了。那么什么叫“有一点艰苦”?也许是和尚享用的铺满地毯的单间吧。不要用副词,除非这些词的确起作用。求你不要以“获胜的运动员咧嘴大笑”之类的表达来报道新闻。

在讨论此类问题之时,让我们停止使用“决定性”地以及其含义模糊的远亲们。每天我在报上都会看到某些局势决定性地变好,另一些则决定性地恶化,但是我从来都不知道好转的程度有多么决定性,或谁做的决定,就如同我从来都不知道某个“特别公平”的结果会有多么特别,或者是否相信一个“具有争议性的真实”的事实。“他是纽约大都会队有争议性的最好投手,”得意洋洋的体育记如此写道,渴望达到帕纳塞斯艺术圣山的高度,而瑞德·史密斯却从没用“争议性”之类的词就达到了。这名投手是该队的最佳投手吗?——这可以通过论证加以证明。如果能,请省略“争议性”这个词。或者他也许是最佳投手?——这样此观点就具有争议性了。恕我坦言,我也不知道。这实际是个难以定夺之事。

形容词

多数形容词也是不需要的。像副词一样,它们四处喷洒在句子中,这些作家也不停下来想想那些概念早已在名词中了。此类散文到处充满险峻的峭壁和花哨的蜘蛛网,或者表达某些颜色已经不言自喻之物的颜色的形容词,如黄色水仙、褐色泥土。如果你想对水仙做一番评价,选一个诸如“艳丽”之类的词。如果你所处的乡间的泥土是红色,尽管用红土来形容。这些形容词才能起到名词本身起不到的作用。

多数作者几乎是无意识地在其散文之壤播撒形容词,使其更茂盛和漂亮;句子变得越来越长,他们充斥其中的有诸如“威严的榆树”、“活泼的小猫”,“老道的侦探”,“沉睡的泻湖”等等。这是习惯性地用形容词,你需要摒弃这个习惯。并非每棵橡树都是节节疤疤的。只为装饰而存在的形容词对于作者是自我放纵,对于读者则是负担。

同样,规则很简单:让形容词起到需要起的作用。“他望了望灰蒙蒙的天和黑压压的云,决定驶回港口。”深暗色的天和云是做出此决定的原因。假如告诉读者房子毫无光彩或者女孩儿美丽很重要,那就直接用“毫无光彩”和“美丽”。这些形容词会具有恰到好处的力量,因为你已学会节俭地使用形容词。

小修饰语

剪掉修饰你感觉如何、怎么想以及你看见什么之类的小词儿:一点儿、某种、颇为、相当、很、太、非常、在某种意义上,还有几十种更多的此类词语。这类词语会冲淡你的风格和说服力。

不要说你有点儿困惑、有种累的感觉、有点儿郁闷、有些儿烦恼,就直说困惑、累、郁闷、烦恼。不要用小小的胆怯去围绕你的散文。好的写作精练而自信。

不要说你不太满意,因为旅馆相当贵。说你不满意,因为旅馆贵。不要告诉大家你相当幸运。那有多幸运?不要将一件事描述为颇为壮观或很了不起。“壮观”和“了不起”之类的词是不受度量的。“很”、“非常”表示强调是个有用之词,但它也常常是个赘词。没有必要称某人很有条理。他要么有条理,要么就是没条理。

更大的问题是权威性问题。每一个小修饰语都会削减读者的部分信任。读者要作者相信自己并且相信自己所说的。不要减弱这个信念。不要有点胆量。要有胆量。

标点符号

这里只是有关标点的一些粗略想法,决没有将其作为入门读物的意思。假如你不知如何句读——其实许多大学生也不知——找本语法书吧。

句号:有关句号没有太多可说的,只是多数作家没有及时用到位。如果你发觉自己无望地陷人长句之中,那也许是因为你试图在合理的情况下让该句做更多的事,也许想要表达两个不相似的想法。解决的最快办法是将一句断为两个甚至三个短句。在上帝眼里,句子接受的最短长度并无规定。好作家当中,短句为主,但别跟我提诺曼·梅勒——他是个奇才。假如你要写长句子,也得首先做个奇才。或者至少要保证,从头到尾,在句法和标点上,所写的句子都要可控,这样读者才知道在弯弯曲曲的路径上的每一步他所处的位置。

感叹号:不要用感叹号,除非你必须要造出某种效果来。感叹号造了一种煽情的气氛,例如描写初入社交界的少女评论只让她个人激动的事件时所表现出来的扣人心弦的激动:“爸爸说我一定是喝了太多的香槟了!”“但说实话,我当时能跳一晚上舞!”这些句子中的感叹号敲击着我们的头脑,使我们感到事情是多么好玩、多么美妙,但我们大家因此而遭的罪比这些句子所给予的好处要多。因此要与之相反,造好句子,使词序起到你要强调的作用。同时也要抵制用感叹词来通知读者你在开玩笑或用讽刺语。“我从未想到水枪也可以上膛!”读者对你提醒这是滑稽的一刻会感到恼怒。同时他们也被剥夺了自己发觉其好玩的乐趣。幽默通过低调陈述可以取得最佳效果,而感叹词却没有任何微妙之处。

分号:这个标点符号有一种陈腐的19世纪气息。我们将它与小心翼翼的平衡句,以及康拉德、萨克雷、哈代深思熟虑的“从一方面看来”和“从另一方面看来”之类的词组联系起来。因此分号对于现代虚构作者来说应该慎用。然而我注意到在本书我所引的段落中分号相当经常地出现,而且我自己也常用它——一般是用来针对句子的前部分增加相关的想法。虽然如此,分号带给读者的,假如不是停止,至少也是个停顿。因此要谨慎使用分号,切记它会将你想奋力达到的21世纪初的动力放慢到维多利亚时代的节奏。要依靠句号和破折号。

破折号:不知为何,这个极为宝贵的工具被广泛认为是不太适用的——就像一桌温文尔雅的好英语晚餐上的土包子。但他却有完全的会员资格,并且能使你摆脱许多尴尬境地。破折号有两个用法。其一是在句子第二部分详述或说明你在第一部分陈述的某个想法。“我们决定继续走——还有100英里路,我们可以及时赶到吃晚饭。”破折号凭其形状本身就可以推进这句话,解释他们为何要继续走。其二涉及两个破折号,将长句中解释性的想法分开。“她告诉我上车——整个夏天她都在要我剪头发——于是我们静静地驶向城里。”本来需要一个单独句子来处理的解释性细节,现在顺便被利索地解决了。

冒号:冒号看起来变得比分号更陈旧,其许多功能已经被破折号所取代。在类似于罗列一连串项 目之前起到纯粹的暂停作用方面,冒号仍然很有效。“介绍手册上说,船将在以下港口停泊:奥兰、阿尔及尔、那不勒斯、布林迪西、比雷埃夫斯、伊斯坦布尔和贝鲁特。”类似这样的功能,哪一个标点符号也抵不过冒号。

语气变化词

学会尽快警醒读者前一句之后发生的任何语气变化。至少有几十个词可以完成此项任务:但、不过、然而、仍然、还、反而、因而、所以、同时、现在、之后、当今、随后,以及更多。如果你改变方向时用“但”开始,那么毫无疑问,读者在理解一句话时会多么容易。但与此相反,如果他们必须等到最后才意识到你改变了语气,那将会多么难。

我们很多人都被教导说每一句话都不该以“但”开始。假如这就是你所学的,那抛弃它——在开始处没有比它更强的词了。它宣告同之前的句子形成完全的反差,这样读者就事先准备好了这一变化。如果你想缓解一下太多以“但”开始的句子,就转为用“然而”。然而,这是一个弱性词,需要审慎地放置。不要以“然而”开始句子——它会像一片湿洗碗布挂在那儿。也不要以“然而”结束句子——到那时它已经失去了其“然而性”。将其置于合理早的部位,就像我刚才一样将其置于三句话之前。这样其突兀性就变成了优点。

“不过”的作用与“但”几乎相同,但其意思更靠近“仍然”。在英语中,这两个词都可以置于句首——“不过他还是决定去”或者“不过他仍然决定去”——来替代概述读者已被告知有关内容的一个长长的词组:“尽管所有危险都已向他指出这一事实,他还是决定去。”查清句中的所有位置,看此类短词可否同时表达与冗长、乏味的从句一样的意思。“我反而乘了火车。”“我还是不得不羡慕他。”“就这样我学会了抽烟。”“因此见到他是容易的。”“同时我与约翰谈过了。”这些轴心词节省了多少膨胀的词语啊!(这里的感叹号表明我真是这个意思。)

缩约词

如果你使用像“I'll”(我将)、“won’t”(不会)、“can’t”(不能)这样的缩约词来自然而然地适应你所写的,你的文风相对于你的性格会变得更温和、真挚。“我将很高兴见到他们,如果他们不生气的话”比起“我将会很高兴见到他们,如果他们不会生气的话”就不那么生硬。(大声朗读后一句,你将听见它有多做作。)并没有什么规定反对这种非正式用语——相信自己的耳朵和直觉。我只建议避免使用那种容易混淆的形式——读者在搞清楚是哪个意思之前很可能已经深入到句子里。经常是其含义并非读者所想到的意思。还有,不要自己造缩约词,这会降低你的文风。坚持用你在词典中能查到的。

概念性名词

在差的写作中,表达某种概念的名词得到广泛使用,而不是诉说某人做了什么的动词。这里有三个典型的僵死句子:

常见的反响是不信任的嘲笑。

茫然的冷嘲热讽并不是对于这个旧体制的唯一反应。

当前校园的敌意是变化的征兆。

这些句子的怪异之处在于句子里没有人。句子里也没有实意动词——只有“是”或“不是”。读者想象不出有任何人进行活动;所有意思都非人格化地包含在某种概念模糊的名词里:“反响”、“冷嘲热讽”、“反应”、“敌意”。变一变这些冷冰冰的句子,叫人动起来:

多数人只是以不信任的态度嘲笑。

一些人对旧体制报以冷嘲热讽;另一些人……

很容易注意到变化——你会看到所有学生有多么气愤。

我修改过的句子也没有跳跃出什么活力,部分原因是我想使劲揉成形的材料是不成形的面团。但至少这些句子里有了真人和真动词。不要陷入拎了一满袋抽象名词的窘境。那样你会沉人湖底,永无出头之日。

蔓生的名词串儿

这是一种新美语疾患,它将两三个名词串在一起,而实际上用一个名词——或者更好一点儿,一个动词——就够了。现在没人走向破产;我们只是有资金领域问题。天不再下雨了,而只是有降雨活动或者雷暴可能性的天气。求求你啦,让雨下来吧。

现今有多达四五个概念性名词附加在一起的情况,就像一个分子链。这里有一个我最近发现的绝好样本:“促进交际技能发展干预”。看不见一个人,也没有任何实意动词。我想这是一个帮助学生提高写作能力的项目。

夸大其词

“客厅看起来就像那里爆炸了一颗原子弹,”写作新手这样描述一场失控了的晚会之后星期天早晨的情景。好吧,我们都知道他是在夸张,想造出点儿滑稽效果,但我们也都知道原子弹根本就没在那儿爆炸,什么炸弹也没有,也许水弹除外吧。“我感觉就好像十架747喷气式飞机飞过我的头顶,”他写道,“我真想跳出窗户自尽。”这些戏谑之语会戏谑出格——这位作者早已出了界线——读者也早已感到昏昏欲睡,无可抵御。这就像被一个不停地背诵打油诗的人缠住了一样。不要夸大其词。你不会真想跳出窗户。生活中真正可怕滑稽的情形多得很。让幽默悄悄潜人,我们几乎听不见它的到来。

可信度

可信度对于作者同对于总统一样脆弱。不要夸大一件事,不要言过其实。如果读者抓住你在竭力扮假为真,哪怕只有一句虚假的陈述,你所写的一切就会遭到怀疑。这个风险太大,不值得一试。

口述

在美国有许多写作是通过口述完成的。行政主管、公司高管、经理人、教育家以及其他官员们所想的是高效利用时间。他们认为将什么“写”下来的最快办法是向秘书口述,也不用再看一眼。这实际是得不偿失——他们节省了几个小时,却毁坏了自己的人格。口述的句子很容易夸大其词、拖泥带水、废话连篇。业务繁忙,避免不了口述的公司高管们至少应该抽出时间编辑一下自己口述的内容,减词加词,保证最后落实到文字的内容能够真实反映他们的特点;特别是那些要给客户的文件,客户们会依照其文风判断其人格和公司的情况。

写作不是竞赛

每一位作者的出发点不同,去往的目的地也不同。然而,很多作者被同一种想法困扰得不知所措,他们认为自己是在与其他也在写作之人或许写得更好之人竞争。这经常发生在写作课上。缺乏经验的学生会不寒而栗地发现,自己同班的同学在校报上都已署过大名了。但为校报写文章并不算什么太高的资历。我经常发现为校报写文章的兔子们往往被勤勉地移向掌握写作技能这一目标的乌龟们所赶超。同样的恐惧也捆绑了自由撰稿人,他们看见其他作者的文章出现在杂志上,而自己的作品却不停地遭到退稿。忘掉竞争,按照自己的步调走。你唯一的竞赛是同你自己比。

潜意识心理

你的潜意识心理活动对写作的贡献比你想的要多。你经常会花一整天的时间试图在词语的灌木丛中杀出一条路,你感到似乎自己羁绊在那灌木丛中没救了。而第二天早上当你重新投入进去,往往会有好办法出现。在你睡眠之时,你那作者的大脑并没休息。一个作者会一直工作。对你周围的动向保持警觉。经过几天、几个月、甚至几年,通过你潜意识心理活动的渗透,正当你的有意识心理活动劳作于写作之时,急需之刻,你所看见、所听见的大部分都会回来。

最快捷的解决办法

令人称奇的是,句子中一个难解的问题经常可以用简单删除的办法加以解决。不幸的是,这个办法对于卡住的作者来说,通常是最后想到的。首先他们会想方设法安置这个麻烦的词语——将其挪到句子的其他部分,竭力重新调整,增加新词来澄清思路,或者润滑任何卡住的部分。这些努力只会使情况更糟,作者最后只得下结论说这个问题无解决办法——这可不是什么欣慰的想法。当你发觉自己处于如此绝境,再看看那个麻烦成分,问自己,“难道我真的需要它吗?”很可能你并不需要。其实它一直在做无用功——这就是它为何一直带给你那么多痛苦的原因。去掉它,看那受苦受难的句子重放生机,自由呼吸。这是最快的治愈办法,也常常是最好的。

段落

保持段落简短。写作具有视觉性——它首先抓住的是眼球,然后才有机会抓住大脑。短段落在你所写的四周留有空气,使其看起来更有吸引力,而那种冗长的段落会打击读者的兴趣,甚至连开头都不愿意读。

报纸段落应该只有两三句长;报纸的版式横向排得比较窄,因而其行数很快就会增加。你也许会想如此频繁换行有损于表达你的观点。显然,《纽约客》对此类恐惧难以自拔——担心读者读上几英里长,毫无喘息之机。别担忧,利会大于弊。

但也不要失控。连续的短段落与一个大长段落同样讨人厌。我想到所有那些侏儒段落——无动词的奇迹——由当代记者们所写,为的是使文章快而易。实际上他们这样切断自然的思路反倒使读者阅读起来更加困难。试比较以下同一篇文章的两种排列——注意看第一眼时和读起来时彼此的区别:


白宫2号律师星期二提前下了班,驱车到一个眺望波托马克河的偏僻公园,结束了自己的生命。

他手里有一把左轮手枪,身体低垂地靠在内战时期的大炮上,没留下遗言,没有解释。

只有朋友、家人和同事的震惊和悲痛。

还有直到星期二之前读起来还像是任何人的幻想的人生故事。

白宫2号律师星期二提前下了班,驱车到一个眺望波托马克河的偏僻公园,结束了自己的生命。他手里有一把左轮手枪,身体低垂地靠在内战时期的大炮上,没留下遗言,没有解释——只有朋友、家人和同事的震惊和悲痛。他还留下了直到星期二之前读起来还像是任何人的幻想的人生故事。

美联社的文稿(左),其段落轻松、简短,而且第三、四句无动词,具有轰动效果,且带有优越感。“啊哈!看,我为你把它弄得多简明!”那位记者向我们说。我的文稿(右)给予记者写出好英语并把三个句子整合为一个逻辑单元的尊严。


划分段落在非虚构文章和书籍写作中是一个微妙而重要的因素——它是一幅地图,不停地告诉读者你是如何组织想法的。研究一下好的非虚构作家,看他们是怎么做的。你会发现几乎所有人都以段落为单位思考,而不是以句子为单位。每一段都有其自身内容和结构的整一性。

性别歧视语

令作者最烦恼的一个新问题是如何处理性别歧视性语言,特别是代词“他一她”。女权运动有助于揭示性别歧视是多么广泛地潜藏于我们的语言之中,不仅是在挑衅性的“他”中,而且在成百上千的带有令人反感的意思或在价值判断上有弦外之音的词语中。这些词包括以恩宠自居的“小妞儿”,暗示二等地位的“女诗人”、二等角色的“家庭主妇”,某种头脑空空的“丫头们”,贬损女性工作能力的“女律师”,故意好色、很少用于男性的“离婚女子”、“女生”、“金发碧眼女郎”。男子遭抢劫也就抢了,女子遭抢劫就会是一个身材匀称的女乘务员或者时髦的黑发女郎。

更有损人格的——也更微妙的——是所有将女性看为男性家庭财产的用法,而不是将她们看作有自我身份,在家庭中扮演同等角色的人:“早期定居者带着妻子和孩子向西推进。”该把这些定居者改成开发边疆的家庭,或者开发边疆的夫妻带着自己的儿子和女儿向西走,或者男男女女们定居在了西部。现今很少有什么角色不是向男女都开放的。不要用暗示只有男人才能成为定居者、农民、警察或者消防员的词语结构。

另一个棘手问题是由女权主义者对一些带“男人”(man)的词不满所提出的,如:“主席”(chairman)和“发言人”(spokesman)\footnote{此处后缀的情形特指在英语语言环境下,请读者参考阅读——编者注}。她们反对的观点是女人也能同男人一样主持好会议,也同样擅长发言。这样就出现了一些新词,如“chairperson”和“spokeswoman”。这些来自60年代的临时词语使我们意识到性别歧视问题,既在词语中也在态度上。但最终这些仍是临时性的词语,有时对这项事业的损害反倒多于帮助。一个解决办法是找另一个词:“chair”代替“chairman”,“company representative”(公司代表)代替“spokesman”。你也可以将名词转成动词:“代表公司,琼斯女士说……”当某种职业既有男性也有女性称谓的形式时,找一个通用的替代词:男演员和女演员都可以称为表演者。

但这还是留下了代词等待解决。“他”就是折磨人的词。“每一个雇员都应该决定他认为什么对他自己以及他的家属是最好的。”我们对这类无数的句子又该怎么办呢?一个办法是将其转为复数:“所有雇员都应该决定他们认为什么对他们自己和他们的家属是最好的。”但这只是在小剂量下才有效。将每一个“他”都转成“他们”,这样的风格很快就会变得淡而无味。

另一个通常的解决办法是用“或”:“每一个雇员都应该决定他或她认为什么对他或她自己是最好的。”但同样,这也得慎用。作者经常会发现在一篇文章中有几种情况,他或她可以用“他或她”,只要自然就好。“自然”的意思,指的是作者已经注意到他(或者她)有这个问题,并尽力在合理的限度内解决此问题。但还是要直面问题:英语语言卡在了阳性通用名词上
(“Man shall not live by bread alone”,人不能光靠面包生存)。要把每一个“他”转成“他或她”,每一个“他的”转成“他的或她的”,就会堵塞语言。

在《写作法宝》的早期版本中,我用“他”来指“读者”、“作者”、“评论者”、“幽默作家”等。我感到假如每次我提到这些人都用“他或她”,这本书读起来就费劲了。(我完全杜绝“他/她”这种形式,斜杠在规范英语中毫无地位。)不过几年来,许多女士写信给我唠叨这个问题。她们说身为作者的她们对总是想象到一个男子在写作、阅读表示反感。她们是对的。我忍受着唠叨。多数唠叨者敦促我用复数形式。我不喜欢复数;复数形式削弱了写作的力度,因为复数不如单数确切,不那么容易唤起视觉想象。我想要每一位作者都想象出有一位读者在竭力阅读他或她所写的。然而我还是发现有三四百处,在这些位置我可以去掉“他”,办法主要是转成复数,而且无大碍;天也没塌下来。在本版中仍有男性代词,但我感到这里的用法是唯一不麻烦的解决办法。

最好的解决办法就是简单地去掉“他”以及其所带的男性特有的内涵,而使用其他代词或者变换句子的其他成分。“我们”可以方便地替代“他”。“我们的”常常可以替换“他的”。(A)“首先他注意到有什么事情发生在他的孩子身上,而后他为此抱怨他的邻里。”(B)“首先我们注意到有什么事情发生在我们的孩子身上,而后我们为此抱怨邻里。”泛指名词可以代替确指名词。(A)“医生经常忽略妻子和孩子。”(B)“医生经常忽略家人。”通过这些小小的变化,可以消除无数错误。

在我的修补中有帮助的另一个代词是“你”。在谈论“作者”做什么以及“他”都遇到些什么麻烦时,我发现在很多地方我都可以直呼作者(“你会经常发现……”)。这并不适用于所有类型的写作,但对于写指导手册或自助手册的作者来说却是个天赐的办法。以本杰明·斯波克博士的口吻对发烧小孩的母亲说话,或者以朱莉娅·蔡尔德的口吻对忘了菜谱的厨师说话,是读者所能听见的最放心的声音。永远都要寻找使自己能够贴近读者的办法。

修改

修改是写好作品的基础,它决定了这场游戏的成败。这个想法让人难以接受。我们在第一稿中都有一种感情投人,我们不能相信自己的作品天生不是完美的。但其不完美的可能性是百分之百。多数作者并非开始就能说出自己想说的,或者尽可能说得好。新孵出的句子几乎总是有错。它不清楚。它没逻辑性。它啰嗦。它太笨重。它装腔作势。它乏味。它充斥赘语。它充满陈词滥调。它缺乏节奏。它可以有几种不同的读法。它不是从前一句接着往下走。它不……要点是,清晰的写作是反复修补的结果。

许多人认为专业作家不需要修改,词语会自动到位。正相反,严谨的作家会不断地修改。我从不认为修改是一个多余的负担,我感谢每一次改进自己作品的机会。写作就像一块好表——它应该走得顺畅,不带任何多余的零件。学生并不苟同我对修改的热衷。他们认为那是惩罚:额外的作业或额外的内场练习。请——如果你是这样的学生——把它当做一件礼物。假如你不理解写作是一个进化的过程,而不是一个完成的产品,你就写不好。没有人指望你一次就写对,甚或两次就写对。

那么我说的“修改”是什么意思呢?我并不是指写一份稿件,然后再写一个不同的文稿,然后再写第三个。多数修改包括重塑、缩紧、精炼你第一次写的原材料。许多部分包括保证给予读者一种他能够轻易从头到尾读下去的叙述的流畅感。不断使自己处在读者的位置上。有什么他应该在句子前面被告知,而你却将其放在句尾了呢?当他开始读A句到B句时,你已经改变了B句的题材、时态、语调、重点,他知道这些吗?

下面看一个典型的段落,想象它是作者的第一稿。其中并没有什么错,它清晰、合语法。但其中到处是粗糙的棱角:作者未能不断地让作者了解时间、地点、语气上的变化,或者变换和激活自己的写作风格。我所做的是,在每一句后面用方括号加上编辑在读第一稿时可能想到的一些建议。之后你会看见我修改过的段落,其中吸收了那些具有纠正性的建议。

在过去的年代,邻居们会互相帮助,他记得。[将“他记得”前置以便建立一种回顾的基调。]事情似乎不再是那样了,然而。[用“然而”表示对比的词必须置于句首。用“但是”开始。同时明确美国这个地理位置。]他想知道其原因是否是因为在现代世界人人都太忙。[以上所有句子长短都一样,而且昏昏沉沉的,节奏也相同;把这句变成问句?]他想到如今人们有这么多事情要做,他们没有时间再保持老式的友谊。[这句基本上重复上一句;去掉或者加具体细节。]在美国的前一个时代,情况可不是这样的。[读者仍在现在状态;调转一下句子说明读者现在是在过去。“美国”一词如果在前面已经插入,这里就不需要了。]他也知道在其他国家情形是大为不同的,他回想起自己在西班牙和意大利乡村生活的岁月.[读者仍在美国。用一个否定连词将读者过渡到欧洲。句子太松散。断为两句?]几乎对他来讲,随着人们变富,房子彼此建造得越来越远,他们似乎将自己从生活中最基本的需求里孤立出来。[讽刺延迟得太长。早一些植入讽刺。突出有关财富的悖论。]然后有另一个想法使他烦恼。[这是本段的真正要点,提醒读者这很重要。避免弱化结构“有……”]正当他近期得病最需要朋友之际,他的朋友们却都离弃了他。[重新调整以“最需要”结尾;最后一个词是回响在读者耳边的词并且给句子以力量。留“得病”到下一句,那是另一个想法。]就好像他们发现他做了羞愧的事而感到内疚一样。[引人“得病”在此作为羞耻的原因。省略“羞愧”,这里已经暗示了。]他回想起在什么地方读过有关世界原始地区的社会,在那里病人都被远而避之,虽然他从未听说过在美国有这样的习俗。[句子开始太慢,而且一直呆滞、乏味。断句为更短的单位。迅速射出讽刺要点。]

他记得邻居们过去总是互相帮助。但那似乎在美国不再发生。其原因是大家都太忙吗?人们的时间真是都被电视、汽车以及健身运动占用而无暇顾及友谊了吗?在以前的年代,情况绝非如此。世界其他地区家庭的生活方式也不如此。甚至在西班牙和意大利最穷的乡村,他回忆道,人们都会带上一条面包串门儿。他的脑中忽然闪念出一个颇具讽刺意味的想法:随 着人们变得富裕起来,他们反倒将自己与生活的丰富性割裂开来。但真正使他烦恼的是一件更令人吃惊的事。他的朋友 离弃他之际正是他最需要他们之时。他这一病倒几乎就好像他做了什么羞愧之事。他知道其他社会有对重病之人避而远之的习俗。但那种习俗只是存在于原始文化群落中。或许真是如此吗?

我的修改并非是所能做的最好的,或是唯一的。这些主要是工匠的技艺问题:改变意群的长短、紧缩语句的流动、突出要点。在抑扬顿挫、语言的细节和新鲜度方面,还大有改进的余地。总体结构同样重要。从头到尾朗读你的文章,切记在前一句中你将读者置于何处了。你可能会发现你写了像下面这样的两句话:

这部剧的悲剧主人公是奥赛罗。又矮小又恶毒,埃古的嫉妒与猜疑在不断膨胀。

埃古这句话本身没有问题。但是作为连接上一句的句子,那它就大错特错了。在读者耳畔回响的名字是奥赛罗,读者会自然而然地认为,是奥赛罗又矮小又恶毒。

当你朗读自己所写时,脑子里想着这些连接处,你会很不安地听见在很多处你使读者迷途、困惑,或者没能告诉读者他所需要的,或者两次告诉他同一件事:这些都是每一份初稿不可避免的松散的结果。你必须要做的是有整体安排——从头到尾要紧凑,行文要简洁、温暖。

学会这项清理过程。我不喜欢写作的艰苦,我喜欢写作的果实。但我热衷于修改。我特别喜欢删节:按删除键,看到不必要的词或词组或句子消失在屏幕上。我喜欢替换一个乏味的词,用一个更精确、多彩的词。我喜欢加强一句与另一句之间的过渡。我喜欢替换单调的句子,增加更愉快的韵律,或更典雅、更有乐感的句子。精练每一个细微之处,我感到自己在靠近我想到达之处,而当我最终到达那里,我知道,是修改而不是写作赢得了这场游戏。

用电脑写作

电脑对于修改和重组是上帝赐予的礼物,或者说技术的礼物。它将词语径直放在你眼前,供你随时考虑——和再考虑;你可以玩转句子,直到弄对为止。段落和页码会不停地自动调整,无论你裁减、变化多少都没关系,然后打印机会将一切整齐打印出来,期间你可以去喝杯啤酒。对作家来讲,几乎没有什么甜美的音乐之声比听见自己的文章全部改好后又被重新打出更美妙了——但不是被作家们用手打的。

本书已经不再需要像较早的版本那样解释如何操作这件称为文字处理器的奇妙机器,它已经进入大家的生活,也无须解释如何将其神奇的功能用于写作、修改以及组织。现在这都是常识了。我只想提醒大家(假如你还不信的话)它所节省的时间和精力是大量的。比起从前用打字机,有了电脑,我更愿意坐下来写,特别是在面对复杂的材料组织工作时。我可 以更早完成任务,也不那么累。这些对作者都是至关重要的益处:时间、产出、精力、享受和掌控。

相信自己的材料

我从事写作这个行当越长,就越意识到没有什么比事实更有意思。人们的所为——还有所说——一直令我惊讶,包括其神奇性、怪癖、戏剧性、幽默,甚至痛苦。有谁能控制所有这些令人震惊之事的发生呢?我发觉自己越来越频繁地对作者和学生说,“相信自己的材料。”但这个告诫似乎难以遵循。

最近我花了一点儿时间,给一个美国小城市报纸当写作指导。我注意到许多记者陷人一种习惯,他们用写特写的风格竭力使新闻更受欢迎。他们的开头包括以下一系列句子:

喔……!

难以置信。

埃德·巴恩斯迷迷糊糊不知道自己在看什么东西。

或者也许只是春倦症。奇怪,四月会对一个小伙子有那么大的作用。

不像是他离家前没有检查车子。

但之后还是一样,他没有记得去告诉琳达。

这很奇怪,因为他总是记得告诉琳达。自从他们早在初中时走到一起,一切就如此了。

那果真是20年前的事吗?

而现在还有小斯库特要担忧。

想想看,那条狗已经露出怀疑了。

这些文章经常从第一版开始,我会读到“接第九版”,但是不清楚写的是什么。然后我会尽职地翻到第9版,发现自己在读一篇有趣的故事,充满确切的细节。我会对记者说,“故事不错啊,但我得读到第9版。你为什么不把它放在开头?”记者的回答是,“哦,开头我在描写色彩。”其预想是,事实与色彩是两个分开的成分。它们不是分开的;色彩对于事实是有机统一的。你的任务是呈现色彩斑斓的事实。

1988年我写了一本有关棒球的书,叫《春季训练》。书中结合了我终身的职业与终身的癖好——这是能够发生在作家身上最好的事情之一;人们如果写自己喜欢的,他们能写得更好,也更享受写作。我选择春季训练作为棒球这个大题目的一角,因为此时是该更新之时了,对球员和球迷都是如此。这项体育运动被还原到其原初最纯粹的状况:在室外进行、在阳光下、在草坪上,没有风琴音乐,年轻的球员们彼此靠近得都能碰着,他们的工资和不满都被仁慈地搁置一边儿六个月。最重要 的是,这是传授和学习的时间。我选择记述匹兹堡海盗队,因为他们在佛罗里达的布雷登顿旧时的球场训练,而且是一个刚开始重建的年轻俱乐部,经理人是吉姆·莱兰,他致力于传授技艺。

我并不想使这项运动浪漫化。我不喜欢棒球电影中击球手打了本垒时的慢动作镜头,来告知我这是一个多么意义重大的时刻。我了解本垒那桩子事,特别是如果他们在第九局的后半局靠两个在外击中赢得比赛。我坚决不让自己的写作走向慢动作镜头——不提醒读者其重要性——或者将棒球称为生命、死亡、中年、失去的青春,或者更纯真的美国的象征。我的前提是棒球是一项工作,一项光荣的工作,我想知道这项工作是如何教和学的。所以我去了吉姆·莱兰和他的教练那儿,我对他们说, “你是教师。我也是教师。告诉我:你怎么教击球?你怎么教把球投给击球员?你怎么教接球?你怎么教跑垒?你怎么在如此残酷的长日程安排中保持这些年轻人的士气?”所有人都慷慨地回答了我,仔细告诉我他们是如何做他们想做的。球员和其他有我想要的信息的男男女女也是如此:有棒球裁判员、物色新秀者、买票员、当地球迷。

有一天我爬进本垒板后的看台去找星探。春季训练是棒球人才的终极秀,整个球场到处都是那些终身评估棒球天才的练达之人。我发现一位六十多岁饱经风霜的人旁边有一个空座,他在用秒表计时并记录着什么。当一局结束的时候,我问他在卡什么时间。他说他叫尼克·卡姆齐克,加州天使队北选才队长,在卡跑垒道上选手的时间。我问他在找什么样的信息。

“嗯,右撇子击球员需要4.3秒抵达第一垒,”他说,“左撇子需要4.1或4.2秒。自然会有些变化一你得考虑人为因素。”

“这些数字说明什么?” 我问。

“嗯,当然双杀平均需要4.3秒,”他说。他说起来就像常识。我从未想过双杀需要多长时间。

“因此这就意味着……”

“如果你看见球员在4.3秒之内到达第一垒,你就会对他感兴趣。”

作为事实,这不言自喻。不需要再加一句指出4.3秒对于一个击球、两个投球还有三个内场球是多么有限的时间。给出4.3秒,读者自己就会感到惊叹。他们也喜欢允许他们自己思考。读者在写作的活动中起很主要的作用,因此必须给他们起作用的空间。尽量不用诸如“令人吃惊地”、“可以预见地”和“理所当然”之类的词语,这些会在读者遇见事实之前就给该事实定了价。相信自己的材料。

随自己的兴趣走

没有什么题材不允许你写。学生常常绕开写靠近自己内心最近的题材一滑板、拉拉队、摇滚乐、轿车——因为他们以为老师会认为这些题材“愚蠢”。对于认真对待人生的人,生活中没有什么方面是愚蠢的。如果你遵循自己的所爱,你就会写好,就会吸引读者。

我读过优美的书籍谈论钓鱼与扑克牌游戏、保龄球与马术表演、登山与大海龟,还有许多其他我觉得自己不会感兴趣的题材。写自己的爱好:烹饪、园艺、照相、编织、古玩、慢跑、航海、戴水肺潜水、热带鸟、热带鱼。写自己的工作:教学、护理、做生意、开店。写自己在大学喜欢而且总想回去研究的领域:历史、传记、艺木、考古。如果你写的时候真诚地与其相连,没有什么题材会是太专业或太古怪的。