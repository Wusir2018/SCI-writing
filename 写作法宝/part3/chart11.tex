\chapter{写人:访谈}
要让人们谈论起来。学会问问题,引他们讲述生活中最有趣、最生动的答案。没有什么比听人亲口讲述他们怎么想、怎么做更能激发写作活力了。他自己的词语总比你的词语好,即使你是本国最优雅的文体大家也是如此。他的词语带有他说话时语气的曲折变化,以及他造句时的独特风格。这些词语包括对话的地域性以及行业的行话。它们还表现出他个人的热忱。这是一个人在同读者直接对话,而不是通过了作者的过滤。作者一旦介入,其他人的经历就变成二手的了。

因此要学会如何进行访谈。无论你写的是何种形式的非虚构作品,其活灵活现的程度与你在写作过程中编织进去多少“引语”成正比。你常常发现自己正在着手写一篇显然是毫无生机的文章——一个机构的历史,或者某个地方性事件,如排水沟——你很怕不能确保读者甚至你自己在阅读的过程中保持清醒。

提起信心来。如果想要寻找人文因素,你就会找到解决的办法。每一个乏味的机构的某个岗位都有对自己所做的事情怀有强烈热忱的人,他们是故事的丰富宝库。在每一个排水沟之后都有一位政客,他的未来悬于能否将其治理好;都有一位寡妇,她一直在此街区居住,愤慨于某位愚蠢的立法者认为排水沟会通畅无阻。找到这些人来讲述你的故事,那就不乏味了。

我经常向自己证明的确如此。很多年前,我应邀为纽约市公共图书馆写一本小书,庆祝其在第五大道上的主楼建成50周年。表面上看,这似乎只是一幢大理石建筑和上百万陈旧书籍的故事。但在那外观之后,我发现图书馆有19个研究部,每一个部门都有一位主任,监管大量的稀奇宝物和馆藏,从华盛顿手写的告别演说到75万张电影剧照。我决定访谈所有 的部门主任,了解其馆藏,了解他们为跟上新知识领域增加了什么,以及他们的办公室是如何使用的。

我发现科技部有专利收藏,规模仅次于美国国家专利局,因而也是这座城市专利律师们的第二个家。而且每天不断有人来此,他们认为自己处于发现永动机的边缘。“每个人都有什么要发明,”部门主任解释道,“却不告诉我们他们在找什么——也许是怕我们自己会去注册专利吧。”结果发现整幢建筑就是一个学者、探索者和怪人的混合体,我的故事表面上看是一个机构的编年史,实际上是有关人的故事。

在一篇有关伦敦索思比拍卖行的长文中,我用了同样的方法。索思比也同样分为不同的部门,如银器、陶瓷、艺术品,每一个部门由一位专家负责,就像纽约市图书馆一样,其生存有赖于公众变化多端的异想天开。这些专家就像一个小学院的院长,每人都有内容和讲述方式很独特的轶事:

“我们就坐在这里,像米考伯\footnote{米考伯(Micawber):英国作家狄更斯小说《大卫·科波菲尔》中的人物,无远虑而老想走运的乐天派}一样等待东西送进来,”家具部主任泰姆韦尔说。“最近剑桥附近有一个老太太写信说她要筹集两千英镑,问我能否去她 的房子那里寻查一下,看看她的家具是否值这么多钱。我去了,屋子里的家具绝对没什么价值。要走之际,我说,‘都看过 了吗?’她说都看了,除了女仆的房间,她觉得没什么好看的。那间屋里有一个很精美的18世纪橱柜,老太太用来储藏毯子。‘如果你卖那件橱柜,’我说,那你就什么也不用担心啦。’她说,‘那绝对不可能——我的毯子往哪儿放啊?’”

我也不用担心了。通过倾听做这门生意的好奇的专家学者,倾听每天早上蜂拥而至的男男女女,他们带来从英国阁楼找到的自己不再喜欢的东西(“恐怕不是安女王时代的物件,太太——更靠近维多利亚女王时代,可惜啊”),我得到了作家想要的所有人文细节。

还有一次,在1966年,我应邀写“每月一书俱乐部”的历史,庆祝它四十岁生日,我想这当中可能什么也遇不上,而只是一些了无生气之事。但我在篱笆的两边都发现了饶有趣味的人文因素,因为图书总是由意志强大的评委组专家选择,然后寄往同样固执的订购读者,而订购读者从不迟疑地将他们不喜欢的书包起来直接寄回去。我得到了一千多页对俱乐部最初 五位评委的访谈记录(海伍德·布龙、亨利·塞德尔·坎比、多萝西·坎菲尔德、克里斯托弗·莫利、威廉·艾伦·怀特),同时增加了我亲自与俱乐部创始人哈里·谢尔曼以及当时仍活跃的评委的访谈。其结果是呈现出长达四十年的有关美国人阅读变化的个人记忆,甚至那些书籍都有了自己的生命,在我的故事中成为人物角色:

“对所有记着《飘》的巨大成功的人来讲,”多萝西·坎菲尔德说,“也许很难想到这本书当时是如何影响人们的。碰到这本书的人只将其看作一部很长很长的详述有关内战及战后故事的书。我们从未听说过其作者,也没有任何人对此书的评论。选这本书有点儿困难,因为其中的一些人物刻画不太真实或令人信服。但是作为叙事作品,该书具有法国人称为‘吸引力’的特点:它使你想要翻页,看下面会发生什么。我记得有人评论说,‘好了,人们可能并不喜欢这本书,但是不能否认你的钱换来了大量的阅读。’其巨大的成功,我必须说,对我们和对其他人一样,都是令人吃惊的。”

以上三个例子都是锁定在人们头脑中的典型例子,好的非虚构作家必须为其解锁。最好的实践方式是出去采访人。采访本身就是最受欢迎的非虚构形式之一,因此你应该尽早掌握它。

应该怎样开始呢?首先,决定你想采访什么人。假如你是大学生,不要采访你的室友。尽管我们完全尊重你有多么好的室友,他们恐怕没有太多我们想听的内容要说。要学非虚构写作技能,你必须将自己推到真实的世界中——你的城镇,或你的城市,或你的国家——并装作你是为了真正出版而写。假如这样做有所帮助,可以决定你想为哪家出版物写。将你的题材选为某人,其工作特别重要,或特别有趣,或特别与众不同,普通读者都愿意阅读有关这个人的事。这并不是指他或她非得是银行总裁。可以来自当地的比萨店或者超市,或是理发学校的老板。可以是每日出海的渔民,或者少年棒球联合会的经理,或者护士。可以是屠夫、面包师,或者——假如你能找到他更好——烛台工匠。找你所在社区的女士们,她们会揭示两性注定要做什么的古老神话。简而言之,选择那些触动读者生活一角的人。

访谈是你可以不断提高的技能之一。尝试之后,你就不会像第一次访谈时那样紧张,但在尽力使受访对象不要太腼腆,而受访对象不善言辞因而不能回答所问之时,你恐怕也不会感到完全轻松。但这项技能的大部分都是机械性的。其余部分则靠直觉——知道如何使另一个人放松,何时推进,何时倾听,何时停止。这些都可以通过实践学会。

访淡的基本工具是纸和削好的铅笔。你觉得这个建议显而易见,有点儿伤自尊?你会惊讶于有多少作者莽撞地追踪猎物而没带铅笔,或者铅笔断了,或者钢笔不好用,而且没有纸。“时刻准备着”这个座右铭对于非虚构作者来讲和对于童子军一样适用。

但要将笔记本收好,需要时再取出。没有什么比一个陌生人带着一个速记员的本子更难让对方放松了。你们两个都需要时间认识彼此。先花一点儿时间聊聊,评估一下你要打交道的是什么样的人,让他或她信任你。

在没做任何功课之前,决不要进行采访。如果你想采访城镇官员,要弄清他或她的选举记录。如果是女演员,要弄清她演过什么影视作品。假如你问的事实是可以事先得到的,你会让人恼火的。

将可能问的问题列一个单子,这样能避免采访变得沉闷所造成的巨大尴尬。也许你并不需要单子,你会临时想到更好的问题,或者接受采访的人可能改变角度而你无法预知。这样你只能靠直觉行事。假如他们跑题太远,拉他们回来。假如你喜欢新方向,跟进,忘掉你有意要问的问题。

许多采访者内心都有恐惧,以为自己在强加给别人什么,觉得自己没有权利侵犯他们的隐私。这种恐惧几乎完全没有根据。所谓的凡夫俗子会很高兴有人要采访他。多数人的生活,假如不是在静静的绝望中,至少也是在绝望的寂静之中,因此他们很欢迎有机会向似乎愿意倾听的外人谈谈自己的工作。

这并不一定意味着一切顺利。你经常会同那些从未接受过采访的人谈话,他们在这个预热的过程中可能感到局促不安,过分敏感,也许不能提供任何可用的东西。过一天再来,访谈会好一些。你们两个甚至会享受这一过程,这证明你没有在逼迫你的牺牲品做他们不想做的事。

讲到工具,(你会问)可以用录音机吗?为何不带一个录音机,把它打开,然后忘掉那些纸和笔?

显然录音机是一个抓取人们所要说的内容的超级机器,特别是对那些出于文化或秉性原因绝不费事写东西的人来说。在诸如社会史和人类学领域,这是很有价值的。我欣赏斯塔兹·特克尔的书,如《艰难时世:大萧条时期口述史》。他“写”这部书的方法是记录对普通人的访谈,将结果拼缀成连贯的统一体。我也喜欢问一答方式的访谈。这种访谈有一种随意的声音,清新自然,避免了作者凌驾于产品之上并将其抛光的可能。

严格来讲,这不是写作。这是一个问问题、修剪、嫁接、编辑记下来的回答的过程,需要占用大量的时间和精力。你以为受过教育的人会一直以线性精确度冲着录音机说话,结果他们在语言之沙上毫无目的地跌跌撞撞,连一个像样的句子也说不全。耳朵可以容忍缺少语法、句法成分、少转折词,但眼晴不能忍受这些东西印出来。录音机用起来似乎简单,实则不然,之后需要做无数遍的编辑加工。

但我警告你远离录音机的主要理由是很实际的。风险之一是你不是常常随身带录音机,你更有可能带的是铅笔;另一个风险是录音机可能失灵。在新闻职业中,记者本来以为带回来“一个真正的好故事”,紧接着他按播放键,却发现根本没有声音。没有比这更令人沮丧的了。最重要的是,作者应该能够看到自己的材料。假如你的采访在磁带上,你就成了听众,需要不停地摆弄那部机器,往后倒带,找你根本找不到的精彩言辞,往前倒带,停止,开始一这简直能将你逼疯。做一名写者。将东西写下来。

我用手采访,用一根削得尖尖的一号铅笔作记录。我喜欢同另一个人相互作用。我喜欢那个人看见我在工作——干一件活儿,而不仅仅坐在那儿让机器为我做事。只有一次我充分利用了录音机,那是为我的书《米切尔与拉夫》准备的,是有关爵士乐手威利·拉夫和德怀克·米切尔的。虽然我对两位都很熟悉,但我感到一个白人作家冒昧地写黑人的体验,有责任要把音调搞准。这并不是说拉夫和米切尔讲另一种英语,他们的英语讲得很好,而且常常很雄辩。但作为南方黑人,他们所用的某些词与习语为自己的传统所独有,这为他们所讲话的增添了丰富性与幽默感。我不想错过任何一点儿这些用法。我的录音机抓住了这一切,这本书的读者可以听见我对二位的描述是准确的。在你有可能违背你所采访之人的文化传统整体性的情形下,要考虑用录音机。

记笔记也有一个大问题:受访的人常常讲起话来比你写得快。你还在记A句,他很快就推进到了B句。你停止A句,追他到B句,同时还竭力在耳朵里抓着A句的其余部分,希望C句会无足轻重,可以完全略过,这样就有时间赶上。可惜,你现在的受访者讲得飞快。他终于说出你花了一个小时一直在诱导他说的全部内容,并且他说起来似乎像丘吉尔那样雄辩。你的内耳塞满了想要抓住的句子,但很快这些句子就溜走了。

告诉他停。就说,“请停一分钟,”然后继续写,直到赶上。你发狂般地抄记的意图是准确引用被访者,没人愿意被误引。

通过练习,你能写得更快,并形成某种形式的速记。你会发觉自己在为常用词发明缩写,并省略了连接性小句子结构。一旦访谈结束,补充所有你能记起来的缺失词语,完成未完成的句子。多数句子还停留在可回忆起来的范围之内。

当你到家后,将笔记打出来——很可能几乎辨认不出来——这样你就可以轻松阅读了。这不但能使访谈同你所搜集的剪报以及其他材料一起更容易使用,而且还能使你静下心来回顾自己在匆忙中写下的大量词语,从而发现受访者真正想说的。

你会发现他说了大量无趣或者不相关、或者重复的事。挑出最重要和最精彩的句子来。你会很想用笔记中所有的词语,因为你做了大量辛勤繁杂的劳动,才将其全部记下来。但这就太放纵自己了。没有理由将读者也置于同样的辛苦之中。你的任务是蒸馏出本质的东西来。

那么你对被采访者的义务又如何呢?你能在多大程度上裁减或摆弄他的词语呢?这个问题使每一个第一次做采访回来的作者感到烦恼——也应该如此。如果记住下面两个标准,解决的办法并不难:简洁明了和公平合理。

你对受访者的道德义务就是准确地呈现他的立场。假如他仔细衡量了一个问题的两端,而你却只引用他的观点的一面,使他似乎倾向于那一面的立场,你就会错误地再现他告诉你的。或者,你也许会由于离开语境引用他的话,或只选某些精彩的话而没增加后来的严肃思考,错误地再现他的观点。你所谈论的内容涉及一个人的荣誉和名声,也涉及你自己的。

在那之后,你的责任就是对读者的。他或她应该获得最紧凑的信息包。多数人在对话中会漫谈,填充一些没有关联的故事和琐事。其中大部分使人愉悦,但毕竟还是琐事。你的访谈应该强有力到只包括要点而没有废话。因此,如果你在笔记的第5页发现一个充分说明第2页中一个要点的评述——一个在访谈的前部分提出的要点——如果你将这两个想法连起来,让第二句话跟在后面,说明第一句话,你就帮大家忙了。这可能会违背访谈实际进行时的真实情况,但你会忠实于受访者说话 的意图。通过各种方法摆弄好引文:挑选、去除、精简、调换位置,留下好的引文放到最后。只要保证处理得不违背真实性就好。不要变换词语,也不要让裁剪了的句子打乱留下的内容所处的恰当语境。

我真是指“不要变换词语吗”?既是也不是。假如说话人选词精当,你就应该出于职业对语言要求的特别自豪感而一字不差地引用。多数采访者对此很马虎,他们认为大概差不多也就够了。实际上还不够:没人愿意看到自己从来不用的词语以自己的口吻印出来。假如说话人的对话很粗糙——句子拖拖拉拉,思路杂乱无章,语言乱作一团,连他自己都感到难为情——作者别无选择,只能清理语言,提供缺失的连接部分。

有时候你尽力忠实于说话者,但太忠实了,反而使你陷入困境。你在写文章的过程中,不折不扣地打出自己记下来的词。你甚至对自己作为这么忠实的抄写员感到心满意足。之后在编辑自己所写的内容时,你意识到有几个引语不太能说明问题。你首次听见这些引语的时候,它们听起来很贴切,你并没多想。而现在再想,你发现其中的语言或逻辑有漏洞。留下漏洞对读者和说话者都不好,对作者也不光彩。通常你只需添加一两个词来澄清。你或许在笔记中找出另一个能清楚地表达同样要点的引语。但同时记住,你还可以找你采访的人,告诉他你想核实一下他说的几件事,让他重述自己的要点,直到你明白为止。不要成为自己引文的囚徒——受到引文美妙的声音诱惑,而不再停下来分析。绝不要让任何自己不明之事出笼入世。

至于如何组织访谈内容,开头应该明确地告诉读者为何受访之人值得一读。他有什么地方值得我们倾注时间和注意力?因此,要在受访者用自己的语言所说的和你用自己的语言所写的之间,尽力取得平衡。如果你连续三四段引用一个人的话,就会显得单调。当你把这些段落分开,以向导的身份定期出现,引语就会更生动。你仍是作者,不要放弃对内容的掌控。但你的出现要有用,不要只插几句干瘪瘪的话,来向读者疾呼你这么做的唯一目的就是要断开一连串的引语。(例如,“他拿烟斗在旁边的烟灰缸上敲,这时我注意到他的手指很长。”“她悠闲地拨弄着芝麻菜沙拉。”)

在使用引语的时候,把它放在句子的开始。不要用一个乏味的词组说受访者说,然后导向引语。

差例子:史密斯先生说,他喜欢“每周进城一次,同自己的几位朋友吃午饭”。

好例子:“我一般喜欢每周进城一次,”史密斯先生说,“同自己的几位朋友吃午饭。”

第二句话有活力,第一句话死气沉沉。没有比“史密斯先生说”这种结构的句子更死气沉沉的了。许多读者会就此打住。如果受访人说了什么,就直接让他说,以温情、人性化的句子开始。

但要注意在何处断开引语。在你能最自然地断开时断开,这样读者便知是谁在说话,不过不要在可能影响节奏和意思之处断开。注意下面三种变体都造成了某种程度上的损伤:

“我通常喜欢,”史密斯先生说,“每周进城一次同我的几位朋友吃午饭。”

“我通常喜欢进城,”史密斯先生说,“每周一次同我的几位朋友吃午饭。”

“我通常喜欢每周进城一次吃午饭,”史密斯先生说,“同我的几位朋友一起。”

最后,不要费力找“他说”的同义语。不要只是为了避免重复“他说”,而用断言、坚称和劝告之类的词语来表示受访者的引语。请千万不要写“他微笑道”或“他咧嘴笑道”之类的词语。我从未听见什么人微笑道。读者的眼睛反正会略过“他说”,因此无须小题大做。如果你特别喜欢语词的变化,选择能抓住对话变化实质的同义词。“他指出”、“他解释说”、“他回答道
”、“他补充说”等,这些词语都带有特定的含义。但假如受访者只是在肯定地说什么,而不是在他已说过的话之后加什么,就不要用“他补充说”。

所有这些技巧只能帮你到此。一次成功的采访,最终与作者的性格与个性有关,因为你所采访之人对有关题目的了解总是超过你。有关如何在不对等的情形下克服焦虑,以及如何学会相信自己的聪明才智,本书第20章“愉悦、恐惧与信心”提供了答案。

恰当的和不恰当的引语使用在新闻中都很常见,其中就有某些知名度很高的事件。有一个是有关珍妮特·马尔克姆污蔑诽谤罪的审判。陪审团发现她涉嫌在《纽约客》传略中“伪造”心理医生杰弗里·M·马松的某些引语。另一个是由乔·麦金尼斯透露的。在他写参议院议员爱德华·M·肯尼迪的传记《最后一个兄弟》中,他“以我推测是他的观点为依据,撰写了某些场景,描述了某些事件”,但其实他从未采访过肯尼迪本人。如此模糊的事实和想象成为困扰严谨作者的一种趋势,也是对此类写作技能的一种侵犯。然而即便对自觉的记者来说,这也是不确定的领域。让我来援引约瑟夫·米切尔的作品作一指导。在散文中无缝编织引语是米切尔在1938—1965年为《纽约客》撰写的出色文章所取得成就的标志,其中许多是写工作在纽约码头附近的人们的。这些文章对我这一代非虚构作家影响重大,是一本基础教材。

米切尔的六篇文章后来逐渐成书,即《海港之底》,成为美国非虚构作品的经典。这六篇文章在20世纪40年代晚期和50年代早期以极为恼人的频率刊登于《纽约客》,刊登时间经常相隔几年。有时我会问在那家杂志工作的朋友何时出此系列的新文章,但他们从不知晓,甚或从不冒昧地猜想。这是马赛克式的作品,他们提醒我说,马赛克师对于如何将一块块碎布拼嵌在一起非常讲究,非得拼嵌对了才行。当新文章终于出现,我明白了为何需要那么长时间:真就非得这样写才对。我仍记得阅读《亨特先生的坟》一文时的激动心情,那是我最喜欢的米切尔文章。该文描述了一位87岁高龄的非裔循道公会教堂的长者,他是斯塔腾岛名为沙地村的一个19世纪黑人采蚝村庄最后一批幸存者之一。由于《海港之底》的连接,过去的记忆成为米切尔作品中的主要人物,赋予其既有挽歌式又有历史性的基调。作为主要描写对象的老人们是记忆的守护者,他们是连接早期纽约的活的纽带。

下面这段引用了乔治·H·亨特对美洲商陆这种植物的描述,是《亨特先生的坟》中许多很长的引语中典型的代表,由轻松、愉快的细节合成:

“春天的时候,当它刚长出来时,根部上端的嫩芽很好吃。这些嫩芽吃起来像芦笋。沙地村的老太太过去都迷信吃美洲商陆的嫩芽,那些南方老太太们。她们说这样可以更新人的血液。我母亲就信这个。每年春天她都指使我去林子里采美洲商陆的嫩芽。我也信这个。所以每年春天,假如想起来,我就去采一些做熟了吃。并不是我有多喜欢吃这些东西——其实这些嫩芽让我直放屁——只是这些东西让我想起过去的日子,想起我的母亲。眼下,斯塔腾岛的这个地方,在远离市区的林子里,可能在某个不知名的地方另一边十五英里处,就在亚瑟杀人的路上面不远,靠近阿登大道上方,有一个转弯处,从那里你有时可以看见纽约摩天楼的楼顶。只能看见最高的摩天楼,只是这些摩天楼的楼顶。但必须是特别晴的天才行。即使这样,你也可能一会儿看得见,一会儿又看不见了。就在这路上转弯处旁边,有一块小沼泽地,沼泽地的边上是我所知的最佳采摘地。今春的一个早上,我去那里想采一些,可是这个春天来得晚,你还记得吧,美洲商陆还没长出来。卷牙长出来了,还有奥昂蒂、春美草、臭菘和矢车菊,就是没有美洲商陆。所以我这儿看看,那儿看看,没注意踩在哪儿,一下子踩空了,等我明白过来,淤泥已经到了膝盖。我在淤泥里挣扎了一分钟,定了定神,这时我刚好抬头向上看,突然间我看见,在很远处,几英里、几英里以外,纽约摩天楼的楼顶在晨光中熠熠生辉。我真没想到,太叫人惊讶了。那就像《圣经》中的幻象。”

好了,其实没人认为亨特先生真的一口气说完了所有这些,米切尔做了大量拼接。然而我毫不怀疑亨特先生在此时或彼时的确说了,所有这些词语和措辞都是他的。这听起来就像他;米切尔并没有依照他所推测的受访者的观点来描写这些情景。他做了文学性的布局,假设自己花了一个下午被带着去墓地转了转;与此同时,由于我了解他出了名的耐心细致和彬彬有礼以及精雕细琢的方法,我猜想这篇文章至少花了他一年的时间四处辗转、聊天、写作、修改。我很少读到质地如此丰富的文章,米切尔的“下午”有那种实实在在的下午的从容不迫的特质。到文章结束的时候,亨特先生已经回顾了在纽约港挖牡蛎的历史,在沙地村逝去的一代又一代人,每个家庭与家庭成员的姓名,种庄稼和做饭,野花和水果,鸟儿和树木,教堂和葬礼,变化和衰落;他已经接触到了生活的方方面面。

《亨特先生的坟》毫无疑问是非虚构作品,但米切尔在这篇文章中改变了已逝去的时光,他以剧作家独特的技法,压缩、突出了其中的故事情节,因而给予读者一种可控的叙述框架。假如他按照事件实际发生的时间顺序讲故事,将自己在斯塔滕岛上度过的日子串联起来,其结果会像安迪·沃霍尔拍的电影一样,一个人睡觉八个小时,电影的时长就得八个小时,以显示其真实性。米切尔通过审慎的操控,将非虚构技巧提升到了艺术的高度,但他从未操控过亨特先生的真实性。这里既没有“推测”,也没有“杜撰”。他做得客观公平。

最后讲一讲我的标准。要写好一篇有力度的访谈,不改一改、删一删被访者的原话是不可能的。任何作家都得这么做。对此有两种不同的意见,彼此呈现出许多细微的差别。纯洁主义者会说约瑟夫·米切尔拿了小说家的魔杖描写事实,而革新者则会说米切尔是一位开拓者——他比别人早几十年就践行了“新新闻学”。作家如盖伊·塔利斯和汤姆·沃尔夫由于在60年代发明此方法而大受欢迎。他们运用充满想象的对话与情感等虚构技巧,给以严肃事实为基础的叙述作品增添了光彩。其实两种观点都有合理性。

我认为,错误的做法是在作品中杜撰引语或猜测某人可能说了什么。写作是一种公共信任。非虚构作家少有的特权是拿整个世界上的真人真事来写。当你让人们说话的时候,对待他们所说的话要像对待贵重的礼物一样。