\chapter{写你自己:回忆录}
在作者可以利用的所有题材中,最熟悉的是自己:自己的过去和现在、自己的思想和情感。然而这些也可能是自己最想竭力回避的。

每当我受邀去学校或大学上写作课,我问年轻学生的第一件事是:“你的问题是什么?你关心什么?”从缅因到加利福尼亚,他们的回答都一样:“我们得写老师想要的。”这句话真令人沮丧。

“这是好老师最不想要的,”我告诉他们。“没有哪位老师想要25份千篇一律的作文。我们要寻找的,我们想看到从你的作文中蹦出来的,是个性。我们要找的是任何使你独一无二的东西。写你自己的所知所想。”

中年人也并没有摆脱这个问题。在作家研讨会上,我遇见一些女士,她们的孩子都已长大,现在她们想通过写作来清理一下自己的生活。我敦促她们运用个人生活中的细节,写最靠近自己的。她们表示反对,说“我们得写编辑想要的”。换言之,还是“我们得写老师想要的”。为什么她们认为自己需要得到许可才能写自己最熟悉的或者她们自己的体验与情感?


让我们再跳到另一代人。我有一位记者朋友,他一辈子踏踏实实地写作,但总是用二手材料阐释他人的事情。几年来我常听他提到他的父亲,一位在堪萨斯一座保守的城里持自由立场、孤立无援的牧师。显然我这位朋友强烈的社会良知就来 自于此。几年前我问他何时开始写生活中对自己真正重要的内容,包括他父亲。有朝一日吧,他说。但这一天总被推迟。

当他到了65岁,我开始不断烦扰他。我寄给他一些感动过我的回忆录,最后他同意把自己的上午时间花在写回顾性的内容上。现在他简直不敢相信自己踏上的是怎样一条解放之路:他发现了许多自己不明白的有关父亲的事情,以及有关他自己的生活。但当他描写自己走过的旅程时,他总是说,“我以前从没有这个胆量”,或者“我总是害怕尝试”。换言之,“我之前并不认为自己得到了许可”。


为什么不呢?难道美国不是“坚强的个人主义者”之地吗?让我们找回那丧失的领土和那些迷失的个人主义者。如果你是写作课老师,让你的学生相信他们自己的生活值得一写。如果你是作者,许可你自己告诉大家你是谁。

说“许可”,我不是指“放任”。我可没有耐心放任松松垮垮的写作技能——60年代那种“一切都随它去”的措辞。要想在这个国家拥有一项体面的事业,能写出体面的文字很重要。但针对你为谁写作这个问题,不要急于讨好。如果你有意识地为老师或编辑写,结果是你为谁也没写。如果你为自己而写,你就会达到你的写作目的。

写自己的生活自然与阅历有关。当学生说他们得写老师想要的,其实常常是指自己没什么想说的——他们的业余生活极为贫乏,主要的活动不是看电视就是逛购物中心,都是人造的现实世界。而在任何年龄,写作这一具体活动都有强大的探索机制。浸入自己的过去,我会发现某些已被遗忘的事情在我需要的时候会准时到位,这常使我感到惊讶。当你的其他资源之井干枯的时候,你的记忆几乎总能提供素材。

不过,许可是一把双刃剑,使用时人人都要牢记军医的警告:过度写自己无论对于作者还是读者的健康都是有害的。自我与利己主义只有一线之隔。自我是健康的,作者若无自我行之不远。利己主义则是障碍。虽然本章无意发放许可来空谈疗法,我还是要提出规则,以确保回忆录中的每一个组成部分都起作用。写自己,要不遗余力地用自信和愉悦的心情去写自己。但要保证所有细节,包括人物、地点、事件、轶事、思想、情感,都能稳步地推动故事的进展。

这就涉及回忆录的形式问题。几乎任何人的回忆录我都读。对我来说,没有其他非虚构形式能如此深入地进入个人经历的根基——进入人生的所有戏剧、痛苦、幽默以及不可预知性。在我读过第一遍的书中,记忆最真切的一般都是回忆录,比如安德烈·艾西蒙的《走出非洲》、迈克尔·J·阿伦的《流放》、拉塞尔·贝克的《长大》、薇薇安·戈尼克的《强烈的依赖》、皮特·哈米尔的《醉酒人生》、莫斯·哈特的《第一场》、约翰·豪斯曼的《排练》、玛丽·卡尔的《说谎者俱乐部》、弗兰克·麦考特的《安吉拉的骨灰》、弗拉基米尔·纳博科夫的《说,记忆》、V.S.普里切特的《门前的出租车》、尤多拉·韦尔蒂的《一个作家的开端》、伦纳德·沃尔夫的《成长》。

给这些回忆录以力量的是聚焦的狭窄度。不像跨度为一生的自传,回忆录也呈现人生,但只精选一小部分而忽略大部分。回忆录作者带我们回到其过去某一段刻骨铭心的时光,比如童年,或者饱受战争或某种社会动荡影响的岁月。贝克的《长大》就是盒中盒。故事记述了遭受大萧条打击的家庭中一个男孩成长的经历,其力度来自当时的历史环境。纳博科夫的《说,记忆》是我所知的最优雅的回忆录,它激起人们对沙皇时代在圣彼得堡度过的金色童年的回忆,那是一个家庭教师与夏宫的世界,是一个俄国革命永远将其终结的世界。这一写作行为冻结于特定的时间和地点。普里切特在《门前的出租车》中回顾了几乎是狄更斯一样的童年,他在伦敦皮毛行当学徒的的那段阴郁时期似乎属于19世纪。然而普里切特的描写并没有自怜,反倒有些自娱。我们看到他的童年与他所出生的特定时期、特定国家以及阶层密不可分,是他成长为了不起的作家的有机部分。

当你尝试回忆录写作的时候就要想得窄一点儿。回忆录不是整个人生的概述,它是通向生活的窗户,很像在构图上具有高度选择性的照片。它看起来像是在随意、甚至任意地回顾往事,实则不然,它是有意的建构。梭罗八年间写过七个不同版本的《瓦尔登湖》草稿。在美国人的回忆录中,没有比这一部更难完成的了。要写好回忆录,你必须成为自己生活的编辑,赋予散落四处、半遗忘的事件以叙事结构与组织思路。回忆录是发明真实的艺术。

艺术的秘诀之一是细节。任何细节都会有效——一个声音、一种味道或者一首歌的名字——只要它在你所提取的那部分生活中起结构性的作用。想一想声音。看一下尤多拉·韦尔蒂是如何开始《一个作家的开端》一书的,看起来薄薄的一本书,却充满了丰富的回忆:

我1909年出生在密西西比州杰克逊城北国会街自家的房子里,是三个孩子中最大的一个,我们都是伴随家里敲打的钟声长大的。门厅里放着一架教堂样式的橡树老爷钟,它撞出鸣锣般的响声,传到客厅、餐厅、厨房、配餐室,连楼梯的板壁都声声作响。整个晚上,钟声钻进我们的耳朵,有时我们甚至在卧廊都会被半夜惊醒。我父母的卧室里有一台小撞钟与之呼应。厨房里的钟不发声而只显示时间,餐厅里的是个布谷鸟钟,下面有坠连在长链条上。有一次我小弟弟爬上椅子够到碗柜的顶部,成功将猫在坠上悬吊了一会儿。我不知道父亲在俄亥俄州住时的老家是否与此有关,不过在18世纪初第一批韦尔蒂兄弟来美国之前,这一家都是瑞士人;反正我们大家一生都特别在意时间。这对于一位未来的小说家来说至少是件好事,它使我学会了深刻地洞察时间的顺序,而且在描述事件时几乎总是想首先处理好时间安排问题。这是我在几乎不知不觉中就学会的许多好事之一,写作中掌控时间的技巧对我是随需随到。

我父亲喜爱所有教育人、愉悦人的工具。他保存东西的地方是图书馆式大桌子的抽屉,在里面,折叠起来的地图之上有一架带黄铜伸缩部件的望远镜,晚饭后他用它在前院观察月亮和北斗七星,还按时观看月食、日食。还有一架折叠式柯达相机,在圣诞节、生日以及外出旅行时用。在抽屉的最里面有放大镜、万花筒、陀螺仪,都放在一个黑色硬麻布盒子里。他会让陀螺仪在绷紧的绳子上跳舞给我们看。他还为自己准备了一套游戏拼接组件,由穿在一起的金属环、连接链和连接锁构成,父亲无论怎么耐心地教,我们都拆不开。父亲对精巧的发明有一种近乎孩童般的热爱。

不久,餐厅的墙上又多了一个气压表,但其实我们并不需要气压表。父亲像乡下孩子一样对天气和天空的变化了如指掌。他每天早晨第一件事就是站在门前台阶上,观察一下天气,然后吸一口气。他是个很棒的天气预报员。

“可我不是……”母亲会极为得意地这么说。

因此我对于天气变化也极为敏感。后来我写小说的时候,环境气氛从一开始就起到很大的作用。天气的剧烈变化以及由此引起的内心情感的起伏,会以戏剧性的形式共同呈现在我的作品中。

注意,我们一开始就马上了解到许多有关尤多拉·韦尔蒂早年的信息——她出生的家境、她父亲的为人。她以钟表声带我们进入她的密西西比少女时代,钟表的报时声在楼梯上下回荡,甚至向外传到了卧廊。

阿尔弗雷德·卡津运用气味作为线索,来追寻自己在布鲁克林区布朗斯维尔地段所度过的童年。从我很久以前第一次读到卡津的《城市里的步行者》之时起,我就记得那是一本感官特点浓重的回忆录。以下这段不仅是用鼻子写作的范例,而且还显示出回忆录如何受作家所创造的地点滋养——那是一种使其街区和传统与众不同的感觉。

至于星期五的晚上,我最喜欢的是街道上的黑暗和空荡,就好像为那一天的休息和拜神做准备,而犹太人则像迎接“新娘”一样对待那一天,就连碰钱都是被禁止的,他们停止一切工作、一切旅行、一切家务,甚至连开灯关灯都停止了——犹太人通过饱受痛苦的心灵,找寻到自己某种远古的中心。我等待星期五晚上街道黑下来,就像其他孩子等待圣诞灯亮起来一样……当我三点回家的时候,烤箱里焙烤的咖啡饼的温暖气味,妈妈手脚着地在厨房地板上刷洗油地毡的身影,使我心中充满温馨的感觉,我觉得自己的感官都伸展出去,拥抱住家里的每一个物件……

我最盼望的时刻六点钟到来,这时父亲下班回家,他的工作服闻起来还有淡淡的松节油和虫胶清漆的味道,下巴上还有银漆的白点闪闪发亮。他把大衣挂在通往厨房的又长又黑的走廊,一只口袋里留着一份折得松散的纽约《世界报》。随后我大脑的另一半会唤起纽约东河那边的一切,新印出来的报纸油墨味道和头版的地球图案浮现在脑海里。这份报纸对我来说承载着与布鲁克林大桥特别的联想。编辑部坐落在俯瞰大桥的公园街上,《世界报》就是在那里的绿屋顶下出版的。纽约港清新、咸咸的空气滞留在我家的走廊里,混合在油漆和新闻纸的气味中。我感到父亲通过每天一份的《世界报》,直接将外面的气息带进家里。

卡津后来跨过布鲁克林大桥,成为美国文学评论界的泰斗。但他心目中的文学体裁并非诸如长篇小说、短篇小说或诗歌之类的文学形式,而是回忆录,或者他所称的“个人历史”——特别是那种他孩提时发现的“美国个人经典”,如沃尔特·惠特曼在内战时期的日记《典型的日子》和他的《草叶集》,梭罗的《瓦尔登湖》,特别是他的日记以及《亨利·亚当斯的教育》。使卡津兴奋的是,惠特曼、梭罗和亚当斯都大胆运用最个性化的形式——纪事、日记、信件、回忆录——将自己写入美国文学的地貌之中,而他通过书写个人历史,创造出同样与美国“值得珍视的联系”,因而也将自己,这位俄国犹太裔的子孙,置于同样的地貌之中。

你可以用自己的个人历史跨越你自己的布鲁克林大桥。回忆录是捕捉美国新移民生活的最佳形式,每一位移民的儿女都从自己的文化中带来与众不同的声音。下面是恩里克·汉克·洛佩斯撰写的《回到巴钦巴》中的一段,典型地反映出被遗弃的过去和被抛在身后的故乡所发出的巨大力量,它给予这种写作形式以丰富的情感。

我是来自巴钦巴的“老墨”,那是一个很小的墨西哥乡村,在齐瓦瓦州,我父亲曾在那里为比亚的军队打仗。他其实是比亚军队里唯一的二等兵。

一般来讲,“老墨”在墨西哥是一个贬义词(简单讲,“老墨”是指那种粗俗平庸的墨西哥人,却装作是狗娘养的外国佬),但我用其特别的含义。对于我,这个词意味着“拔了根的墨西哥人”,而那正是我一生的写照。虽然我的整个成长和教育过程都在美国,我却从未真正感到过自己是美国人;而当我在墨西哥,有时又感觉像一个错了位的外国佬,有一个古里古怪的墨西哥人的名字。人们可能会下结论说,我不是文化分裂型的墨西哥人,就是有文化的分裂型美国人。

不管怎么说,这种精神分裂早就开始了,当时我父亲和比亚队伍里的许多人逃过边境,以躲避正在压近的、最终击败比亚的墨西哥联邦部队。母亲和我乘平板马车穿越炽热的沙漠平原,在父亲仓皇逃离几天之后,在得克萨斯的埃尔帕索同父亲会合。随着比亚的人每天蜂拥涌入埃尔帕索,很显然工作会越来越少而且不稳定,于是父母打点好我们仅有的几件家具,赶第一辆公共汽车去了丹佛。父亲希望搬到芝加哥,因为那个城市的名字听起来很墨西哥化,但是母亲少得可怜的积蓄都不够买票去科罗拉多。

在那里,我们搬进了讲西班牙语的居民区,那里的人特意称自己为西班牙裔美国人,他们憎恶突然从墨西哥迁徙过来的同族兄弟,嘲讽地称其为“南方佬”……我们这些“南方佬”开始在这个大居民区的一个次街区聚集起来,而正是在那里,我才痛苦地意识到父亲是比亚军队里唯一的二等兵。我的大多数朋友都是上尉、上校、少校甚至将军的儿子,只有少数几位父亲承认他们只是中士或下士……还有另一个现象越发加重了我的苦闷,那就是比亚的丰功伟绩是家里谈话中从不间断的话题。我的整个童年似乎都笼罩在这一阴影之中。几乎每晚的饭桌上,我们都会听到重复了无数遍的对这场战役、那次战略的叙述,或者某种罗宾汉施善似的伟大行为……

就好像要加深我们对比亚主义的印象,父母教我们《爱德丽特》和《他们将大炮送到巴钦巴》这两首墨西哥革命时期最著名的歌曲。大约二十年后(我在哈佛法学院任职的那段时间),沿着查尔斯河边散步,我还会不由自主地反复哼唱“他们将大炮送到巴钦巴,送到巴钦巴,送到巴钦巴”,这就是我能记住的那首带有强烈造反激情的歌曲全部。虽然我出生在那里,但我总认为 “巴钦巴”是一个虚构的、杜撰的、路易斯·卡罗尔\footnote{路易斯·卡罗尔(Lewis Carrol,1832—1898):英国数学家、作家,著有童话《爱丽丝漫游仙境》,其中有许多杜撰的地名。}式的名字。因此,当我八年前首次回到墨西哥,来到齐瓦瓦南面的一个十字路口,看见一块老路牌“巴钦巴18公里”时,简直惊呆了。还真有这 么个地方——我内心大喊——巴钦巴是一个真实存在的城镇!驾车拐到狭窄、铺得很差的路上,我加大马力,开向儿时就一直歌唱的镇子。

对于加利福尼亚州斯托克顿市中国移民的女儿汤婷婷来讲,一个孩子在异国他乡开始上学,这样的人生经历充满了腼腆与尴尬。下面这段选自《女勇士》,标题“找到自己的声音”十分恰当。注意汤婷婷是如何生动地回忆起自己早年在美国那令人伤痛的事情和情感的:

我上了幼儿园,第一次必须说英语,这时我变得沉默寡言。甚至现在当我想说一个随意的“你好”,或者在前台问一个简单的问题,或者向公车司机问路的时候,还会有一种莫名的沉默——一种羞愧将我的嗓音撕成两半。我会僵直地站着……

第一年,我沉默无语,在幼儿园不跟任何人说话,去厕所也不请假,成绩也不及格。我姐姐也是三年没说话,在操场上沉默,午餐时同样沉默。除了我们家的女孩之外,还有其他沉默寡言的华裔女孩,但她们大都比我们克服得早。我喜欢沉默寡言。开始我还没意识到自己该说话,或者该通过幼儿园的考试。我在家里说话,也对班里一两个华裔孩子说话。我做动作,甚至开玩笑。我用玩具碟子托着杯子喝水,当水溅出托在玩具碟子上的杯子时,大家哄堂大笑,用手指着我,于是我就再演一遍。我不知道美国人不用碟子托着杯子喝水……

当我发现自己必须说话的时候,上学成立痛苦的事,沉默寡言也成了痛苦。我不说话时,而且每次不说话时,都感到心情不好。 不过我在一年级的时候也朗读课文,听见自己喉咙里吱吱地发出最空洞的低语。其他华裔女孩也不说话,因而我知道这种沉默寡言同作为华裔女孩的身份有关。

童年的低语现已成为用智慧与幽默向我们诉说的成年作家的声音,我对于在我们之中有这样的声音心存感激。一位华裔女孩就这样被扔进美国的幼儿园,被指望成为美国女孩,除了华裔美国女性之外,无人能够使我感到作为华裔女孩会是什么样。回忆录是使文化差异变得富有意义的一种方式,这些差异在当今美国可能是日常生活中痛苦的事情。在下面这篇《为我的印第安女儿》一文中,想一想刘易斯·P·约翰逊所描述的对身份认同的追求。约翰逊在密歇根州长大,是帕塔瓦米族奥塔瓦人公认的最后一位酋长的重孙子:

在大概35岁的某一天,我听说有一种帕瓦仪式。我父亲过去常参加这种仪式,于是我带着极大的好奇心和异常的兴奋,去发现我该继承的那部分传统。我打定主意,为这件大事做准备,让朋友在他的铁匠铺为我造一支矛。钢矛质地精良,蓝光闪闪,耀眼夺目。矛杆上羽毛鲜亮,威风凛凛。

在南印第安纳州一个尘土飞扬的集市广场,我发现有些白人一身印第安人打扮。据说他们是“癖好者",也就是说,他们有在周末装扮成印第安人休闲的癖好。这时我感到自己拿着矛很滑稽,于是就离开了。

过了好多年我才有勇气告诉别人那个周末我有多么窘迫,并且在其中看到了某种幽默性。但从某个方面来说,不管当时那个周末如何寂静,它都警醒了我。我意识到我并不知道自己是谁。我没有印第安人的姓名,不说印第安语,不知道印第安人的习俗。我恍惚记得奥塔瓦词儿“狗”,但那是儿语kahgee,不是全称muhkahgee,后来我才学会它的全称。更恍惚的是 ,我记得曾有过一次起名仪式(给自己起名)。我记得周围到处都是舞动的腿,四下里尘土飞扬。那曾在哪儿?我曾是谁?“Suwaukquat,就是树木开始生长的地方,”我问的时候,母亲这样告诉我。

那是在1968年,而我并非是这个国家唯一感到需要记住自己是谁的印第安人。还有其他人也有同感。他们举行帕瓦仪式,真正的仪式,后来我终于找到了他们。我们共同探寻自己的过去;对我来讲,这一探寻的顶点就是那次最长的行走,即1978年华盛顿大游行。也许因为我现在知道了作为印第安人意味着什么,我对别人仍不知道感到吃惊。当然,我们印第安人现在剩下的也不多了。一个普通人在自己普通的一生中认识一个普通的印第安人的机会不会太多了。

回忆录中关键的因素当然是人。声音、气味、歌曲以及卧廊只是带你到此而已。最终,你必须召回生活中让你印象深刻的人们。是什么让你对他们难以忘怀——什么天赋的聪慧,什么怪癖?在回忆录这一大鸟舍中,有一只典型的怪鸟是约翰·莫蒂默的父亲。他儿子在《抓住残骸》中回忆起这位盲人律师。这部回忆录成功地刻画出一位既温情脉脉又滑稽可笑的父亲。莫蒂默本人也是律师,同时还是一位多产的作家和戏剧家,以作品《法庭的鲁波尔》著称。他写道:父亲失明时,还坚持帮人打官司,就好像什么也没发生似的。为此母亲成了给他读辩护状和为他受理的案子做记录的人。

我母亲成为法庭上家喻户晓的人物,就像法警或庭长一样出名,带父亲从一个法庭到另一个法庭,耐心地微笑着。父亲则用带有云纹状图案的马六甲白藤手杖点着铺设平整的地面,向母亲或者向他的诉状律师大声责骂,或者同时责骂他们两个。战争早期,当他们在这个国家最终安顿下来,母亲就每天开上四英里的车,送父亲去亨利火车站,带他上车。父亲安坐在车厢一角的座位上,穿着打扮像温斯顿·丘吉尔:黑色外套,条纹裤子,领结扎在燕子领上,长筒靴上带有鞋罩。父亲会要求母亲用响亮而清晰的嗓音朗读当天离婚案的证据。随着火车嘎嘎作响地停靠在梅登黑德附近,一等车厢突然寂静无声,此时母亲正在读私人侦探调查到的有关通奸行为细节的报告。一旦她降低嗓音读到弄脏的亚麻床单,七零八落的男女衣物,或是轿车里的不轨行为,父亲就会大叫,“大声点儿,凯丝!”这时同车的旅客就能享受另一集耸人听闻的连载故事。

大家希望回忆录中最有意思的人物最终会是写回忆录的人。这位男士或女士从生活的山丘峡谷中都悟到了什么?弗吉尼亚·伍尔夫是高度个性化写作形式的积极实践者。她用回忆录、日志、日记、信件来捋清自己的思绪与情感。(我们经常开始出于义务写一封信,写到第三段时却发现还真有想对对方说的。)弗吉尼亚·伍尔夫一生所写的私密内容,对于那些同样与天使和魔鬼较量的其他妇女,有极大的益处。肯尼迪·弗雷泽曾为一本有关伍尔夫儿时遭性侵犯的书写过一篇书评,她在其中明确指出伍尔夫对她的影响。该文以回忆录形式开头,其坦诚与脆弱引人注目:

有一段时间,我感到生活似乎异常痛苦,读其他女作家的人生遭遇是少数的几种解决办法之一。我不幸福,并为此羞愧。我被生活所累。在我三十出头的几年里,常常坐在扶手椅里阅读他人的生活。有时读完之后,我会坐下来再从头通读一遍。我记得那一切难以言状的强烈情绪,以及一种鬼鬼祟祟的感觉——就好像唯恐自己被别人从窗外看到。甚至现在,我都感到自己应该假装只是在读这些女人的小说或诗歌——她们有意选给读者看的人生部分,化为艺术的那部分。但这却是谎言。我真正喜欢的正是那些私密的内容——日志和信件、自传和传记,只要她们似乎讲的是真话就好。我那时感到很孤独,只关注自己,十分自闭。我需要所有这些喃喃的合唱,这种不间断的真实故事,帮我渡过难关。她们对我来说就像母亲和姐妹,这些文学妇女中的许多人早已故去;她们似乎伸出手来帮我,超过我的家人。像其他许多人一样,我很小就来到纽约,为的是塑造自我。而像许多现代人一样——特别是现代妇女——我被弹射出自己的生活环境……(作家们的)成功当然给我希望,然而我最喜欢的却是她们生活中绝望之时的点点滴滴。我在寻找方向,收集线索。我特别感激的是有关这些女人的隐秘、羞愧的事情——她们的痛楚:堕胎、不匹配的婚姻、她们吃的药、她们喝的酒。还有,是什么使她们成为女同性恋者,或者爱上男同性恋者,或者爱上有妇之夫?

当你写个人历史的时候,必须贡献的最佳礼物就是你自己。要允许自己写自己,并享受这一过程。