\chapter{写艺术:评论家与专栏作家}
艺术无处不在,它丰富着我们每天的生活,无论我们亲自实践——如演出、跳舞、作画、写诗、弹奏乐器——还是到音乐厅、剧院、博物馆以及画廊去观赏,都是如此。我们还想阅读有关艺术的评论,无论艺术在哪里,我们都想同当今文化潮流保持接触。

此类写作任务的一部分是新闻性质的,比如采访新交响乐指挥、与建筑师和馆长一同参观新博物馆,这些与本书所讨论的其他写作形式需要同样的写作方法。写新博物馆是如何设计、筹资和建造的,从原则上讲,与解释伊拉克如何差点儿造成原子弹并没有什么差别。

但要从内行的角度去写艺术——评估新作品、评价演出、辨认什么好什么差——则需要一套特别的技艺和特别的知识储备。简言之,这需要你做一名评论家,每一位作者在个人生涯的某一段,都几乎想做评论家。小城的记者们梦想着,有朝一日他们的编辑会召唤他们报道已经在本地剧院卖票的一位钢琴家、一个芭蕾舞团,或者一家保留剧目轮演剧团。他们会匆匆忙忙翻出大学期间辛勤学来的词语——“凭直觉”、“敏感性”、“卡夫卡式的”——并且还会向全镇的人显示自己能从安特雷沙芭蕾舞动作中分辨出滑奏,在易卜生的作品中分辨出比易卜生本人所能想到的更多象征来。

这只是此类强烈欲望的一部分。评论是记者最富于想象力的花式慢舞的舞台,也是其风趣之声誉诞生之处。美国俗语富于警句妙语,如“她经历了从A到B的全音阶感情波折”,这些作者包括多萝西·帕克和乔治·S·考夫曼,他们的声誉应该部分地归功于创造了这些词语;而以评论毫无天赋的蹩脚演员的代价去留名,这对于所有人的诱惑都实在太大了,最圣洁的人除外。我特别喜欢考夫曼的暗示语,他是这样评价雷蒙特·马西在《艾贝·林肯在伊利诺伊》中演主角演过了头的:“马西他不演到被暗杀绝不罢休。”

但真正的风趣罕见,正如射出去一支带刺的弓箭,会有一千支落在弓箭手的脚下。假如你要写正经的评论,这个办法则过于容易了,因为留存下来的警句妙语都是狠话。埋葬恺撒要比赞美他容易得多——对埃及艳后克娄巴特拉也是如此。但是要用听起来并不平庸的话来说明你为何认为一部剧好,却是这个行当里最难做的工作之一。

因此不要上当受骗,以为评论是通往荣耀的捷径。此项工作所享有的权利也并非猜想的那么广。也许只有《纽约时报》的每日剧评人才能成就一部创作或扼杀一部创作,而乐评人则几乎没有这种权利,他们评论的那一串音符,其声音很快就消失在空气中,人们再也不会以同样的方式听到它。文学评论人也没能使畅销书排行榜免于成为丹尼尔·斯蒂尔之类作家的容身之地,他们并没有凭直觉发现他的感受力。

因而我们需要区分“评论家”与“评介者”。评介者为报纸或大众杂志写作,他们报道的基本上是某个产业的产品,比如影视产品,以及不断增多的出版业产品,如泛滥成灾的烹饪、健康、指南类图书,“本人口述”类图书,礼品图书,以及其他类似的商品。作为评介者,你的工作更多的是报道,而非作出美学评价。你代表着想了解情况的普通男女:“新电视剧演的是什么?”“这部电影对孩子是否健康?”“这本书真能改善我的性生活吗,或者教我如何做巧克力慕斯蛋糕?”想一想,假如你要花钱看电影、雇保姆、在好餐馆享用盼望已久的晚餐,你都需要知道什么。显然你会使自己的介绍更平实,不那么复杂,远不像你在评论新编排的契诃夫的剧作时那样。

不过我提出了几种情形,既适用于好的评介也适用于好的评论。

其一:评论家应该喜欢、最好喜爱所评的媒介。如果你认为电影愚蠢,就别写它了。应该能有一位电影迷带给读者丰富的相关知识、激情以及个人之见。评论家没有必要喜欢每一部电影,他的评论只是个人的看法。但他应该抱着喜欢的愿望去看每一部电影。假如他经常失望多于愉悦,那是因为该电影没有发挥出最好的潜能。这大不同于那种以憎恨一切为荣的评论家。他的厌烦来得比你的“卡夫卡式”之类的词语还要快。

其二:不要透露太多的情节。恰到好处地告诉读者,让他们自己决定是否是自己喜欢的故事,但不要告诉太多,不然会扼杀读者的兴致。常常一句话就能起作用。“这是一部关于一位难以捉摸的爱尔兰牧师的电影,他雇了三个孤儿做帮手,三个孩子都打扮成传说中能指点宝藏的矮妖精,他们去一个村庄搜寻,那里有一位吝啬的寡妇藏了一罐金子。”我不会被抽打着去看那部电影,我已经在舞台上和荧屏中看够了“小矮人”,但还是有很多人并不认同我的怪想法,他们愿意成群结队地去看这部电影。不要泄漏故事的每一个悬念,这样会搅了他们的乐趣,特别是不要透露有关桥下侏儒的滑稽情节。

第三:运用特别的细节。这样可以避免流于笼统,而笼统意味着什么也没说。“这个剧总是令人着迷”就是典型的评论家用语。但它是如何令人着迷的呢?你对令人着迷的看法不同于别人。要举几个例子,让读者在自己的着迷秤上衡量。下面是两篇约瑟夫·洛西执导的电影的影评节选:(1)“在试图文明化与自律化的同时,电影拒绝了粗俗的可能性,并误把无流血冲突当成品味。”这句话很含混,它向我们提供了一丝电影的气氛,但毫无可视觉化的意象。(2)“洛西追求一种在灯影下发现征兆、在桌案摆设上发现意义的风格。”这句话很精确,它使我们知道了这部电影的艺术特点何在。我们几乎能看见摄影机小心谨慎、慢慢地停留在家庭水晶摆设上。

在书评中,这意味着允许作者用文字本身来说明问题。不要说汤姆·沃尔夫的风格是华而不实、与众不同;要引用几句他华而不实、与众不同的话,让读者自己看到这些句子有多么怪异。评论戏剧演出时,不要告诉大家舞台布景有多么“醒目”;要描述其变幻多样的层次,或者舞台布景是多么巧妙地被照亮,或者它如何有助于演员出场、退场,如何不同于传统的方式;要将读者置于剧院的座位上,帮助他们看到你所看见的。

最后一条:要注意的是,评论家激动时会用诸如“心旷神怡”、“灿烂辉煌”之类的词语,要避免此类虚夸性的形容词占到不恰当的比例。好的评论需要用精炼而生动的风格来表达你的观察和思考。花哨的形容词带有气喘吁吁式散文的特定风味,《时尚》杂志就喜欢用此风格透露其最新的时尚发现:“我们才听说这无与伦比且令人心驰神往的科苏梅尔小海滩!”

关于此类作品的评介以及写作的简单规则到此为止。评论又该是什么样的呢?

评论是一项严肃的智力行为,其任务是评估严肃的艺术作品,并将其置于该媒介内或该艺术家从前作品的背景之下。这并不意味着评论家必须将自己限制在只评估目标高远的作品上,他们也可以选一些商业性作品,如《法律与秩序》,来点评美国社会和价值观。但总的来说,他们不想在文化产品兜售者身上浪费时间。他们自认为是学者,所感兴趣的是自己领域中思想的作用。

因此,如果你想当评论家,就要深人到你希望自己有所长的领域的文献中去。如果你想当戏剧评论家,则尽可能多看戏剧,好坏、新旧都要看。要通过阅读经典或者观看经典戏剧重演,来补习对以往戏剧的了解。认识你的莎士比亚和萧伯纳、你的契诃夫和莫里哀、你的亚瑟·米勒和田纳西·威廉姆斯,了解他们是如何标新立异的。了解伟大的演员和导演,发现他们在方法上的差异。了解美国音乐剧历史,包括杰罗姆·克恩、格什温兄弟和科尔·波特,罗杰斯、哈特和海默斯坦因,弗兰克·莱赛和斯蒂芬·桑德海姆,阿格尼斯·德·米尔和杰罗姆·罗宾斯等人的特别贡献。只有这样,你才能将每一部新话剧或音乐剧置于一个更悠久的传统之中,分清楚哪些是开拓,哪些是模仿。

我可以为每一门艺术列出同样的清单。一位电影评论家在评论罗伯特·奥尔特曼的新电影时,假如没看过奥尔特曼早期的电影,那他对认真的电影观众就提供不了多大的帮助。一位音乐评论家不仅应该知道巴赫和帕菜斯特里纳、莫扎特和贝多芬,还应该知道勋伯格和艾甫斯,以及菲利普·格拉斯这些理论家和特立独行者,还有实验派。

显然我现在所假设的这群人是修养比较好的读者。作为评论家,你可以预先假设与你为之写作的人们共享某些知识领域。你不必告诉他们福克纳是南方小说家,但假如你要评估一位南方作家的第一部小说,并且衡量福克纳对这位作家的影响,则你必须提出富有启发性的想法,将其付诸笔端,这样你的读者才可以品鉴其气味。他们可能不同意你的观点,那是他们智力趣味的一部分,但他们会欣赏你的秉性以及你得出结论的路径。我们喜欢好评论家,既因为他的观点,也因为他的性格。

没有什么媒介像电影一样赋予我们与评论家共同前行的乐趣。这里有非常广泛的共享领域。电影与我们的日常生活和态度、我们的记忆和神话交织在一起——《卡萨布兰卡》中四行不同的台词进入了《巴特勒特常用妙语词典》——我们依赖评论家为我们打通那些联系。影星们在一部接一部的电影屏幕上展现自己,有时他们来自观星者从不知晓的星系,而评论家所提供的一项典型服务就是让这些影星暂时静止,以便我们观察。在电影《黑暗中的哭泣》里,梅里尔·斯特里普扮演了一位澳大利亚妇女,她在一次野外旅行中犯了杀婴罪。莫莉·哈斯克尔在评论这部电影时,深入思考了斯特里普“假扮的快感,包括奇异的假发、违背常理的穿戴以及外国口音,还有扮演超出观众正常同情范围的妇女形象”。正像好的评论家应该做的那样,莫莉将其置于特定的历史背景中:

老牌影星的光环发自于自我意识,能把一种核心的自我身份投射于每一个角色。贝特·戴维斯、凯瑟琳·赫伯恩、玛格丽特·萨拉文的表演无论怎样变化多端,我们都能感到所面对的是某种可知、熟悉、恒久的东西。他们有可辩别的声音、说台词的方式,甚至他们的某些表情从一部电影到另一部电影也保持了恒久的特质。喜剧演员可以作出模仿,观众要么对此反映强烈,要么就全无反应。但斯特里普就像变色龙,她连根切断了你的这类反应,而从不呆在同一位置上等你定位。

贝特·戴维斯拓展了演员类型的界限,她既喜欢演以服饰为特色的电影(《贞洁女王》),也喜欢演以时代为特色的电影(《老保姆》);但无论如何,她总是贝特·戴维斯,而且没人会想要另一个她。像斯特里普一样,她甚至敢演不讨人喜欢的、道德不定的女主角,她所扮演的最杰出的角色就是《信件》中庄园主的妻子,冷酷地谋害了自己诡计多端的情人,而且毫无忏悔之意。其中的区别是,戴维斯与角色融为一体,在其中倾注了自己的激情和力度。女主角冷酷傲慢,而且像美狄亚\footnote{美狄亚(Medea):希腊女神,精于巫术,曾帮助伊阿宋取得金羊毛,与其私奔,后被遗弃,愤而杀死亲生儿女。}一样无情,这可能是奥斯卡评奖委员会为何没颁奖给她的缘故,她本来该得奖,但奖项却颁给了主演《基蒂·福伊尔》的更甜美温柔的金杰·罗杰,而其实戴维斯让我们对其内心之火的反应更加强烈。很难想象斯特里普这样的演员,能与其角色保持安全的距离,既能上升到如此高度,又能堕入到如此深渊。

以上这一段很巧妙地将好莱坞的过去与现在联系在一起,留下我们自己去揣摩梅里尔·斯特里普的后现代冷漠,同时也告知我们需要了解的有关贝特·戴维斯的一切。扩而展之,它也向我们讲述了整整一代巨星同戴维斯一道称霸这一影星制度的黄金时代,诸如琼·克劳福德和巴巴拉·斯坦威克,她们并不介意在屏幕上遭人恨,只要在票房上惹人喜欢就好。

再转向另一种媒体,来看看迈克尔·J·阿伦的《客厅大战》,这是阿伦60年代中期撰写的电视评论专栏集。

越南战争经常被称为“电视战”,意思是,它是主要由电视带到人们面前的战争。人们的确看电视,他们确实看到了。他们看狄克·范·戴克,并成为他的朋友;他们看乐于思考的切特·亨特利,发现他的确乐于思考;他们看风趣的大卫·布林克利,发现他的确风趣。他们看越南。他们看越南,就像孩子跪在走廊,眼睛对着钥匙眼儿,看两个大人在锁着的房间里争论——钥匙眼儿的开口很小,人影模模糊糊,多数看不见;嗓音分辨不清,被隔绝的威胁毫无意义;孤立的几瞥,肘部弯曲的地方,一个男子的夹克(那人是谁呢?),脸,一个女人的脸。啊,她再哭泣。我看见了眼泪。(说话声在含混不清地继续。)我数着眼泪,两滴,三滴。两次爆炸袭击,四次“寻找目标即刻摧毁”的任务,六次政府公告。如此美貌的女子。我徒劳地寻觅另一个大人,但是,嗨,钥匙眼儿太小,总是有点儿看不准。看!凯将军在那儿。看!有几架飞机安全返回提康德罗加。我想知道(只是有时候),经营电视节目的人如何看待战争,因为他们为我们提供了钥匙眼视窗,我们则报他们以收视率。而现在,在这关键时刻,他们回敬以这钥匙眼儿般的视窗。我想知道他们是否真的认为这些支离破碎的肘弯、脸庞、旋转的裙子(谁是那另一个人呢?)画面,还有这屋子里发生的一切,是我们这些孩子看得下去的。

这是最佳状态的评论:风格化、暗示性、令人不安。评论就应该令我们不安,因为它触动了一套信仰,迫使我们重新审视自己。吸引我们注意力的是钥匙眼儿这个比喻,如此精确但又如此神秘。但剩下的根本性问题是,这个国家最具影响力的媒体如何向人们报道正在进行的战争,而且这场战争仍在加剧。这个专栏登载于1966年,那时多数美国人还在支持越南战争。假如电视报道扩大了那个钥匙眼儿,不但向我们放出“旋转的裙子”,还放出割断的头颅和燃烧的孩子,那么美国人是否会更早起来反对越战呢?现在要知道这一切已为时过晚,但至少有一位评论家一直在关注它。当我们所坚信的不言自喻的真理不再真实时,评论家应该是第一个告知我们大家的人。

与其他艺术相比,有一些艺术更难用文字的形式来捕捉,其中就包括舞蹈,这门人体运动的艺术。作家如何把所有这些优美的跳跃和快速的旋转凝固下来?另一种是音乐,通过耳朵欣赏的艺术,而写作者却苦于如何用我们看得见的文字来描述音乐。他们最多只能部分成功,许多音乐评论家靠藏匿在一整套意大利音乐术语藩篱之后,建立起自己漫长的职业生涯。他会发现一位钢琴家所演奏的“自由速度乐段”有点儿过了,一位女高音的“应用音域”中有一丝尖声。

但即使在音符的瞬间世界里,好的评论家仍可凭借一手好英语,向读者传达所发生的一切。1940-1954年间,《纽约先驱论坛报》的音乐评论家维吉尔·汤姆逊就是一位优雅的践行者。他本身是一名作曲家,博学、文雅,从没忘记读者都是有七情六欲的人,因而他的评论充满激情,能够感染读者,他的风格充满了活力与惊喜。他同时也无所畏惧。在他任职期间,没有哪位音乐人能够逃脱他严谨犀利的评论。他从没忘记音乐人也是有七情六欲的人,因此会毫不犹豫地将这些音乐巨人拉回人间:

不寻常的是,音乐人士彼此间很少谈论托斯卡尼尼在速度、节奏以及音调的适宜度方面的对错。同其他音乐人一样,托斯卡尼尼在这些方面经常反应敏捷,也常常错误百出。而似乎更重要的是他所一直拥有的处理乐曲的能力。一旦发现观众的注意力趋于摇摆,他就会毫无顾忌地加快音乐节奏,牺牲清晰度,忽视基本的韵律,而是让音乐像他的指挥棒一样,不停地转。没有哪支乐曲非得表现特定的含义,每一首乐曲都得激发起听众的热情而使他们欣然接受。这就是我所称的“叫好的技巧”。

这里没有“自由速度乐段”或“应用音域”之类的术语,也没有盲目的偶像崇拜。但这一段落却捕捉到使得托斯卡尼尼伟大的精髓:为演奏界增添了色彩。假如认为其精髓竟包含如此粗糙的成分会让乐迷感觉受到冒犯,他们可以去欣赏这位音乐大师的“抒情色调”或者“交响齐奏”。我则会跟从汤姆逊的判断,而且我认为大师也会如此。

幽默是评论中的一种润滑剂。幽默能使评论家工作时采取迂回的姿态,对其所写的评论自身也带有娱乐性。但此类专栏的写作应该是一个有机的整体,而非像猛击兔子后颈那样的一时机巧。詹姆斯·米切纳的书历来抵挡得住书评者对其作品说任何坏话;这些作品严肃认真,无懈可击。然而,约翰·伦纳德在评论《盟约》一书时,却拐弯抹角地用暗喻伏击了米切纳一把:

有关詹姆斯·米切纳必须要说的是,他使你疲惫不堪。他使你不知不觉地进入默许的状态。一页又一页平淡无奇的散文,就像被击败的大军,穿过你的视线。这是一次从平凡到虔诚的南非大迁移。两耳之间的大脑,可看作是在姆济利卡齐破坏之后,或者是布尔战争期间英国“焦土”政策之后的南非大草原。鸟儿不再歌唱,羚羊死于干渴。

然而米切纳先生就像鞋子一样忠实于主人。在《盟约》中,就像在《夏威夷》、《一百周年》和《切萨皮克》中一样,他的眼光源远流长。他始于15000年以前,止于1979年末。无论我们想还是不想,他都打算让我们了解南非。他经常一本正经、不偏不倚地表现荷兰人的观点,而就像这些荷兰人一样,他自己也是顽固不化;他忍受自己的逆境,驱赶事实这头牛,直到它倒下。

300多页之后,读者一声叹息,服了。当然,假如我们打算花一周读一本书,这本书应该是普鲁斯特或者陀思妥耶夫斯基写的,而不是由米切纳先生从文档卡片中编订出来的。但是已经没有回头路了。这与其说是在读虚构作品,还不如说是在遭罪,我们尽由这位骑在我们肩头上的学究牵着走。也许这么学习对我们会有好处。

我们的确学到东西了,米切纳不骗人。他的个人盟约不是与上帝所结,而是与百科全书所结。假如15000年前非洲灌木丛中的桑人用了毒箭,他就会描述那些毒箭,并且指出毒药的来源。

一篇好的评论该如何开始呢?你必须马上让读者面向将要进入的特殊世界。即使这些人学识广博,也需要你告诉或提醒他们某些事实。你不能把他们扔下水里然后指望他们游泳自如;他们需要热身,水需要热起来。

对于文学评论来讲尤为如此。前面过去的太多,无论他们决定顺流而下还是逆流而上,所有作家都是文学长流中的一部分。本世纪没有哪位诗人的创新性和影响力超过T.S.艾略特,然而令人吃惊的是,1988年他的100岁生日悄然过去,却没有引起多少公众关注。辛西娅·奥齐克在《纽约客》上的一篇评论文章的开头注意到了这件事。她指出,这位诗人对她那代人具有“巨大而先知性的存在”意义,而今天的大学生对其却几乎毫不知晓:“(对于我们,)在那个文学类似于永恒的时代,T.S.艾略特……似乎就是纯粹的顶点,一位巨人,绝不亚于一颗永恒的天体,像太阳和月亮一样固定在苍穹上。”

奥齐克是多么灵巧地把水热了起来,引导我们回到她大学时代的文学景象,我们因此理解她对于自己要述说的故事近于被人遗忘所表现出来的惊诧。

通向艾略特诗歌的门径开启起来并非易事。其诗行和主题理解起来也不容易。但年轻人还是扑入这些门径,受其陌生的魔法诱惑,被其带给人快感的厌倦之绸所束缚。“四月是最残忍的月份”——带着阴森森的韵律,艾略特的嗓音从学生的留声机里传出来——“滋生/丁香花出自死寂的土地,混合/记忆与欲望。”那高贵的英国口音,平淡、精确、稳重、冷漠、令人惊诧的高傲、阴郁的被动,在令人敬畏的英语系和令人崇拜的宿舍与房间萦绕,在那里,墙上钉着毕加索的画像,庞德、艾略特、《尤利西斯》、普鲁斯特,杂乱无章地在痴迷的年轻人胸中彼此推挤。那嗓音就像诗人自己,几乎是天赋神权,无动于衷,就像茫然、机械、痛苦的线轴,缠绕迂回地遍及全国各个大学校园。“安静 安静 安静”,“没有轰鸣,只有呜咽”,“一位老人在干枯的月份”,“我要穿着卷起裤脚的裤子”——这些就是四五十年代文学迷们吟诵的诗句,他们在自己最早的诗行里虔诚地模仿艾略特的腔调,模仿其节制、凝重、神秘的特征,其扩散的疏远以及纹丝不动、支离破碎的绝望。

这一段的精彩之处在于,记忆中的细节、学者的考究魔法般地将艾略特本人召回,重新让其巨大的实体存在遍布全美各大校园。作为读者,我们被重新带回这位大祭司的巅峰时刻,向展现在前方的完美坡道进发。许多学者不喜欢奥齐克的文章,他们认为奥齐克对这位诗人声誉的兴衰夸大其词。但对我来讲,那正是其文章的有效性所在。一点儿争议性都激发不了的文学批评几乎不值得一写,很少有大型体育运动能像一场学术争吵那样使人快乐。

当今,评论在报刊写作中有许多近亲:报纸或杂志专栏、随笔、社论、书评文章。评论者在其中从一本书或一种文化现象推演到某种更大的观点。(戈尔·维达尔将一种高度的放肆与幽默感带到这种形式之中。)许多在好的评论中行之有效的原则同样适用于专栏写作。比如政治专栏作家必须热爱政治,并热爱其与古代千丝万缕的联系。

但是所有这些形式的共通点是,它们都包括作者的观点。即使是用“我们”这一人称的社论也显然是由“我”撰写的。你作为作者的关键是要坚定地表达自己的观点。不要由于最后一刻的含糊其辞和回避矛盾而消减其力量。日报中最乏味的句子是在社论的最后说出诸如“新政策是否奏效现在下结论还为时过早”或者“这项决定的有效性还有待见证”这样的话。假如为时过早,就不要以此来烦扰读者。至于有待见证,一切都有待见证。要坚定信念,站稳立场。

许多年前,我曾为《纽约先驱论坛报》撰写社论,社论版的编辑是一位来自得克萨斯的脾气暴躁的大块头,名叫L.L.恩格尔金。我尊敬他,因为他不装腔作势,憎恨对一个题目纠缠不休。每天上午我们都碰面,讨论第二天要写什么社论、持什么观点。对此我们经常不太确定,特别是一位算得上是拉美专家的作者。

“怎么看乌拉圭的政变?”编辑问。

“那可能代表经济的进步,”那位作者这样回答,“也可能会使整个政治形势不稳定。我也许可以先指出可能的益处,然后——”

“好了,”这位来自得克萨斯的家伙打断他,“咱们不能撒尿顺着双腿流。”

这是他经常提的要求,是我所收到的最不文雅的忠告,但是在经历长期的评论和专栏写作生涯之后,每当我尽力想就深有感触的事物发表自己的观点时,都会感到这个忠告也许是最好的。