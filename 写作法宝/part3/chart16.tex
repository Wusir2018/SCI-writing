\chapter{体育}
我自儿时起就着迷于体育栏目,早在明白“电子回路”之前,就已经明白“打本垒”了。\footnote{电子回路(electrical circuit)、打本垒(circuit clout),两个词组中都有circuit(回路),作者在此作文字游戏。还有之后列举的体育行话其实每一个几乎都有一个更普通的称谓,作者讽刺一些记者嗜好行话。这些行话与普通称谓的区别属于语内区别,在译语中不一定有此区别。}我知道脚尖触在投球板而面向左的“投手”或“投球手”叫“左投手”或“左投球手”。左投手总是身子“细长”,而左投球手总是身子“厚实”,可除了花生酱之外(以区别于“像奶油一样柔滑”),我从未听过“厚实”用于形容其他东西,我也不知道厚实的人会长什么样。当投手猛一下投出一个“老棒球”时,击球者会竭力调整倾斜度。他一旦成功,就可能击中一个利落的“安全打”到外场,赢得一个“意外本垒”,或者至少“结紧球数”。假如他不成功,他也可能“击触地球”而获“双杀”,气喘吁吁地用鼻音报出进攻方“连续得分”的消息,使自己的队夺旗希望渺茫。

我可以继续下去,发掘每一项体育运动的行话,从其母矿中采出各种在其母语中找不到的词语。我可以写出篮球队员、冰球队员、摔跤队员、划桨手、橄榄球队员各自称谓的行话。我可以热情洋溢地写出有关颇具传统的“猪皮橄榄球”的文章——比任何一位猪农都更激情高涨——描述秋季经典赛季处于兴奋之中疯狂的观众。简言之,我可以写出体育英语来替代好英语,就好像它们是两种不同的语言。但它们不是。就像写有关科技或其他领域的文字一样,什么也替代不了最好的英语。

你也许会问,“左投手”(southpaw)有何错?我们难道不应该对有这么生动的词心存感激吗?用twirlers(投球手)和circuit clouts(打本垒)来替代旧有的pitchers(投球手)和home runs(打本垒),难道不是一种调剂吗?回答是,这些词的币值已经贬得低于所要取代的硬币。这些词源源不断地从每一个记者席上的每一位新闻记者(体育记者)手中写出。

第一位想出“左投手”这个词的人有权利感到高兴。我想他会由于自己所发明的新玩意儿而会心地微笑。但那是多久以前了?由于几十年的重复,“左投球手”给英语所添加的色彩早已褪去,同时褪色的还有构成每日体育报道结构的其他习语。这些词语给人一种倦怠感。我们读这些文章是为了弄清谁赢了,而这些词语并没有给我们提供享受阅读的乐趣。

最好的体育记者知道这一点。他们避免那些疲惫不堪的同义词,而是在句子的其他部分争取新意。你可以查瑞德·史密斯的栏目,而发现不了“击球员触地获双杀”这样的词语;史密斯不会害怕用“让击球员击中双杀”这样的词语。但是你将发现他所用的成千上万的与众不同的词语,都是好英语,它们选择精准,恰到好处,其他体育记者却写不出来。这些词语使我们愉悦,因为作者注重的是新闻体中新鲜的意象,而其竞争对手却满足于一脉相承的旧东西。这就是瑞德·史密斯为何在半个世纪的写作生涯之后,仍然是该领域的国王的原因,也是其竞争对手为何(就连他们自己也这么说)早已提前退出比赛。

我仍记得瑞德·史密斯栏目中的许多词语,其幽默和独创性令人惊喜。史密是一位虔诚的垂钓者,看他装鱼饵于钩上,钓上一条滑溜溜的鱼,而一位体育主管就像鱼一样大喘着气,那可真是一大乐趣。“在多数专业体育运动中,几乎一切都已掉出了这一独断行当的底线而难以控制,”他这样写道,指出球队老板的贪婪大有超过体育监管者的监管勇气的趋势。“第一位也是最强硬的 (棒球)巨头是芒廷·兰迪斯。他于1920年掌权,管理手段一直强硬,直至1944年去世。但假如说棒球始于霸王小凯撒,那么它止于无准备之王艾特尔雷德。”史密斯是我们保持宽广视野的每日守护者,他教我们诚实做人。而这主要是因为他写了一手好英语。其风格不但优雅,而且坚强有力,足以传达坚定的信念。

阻碍多数体育记者写出好英语的是一种误解,那就是对于写好英语,他们根本连试都不应该试。他们在充斥着陈词滥调的环境中长大,便认为那些词语就是自己行业中所必需的工具。他们还害怕重复最易唤起读者视觉印象的词语,如“击球员”、“跑垒者”、“高尔夫球员”、“拳击手”。如果可能,他们总是要找一个对应的同义词。而且一般来讲,假如尽力找,都能够找到。下面来自一份院报的节选就是典型的例子:

鲍勃·霍恩斯比以6比4和6比2击败了达特默思的杰里·史密瑟斯,带领网球队员攻克超强的对手取胜,拓展了其团队的疆域。这位瘦高的小将充分利用强势发球机会,打翻那位绿军头领。这位土生土长的孟菲斯小子以最佳状态赢得了前四场比赛,在这四场比赛中两次回发那位印第安小子的球得分。这位埃克塞特毕业生随后受挫,那位汉诺威支柱重振旗鼓,捞得三场胜利。但这位得分手决不会就此罢休,那位扬基佬想以4比4平第一局的图谋宣告失败,他在第六局末平分点时,遭遇半场斜线拦击,结果被赶超。这位红发小伙儿实在是太顽强了,而且……

鲍勃·霍恩斯比究竟如何?还有杰里·史密瑟斯又怎么样?仅在这一段之中,霍恩斯比就变成瘦高小将、土生土长的孟菲斯小子、埃克塞特毕业生、得分手、红发小伙儿,而史密瑟斯成了绿军头领、印第安小子、汉诺威支柱、扬基佬。读者在这些纷繁的名字掩盖之下实在是分不清楚谁是谁,或许他们也不屑区分。他们只想弄清赛事情况。所以千万不要怕重复球员的名字,要保持细节的简明。没有必要只是为了避免累赘,而将“一局”或“一个回合”的名称转换成其他同义词。这样治疗的效果反倒比病患本身糟糕。

另一个癖好是数字。每一个体育迷在生活中都满脑子数据,其中的数字交叉纵横张口即来。许多棒球迷在学校连简单的算术都不及格,在球场却成了速算天才。当然啰,有一些数据会比另一些更重要。假如投手赢得了第20场球,高尔夫手打了一个61杆,赛跑运动员一英里跑到了3分48秒,请提出来。但不要太过了:

11月1日,亚拉巴马州,奥伯恩(合众国际社)——帕特·沙利文,奥伯恩的二年级四分卫球员今日获得两次达阵,传球两次,以38比12击败佛罗里达队,这是本赛季排名第九的佛罗里达鳄鱼队第一次受挫。

佛罗里达队的约翰·里夫斯打破两项东南联会的纪录,平了一项纪录。这位来自佛罗里达坦帕市的高个子二年级学生赢得369码过人,将自己六场比赛总成绩推进到2115分。这打破了东南联会赛季1966年由冠军海斯曼10场比赛赢得2012分所创的纪录。里夫斯尝试了66次传球,这是东南联会的纪录,平了今秋由密西西比队阿尔奇·曼宁所创的33次成功传球纪录。

幸运的是,对奥伯恩来说,里夫斯传球中九次被拦截——打破1951年由佐治亚队齐克·布拉特科夫斯基抗击佐治亚理工学院8次传球失利的纪录。

里夫斯的表现使他离东南联会赛季由佐治亚的弗兰克·辛克维奇在1942年11场比赛中所创造的2187次攻击总纪录只差几码。而他两次达阵传球抗击奥伯恩,比肯塔基的贝比·帕里利1950年所创的23分东南联会纪录只差一个达阵传球……

以上是一篇六段报道文章中的前五段,醒目地登在我所在地纽约的一家报纸上。纽约离奥伯恩远着呢。在这篇报道中有一种对数字的兴奋度不断增长——一位数字狂敲打着数字。可是谁能读这样的文章?谁又愿意读这样的文章呢?只有齐克·布拉特科夫斯基\footnote{齐克·布拉特科夫斯基(Zeke Bratkowski, 1931— ):美国20世纪五六十年代联盟四分卫美式橄榄球队队员,当过多支球队的教练和进攻协调员。}吧,但就连他最终也会逃之夭夭。

体育是当今向非虚构作者敞开的最富饶的领域之一。许多以“严肃”书籍著称的作家都曾以体育竞技观察者的身份写过一些这方面最扎实的书籍。约翰·麦克非的《比赛的水准》、乔治·普林顿的《纸狮子》、乔治·F·威尔的《工作中的男人》,这些有关网球、职业美式橄榄球和棒球的书籍带我们深入到球员的生活之中。这位奇特的家伙、这位获胜的运动员是谁?有什么内在动力驱使他不断前行?棒球文学的经典之一是《波士顿球迷告别基德》。约翰·厄普代克记载了泰德·威廉姆斯1960年9月28日的最后一场比赛。当时42岁的“基德”在沼泽路球场赧后一次为棒球现身,他将球击过了围墙。但在撰写此文之前,厄普代克已经从《这位激动易变的球员》中提取了精华:

……我觉得,在所有团体项目中,棒球比赛那幽雅的间歇性运动、巨大静谧的场地、身着白色球衣蓄势待发的球员,以及冷静的数学运算,似乎最适合独立之人,也最受这样的人青睐。它从本质上是一种孤独的游戏。我们这代人所见的棒球手在自己体内所聚集的辛酸没有其他球员可比。他们如此刻苦地完善了自己与生俱来的技能,又如此不断地将能力集中体现在本垒板上,让喜悦之情涌上心头。

这篇文章的深度在于,这是作者的作品,而不是体育记者的作品。厄普代克知道有关威廉姆斯在本垒板上无可匹敌的能力,这已经没有什么可多说的了:那出名的转身,一双眼睛可以看见以90英里时速飞驰而至的棒球的轨迹。但此人的神秘性仍未解开,甚至在他运动生涯的最后一天都是如此。也就是在这里,厄普代克引导了我们的注意力,暗示棒球适合这样一位独处的明星,因为它是一种孤独的游戏。棒球孤独?我们这项伟大的美国宗族仪式?思考一下吧,厄普代克如是说。


厄普代克心中的某个东西触及了威廉姆斯心中的某个东西:两位孤独的匠人在众目睽睽之下竭尽全力。要找出这种人性的结合点。记住,运动员是那些在赛季成为我们生活一部分的男男女女,他们演绎了我们的梦想,或是满足了我们的其他需求,我们需要让这种结合得到尊敬。收敛炒作,给我们以可信的英雄。

在罗伯特·克里默的优秀传记《贝比》中,甚至贝比·鲁斯这么一个奇才也沿着净化的坡路被请下奥林匹斯圣山,而转变为实实在在的人,尽管他的胃口大得像他的腰围一样。同样的特点也体现在后来克里默所写的《施滕格尔》一书中。在此之前,读者都心甘情愿地接受过去对凯西·施滕格尔的典型描述,即一位开始上年纪的丑角,吐词不清,却赢得十次锦标赛。克里默写的施滕格尔要有趣得多:他是一位复杂的人,并非人们的笑料,他的故事就是棒球本身的故事,其历史可以一直回溯到19世纪的美国乡村。

诚实的描述只是诸多新现实中的一种,而这一领域曾是神话世界。现在体育是反映社会变迁的一个重要前沿,而且国内一些最恼人的问题——滥用毒品及类固醇、群体暴力、妇女权利、少数民族管理、电视播放合同等——都会在体育场中、看台上以及更衣室里产生作用。如果你要写美国,这儿就是你可以支起帐篷大干一场的地方。仔细审视一下各类学校的运动员由于受经济诱惑而引发的事件,那就不只是单纯的体育事件了,而是涉及我们如何教育下一代的价值观以及要优先考虑什么的问题。美式橄榄球王与棒球王在经济上稳坐宝座。有多少教练的薪酬超过大学校长、中学校长以及教师的收人?

钱是美国体育中一个赫然逼近的怪物,其阴影随处可见。有关可憎的巨额薪酬的消息充斥着体育栏目,而现在这一栏目所登载的金融消息几乎同金融栏目一样多。球员赢得一场高尔夫或网球锦标赛将获得的钱数,赫然登载在报道的头版头条中,而且是在其得分之前。赚大钱也带来了巨大的情感困扰。今日多数的体育报道与体育无关。首先我们被告知是谁的感情受到了伤害,因为运动员被球迷喝倒彩,球迷认为身价1200万美元的球员击球率就应该高于225,并且要追赶着接到朝他这个方向打过来的腾空球。网球比赛这桶金利润巨大,球员绷得就像他们的高科技球拍一样紧,这些百万富翁随时都会向裁判和司线员抱怨和咒骂。在美式橄榄球和篮球中,球员的报酬是天价,脾气也有过之而无不及。

现代运动员的自我也随之在现代体育记者的笔端被抹掉。我感到震惊的是,如今有许多体育记者以为他们就是赛事的中心,认为他们自己的想法要比他们所报道的比赛更有意思。我怀念从前的日子,那时候的记者会很谦逊且直截了当地说谁赢了,而今天,谁赢的信息则姗姗来迟。有一半的体育记者自以为是莫泊桑,是在头版头条精心制造悬念的大师。其余的以为自己是弗洛伊德,私下里能洞察运动员的心理需求与受伤的情感。另一些人还会在一旁实施外科整形和关节内窥镜手术,评估由核磁共振影像扫描才能确诊的投手的肌腱套是否撕裂,他们在这方面比队里的医生反应还快。“他的状况每天都在变,”他们这样下结论。谁的状况不是每天都在变?

未来的莫泊桑们专门写比赛之前发生的事件,他们守候在运动队俱乐部会所收集这些事件,寻找“趣闻”。任何新闻不管多琐碎乏味,只要能够构筑进巴洛克大厦,刊登为报刊的头版头条,就都成了珍宝。下面的例子是我编的,但每个体育迷都能认出其体裁:

两周前,亚历克斯·罗德里格斯的祖母做了一个梦。她告诉亚历克斯,她梦见亚历克斯和他在扬基队的部分队员去一家中餐馆吃晚饭。吃甜点的时候,罗德里格斯叫服务生给他拿一份签饼\footnote{fortune cookie,又称“福饼”,国外中餐馆的特产,是一种很脆的饼干,里面藏有一张小纸条,印有中英文对照的签语。}。“有时候这些东西真能一语中的,”他的祖母回忆亚历克斯对德里克·杰特这样说。打开签饼里的纸签,他看见有这样的字句:“你会很快做出强有力的举动来挫败对手。”

也许亚历克斯·罗德里格斯昨晚在扬基体育场踏上本垒板,面对红袜队王牌柯特·西林时;正想着祖母的梦呢。他在2004年是3比27对西林,陷入他在此赛季最长的下滑困境。不用谁来告诉他,球迷们正盯着他看,他已经听见球迷在喝倒彩。挫败敌人的最佳时机。第八场的后半局,两个人都在垒上,红袜队3比1领先。时间在流逝。

罗德里格斯击球数已快满,这时他从西林那里得到一个齐腰高的滑行曲线球,在关键时刻击了出去。球升上去形成高高的弧线,你看着罗德里格斯就知道,他以为球会飞向棒球场左侧的席位。一阵强风席卷体育场,但签饼中所说的“强有力的举动”不容置疑。马里亚诺·里韦拉在第八场的前半局阻挡住红袜队之时,记分板上显示扬基4,波士顿3。奶奶,谢谢!

未来的弗洛伊德们也不甘示弱,他们大吹大擂,无法平静。“在安德烈·阿加西斯昨天进人球场对阵小他20岁的敌手之前,就该有人告知他必死无疑,”这些研究人性动机的专家们这样写道,一面用诸如“可以预测,毫无希望”之词,而这本不该是记者用语,他们这么做是为了显示自己面对失利的运动员时的优越性。“昨晚纽约大都会棒球队打外场,铁了心要另找一个滑稽可笑的方式输球,”在最近一次低潮赛季,我所在的当地报纸报道该球队的记者用如此惯用的腔调不停向我说,极尽讽刺之能事,而非以事实说话。如果你要写体育,记住,你所写的男男女女是在做极难的事情,他们都有职业自豪感。你所做的工作也有自己的信誉度。其中之一就是你并非事件的主人公。


瑞德·史密斯对自以为是的体育报道毫无耐心。他说,记住棒球是小孩玩的比赛,始终有助于我们理解此项比赛。这也同样适用于美式橄榄球、篮球、曲棍球以及网球等大多数其他比赛。从前的小男孩、小女孩玩过这些球类,他们长大后成为体育版的读者,而在心中觉得自己仍年轻,仍在棒球场、网球场、篮球场上,仍在比赛。当他们翻开报纸,想要知道的是球员们打得如何,比赛的结局如何。请告诉我们这些。

体育记者的一个新角色是让大家知道真正从事体育运动是什么样的感觉:当一名马拉松运动员或是足球守门员,滑雪运动员、高尔夫选手或是体操运动员。现在时机已经成熟,大众对人体潜能所能达到的高度兴趣之大前所未有。美国人掀起了健身热潮,他们在健身器材上健身,精确地调整体重,测量肺活量以及心脏压力。对于非虚构作家来讲,这些周末勇士提供了一批崭新的读者:这些体育迷也是业余运动员,他们很想钻进运动员脑子里,一窥他们处于最佳状态时的情形。


高速度是许多体育项目中的刺激核心,它是普通人只能尽力想象出来的典型感受。作为开到每小时65英里就发抖的车主,我从未了解过驾驶跑车是什么感觉。我需要作家莱斯利·黑兹尔顿把我系在一级方程式赛车的车座上。“每当我开快车,”她写道,“就有一种意识,我在逾越自然规律,车子的运行速度超过了自然所赋予身体的运行速度。”黑兹尔顿说,这种意识真正始于驾驶员亲历重力之时,那是一种外力,它“以特别的压力作用于你,就好像你的身体先被推动,而内脏随后跟上”:

赛车手所要应对的重力巨大,要承受三四倍于普通重力的压力。一级方程式赛车从静止到每小时100英里只需三秒钟。而在那第一秒中,赛车手的头部被猛烈后推,脸部撑开,让他显出一种可怕的笑容。

在另一秒之内,他已经两次换挡,每一次的加速都将他挤进车座上。三秒钟之后,赛车从每小时100英里加速到200英里,赛车手的周边视野完全模糊起来。他只能看清正前方。800马力的引擎以130分贝吼叫,每一个汽缸都要完成四次循环燃烧,每分钟10000次,这意味着他所感觉到的振动就是这个速率。

赛车手脖子和肩膀上的肌肉承受着巨大的拉力,随着重力将头从一边甩向另一边,他竭力保持眼睛平视。强大的加速力将血液堆积在腿部,传输到心脏的血液减少,这意味着心脏供血减少,压迫脉搏增快。一级方程式赛车手的脉搏经常高达180,甚至200,在几乎整整两个小时的赛程中,赛车手的脉搏有85\%的时间保持在这一最快频率上。

随着肌肉需要更多供血,呼吸开始加快,速度则完全让人窒息,整个身体进入一种长达两个小时的紧急状态。正常心跳一次的时间内,赛车就要跑完一个美式橄榄球场那么长的距离。赛车手嘴里发干,眼睛膨胀,大脑以惊人的速度处理信息,因为速度越高,反应时间就越短。反应不仅要快,而且要异常准确,无论身体压力多大都必须如此。几分之几秒也许只是时间上小小的片段,但关系到一场赛车的输赢,或者是否遭遇一场车祸。

简言之,一级方程式赛车手处于巨大的身体压力之下,他必须保持几乎超常的警觉。显然,其肾上腺素在涌动……但除了要有顶级运动员的健康身体状况之外,他还要有棋手般的头脑,综合遥测数据、计算超车点、实施赛车战略。所有这些说明了为何速度对我们大多数人来讲都是十分危险的,我们根本既不能在身体上也不能在精神上应对这一切。



从心理上讲,赛车中所发生的一切还要更加复杂。肌肉、大脑中的化学物质、物理定律、振动、比赛条件,所有一切结合在一起,才能从身体里产生一种高度的兴奋和紧张,使赛车手保持绝对的头脑清醒、神情警觉,以及亢奋。

虽然黑兹尔顿一直用的是“他”,但其文章中的人称代词应该是“她”。在当今体育运动及体育新闻遭受侵蚀的形势之下,“她”代表了一种巨大的进步。优秀女运动员出现在从前经常是男性所垄断的地盘上,女记者享有进人男性更衣室的平等权利,她们还拥有新闻记者的其他常规权利。请思索一下这方面无论是在行动上还是在观念上的诸多进步。下面这篇由女记者贾尼斯·卡普兰所写的文章就体现了这些特点:

要了解女性在体育方面取得的成绩有多好,得先了解仅在10年以前女性在这方面有多差。70年代初,人们所争论的不是女性在体育方面能做多少,而是普通女性从根本上是否适合当运动员。

例如,据说参加马拉松对儿童、老人和妇女有害。难度巨大的波士顿马拉松直到1972年才正式向女性开放。那一年,尼娜·库西克同性别歧视还有比赛中途发作的腹泻抗争,最终成为女子组的第一位冠军。知道这件事的人都涌动着自豪感,同时夹杂着一点儿难堪。自豪是因为库西克的胜利证明女性也能跑26英里,难堪是因为她跑的3小时零10分比男子最好成绩慢了50分钟还多。50分钟啊!这在赛跑中简直就是永恒了。显而易见的解释是,女性之前几乎没有跑过马拉松,她们缺乏训练和经验。这的确是显而易见的解释,但谁又相信呢?

再返回到今年。女子马拉松将首次成为奥林匹克比赛项目。最具竞争力的选手之一很可能是琼·贝努瓦,她保持的女子马拉松纪录是2小时22分。自从第一位女性在波士顿参加赛跑后的十几年里,女子的最好成绩已经提高了约50分钟。这是另一个永恒。

与此同时,男子马拉松成绩只提高了几分钟,于是这项戏剧性的进步开始回答训练与荷尔蒙之间的关系问题:女性比男性更慢、更弱,是因为生理结构差异呢,还是因为文化偏见和我们并没有机会去证明我们都能做什么?……男女之间的鸿沟能否完全弥合,好像并非重点。重要的是,女性在做她们从未梦想自己能做的事:认真对待自己和自己的身体。

在这场改变意识的革命中,一个关键事件是70年代中期比尔斯·琼·金和博比·里格斯的网球比赛 。“这场比赛被宣布为性别大战,”卡普兰在另一篇文章中回顾道,“而且后来证明的确如此。”

也许从来没有一场赛事比这场更不关注于体育问题本身,而是关注于社会问题。这场比赛的大问题是妇女:我们归属于何处,我们能做什么。不要提最高法院的决定与《平等权利修正案》的裁决;我们所期盼的是两位运动员以至关重要的方式来了结妇女平等权利问题。在体育比赛中,一切都写得清清楚楚、确确实实,只有一个赢者和一个输者,一切都无可争辩。

对许多妇女来说,在比尔斯·琼的胜利中有一种个人夺冠的感觉。这一胜利释放了全国妇女身上的某种能量。年轻女性在大学体育活动中要求自已的权利,而且得到了更重要的地位。在许多专业体育赛事中,奖励女性的资金大增。小女孩开始参加少年棒球联合会,加入男孩的球队。这些都证明男女生理上的差异并非从前人们想象得那么大。

美国体育总是与社会历史交织在一起,最优秀的作家是那些将其结合起来的人们。“将篮球赛变为避税的表演活动并不是我的主意,”比尔·布拉德利在《奔波的生活》中这样写道,那是一部记述他跟随纽约尼克斯队征战的赛季的编年史。前参议员布拉德利的书是现代体育写作的范例,因为该书反思了一些正在改变美国体育的破坏力量——球队老板的贪婪、对体育明星的膜拜、不能接受失败,等等:

在范离开后,我意识到无论球队老板多么善良、友好而且发自内心地感兴趣,最终大多数球员对他们来说也只是会贬值的财产而已。

大多数运动员的自我认定来自于外在的资源而非内在的动力。专业运动员在保持身体技能的同时,还是名人——他们被溺爱、原谅、赞美、崇拜。只是到了他们事业的末期,这些明星们才意识到,他们对自己的认同感还不够。

获胜的球队就像征服大军,一路上占有一切,似乎在说只有获胜才是唯一重要的。但是胜利也有很狭隘的含义,它可能成为破坏力。失败的滋味也有其自身的丰富体验。

布拉德利的书也是一本优秀的旅行日记,它捕捉了专业运动员游牧般生活中的疲倦与孤独——无数的夜间航班和巴士旅行,沮丧的日子,以及在汽车旅馆和航站里无尽的等待:“飞机场成为我们的通勤站,我们在那里目睹了许许多多戏剧性的个人经历,我们自己变得冷漠、生硬。对外人,我们过着浪漫的生活;而对自己,我们每一天都得不断地奋力去应付生活中的复杂问题。”

这些就是你在写体育的过程中所要寻找的价值:人与地点,时间与转变。这里有一个意味深长的人员清单,每一种体育都需配备这样的人。清单来自G.F.T.赖亚尔的讣告。他以笔名迈纳为《纽约客》报道了半个多世纪的纯种马比赛,直到几个月前在92岁高龄去世。讣告说赖亚尔“后来认识了与赛马相关的形形色色的人——马主人、饲养员、管事、裁判、计时员、同注分彩管理员、私人侦探、训练师、厨师、马夫、赛马优胜预测人、遛马人、发令员、乐师、职业赛马骑师,以及他们的代理人、下注者、挥金如土的赌徒和阔绰的赌客。”

要夺取田径场、赛马训练场、体育场以及冰球馆。仔细观察,深度采访,倾听老手,反思变化。要好好写。