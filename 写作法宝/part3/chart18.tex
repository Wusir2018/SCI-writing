\chapter{幽默}
幽默是非虚构作者的秘密武器。其之所以秘密,是因为很少有作家意识到幽默常常是他们最好的工具,而且有时是阐述重要观点的唯一工具。

如果你惊讶地认为这是一个悖论,也并不奇怪。幽默作家们通过自己的生活阅历意识到,许多读者根本不知道他们到底是做什么的。我记得有一位记者打电话问我,为何会想到为《生活》杂志写一篇戏讽。最后他说,“我该称你为幽默作家么?或者说你也写过什么严肃作品吗?”

回答是,假如你一心想写幽默,那么几乎你所写的一切都是严肃的。很少美国人明白这一点。我们对国内的幽默作家不屑一顾,认为他们是在玩弄把戏,因为他们从来都安不下心来做“真真切切”的工作。普利策奖都颁发给诸如欧内斯特·海明威和威廉·福克纳那样的作家,他们严肃认真(上帝知道),因而被认定为文学家。这些奖项很少颁给诸如乔治·埃德、H.L.门肯、林·拉德纳、S.J.佩雷尔曼、阿特·布赫瓦尔德、朱尔斯·费弗、伍迪·艾伦以及加里森·凯勒那样的人。他们似乎都是无所事事之徒。

他们当然并非无所事事。他们在目的方面同海明威或福克纳一样严肃认真,都是民族的财富,能让这个国家看清自己。幽默对他们是紧迫的工作,是试图以特殊的方式说重要的事情,是常规作者以常规方式说不出的——或者假如他们说得出,那也太常规了,没人会读它。

一幅强有力的报刊漫画抵得上一百篇一本正经的社论。一张加里·特鲁多的《杜恩斯比利》连环漫画抵得上一千个词的说教。一部《第22条军规》或者《斯特兰奇洛夫博士》比所有试图演示战争“实况”的书籍和电影都更强有力。战争的心态明天就可能毁灭我们所有人,这两部喜剧的创作对于任何想警告我们防止这种心态的人来说,至今仍是参照标准。约瑟夫·海勒和斯坦利·库布里克强化了有关战争的真谛,恰如其分地捕捉了战争的疯狂,也让我们认识到战争的疯狂。这个玩笑已不再是玩笑。

这种以强化的方式揭露某种战争狂的举动,其强化的程度达到了使其本身看上去就是一种疯狂 ,但这正是严肃的幽默作家竭力想要达到的。下面的一个例子说明幽默作家是如何执行自己的神秘工作的。

60年代的某一天,我忽然发现美国有一半的女人无论大小都戴起卷发夹来。这简直就是一种奇特的新枯萎病。更令人不解的是,我不知道这些女人何时会将卷发夹取下。根本就没有迹象表明她们会取下这些发夹——他们戴着发夹去超市、去教堂、去约会。是什么美妙的东西使她们一直保持这样美妙的发式呢?

我费了一年的时间想办法要写一写这个现象。我可以说“这太不成体统了”或者“这些女人没有自尊吗?”,但那就成了说教了,而说教是幽默的死敌。作者必须找到某些喜剧性的方法——讽刺、戏讽、反讽、挖苦、打油诗——他可以用这些来掩饰其真实目的。但他常常找不到合适的方法,因此目的也就达不到。

幸运的是,我的不眠之夜最终得到回报。我在报摊上浏览,看见四本杂志并排放着:《发式》、《名人发式》、《美发》以及《高发髻》。我将四本全买下来——报摊主大为吃惊——我发现这些都是专门开发美发文章的杂志:脖子以上的人生,但不包括大脑。杂志上有卷发筒详细位置的图解,姑娘们可以将自己的卷发筒问题邮寄给专栏的编辑寻求指导。这正是我需要的。我杜撰了一份杂志叫《卷发》,写了戏讽读者来信与回复的系列,登载在 《生活》杂志上。信是这样开头的:

亲爱的《卷发》:

我15岁,在同伴儿中算是漂亮的。我戴浅粉红色卷发筒,超大号。我同一个男孩一直要好有2年半了,他从未见过我不戴卷发筒。有一天晚上我摘下了卷发筒,结果我们大吵了一通。“你的脑袋看起来很小啊,”他告诉我。他叫我侏儒,而且还说我骗了他。我怎么才能叫他回心转意呢?

伤心人儿

纽约,斯皮昂克

亲爱的伤心人儿:

你做出这么蠢的事儿,只能怪你自己。最新《卷发》调查表明,94\%的美国女孩现在每天有21.6小时头发上戴卷发筒,每年359天。你想与众不同,却因此失去了男友。听我们的劝告,戴上超超大号卷发筒(这个型号也有你最喜欢的浅粉红色),你的脑袋会比以前更大,你会加倍可爱。千万不要再摘掉了。

亲爱的《卷发》:

我的男友喜欢用手指捋我的头发。问题是他的手指老是夹在我的卷发筒上。有一天晚上发生了一件极为难堪的事。我们在看电影,不知怎的我男友的两个手指被夹住了(正好是中间那个卷发筒上夹着的卷发夹),说什么也拔不下来。我感到在众目睽睽之下,他的手还夹在我的头发里,我们就这样离开了影院,而且在回家的公共汽车上,好几个人都以“异样的眼神”看我们。幸运的是,我在家联系上了发型师,他马上带着工具过来,给可怜的杰瑞松了绑。杰瑞气极了,说他再也不跟我约会了,除非我戴的卷发筒不再有这种夹人的怪癖。我觉得他并不讲理,可是他“说话算话”。你能帮帮我吗?

急疯了的人儿

野牛城

亲爱的急疯了的野牛:

我们很抱歉地通知你,目前还没有研发出不会偶尔夹住弄乱别人头发的男孩手指的卷发筒。不过卷发筒行业正在尽力解决此问题,因为这种抱怨时有发生。与此同时,干吗不叫杰瑞戴上连指手套?这样你也高兴,他也安全。

还有很多这样的信,也许我甚至对伯德·约翰逊女士的“美容”活动都做出小小的贡献。但要点是:一旦你读了这篇文章,就不会再以同样的方式看待卷发筒这件事了。你受了幽默的震动,开始以崭新的眼光看待自己日常环境中以前熟视无睹的怪诞之事。在此事例中,这一问题的对象并不重要,卷发筒并不会毁坏我们的社会,但是这种写作方法的确适用于重要的问题对象,或是几乎任何对象,只要你能找到正确的喜剧性框架来描述它。

在1968—1972的五年间,我在老牌的《生活》杂志中,运用幽默手法涉及了几个似乎不相干的题目,比如军事力量以及核试验的滥用。其中一个专栏是围绕在巴黎针对越南战争的和谈过程中,有关桌子形状问题的琐碎争吵。八周的谈判毫无结果,形势变得简直无法容忍,只有通过嘲讽来解决。于是我就作出种种尝试来描述这场淡判,拿我的饭桌作比喻,试着同我自己的饭桌达成和解。其中的办法包括每晚改变桌子的形状,或者降低不同人椅子的高度,以便给他们低一点儿的“身份”,或者旋转他们的椅子,这样我们其他人就不必“认出”他们了。这都是在巴黎实际发生的事情。

使这些文章产生效果的要素是每一篇都紧跟特定的嘲讽写作形式。幽默看起来似乎是一种大肆夸张的行为,但无论是在文体上还是心态上,假如我们不将这些卷发筒信件确定为一种特别的报刊写作形式,这些文章就不会成功。掌控能力对于幽默的效果至关重要。不要用诸如“混饭吃的公职人员\footnote{Throttlebottom,这是看起来很怪的英语单词。}”那样滑稽的名字。不要一而再再而三地开同样的玩笑,读者更欣赏一个玩笑只说一次。相信读者的鉴赏力,他们知道你在做什么,不要操心其他的事。


我为《生活》写的专栏使读者大笑。但是,这些文章有严肃的目的,他们在提醒大家:“有一些疯狂的事情在这里发生,其中一些侵蚀着我们的生活质量,另一些对人生构成威胁,而人们却认为一切正常。”现今,荒诞无稽之事第二天就会变得习以为常。幽默作家在竭力告诉大家荒诞无稽之事就是荒诞无稽。

我记得60年代晚期学潮期间比尔·莫尔丁的一幅漫画,当时步兵和坦克应召在北卡罗来纳的一所学院维持秩序,伯克利的本科生也被直升机喷洒的泪毒气所驱赶。那幅漫画上,一位母亲向征兵局请求道:“他只是一个孩子——请叫他离开校园。”这是莫尔丁将这场特别的疯狂行为具体化的方式,而且瞄得很准——事实上正中靶心,因为就在漫画出现不久,四名学生就在肯特州立大学遭到枪杀。

靶子每周都会变化,但是社会永远不缺乏幽默作家与之战斗的新的疯狂和危险。林登·约翰逊在其灾难性的越战任期内,部分是被朱尔斯·费弗和阿特·布赫瓦尔德拉下台的。参议员约瑟夫·麦卡锡与副总统斯皮罗·阿格纽也是部分被沃尔特·凯利通过其《跳跳乐》连环漫画拉下台的。H.L.门肯将一大群著名的伪君子拉下高位,而坦慕尼协会会堂的特威德“老板”也是部分地被托马斯·纳斯特的漫画所推翻的。喜剧演员莫特·萨尔是在艾森豪威尔时代,当美国处于镇静状态不想被打扰之时,唯一保持清醒的人。许多人认为萨尔是一个愤世嫉俗之人,但他认为自己是一个理想主义者。“如果我批评某人,”他说,“那是因为我对这个世界抱有更高的希望,想用好的来替换坏的。我并不是在说‘垮掉的一代’所说的:‘走开,因为我不参与。’我是在说:我在此,我参与。”

“我在此,我参与”:如果你要写严肃的幽默,就将此作为自己的信条吧。幽默作家所参与的潮流比大家所想的更深。他们必须逆潮流而上,说大众与总统也许不愿意听的事。阿特·布赫瓦尔德和加里·特鲁多每周都大胆行事。他们说该说的,一般专栏作家都会这么做。使他们得以解脱的是,政客们不善于幽默,因而也就比一般民众对幽默所讽刺的东西更迷惑不解。

但幽默除了有关时事之外,还有许多其他用途。那些用途并不紧迫,但有助于我们看待心灵、家园、家庭、工作,以及其他那些从早到晚都可能遭遇的挫折,还有那些更长久的问题。我曾采访过《勃郎黛》漫画系列创始人奇克·扬,当时他已经为那份每日漫画系列撰写和绘画了40年之久,总共14500幅漫画。他的作品是所有漫画系列中最受欢迎的,读者遍及世界每一个角落,达到6千万人。我问扬为什么他的漫画这么经久不衰。

“我的漫画经久不衰,因为简单,”他说,“它们基于四项人人都做的事情:吃饭、睡觉、赚钱、养家。”在此四项基础上,其漫画的喜剧性变化之多就像生活一样。其中的人物大梧努力从老板迪瑟斯先生那里赚钱,但永远也超不过勃郎黛努力花钱的速度。“我尽量使大梧处于人们习惯了的世界中,”扬告诉我。“他从不干像打高尔夫球之类的特别的事情,去他家的人都是那些普通人家必须打交道的人。”

我引用扬的四项主题来提醒大家,多数幽默,无论它看上去多么奇妙,都离不开最基本的事实。幽默并非依靠自身微弱的新陈代谢就能单独存活的组织。它赋予已经写得很好的作家一种特别的视角。这些作家并非在写本质上就可笑的人生,而是在写本质上就严肃的人生,但他们的眼光所落之处,却是真诚地希望被命运的逆转所嘲弄——就像斯蒂芬·李科克所说的“那种在我们的抱负和成就之间奇怪的不协调”。E.B.怀特也持同样的观点。“我不喜欢‘幽默作家’这个字眼,”他说。“我觉得这似乎在误导读者。幽默是在某一些但并不在另一些严肃作品中出现的副产品。我受唐·马奎斯的影响比海明威更大,受佩雷尔曼的影响比德莱塞的影响更大。”

因此我向幽默写作者建议几条原则。要掌握写得好、写得“直截了当”的本领,从马克·吐温到拉塞尔·贝克,这些幽默作家首先是超级作家。不要猎奇,不要鄙视似乎太普通的事物,通过发现你所了解的真实事件中的滑稽之处,你能拨动更多的琴弦。最后,不要滥用笑这一工具,使人发笑的幽默建立在惊诧之上,而你能使读者惊诧的次数是有限的。

遗憾的是,对作家来讲,幽默是捉摸不定、充满主观性的。没有两个人会因为同样的事情发笑,被一家杂志作为废物退稿的文章很可能被另一家当做珠玑发表。退稿的原因也同样令人难以捉摸。“它就是不行,”编辑会这么说,而且没有进一步的解释。偶尔这种被退稿的作品还可以用,其中的某些缺陷可以修改,但是夭折率还是很高。“幽默可以像一只青蛙一样被解剖,”E.B.怀特曾写道,“但在这个过程中青蛙就会死亡,除了纯理科人士外,其内脏对其他人的头脑都会是很沮丧的。”我可不喜欢死青蛙,但是我想知道,通过扒拉其内脏,可否至少学到一点东西。于是我在耶鲁大学教书的时候,有一年决定开一门幽默写作课。我警告学生也许幽默是学不来的,结果可能会扼杀自己所喜欢的幽默。幸运的是,幽默不但没有夭折,而且还在一本正经的学期论文这片荒漠之上开花结果。这样我在下一年又开了那门课。现在让我简明地回顾一下我们的幽默学习之旅。

“我希望指出,美国幽默具有光荣的文学历史,”我在为选课的学生准备的备课笔记中这样写道,“我还希望探讨某些幽默的开拓者对其后辈的影响……虽然在幽默中,‘虚构’与‘非虚构’的界线模糊,但我将本课程作为一门非虚构写作课:你们所写的将以外在的事件为基础。我的兴趣不在那种纯粹想入非非与漫无目的的‘创意写作’上。”

我的课堂先是读早期作家的节选,以说明幽默作家可以运用广泛的文学形式,或者发明自己的形式。我们从乔治·埃德的《俚语中的寓言》开始,其中第一篇发表于1897年的《芝加哥纪事》,当时埃德是该报的记者。“他就这样若无其事地坐在一张白纸前,”琼·谢泼德在选集《乔治·埃德的美国》中精彩的前言里这样写道,“这时他突发奇想,要用寓言形式写点儿什么,就用当下的语言及俗语,换句话说,也就是俚语。为了让大家明了自己还不至于只能用俚语来写作,他决定大写所有可能引起猜疑的词语。对于人们可能误以为他不识字,只会用俚语,他怕得要死。”

他本不必担心,因为到了1900年,这些寓言大受欢迎,他每周都能赚到1000美元。下面是《瞧见灵光的下属之寓言》:

从前有一个雇员,他累得像死狗,可得到的还不够糊口。他怒斥工时长,薪水太少,还帮着发起了员工保护协会。他为的是苦力工,反的是吸血鬼。

为了消他的火,老板给了他好处。打那以后,他看了工资条就出汗,于是乎,他似乎的确感到有好多五大三粗、好吃懒做的家伙在车间周围整天闲逛,逃避干活。他学会了每当办公室的勤杂员咯咯笑就打响指。至于想要薪水涨到9美元外加夏天一周假期的忠实的老会计,最多只得到了一顿劝告,让他知足常乐吧。

对这个人来讲,一天中最悲哀的时刻是全体工友晚上6点全部下班的时候。把10个小时的工作时间就当做一整天,这似乎太无耻了。至于争取周六半天放假的倡议,那简直就是拦路抢劫。那些从前和他在同一条船上的人,现在得叫他先生,而他把这些人像囚犯一样都编了号。

有一天,一个底层雇员壮着胆儿提醒这位苛刻的工头,说想当初他也曾是这些工薪阶层的患难兄弟。

“算你说对了,”工头老板如是说。“不过当我献身于低薪兄弟运动之际,还从未进过主任办公室,从未见过那‘美德赢红利’的美妙画面。不知这是否向你解释清楚了,因此我也就只能对你说,站在我们这边的所有人马,一定有能力在这个薪水问题上提高思想觉悟。”

寓意:为了教育目的,每一位雇员都应该被公司买通。\footnote{本篇原文用词多为美国俚语、俏皮话,营造出一种幽默的气氛,译语只能尽力再现某种滑稽讽刺的气氛。}

在这历时百年的宝物中所蕴涵的普适性真谛,至今千真万确,几乎所有寓言都是如此。“作为幽默作家,埃德是第一位影响到我的人”S.J.佩雷尔曼告诉我。“他对历史具有社会责任感。他对世纪之交印第安纳州山地人生活的描绘,比对人们消耗的煤炭要花多少钱的任何研究,都更具有纪实性。他的幽默根植于对人和处所的感知能力。他所具有的活力和辛辣的风趣是早期美国幽默作家所没有的。”

从埃德往下,我要讲的是林·拉德纳。他的经典语句“住嘴,他解释道”,部分说明戏剧性对话是幽默作家可以利用的另一种形式。我很容易着迷于拉德纳的滑稽剧。他写这些剧很可能只是自娱自乐,但他也讽刺挖苦戏剧创作中的金科玉律,包括运用大量斜体字句来说明舞台上所发生的事件。拉德纳的戏剧《室内装潢商》的第一幕包括十行对话,但对话与所列人物并无关系,其中还包括九行不相干的斜体说明,其中结尾说明是:“幕布放下7天,以表明过了一个星期。”在其创作生涯中,拉德纳能将幽默的强大效力运用于许多文学形式中,比如棒球小说《阿尔,你了解我》。他的耳朵洞悉美国人的虔诚与自欺欺人。

下一个我要发掘的是唐·马奎斯的《阿奇与梅希塔布尔》,这位具有影响力的幽默作家也用一种非传统的形式——打油诗——来表达他的要旨。马奎斯这位《纽约太阳报》的专栏作家,恰巧找到了一种新办法,来解决报人赶截稿期并且必须用有条不紊的散文体来呈现事实这一苛刻要求,就像埃德遇上了寓言形式一样。他1916年创造了蟑螂阿奇。阿奇深夜在马奎斯的打字机上砰砰地打出自由诗,这些诗一律没有大写字母,因为他不够有力气,按不动切换键。阿奇的诗描述了他与一只名叫梅希塔布尔的猫的友谊。诗中充满哲理,这从他们那衣衫褴褛的外表上是看不出来的。马奎斯用他的诗彻底削掉了那些抱怨戏剧现状的老演员们的傲气,这一点任何严谨的散文都比不上。马奎斯在长诗《老演员》中,借用阿奇描述了梅希塔布尔见一个叫汤姆的剧院老猫的情景:

我来自那古老的家族

一个剧院猫的家族

我的祖父

与福雷斯特为伍

他说自己是真正的老演员……

马奎斯熟悉这一切,他借用猫来逐渐改变自己对此类乏味之人的不耐烦。这种不耐烦具有普遍性,无论对哪类老者都是如此,就像老者的普遍特征是抱怨自己这一行每况愈下。马奎斯实现了幽默经典功能的一种:将气愤转入一种渠道,在那里我们可以嘲笑弱点,而不是怒斥它。

我的幽默之途的下几位是唐纳德·奥格登·斯图尔特、罗伯特·本奇利、弗兰克·沙利文。他们极大地扩展了“自由联想”幽默的可能性。本奇利增加了温暖与脆弱的纬度,但这并不存在于埃德与马奎斯的作品中,他们一头扎人非人称表现形式,如寓言和打油诗,以便自己藏匿其中。在一头就能扎入主题方面,没人能超过本奇利:

阿西西的圣·弗朗西斯(除非我将他与圣·西米恩·斯泰莱特搞混了,这很容易,因为两个名字都以圣开头)非常喜欢鸟儿,他经常在鸟儿站在自己肩上并在自己的手腕上啄食时与其合影。这没什么,只要圣·弗朗西斯喜欢就好。我们大家都有自己的喜好与憎恶,我自己对狗更情有独钟。

也许这些作家都只是在为S.J.佩雷尔曼铺路。如果是这样,佩雷尔曼也感激地认可了他们的贡献。“你必须从模仿学起,”他说。“在我20年代晚期的作品中,我模仿林·拉德纳的成都都够被抓起来了——并不是在内容上,而是在方式上。这些影响逐渐都会消退。”

但佩雷尔曼自己的影响却没那么容易消除。到1979年去世时,他写了有半个多世纪,用语言穿过了一个个令人屏息的圆圈,而这片林中仍有许多作家与喜剧演员,他们被他文体的分量所吸引,而且从未完全摆脱出来。其实并不需要什么侦探,人们就能看见佩雷尔曼的那只手不仅在伍迪·艾伦身上,而且体现在英国广播公司的《傻瓜秀》与《蒙提·派森》节目之中,在鲍勃与雷的广播滑稽剧中,在格劳乔·马克斯的即兴风趣之中——他的影响在这里更易于追根溯源,因为佩雷尔曼撰写了马克斯兄弟早期的几部电影。

佩雷尔曼的创造使人们意识到,当作者的大脑以自由联想的形式工作时,它可以从正常飞向荒诞,并通过令人意想不到的角度,捣毁以前所有陈腐的想法。在这一让人不断惊奇的过程中,他嫁接了炫目的文字游戏,使之成为他的标志,那是一种丰富、玄妙的词汇,一种基于阅读与旅行所获得的博学。

但是假如他没有目标,即便如此多才多艺也不足以使他维持太久。“一切幽默都必须关乎某些事情——必须具体地接触人生,”他这么说,而且欣赏他的风格的读者虽然可能忽略他的动机,但某些浮夸的形式都会在佩雷尔曼作品之后化为废墟,就像大歌剧在马克斯兄弟的《歌剧院之夜》之后就从未完全恢复元气,或者银行业在W.C.菲尔德的《银行妙探》之后一样。他总有办法对付骗子与无赖,特别是在百老汇、好莱坞、广告业以及商品促销这些领域。

我仍记得自己十几岁时第一次被佩雷尔曼的语句震撼到的情形。他的语句不同于任何我所见过的,使我激动不已:

汽笛鸣响,一会儿我就嘎嚓嘎嚓地出了大中央火车站梦幻般的尖顶。我只嘎嚓出几英尺,忽然发现我走了而火车却没动,于是我只得跑回来,等火车出发……在芝加哥只有两个小时,我就要见不到这座城市了,这一想法反倒使我镇静下来。我注意到迪尔伯恩大街车站又增添了一层新污垢,心中暗喜,当然我并没有自以为是地认为,那同我的这次造访有任何关系。

女人们喜爱这位鲁莽的爱尔兰冒险家,他不是打架滋事就是大吃大喝。有一天晚上,他正在朴次茅斯的一家叫“小酌”的酒馆焦躁不安,忽然听到一位强壮的枪炮军士长醉醺醺地随意闲聊……第二天早上,配有36门炮的“赫尔圣女”号护卫舰全体官兵瞬间出了巴斯进入河床,乘风破浪,顺流而上,直驶孟买,联姻的向往之地。\footnote{此处原文巧用双关语,如Maid(仆人、处女、圣女),Bath(巴斯城、浴缸),Bed(河床、床),因此含义也可以理解为:处女出了浴缸就上床。但译文只能选其一。}我曾祖父是船上的乘客……五十一天后,几乎完全以宝石饰针和随猎枪扳机应声而落的雷鸟为生的他,终于看见了伊什珀明的塔,那里是无理数与余弦的圣城,狂热的伊斯兰勇士教派的圣地。


我的课堂幽默概论止于伍迪·艾伦,他是这个行当中最有头脑的实践者。艾伦的期刊文章现今收集在三本书中,组成自成一体的幽默集,其独到之处在于既具有知性又滑稽可笑,不仅探讨众所周知的死亡与焦虑主题,而且包括一些重头的学科与文学形式,诸如哲学、心理学、戏剧、爱尔兰诗歌,以及对文本的解读(“犹太教哈西德传说”)。《有组织犯罪一瞥》是一篇对所有描述黑手党文章的戏讽,也是我所知道的最滑稽的文章之一,而《施密德回忆录》——希特勒理发师的回忆——则是对这位只干自己活儿的“德国老好人”的致命一击:

我被问到自己是否意识到所做之事的道德含义。正像我对纽伦堡特别法庭所说的那样,我当时并不知道希特勒是一名纳粹分子。事实上,多年来我一直认为他为电话公司工作。当我最终发觉他是个多么可怕的怪物时,为时已晚,因为我买了家具,得挣钱分期付款。有一次,在战争快结束的时候,我的确想松一松“元首”的围脖布,往他背上掉一点儿头发茬儿,可就在那最后一刻,我害怕了,退却了。

本章简短的选段只能转达出这些幽默大师们大量的艺术创作中的一点点,但是我要让学生知道 ,他们所学习的领域具有悠久的传统、严肃的目的和大量的勇气,这一传统至今仍存活于诸如加里森·凯勒、弗兰·勒波维茨、诺拉·依弗郎、卡尔文·特里林以及马克·辛格这些作家的作品中。辛格是《纽约客》悠久的作家家族中的当代明星——这个家族包括圣·克莱尔·马克威、罗伯特·刘易斯·泰勒、莉莲·罗斯、沃尔科特·吉布斯——他用冷面幽默来对沃尔特·温切尔这样的公害行刺,在匕首穿破皮肤之处几乎不留任何痕迹。

辛格的致命饮剂是由成百上千种稀奇古怪的事实与引语调制而成的。他是一位坚强的记者,他的风格是对自己的兴趣所在几乎不加压抑。这一招对那些不断考验老百姓的耐性的冒险家们特别有效。他在《纽约客》上对开发商唐纳德·特郎普的简介证明了这一点。注意到特郎普“所追求并达到的顶级奢侈生活,那种从未被任何抱怨所打扰的生存状况”,辛格描述了一次赴玛尔拉格的访问,那是马乔里·梅里韦瑟·波斯特和E.F.赫顿建于20年代的具有西班牙摩尔式威尼斯风格的豪宅,有118个房间,后来由特郎普改造成棕榈海滩健康俱乐部:

显然,特郎普的身心健康哲学根植于这样一种信念,即长时间接触特别漂亮的年轻健康俱乐部服务员会在男性顾客心中逐渐灌输一种热爱生活的愿望。按照这一理念,他将自己的作用局限于在雇用主要员工时亲自行使袋中否决的权利。特郎普领我来到主要健身房,在那里托尼·班尼特正在跑步机上快走,他每个演出季都来玛尔拉格做几场特约演出,被称为“常驻艺术家”。特郎普又向我介绍“我们的常驻内科医师,金格·李·索撒尔医生”——一个按摩学院的新毕业生。金格在给一位心存感激的俱乐部成员按摩酸痛的背部,我趁她听不见,问特郎普她在哪里受的培训。“我也不清楚”,他说。“护滩使者医学院?差不多的地方?告诉你实情吧。我一看到金格医生的照片,就不必再看简历了,也不用看别人啦。难道你要问我’我们聘用她是因为她在西奈山受训了十五年吗’?回答是不。我来告诉你为什么:因为等到她在西奈山花上十五年,我们就不想看她了。”

在当下所有的幽默作家中,加里森·凯勒对于社会变迁的眼光最准,在拐弯抹角地陈述观点方面,头脑也最富创意。他一次又一次地为我们提供欣赏旧体裁穿新装的快乐。美国当下对吸烟者的敌意是大势所趋,任何一位作家只要留意都会注意到这一点,并且会以相应的严肃态度撰文评论。而以下的评论方式却是凯勒所独有的:

在内华达山区唐纳山口的箱型峡谷里,美国最后一批吸烟者被两位联邦烟草管理员乘直升机锁定,就在正午之前,他们俩发现了喷出来的一点点烟圈。其中一位区队长通过空对地无线对讲机叫来了地面队伍。穿迷彩服的六个人是打击与禁止吸烟慢跑组织的成员,他们快速穿越崎岖的地带,包围了那一小撮藏匿之人,用催泪瓦斯制服了他们,迫使他们在八月炎热的阳光下脸冲下躺在沙砾上。他们共有三女两男,都在四十四五岁。自从第二十八条修正案通过之日起,他们就一直潜逃。

凯勒头脑中的这个体裁自从30年代的迪林杰时代就成为美国报刊的必需品,其中回荡着黑帮和美国联邦调查局人员、监视与交火,他对这种形式的喜爱在其作品中显而易见。

凯勒发现了如此完美的表达框架,他显然也喜欢将其运用于其他情形之中,如描述第一届布什政府对于储蓄与贷款业的紧急财政援助。下面是他《储蓄与贷款是如何被拯救的》一文的开头:

当一大群野蛮的匈奴人入侵芝加哥时,总统正在阿斯彭打羽毛球;一位记者的婶婶住在埃文斯顿,他一边向俱乐部会所跑,一边向他喊,“匈奴人在芝加哥杀人了,总统先生!你有何评论?”

布什先生虽然冷不防被这一入侵的信息所惊动,但还是冷静地说,“我们一直在密切关注整个匈奴人的情况,现在看来一切都朝着有利于我们的方向发展,不过我希望再过几个小时我们可以向你公布更确切的消息。”总统看起来很关心这件事,但也很放松,当然是下巴颏上扬,一副一切尽在掌控之中的样子。

这篇文章接着描绘了那些贪婪的野蛮人蜂拥人城,“烧毁教堂、演艺中心还有历史古迹,他们拽走和尚、处女还有副教授……贩人为奴,”然后抢劫储蓄与贷款办事处,而布什总统却对此无动于衷,因为“在购物中心出口处的投票亭显示,人们认为他对这一事件的处理还可以”。

总统决定对试图抢夺储蓄与贷款业这件事不加干预,而是投入1660亿美元,这笔钱不是任何种类的赎金,而是作为政府的正常扶持,这简单明了,并没有什么违反常规的,于是匈奴人和汪达尔人带着财物骑马而归,哥特人也朝着密歇根湖之北扬帆离去。

凯勒的讽刺让我佩服得五体投地。首先其幽默行为如此具有创意,而且其表达公民愤怒的方式是我没能找到的。我所能够聚集的只是无助的气愤,这个行业被这群贪婪之徒洗劫一空,我们的子孙一直到老都将为布什所拯救的这个行业而付出。

但是没有哪条准则说幽默必须观点明确。想入非非的纯粹的胡说八道“永远是乐事”,就像济慈在诗中所暗示的那样。我喜爱看作家天马行空,乐此不疲。下面两段选自伊恩·弗雷泽和约翰	厄普代克最近的作品,绝对异乎寻常,美国早期黄金时代的作品中没有哪篇能比得上这些更滑稽可笑了。弗雷泽的这篇叫做《与妈妈约会》,它是这样开始的:

在当今快速、无常、无根的社会中,人们见面、做爱、分手,彼此间没有真正的接触,因而每一位男子汉与其母亲的固有关系就弥足珍贵,不可忽视。这里有一位成年、阅历丰富、充满爱心的女人——她无须你到晚会或者单身酒吧与之见面,她也无须你千方百计地与之结识。有千百次机会,你和母亲自然而然地凑在一起,没有一般求爱所伴随的紧张——只有你们两个人,单独在一起。你所需要做的只是动一点儿脑筋,充分利用当前的机会。比如,妈妈开车带你去城里买新裤子。首先,在车载收音机上调出一个好电台,她喜欢的。进入高速公路,车匀速行驶——轮胎在路面上哼鸣,空调调到最大。然后探过前排座椅看着她,说类似下面的话:“你知道吗,妈妈,你的身材保持得真不错,别以为我没发现。”或者假设她进到你房间,给你拿来双干净袜子。你抓住她的手腕,拽到跟前说,“妈妈,你是我所见过的最迷人的女人。”也许她会叫你住嘴别说傻话,但我敢向你保证一件事:她决不会告诉你爸爸。也可能她觉得不好说出“亲爱的,派珀刚才向我调情来着”这样的话,或者也可能她心中暗喜,但无论什么原因,她会始终守口如瓶,直到有一天她不再羞于告诉世人你对她的爱。

厄普代克的作品《快乐的服饰》虽然几近于一种蹦极行为,将他带到谷底,离岩石只差几英寸,却具有一种打动人心的现实感。它不但调侃有关埃德加·胡佛那些关乎国家的种种黑暗疑团,也涉及近年来人们记忆中的那些美国高官们——已去世的总统和他的内阁。在琐碎的表面之下,使其奏效的是厄普代克一丝不苟的探索。可以肯定,所有的细节——名字、日期以及时尚术语都千真万确:

对于我们这些仍具有活力、对当下服饰仍感兴趣的人来说,最近读到波士顿的《环球》上“纽约社会名流”苏珊·罗森斯蒂尔的断言,那可真令人心生悲叹。据她说,1958年,约翰·埃德加·胡佛在一家广场酒店的套间里身穿女性服装,来回炫耀:“他身穿一件松软的黑色衣裙,非常松软,镶有荷叶边,脚上穿有蕾丝长筒袜和高跟鞋,还戴着一个黑色卷曲假发。”在艾森豪威尔第二任期间,一些高官们身穿异性服装,显示出特别的光彩与兴奋;未来的几代人对此会迷惑不解,他们只能想象,那种过时的黑色松软的衣裙,配上过分装饰的荷叶边以及配套的假发,曾经是非常时尚的装束,而事实上,我们大家都以为约翰·埃德加是一个类似于老古董式的人物。我一想到这些就感到可悲可叹。

比如总统艾森豪威尔,我们一贯正确的、可敬的艾克,就决不会被人逮着穿蕾丝长筒袜;即使他的确长了适宜的腿,也决不会这么做。我记得圣·劳伦在1958年为迪奥设计的时装系列发布不到一个月,艾克就身着钴蓝色的毛帐篷形长袍,脚穿白色后开口高跟鞋,头戴假发髻,闪亮登场。就在当天,如果我没记错的话,他派遣了五百名海军陆战队队员到黎巴嫩,而他自己的头发丝毫没乱。当时他就是穿着这套装备——或许是前一年束有腰带的A字形女装?他还炫耀自己带花图案的真丝围巾,而当时围巾仍被认为只适合老太婆。但是对于衣裙的边沿,他还是很保守的;当1959年圣·劳伦将裙子底边设计上提到膝盖,总统等了三个月要国会就此事作出决定,后来他实在是失去了等待的耐心,转而亲自动笔写信给巴黎世家。从那时起,一直到他任期的最后,他坚持穿中性暗褐色和米色的高腰日礼服。

而在另一方面,约翰·福斯特·杜勒斯则更喜欢紧身睡衣的款式和菘蓝色裤子和套装,他喜欢布料上带一点儿光亮,还有大量的环饰,上梳的金色假发以及带绒球的拖鞋式女鞋。尽管他坚决反对共产主义,却反倒偏好红色,不过据可靠消息,舍曼·亚当斯至少有一次把福斯特拉到一边,强调说鲜亮的颜色不适合大骨架的人。舍曼的名声虽然被骆马绒给毁了,但他在我心目中却迟迟不肯离去,那家伙围着稀奇古怪的鸵鸟毛长披肩、轻轻浆洗的柠檬色巴里纱,真令人着迷……

最终,乐趣是所有幽默作家都必须传达的,也就是大家都特别开心。这种激发出胆大妄为的想法就是我要我在耶鲁的学生所探讨的。开始我告诉他们用现有的幽默形式写,比如讽刺、戏讽、挖苦等等,不要用“我”或者以自己的经历来写。我向整个班级布置了同一个题目,指出我最近在报纸上发现的某些荒谬之处。学生们大胆地投身于自由联想、超现实主义以及荒谬杜撰。他们发现可以摆脱逻辑的束缚,用现有的幽默形式阐明严肃的观点,并乐此不疲。他们深受伍迪·艾伦的无逻辑关联陈述的影响。(“为此,拉比猛击头部,按照希伯来律法,这是表示关心的最微妙的办法之一。”)

大约四周之后,学生们开始疲倦。他们意识到自己能够写出幽默,但同时也意识到每周都用他人的口吻来不断发明笑料有多么累人。该是放慢他们的新陈代谢的时候了,让他们开始用自己的口吻写,写自己的生活。我宣布暂停阅读伍迪·艾伦,并且说当他们需要再读他的作品时,我会告诉他们。那一天一直没有到来。

我采用了奇克·扬原理,只写自己知道的,然后开始阅读那些用幽默作为经脉不知不觉贯穿于自己作品中的作家。其中一篇是E.B.怀特的《埃德娜之眼》。怀特回忆起在缅因州农场一边等待埃德娜飓风,一边连续几天听收音机空洞地报道飓风迫近的情形。那篇文章完美无缺,充满了睿智与温和的风趣。

另一位我发掘出来的作家是斯蒂芬·李科克,加拿大人。我回忆起小时候对他的印象是滑稽可笑,但忐忑不安地担心他会不会变得只是“荒谬可笑”而已,这种心态常常发生在你去看望自己的老朋友之时。不过他的作品经受住了时间的侵蚀。《我的金融生涯》一文使我记忆犹新,他在其中描述了试图以56美元开设一个银行账户的故事。该文至今似乎仍是幽默的范例,它描述了我们大家在与银行、图书馆以及其他刻板的机构打交道时有多么惊慌失措。重读李科克提醒我幽默作家的另一个功能,那就是以受害者或呆子的身份再现自己在多数情况下的无助。这对读者是一种疗法,使他们感到优于作者,或者至少认同于同病相怜的受害者。描述普通生活的幽默作家决不会没有素材,这一点几十年来已经令人愉快地被厄玛·邦贝克证明了。

这就是我们耶鲁幽默班行进的方向。许多学生写自己的家庭。我们也遇到问题,主要是夸张的用法,但我们逐渐解决了这些问题,尽力获得掌控能力。另外还有删除多余的句子,这类被删除的句子往往重复解释一遍本来已经暗示过的滑稽点。难以抉择的是判断夸张在什么程度上是被允许的,在什么程度上是过量的。有一个学生写了一篇幽默文章,描述他祖母的厨艺多么糟糕。但当我赞许这篇文章时,他却说他祖母实际上是个非常棒的厨师。我说我很遗憾听他这么说——不知怎地,这时那篇文章似乎就不那么可乐了。他问我有什么区别吗?我说在这篇文章里并没有区别,因为我已经欣赏过了,当时并不知道那不是真事,但我认为假如他从真事儿而不是杜撰开始,他会写得更持久——这当然事詹姆斯·瑟伯作为美国主要幽默作家能够长久的一个秘密。在瑟伯的《床翻了的那个晚上》中,我们知道他稍微夸大了事实,但我们也知道那天晚上阁楼上的床的确出了点事儿。

简言之,我们的课开始于努力创造幽默,希望在这个过程中传达一点儿真谛。我们结束于努力揭示真谛,希望在这个过程中加上幽默。最终我们意识到,这两者是交织在一起的。