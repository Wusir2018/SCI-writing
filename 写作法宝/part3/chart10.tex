\chapter{作为文学的非虚构写作}
几年前的一个周末,我在布法罗的一个作家研讨会上发言,会议是由那个城市的一群女作家组织的。这些女士对自己的行当是认真的,她们写的书和文章扎实、有用。她们问我是否愿意在那周的早些时候参加一个无线电广播谈话节目,这样可以宣传本次会议一他们与主持人在录音室,我在纽约我的公寓里,我们电话连线。

约定的夜晚到了,我的电话响了,主持人开播,他以自己那一行业热情高涨的情绪向我问好。他说录音室和他在一起的有三位可爱的女士,他很想知道我们大家如何看待目前的文学状况,我们对也是文学成员并且自己也有文学抱负的他的听众有什么建议。这个热诚的介绍就像一块石头掉在我们当中,三位可爱的女士谁也没说什么,我认为这里是恰当的反应。

沉默了一会儿,我终于说,“我觉得我们应该禁止再提‘文学’、‘文学性’、‘文人’之类的词。”我知道主持人简略地了解我们都是什么样的作家,我们都想讨论什么。但他没有什么别的参照。“告诉我,”他说,“你们大家对今天美国的文学体验有什么高见?”针对这个问题又是一阵沉默。最终我说,“我们在这儿要谈的是写作技艺。”

他不知如何是好,于是便开始求助于作家的名字,如欧内斯特·海明威、索尔·贝洛、威廉·斯蒂伦,这些人我们当然认为是
文学巨匠。我们说这些人碰巧都不是我们的榜样,我们所提到的是刘易斯·托马斯、琼·迪迪翁、加里·威尔斯。他从未听说过这些人。其中一位女士提到汤姆·沃尔夫的《太空英雄》,他也没听说过。我们解释说我们羡慕这些作家,是因为他们有驾驭当今事件和问题的能力。

“但是你不想写文学性的东西吗?”我们的主持人问。三位女士说她们觉得自己已经满足于所做的事。这使得节目又停顿下来,于是主持人开始接听众的电话,所有听众都对写作技艺感兴趣,想知道我们是怎么做的。“然而,在夜深人静之时,”主持人对几位听众说,“你就从未梦想过写一部伟大的美国小说吗?”他们没想过。他们没有那样的梦想——在夜深人静之时,或在任何其他时候,他们都没想过。这就是那种惯常的蹩脚无线电谈话节目之一。

这个故事概述了所有非虚构写作实践者都会看到的一种情形。我们这些人在尽力写好我们所居住的世界,或教学生写好他们所居住的世界,但我们都陷入一个时间弯曲之中,在那里文学从其定义本身仍就包括被确定为19世纪“文学性”的形式:长篇小说、短篇小说、诗歌。但现今作者所写、所卖的,图书、杂志所出版的,读者所需求的,绝大多数是非虚构作品。

这种变化可以通过各种例子得以说明。其一是“每月一书俱乐部”的历史。该俱乐部在1926年由哈里·谢尔曼创建之时,美国人很少有接触优秀新文学的途径,而主要是读如《宾虚》之类的破烂书。谢尔曼的想法是把所有有邮局的城镇都等同于一个书店,而后他开始将最好、最新的图书寄给他在全国各地新招募的读者。

他所寄的大部分图书是小说。1926-1941年,俱乐部主选书单偏重于小说家:埃伦·格拉斯哥、辛克莱·刘易斯、弗吉尼亚·伍尔夫、约翰·高尔斯华绥、埃莉诺·怀利、伊格纳奇奥·西隆尼、罗莎蒙德·莱曼、伊迪思·华顿、萨默塞特·毛姆、薇拉·凯瑟、布思·塔金顿、艾萨克·丹森、詹姆斯·古尔德·科曾斯、桑顿·怀尔德、西格里德·温赛特、欧内斯特·海明威、威廉·萨罗扬、约翰·P·马昆德、约翰·斯坦贝克,还有许多其他作家。那是在美国的“文学”高潮。“每月一书俱乐部”的成员几乎没有听见第二次世界大战的临近。直到1940年,一本名叫《米尼弗太太》的书才将这一消息带到俱乐部成员的家中,那是一本关于早期不列颠之战的小说,冷峻得使人上唇紧绷。

所有这一切都随着珍珠港事件发生了变化。第二次世界大战将七百万美国人送往海外,开阔了他们对于现实的视野:新地方、新问题、新事件。战后,这个趋势又由于电视的出现得到加强。每天晚上在自家客厅目睹现实的人们对小说家的慢节奏和随意幻象失去耐心。一夜之间,美国变成了一个只注重事实的国家。1946年以后,“每月一书俱乐部”的成员主要需求的是——因而收到的也是——非虚构作品。

杂志的传播也顺应同样的潮流。《星期六晚邮报》一直提供给读者的是比重很大的短篇小说餐食,这些小说家的名字似乎都有三个词——克拉伦斯·巴丁顿·凯兰、奥克塔沃斯·罗伊·科恩——但到60年代初,该报颠倒了这个比重。该杂志的百分之九十现在分配给非虚构文章,只保留一篇由三个词名字的作家撰写的短篇小说,来保持那些忠实于此的读者不至于感到被遗弃。这个时代是非虚构作品黄金时代的开始 ,特别体现在《生活》杂志中,它每周都刊登技艺精湛的文章。《纽约客》则通过现代美国写作原创性地标,如雷切尔·卡森的《寂静的春天》和杜鲁门·卡波特的《冷血》,提升了非虚构写作的形式。《哈珀》杂志也约到许多出类拔萃的作品,如诺曼·梅勒的《夜幕大军》。非虚构作品成为新美国文学。

当今生活中没有哪个方面——无论是现在还是过去——没有被人们以极认真优雅的写作风格提供给普通读者。在这种纪实文学中,增加了所有从前被认为只是学术性的学科,并使其成为非虚构作家和兴趣广泛的读者的领域,如人类学、经济学、社会史。增加的所有这些结合历史与传记的书籍,近年来使美国文坛名声大震:大卫·麦卡洛克的《杜鲁门》和《海之间的路》、罗伯特·艾伦·卡罗的《权力掮客:罗伯特·摩西和纽约的衰落》、泰勒·布兰奇的《分水岭:马丁·路德·金岁月的美国:1954— 1963》、理查德·克鲁格的《报纸:纽约先驱论坛报的生死》、理查德·罗兹的《原子弹诞生记》、托马斯·L·弗里德曼的《从贝鲁特到耶路撒冷》、杰伊·安东尼·卢卡斯的《共同基础:美国家庭生活的动荡十年》、埃德蒙·莫里斯的《西奥多·雷克斯》、尼古拉斯·莱曼的 《乐土:黑人大迁移及美国的变迁》、亚当·霍克西尔德的《利奥波德国王的鬼魂:殖民非洲的贪婪、恐怖与英勇故事》、罗纳德·斯蒂尔的《沃尔特·李普曼与美国世纪》、玛丽安·伊丽莎白·罗杰斯的《门肯:美国的反传统文评家》、大卫·雷姆尼克的《列宁墓:苏联帝国的最后岁月》、安德烈·德尔班科的《梅尔维尔》、马克·史蒂文斯和安娜琳·斯旺的《德·库宁:美国大师》。简而言之,我的非虚构新文学作家名单将包括所有承载了信息,并以有力、清晰与人文的特点呈现这些信
息的作家。

我并不是说虚构作品死亡了。显然,小说家可以带我们进入其他作家无法带入的地方:进入深沉的情感世界和内在的生活。我所说的是,我没有耐心听势利之人说非虚构写作只是给新闻体写作换了一个名称,以及新闻体写作无论用什么名称都是一个脏词儿。我们在重新界定文学,让我们也重新界定一下新闻体写作。新闻体写作就是先出现在期刊上的写作,与读者无关。刘易斯·托马斯的最早两本书,《细胞生命的礼赞》和《水母与蜗牛》,就是首先为《新英格兰医学杂志》以散文形式所撰写的。历史地讲,好的新闻体作品在美国已成为好的文学。H.L.门肯、林·拉德纳、约瑟夫·米切尔、埃德蒙·威尔逊,以及其他几十位主要的美国作家,在他们于文学殿堂内被经典化之前都是职业记者。他们只是尽力做好自己的事,从不担心自己所为是如何被界定的。

最终每一位作家都必须遵循自己感到最惬意的路径。对多数学习写作的人来说,这条路径就是非虚构写作。它使得人们能够写自己知道的事,或者自己能够观察或发现的事。这一点特别适用于年轻人和学生。我们更愿意写有关自己生活的题材,或者自己有能力写的题材。动力是写作的核心。如果说非虚构写作是你所写或者所教的最佳领域,不要被迷惑而以为这是一种低等的类别。唯一重要的区别只在于它是好的写作还是差的写作。好的写作无论采取什么形式,无论我们管它叫什么,都仍然是好的写作。