\chapter{商务写作:工作中的写作}
如果你在工作中需要写作,这一章就是为你准备的。就像科技写作一样,焦虑是问题的关键,而人性化和清晰的思考则是解决问题的关键。

虽然这只是一部关于写作的书,但此书并不只是写给作者的。其中的原理适用于每一个在自己的日常工作中需要写一点东西的人。备忘录、商业信函、行政报告、经济分析、市场营销计划、给老板的说明、电传、电子邮件、报事贴——每天流转于办公室的各种各样的纸片文书都是写作的一种。要认真对待这些文字。无数事业上的起伏都起因于雇员有没有能力流畅地陈述事实、概述一次会议的内容,或者阐述一种想法。

多数人是为各种社会机构工作的:商业企业、银行、保险公司、律师事务所、政府机构、教育系统、非营利组织以及其他单位。这些人中的多数是管理者,他们所写的文书都将公之于众:总裁致持股人、银行家说明程序变更、校长写致家长信。他们无论是谁,都有害怕写的倾向,以至于其字里行间缺乏人文关怀,而其机构也是如此。在此情形之下,很难想象这些机构都是确确实实存在的地方,每天都有真真切切的男男女女来这里上班。

但即使大家都为某个机构工作,写文书也不必千篇一律。其实机构可以温暖起来。行政管理者可以变得人性化起来。信息可以清晰而不是虚浮地传达到人。你只需要记住读者认的是人,而不是那些抽象的词语,如“可获利性”、“运用”、“实施”等,或者死板的句子结构,人们在其中看不见任何人在做什么,如“前期可行性研究尚处于文件阶段”。

乔治·奥威尔就这一问题给出了范例。他将《圣经》中《传道书》的著名诗句翻译成了现代官僚模糊语。原句如下:

我又转念,见日光之下,快跑的未必能赢,力战的未必得胜,智慧的未必得粮食,明哲的未必得资财,灵巧的未必得喜悦;所临到众人的,是在乎当时的时机。

奥威尔的译文如下:

对当代现象的客观考虑迫使我们得出以下结论:在竞争性活动中的成功与失败并没有显示出与内在能力相对称的趋势,而相当大的不可预测成分必须始终如一地考虑在内。

首先注意这两段话外观看起来如何。上面第一段吸引我们去阅读。该段词语简短,周围空气充沛,整段能够传达出人们说话的节奏。第二段堵满了冗长的词语,它立即向我们显示出一个头脑笨拙的人在思考。我们不想跟一个用这么窒息的语言表达思想的人去任何地方。我们甚至连读都不想读。

同时还要注意这两段话所讲的内容。在第二段中,那些简短、生动的日常生活中的意象,如赛跑、打仗、面包、财富,全无踪影,而顶替这些的是摇摇晃晃、冗长无力、含义笼统的名词。任何表达某人做了什么事的意思都不见了,如“我又转念”,或者他对人生中一大谜团、命运之不可捉摸感悟到了什么,如“见”,这些词语全都无影无踪。

让我来举例说明这种顽疾是如何传染给大多数在工作中需要写作之人的。我要用校长作为第一个例子,不是因为校长们此类问题最严重(他们并非如此),只是因为我碰巧有这么一个例子。我指的是所有机构中的所有人,他们的语言丧失了人性化,没人知道他们那里管事的人在说什么。

我同校长的接触始于我接到康涅狄格州格林威治校监欧内斯特·B·弗莱希曼的电话。“我们想请你来为我们‘去行话’,”他说。“我们觉得,如果我们学校的高层不能打扫干净自己的写作,我们就教不好学生写作。”他说要寄给我一些本系统内的典型材料。他的想法是要我分析这些写作材料,然后开一个讲习班。

吸引我的是弗莱希曼博士和他的同事们自觉自愿地将自己暴露无遗的胆识;这种自我暴露弱点的行为自有其力量。我们定了个日子,很快一个鼓鼓的信封邮到了。其中包括各种各样的内部备忘录,还有来自城里16所小学、中学、高中邮给家长的油印通讯。

这些通讯看起来欢快、亲切。显然整个教育系统在努力与孩子们的家长热情地沟通。但是只看第一眼,我就发现一些冷飕飕的问语,比如“优先评价程序”、“调整过的科目分类日程表”,而其中有一位校长保证,他的学校将提供“增强了的正面学习环境”。同样明显的是,该系统并没有像他们所想的那样在热情地沟通。

我研究了校长们的材料,将其分为好例子和差例子。我们定了一个早晨在纽约市的格林威治见面,40位校长和课程负责人准时集合在一起,急切地想要学习。我告诉他们,我所能赞许的只是他们为此不怕降低自己身份的勇气。在这场所谓孩子们为何不会写的呼吁中,弗莱希曼博士是我所遇见的第一个承认不仅青少年用语拖沓沉闷,而且成年人也是如此的人。

我告诉校长们,要把那些管理学校的人看做是同我们一样的人。我们对人们装腔作势的行为、别出心裁的词语心存疑虑。社科领域的人杜撰出这类词语来是为了混淆普通人的视听。我要求大家一切顺其自然。如何写与如何说都将展示出自己的个性。

我让大家听自己是如何向社区展示个性的。我复制了某些差例子,改了其中学校与校长的名字。我解释说,我首先要向大家大声朗读其中的部分例子,之后再来看看是否能用朴实的语言将其改好。以下是我选的第一个例子:

亲爱的家长:

我们建立了特别的电话通讯系统,为家长交流信息提供更多的机会。今年我们将进一步强调通讯的目的,并运用各种办法来实现这一目标。你们从作为家长的独特身份所提供的信息将有助于我们计划和实施一项满足您的孩子的需求的教育计划。家长与父母间的公开对话、反馈以及信息共享将使我们能够与您的孩子一起以最高效的方式把教学工作做好。

校长

乔治·B·琼斯博士

我可不想收到这种通讯。我作为家长也许可以提供很独到的信息,但在这种情况下我却不想沟通了。我想被告知,学校将使我给老师打电话更容易,他们希望我能经常打电话来讨论孩子在学校的表现。事与愿违,结果家长收到的是连篇的废话,什么“特别的电话通讯系统”、“进一步强调通讯的目的”、“计划和实施一项教育计划”。至于“公开对话、反馈以及共享信息”,这实际上是用三种方法说同一件事。

琼斯博士显然是好意,而且他的计划是我们大家都想要的:都希望有这么一个机会拿起电话告诉校长,尽管上周二在操场上发生了不愉快的事儿,强尼还是个好孩子。但琼斯博士听起来却不像是我想要通话的人。其实他听起来根本就不像真人。他的话就像是从电脑里放出来的。他在浪费他自己这个丰富的资源。

我选的另一个例子是学年初寄给家长的《校长致辞》,它包括两个截然不同的段落:

从根本上讲,福斯特是一所好学校。在某些科目或技能学习领域需要帮助的学生在这里能受到特别的关注。在即将到来的学年里,我们将努力提供强化了的正面学习环境。孩子与教职员工一定会在有助于学习的环境中工作与学习。我们需要广泛、多样的学习材料,要求对个体能力与学习方式给予充分重视。学校与家庭的配合在学习过程中极为重要。我们大家都应该认识到每个孩子想要的教育目的的重要性。

及时沟通今年给孩子们制定的计划情况,让我们知晓您的问题以及您的孩子可能有的特别需求。在前几周里我见过你们当中的许多人。请继续光临并介绍您自己或者谈谈福斯特。我期盼我们大家都能度过卓有成效的一年。

校长

雷·B·道森博士

在第二段向我致辞的是一位平易近人的人,而在第一段我却听到一位教育家在说话。我喜欢第二段里真正的道森博士。他说起话来,言辞温暖、适宜:“及时沟通情况”、“让我们知晓”、“我见过”、“请继续”、“我期盼”。

形成对照的是,第一段中的教育家道森从不用“我”,甚至连“我”的含义都不带。他坠回到自己的专业术语中,那样他才有安全感,但却没有停下来注意到他实际上并没有告诉家长任何事。什么是“技能学习领域”,又如何与“科目”相区别?什么是“强化了的正面学习环境”,又如何与“有助于学习的环境”相区别?什么是“广泛、多样的学习材料”:铅笔、课本、幻灯片?究竟什么是“学习方式?”什么“教育目的”是“想要的”?

简而言之,第二段温暖、有人情味,而第一段则迂腐、含糊。我反复遇见这种样式。每当校长们写信给家长通知他们一些日常细节时,他们写起来就带有人情味:

校门前的交通似乎又开始拥堵了。如果您还方便的话,放学时请来学校后边接孩子。

请您找自己的孩子谈一谈他们在餐厅的行为举止问题,我将不胜感激。假如您看见自己的孩子在吃饭时的表现,准会大吃一惊。时不时查一下,看他们是不是欠午饭钱。有时孩子们迟迟不还钱。

可是这些教育家们一旦写如何实施教育计划,他们个人就消失得无影无踪:

在这份文件中您会看见被认可和优先的计划宗旨和目标。目标的评价程序也依照可接受的标准建立起来。

在实施这项练习之前,学生对选择题接触很少。大家感到,与学生正在学习的单元同步的习题具有非常正面的效果,考试成绩证明了这一点。

听我读了各式各样的好的和差的例子之后,校长们开始听见真实的自我和作为教育家的自我之间的区别。关键问题是如何弥合这一鸿沟。我列举了我所倡导的四个理念:清晰、朴实、简明、人文。我解释了如何运用主动动词和避免“概念名词”。我告诫他们不要用教育学的术语作为拐杖,几乎所有科目都可以用好的语言表达得通俗易懂。

这些其实都是基本原则,可校长们赶紧记下来,就好像之前从未听说过一样——也许真没听说过吧,或者至少好几年没听说了。也许这就是官僚散文体在各类官僚机构为何如此泛滥的原因吧。一旦管理者升到一定的高度,没人再去向他指出简单陈述句之美,或者指给他看他所写的有多么臃肿不堪,充斥着浮夸的词语和笼统的概念。

最终我们的讲习班开始实际行动。我分发了准备好的复本让校长们重写那些盘根错节的句子。这可真是严酷的时刻。他们是第一次遭遇敌手。校长们在本子上划拉着,潦草地写出自己所划拉出来的句子。有一些人什么也没写。还有一些人把纸弄得皱皱巴巴。他们开始看起来像作者了。整个房间被一片可怕的寂静所笼罩,只是偶尔被有人划掉句子或团皱纸张打断。他们开始听起来像作者了。

这一天接下来的时间,他们才慢慢放松下来,开始用第一人称写作,开始用主动动词。有那么一会儿,他们还是放不下冗长的词语和模糊的名词(如“家长通讯反馈”)。不过逐渐地,他们写的句子开始有人情味了。我叫他们处理“目标的评价程序也依照可接受的标准建立起来”这句话,其中一位是这样写的:“年末我们将评估大家的进展。”另一位:“我们会看到大家所取得的成效。”

家长想要的就是这样质朴的话语。持股者从公司想要的、顾客从银行想要的、寡妇从负责其社会保险机构想要的,也都是如此的话语。人人都有对人际交往的深切渴望和对大话连篇的憎恶。最近我收到一封致“亲爱的顾客”的信件,来自为我提供电脑的公司。信是这样开始的:“我们将转移我们终端用户的订单入径和供货介绍程序到新的电子营销中心,生效期为3月30日。”我最终总算琢磨清楚,原来他们换成了新的800电话号码,而那个终端用户就是我。任何机构如果不屑于把文书写得清楚和有人情味,都将丧失朋友、顾客或金钱。让我向公司经理们以一种方式敬一言吧:预期利润率将会经历亏空。

这里有一个例子,显示出公司如何用做作的语言抛弃了人文关怀。这个例子是一家大股份有限公司发布的“顾客公告”。顾客公告的唯一目的就是向顾客提供有用的信息。这份公告是这样开始的:“各公司都在增大力度转向承受力计划技术,以便确定未来何时处理负载量将会超过处理负载能力。”这句话丝毫无助于顾客,而是由奥威尔所讥讽的名词如“承受力”和“负载能力”等凝结在一起,顾客根本就无法从这种表达中想象出什么过程。什么是承受力计划技术?谁的承受力在计划之中?由谁在计划?第二句是这样的:“承受力计划增加了决策过程的客观性。”更多的死板名词。第三句是这样的:“在信息系统资源重点领域,管理被赋予了增强性的决策参与。”

每一句顾客都得停下来翻译。这份公告还不如就用匈牙利语写得了。顾客从第一句开始——有关承受力计划技术那句,翻译过来的意思是:“这有助于知道你何时会给自己的电脑施加超过其处理能力的工作量。”第二句“承受力计划增加了决策过程的客观性”——其意思是你在决定前应该搞清楚事实。第三句——有关增强性的决策参与——其意思是“你对自己的系统 知道得越多,系统就越高效”。但它也可以表示其他好几个意思。

但是顾客不会继续翻译了。很快他就会找另一家公司。他想,“如果这些家伙那么聪明,他们干吗不告诉我他们是做什么的?也许他们并不那么聪明。”公告继续说,“为了避免将来的成本,生产力得到了增强。”这似乎是指产品会免费——所有成本都被避免了。接着公告向顾客保证,“系统以功能性运转。”这指的是系统工作正常吧。但愿如此。

最终,在最后我们才得到一丝人文关怀。公告的作者问一位满意的顾客他为何选这个系统。顾客说他选此系统是因为该公司有服务上乘的声誉。他说:“电脑就像一支复杂的铅笔。不管它是如何运转的,一旦坏了,你就需要有人来修理。”在之前所有那些垃圾之后,注意这句话是多么清新:体现在其语言中(舒适的词语),可视觉化的细节中(铅笔),以及人文关怀中。在技术程序描述中去除冷冰冰因素的办法是,将其与我们大家都熟悉的经历联系在一起,如:东西坏了的时候等修理工。这使我想起在纽约地铁看见的一幅告示,它证明即使偌大一个都市官僚机构也可以带有人情味地向其市民说话:“如果你常乘地铁,可能已经看到一些标记引导你通向从未听说过的线路。其实这些只是你非常熟悉的线路的新名称。”

然而,要在美国公司里推行朴实的话语并非易事。这里有太多的虚荣。各层的经理禁锢于这样一种理念,即简朴的风格反映出简单的头脑。其实,简朴的风格是勤工作、勤思考的结果;含混的风格反映出某人思维含混或者态度傲慢,或者太笨拙,或者太懒惰,组织不好自己的思路。记住,你所写的常常是你得以表现自己的唯一机会,而你文字的对象的生意或者金钱或者好处恰是你所需要的。假如你所写的华而不实,或者大话连篇,或者朦胧模糊,你本人就会被看成这样的人。读者别无选择。

在格林威治讲习班之后,我大着胆子出去到一些大公司开了讲习班,他们也邀请我为他们“去行话”,这样我就了解到美国公司的情形。“我们甚至连自己的备忘录都看不懂了,”他们告诉我。与我合作的人们撰写大量的材料用于公司内部以及外部交流。内部材料有内部报刊和公告,其目的是告诉员工在他们的工作“场所”发生的事情,给他们以归属感。外部材料包括发给股东的光纸印刷杂志和年度报告、行政主管要的讲话稿、新闻发布稿,还有给顾客的产品说明书。我发现几乎所有这些材料都缺乏人情味,而且许多都看不透。

比如下面这段就是公告中的典型句子:

与以上改进产品同时发布的还有对系统支持程序的变更,这一程序产品同NCP结合在一起使用。所增加的功能性改进产品中还有动态重新配置与交互系统通讯设各。

在这样的工作中,作者无快乐可言,读者当然更不快乐。这像是出自电影《星球旅行》的语言。如果我是员工,我可高兴不起来,也了解不到什么。这些话语鼓不起我的士气。我不会再读这些公告了。我告诉公司的文书们,必须找到这些优异成绩背后的人。“去找构想出这个新系统的工程师,”我说,“或者设计这个系统的设计师,或者安装这个系统的技师,让他们用自己的话告诉你他们是如何想到这个主意的,或者他们是如何共同构想出这个主意的,或者这个主意将如何在真实世界里为真人服务。”要使任何机构变得温暖起来的办法,就是找到丢失的“自我”。记住:“自我”是所有故事中最有趣的因素。

公司的文书解释说,他们其实常找那位工程师交谈,但却不能使他正常说话。文书还给我看了典型的引语。工程师们用晦涩的语言说话,随处都是首字母缩写(如“副系统支持只适用于同VSAG或TNA使用”)。我说,文书得不断地找工程师,直到他最终把自己说明白。文书说工程师不想叫别人明白,假如他说得太简明,他会在同事眼里看起来像蠢货。我说,他们应该对事实与读者负责,而不该对工程师的虚荣负责。我敦促他们自己要有写作的自信,而不能放弃掌控的权利。他们回答说,在等级分明的公司里,书面报告需要高层逐级审批,这说起来容易做起来难啊。我感觉到他们心里有一种潜在的恐惧 :要按公司的惯例行事,为了保住工作,不要冒险去使公司更人性化。

为了让自己听起来有威信,高层主管同样成为牺牲品。有一家公司每月发布公告,为的是使高层“管理人员”能够同中层管理者和低层员工分享其所关心的事情。每一期的重要内容都是分公司副总裁告诫性的致辞,我暂且称他为托马斯·贝尔。从他的每月致辞中判断,他是个空话连篇的蠢货,实际上说不出什么,却大放厥词。

当我提到这个例子时,文书则说,托马斯·贝尔其实是个谦虚谨慎的人,是位好经理人。他们指出,致辞并非他自己所为,是别人为他写的。我说贝尔先生是被帮了倒忙——文书应该每月都去找他(如果需要,带着录音机),在那呆着,直到他用到家后对贝尔太太说话时同样的语言来谈论他所关心的问题。

我意识到,美国的大多数经理人并不按照自己真正的写作风格或说话方式写作。他们放弃了使自己与众不同的特质。假如说他们对公司似乎很冷漠,那是因为他们默认了人性被抽干、被烘干的程序惯例。他们盲从于高科技,却忘记无论好坏,自己最有力的工具是词语。

如果你为某个机构工作,不管什么工作,不管职位高低,要写就写出真实的自我。你会在其他机器人般的人群中间作为真人独树一帜,你所立的榜样也许能说动托马斯·贝尔写出自己的东西。