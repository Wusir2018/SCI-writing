\chapter{科学与技术}
如果你在人文学院找一班写作课的学生,给他们留作业,叫他们写有关科学的某个题目,教室里准会响起一片可怜兮兮的嘀咕声。“别啊!别写科学!”学生会嘀咕道。他们都有一种共同的烦恼:怕科学。他们在年纪很小时就被某位化学或物理老师告知自己没长“理科脑子”。

找一位化学家或者物理学家或者工程师,叫他们写一份报告,你会看见他们简直要疯掉:“不!别叫我们写这个!”他们会这么说。他们也有一种共同的烦恼:怕写。他们在年纪很小时就被某位语文老师告知自己没有“词语天赋”。

两者其实都是不必要的害怕,但它们拖拽着人们的生活,因此,在本章我愿意帮助你平息无论是哪一种害怕。本章基于一个简单的原则:写作并非是语文老师所独有的特殊语言。写作是付诸文字的思维。只要你想得清楚,就能写得清楚。打破迷信,科学只是另一种非虚构项目。打破迷信,写作只是另一种科学家传播知识的方式。

在这两种害怕中,我曾经怕科学。我的化学曾挂科。教我的女老师教过三代学生,已然是一位传奇人物,其传奇性在于她可以教会任何人化学。时至今日,我在这方面比詹姆斯·瑟伯的奶奶也强不了多少。瑟伯在《我的人生与艰难时世》中回忆道,奶奶以为,“电是从墙上的插座中无影无踪地滴进全屋的。”但作为作家,我已经学会如何将科技内容传达给普通读者。这只是一句之后再写一句的问题。然而这个“之后”很关键。你必须迫使自己卖力地将一个个句子都写成线性结构,这至关重要。这可不是想象跳跃或暗含真谛之处。这里的统治家族是实事陈述和逻辑演绎。

我给学生留的科学作文很简单。我只是叫他们描述某些东西是如何运转的。我不要求文体风格或者任何其他优雅性。我只要他们告诉我,比如说,一架缝纫机是怎样工作的,或者水泵是怎样抽水的,或者苹果为何往下掉,或者眼睛怎样告诉大脑它看见了什么。描写任何过程都行,而且“科学”可以笼统地定义为包括技术、医学和自然。

新闻写作的一条原则是“读者一无所知”。从原则的角度来讲,这句话倒并非讨好之语,但科技写作的作者务必牢记这一点。你不能认为你的读者知道你自以为大家都知道的东西,或者读者还记得曾对他们解释过的东西。看了几百次的演示之后,我还不敢肯定我能像航班乘务员所演示的那样套上救生衣:也就是“简单地”将手臂穿过带子,“简单地”将两个棒形纽扣直接向下拽(或许是向边上拽?),还有“简单地”将其吹起来——但不要过早。唯一我能确保的一步是过早地将其吹起来。

描述一个程序如何运作很有价值,其原因有二。它迫使你确信程序是如何进行的。然后它迫使你带领读者重新经历之前已经明了的想法和演绎。我发现这对许多思维混乱的学生来说是一种突破。有一个是来自耶鲁大学的很聪明的大二学生,他在期中还是满篇言辞模糊,后来有一天却兴高采烈地来到班上,问我可否向大家读一下他自己有关灭火器如何灭火的作文。我肯定我们当时都预感文章会一团糟。但他的作文运笔简洁严谨,清楚地解释了三种不同的火是如何用三种不同的灭火器来熄灭。我真是高兴极了,他一夜间就学会了有序写作,他自己也高兴极了。到了三年级结束的时候,他写了一本“如何做什么”的书,结果发行量比我写的书都好。

许多其他思维模糊的学生尝试了同样的办法,之后写起来就清晰了。试试吧。科学和技术写作原则适用于所有非虚构写作。这一原则引领一无所知的读者,一步一步地掌握他们原本认为自己并没有能力掌握,或者害怕因为自己太笨而理解不了的科目。

把科学写作想象为一座倒金字塔。先从塔底下以读者一定知道的事实开始,然后再介绍更多内容。第二句拓宽上一句所陈述的内容,使金字塔更宽,然后第三句拓宽第二句。这样你就可以逐渐超越事实,进一步移动到内涵与推断——这里的新发现是如何改变已知的事物,它可能打开什么新的研究渠道,该研究可能适用于何处,这个金字塔能变多宽是没有限制的,但是只有从一个狭窄的事实开始,读者才会明白宽广的内涵。


小哈罗德·M·施梅克刊登再《纽约时报》第一页上的文字就是一个范例。

华盛顿——在加利福尼亚有一只黑猩猩,会画“连城”游戏。培训师对这种习得现象很是兴奋,但他们对另一件事更为惊叹。他们发现能从这只动物的大脑看出某一个特别举动做得是对还是错。这取决于黑猩猩的注意力状况。当这只受过训的动物集中注意力时,他做出的举动就是正确的。

好,这个事实还挺有意思。但它凭什么登上《纽约时报》第一页呢?第二段告诉了我:

重要的是,科学家们能够辨认出黑猩猩的注意力状况。他们通过对脑电波信号进行详细的电脑分析,可辨别出被称为“精神状态”的现象。

但这在以前可能吗?

比起简单地探测兴奋、困倦或睡眠之类的粗略状况,其成果与希望要大得多。这是向认识大脑如何运作迈出了新的一步。

为何是新的一步呢?

加利福尼亚大学洛杉矶分校的黑猩猩及其研究小组已经从画“连城”游戏阶段毕业,但是研究脑电波的工作仍在继续。这项工作已经揭示出一些空间飞行期间大脑行为的惊人秘密。其成果有可能应用在解决地球上的社会与家庭问题,乃至改进人类的学习能力上。

真棒。这项研究的应用领域简直太宽泛了:太空、人类问题和认知工程。但它只是一个孤立的研究课题吗?其实不是。

这是正在全美国和全世界的实验室中进行的现代大脑研究蓬勃发展的一部分,所涉及的生物各式各样,从人到猴子,再到鼠类、金鱼、扁虫和日本鹌鹑等。

我开始看见整个形势的发展,但目的又是什么呢?

这些研究的最终目的是了解人类大脑——那个不可思议的三磅重的软组织集合体。大脑可以想象出宇宙最远的太空边缘以及原子最终极的核,但却探查不清自己的功能是如何运作的。每一项研究计划都是从一个巨大的谜团中啃掉一小块。

现在我知道加利福尼亚大学洛杉矶分校的黑猩猩在国际科学领域处于何种地位了。知道了这个之后,我就准备好去了解这只黑猩猩的具体贡献。

在这个教黑猩猩做画“连城”游戏的例子里,通过仪器在纸上记录下来的代表动物脑电波的波浪线中,即便是受过训练的眼晴也看不出什么特别之处来。但是通过电脑分析,人们有可能辨别哪些痕迹显示这只动物将要做出正确的举动,而哪些之前犯了错。

其中的关键部分是主要由约翰·汉利博士研发的电脑分析系统。能预兆正确答案的精神状态是那种可以描述为因受过训练而注意力高度集中的状况。假如没有电脑来记录并分析大量复杂的脑电波,这种精神状态的“征兆”就无从发现。

这篇文章在报纸上连续用了四栏,描述这项研究的潜在用途——探究家庭关系紧张的原因,减小司机交通高峰期的焦虑——而且最终还涉及医学与心理学的许多领域正在做的研究工作。但文章开始于一只黑猩猩做画“连城”游戏。

你可以通过帮助读者认同正在进行科学工作,来解除科学写作中的许多谜团。同样,这意味着寻找人的因素——而且如果你非得以黑猩猩为例,至少那是在达尔文进化阶梯上仅次于人的最高一档。

其中有一种人的因素就是你自己。用你自己的经历去连接读者与你所描述的话题。在下面讨论记忆的文章中,注意作者威尔·布拉德伯里如何以个人独特的手法让我们理解一个复杂的话题:

直到现在,我还能看见沙土像一片乌云般打向我的眼睛,我听见父亲用沉着的嗓音叫我快把蜇眼的沙土哭出来,我感到胸中燃烧着愤怒和羞辱。距离我的玩伴在抢我的玩具救护车时对着我的脸扬起一把沙土那刻起,30多年过去了,然而沙子和救护车的影像,父亲说话的声音,以及我受伤时情感的跳动,至今清晰可见。这些是我最初的记忆,这些最初的影像、语言、情感玻璃镶嵌于马赛克图案之中,后来通过大脑最基本的功能——记忆,我认识到那些就是自我。

如果大脑没有这个储存和回忆信息的神奇功能,那么这些醒与睡、表达我们如何感知事物以及进行复杂活动的大脑关键系统,就只能起到比模糊地感知每一刻的印象多不了多少的作用。人也不会有自我的真正感觉,因为他没有过去的画廊供审视、借鉴、欣赏,也不能在需要的时候藏匿其中。然而经过了几千年对自己行为特点的总结、细读和误读,人类渐渐对断开和储藏那些不断流逝的时间点滴的过程有了些许了解。

期中一个问题是,决定记忆的是什么,拥有记忆又意味着什么?例如,亚麻籽油有一种记忆,把它暴露于日光下,哪怕只有很短的时间,当它第二次曝光时,就会改变常态和感光度。它会“记住”第一次与光接触的特性。电子与流体线路也有记忆,情况更复杂一些。把这些线路安装在电脑里,就能够储存和提取大量的信息。而人体至少有四种记忆……

这是一个好开头。谁还没有一串串生动的、从极小的年龄回忆起来的印象呢?读者急于了解这种储藏和提取记忆的成绩是怎样取得的。亚麻籽油这个例子就足以激励人们去了解“记忆”到底是什么,而后作者又回到人的参照框架中来,因为是人安装了电脑线路,而且自身还有四种记忆。

另一种个性化的写作方法是将科学故事编织到某些人身上。伯顿·卢艾什多年来为《纽约客》撰写的系列文章,被称为“医学年鉴”,就具有这种吸引力。这些文章是侦探小说,其中总有一位受害者——某个普通人得了一种不知名的病——还有一位密探痴迷于寻找罪魁祸首。下面是其中一篇的开头:

1944年9月25日,星期一早上大约8点钟,一位衣衫褴褛、漫无目的的82岁老人倒在了哈德逊车站附近的人行道上。一定有很多人注意到了他,但他独自躺在那里好几分钟,神志不清。他由于腹部痉挛而蜷缩着身子,痛苦地发出呕吐声。之后一名警察出现了。若不是弯腰仔细打量这位老人,警察还以为又碰上一个醉汉。大早上在城里这个区域遇见醉倒的流浪汉是常事。但是他这个想法并没有持续多久。这位老人的鼻子、嘴唇、耳朵还有手指都呈现出一种天蓝色。

到中午,十一位变蓝的男子被送进了附近的医院。但别害怕:野外流行病学家奥塔维奥·佩利特里医生在现场,他正打电话给疾病预防局的莫里斯·格林伯格医生。两人逐渐将病人的各种病症联系在一起,发现目前的病状似乎违背医学史上的任何记载,而直到后来这种病例才被确诊,罪魁祸首原来是一种极为罕见的中毒现象,就连许多标准的毒理学教科书都没提过。卢艾什的秘诀就是古老的讲故事艺术。我们受故事中追寻与悬疑的吸引。但他并没有以中毒医学史开头,或者谈论标准的毒理学教科书。他向我们描绘了一个人,又不仅是一个人,而是一个变蓝了的人。

另一个帮助读者了解不熟悉事物的方法是将人们与他们自己所熟悉的景致联系在一起。将抽象的原则简化为他们可以视觉化的意象。建筑师莫舍·萨夫迪构想出生物环境模型,他用这一模型设计了1967年蒙特利尔世界博览会创新性人居综合体。他在《超越生物环境模型》一书中解释说,人类假如花时间看一看自然是如何做的,就会把自己的家园建设得更好,因为“自然创造形状,而形状则是进化的副产品”:

人们通过研究植物和动物的生命、岩石和水晶的构成,可以发现形成它们各自形状的原因。鹦鹉螺进化成自己独特的形状,由此随着它的壳长大,头就不会卡在开口处。这就是我们所知的日晷式生长,其结果是长成螺旋状。从数学的角度看,这是其唯一的生长方式。

要用特定的材料获得强度也是同样的道理。看一看秃鹫的翅膀,其翼骨的构成,进化出一种复杂的三维几何形状,这是一种空间框架,很薄的骨头在末端变厚。秃鹫的主要生存问题就在于如何增强翅膀的力量(当这种鸟儿飞翔时,下摆运动量巨大)而不增加体量,因为那样会限制其活动。通过进化,秃鹫有了难以想象的最有效结构——翼骨中的空间框架。

“生命的每一个方面都有对形状的反应,”萨夫迪写道,指出枫树和榆树有宽宽的叶子,以便在温带气候下最大限度地吸收阳光;橄榄树叶则旋转式生长,因为它必须保存水汽而不吸收热量;仙人掌则垂直转向光照。我们都能想象出一片枫叶和一棵仙人掌来。对每一个理性原则,萨夫迪都给我们一个简明的例证:

经济性与生存是自然界的两个关键词。脱离具体环境来观察,长颈鹿脖子的长度似乎并不经济,但实际上它很经济,因为长颈鹿的大部分食物都在高高的树上。我们所理解的美,我们所欣赏的自然界中的美,从来都不是任意的。

还可以再举一个黛安娜·阿克曼所写的有关蝙蝠的例子。我们大多数人对蝙蝠都只知道三个事实:蝙蝠是哺乳动物;我们不喜欢蝙蝠;蝙蝠有某种雷达,能使它们在黑夜中飞行而不撞上东西。显然,写蝙蝠的人必然要很快回来解释“回音定位”机制是如何运作的。在下面这段中,阿克曼给了我们极为精准的细节描写,而且很容易同我们已知的事情关联起来。阅读这一运作过程成了乐趣:

假如想象一下蝙蝠用高频声音向猎物发出呼唤和口哨,也就不难理解回声定位了。我们的耳朵在最年轻、最敏感的阶段能够分辨每秒2万次的振动,但蝙蝠可以发出高达到每秒20万次的振动。蝙蝠并不是持续不断地发声,而是间歇性地发声——每秒20$\sim$30次。蝙蝠接收返回的声音,当返回的声音变得更快、更响,它便知道被追踪的昆虫飞近了。通过判断回声之间的时间差,蝙蝠可以分辨猎物运动的快慢以及方向。一些蝙蝠的敏感度足以察觉到在沙子上走动的甲壳虫,还有一些蝙蝠可以察觉蛾子停在叶子上扑扇翅膀。

这就是我所思考的敏感度问题,我请出的作家所给的两个例子再精彩不过了。但是对此,我的羡慕超过感激。我仍然在想:有关蝙蝠的敏感度问题,她还收集了多少其他例证才能选出这两个呢?几十个?几百个?要在开始尽可能多地手机材料,然后适当地呈现给读者。

当蝙蝠靠近猎物时,它会发出频率更高的声音,以便确定猎物的方位。而且,从砖墙上弹出的稳定、坚实的回音与从摇晃的花儿上弹回的轻盈、流动的回音之间有质的差别。通过向这个世界发出叫声并听其回声,蝙蝠就可以在脑中形成周围环境以及其中物体的图像,包括质地、密度、动作、距离、大小,也许还有其他可能的特征。多数蝙蝠都向自己的周遭环境发声,我们只是听不见而已。站在一片寂静的树丛里,四处都是蝙蝠,这想起来真是怪怪的。蝙蝠一生都在叫唤。它们向亲爱的叫唤、向对手叫唤、向晚餐叫唤、向忙忙碌碌的大千世界叫唤。这些蝙蝠的叫声,一些快,一些慢,一些响,一些柔。长耳蝙蝠无须叫唤,它们只要悄悄低语就能真切地听见自己的回声。

另一个使科学人人可知的办法是像普通人那样写作,而不是像科学家那样写作。这是一个老问题,即做你自己。你所应对的学术性科目,在一般情况下是以枯燥乏味、墨守成规的方式报道的,但这不能成为你不用出色、生动的语言写作的理由。洛伦·艾斯利是一位博物学家,他无惧于大自然的威胁,在《无边的旅途》中传达给我们的不但有他的学识,还有他的热情:

我一直是章鱼爱好者。这种头足动物很古老,它们变化多端,悄悄地经历过许多形状上的演化。它们是最聪明的软体动物,我总是感到我们真该庆幸这些章鱼从未上过岸,但是也有其他一些动物上了岸。

这倒没有必要恐惧。有一些生物的确怪异,但我却发现这个情况令人鼓舞,而不是相反。看到大自然仍在忙于试验,仍充满活力,并没有因为泥盆纪的鱼最终变成了有两条腿并且头戴草帽的家伙而完结或者满足,这给人一种充满信心的感觉。在海洋这个大染缸里还有其他东西在发酵和生长。知道这一点是值得的。很值得知道未来同过去一样有很多可以期待。唯一不值得知道的是确认人自己在其中的作用。

艾斯利的天才之处在于,他帮我们感觉到当一名科学家是什么样的。在他的写作中关键的事务是一位博物学家与大自然的恋爱,就像刘易斯·托马斯所写的细胞生物学家对于细胞的恋爱一样。“看着电视,”托马斯在他优雅的《细胞生命的礼赞》一书中写道,“你会想到我们生活于走投无路之中,完全处于危险之下,四周围绕着追人的细菌,只有通过不断杀灭细菌的化学技术,才能保护我们免于感染与死亡。我们幸运地把混合了除臭剂的喷雾剂喷进鼻子、口腔、腋窝、隐秘的犄角旮旯——甚至喷进电话听筒里。”但正当我们感到最紧张的时候,他说,“在庞大的微生物世界里,我们人类一直处于一个相对次要的位置。感染了脑膜炎球菌的生命危险,即使没有化学疗法,也比脑膜炎球菌不幸地抓到一个人的危险要小得多。”

刘易斯·托马斯科学地证明,科学家也能像其他人一样写好。要写好不一定非得当“作家”。我们想到了作为作家的雷切尔·卡森,她以《寂静的春天》一书发起了环保运动,但卡森却并不是作家,她是海洋生物学家,她就能写好。她能写好,因为她想得清楚,而且对自己的专业充满热忱。查尔斯·达尔文的《比格尔航行》不但是一部自然史经典,而且也是文学经典,书中的句子生动、有力地阔步前行。如果你是爱好科技的学生,不要想当然地认为英语系对“文学”具有垄断权。每一门科学都有自己的优秀文学。阅读在你感兴趣的领域写得好的科学家的作品——普里莫·利瓦伊的《元素周期表》、彼得·梅达沃的《柏拉图的理想国》、奥利佛·萨克斯的《错将妻子当帽子的男人》、斯蒂芬·杰伊·古尔德的《熊猫的拇指》、S.M.乌拉姆的《数学家历险记》、保罗·戴维斯的《上帝与新物理学》、弗里曼·戴森的《武器与希望》——并且把他们的作品作为自己的样板。模仿其线性风格,避免技术术语,不断将晦涩难懂的过程与读者可以想象出来的东西联系起来。

这里有一篇文章,名为《晶体管的未来》,刊登于《科学美国》,由罗伯特·W·凯斯撰写,他拥有物理学博士学位,也是半导体和信息处理系统专家。约有80\%拥有物理学博士学位的人不能写清楚皮氏培养皿是什么样,但这并不是因为他们不能,而是因为他们不愿意。他们不愿意屈尊去学语言这种简单的工具,而这种仪器却像物理实验室里的任何一台仪器那样精密。下面是凯斯的开头:

我现在写这篇文章所用的电脑有大约1000万个晶体管。作为一个人所能拥有的产品,这个数量相当惊人,而其价钱却低于硬盘、键盘、显示屏、机箱费用。比较来说,1000万个订书钉的价钱大约同一台整机一样。晶体管变得如此廉价,是因为在过去40年间,工程师们学会了在单张硅晶片上蚀刻更多的晶体管。已有的生产成本因此得以摊销在硅片上不断增多的晶体管上。

这一趋势能持续多久呢?学者与业界专家过去曾多次宣称,硅晶片存在某些物理极限,微型化也无法超出这一极限。但他们却多次被以下事实所困惑,即人们至今也不能确定硅片上蚀刻晶体管数量的极限。自晶体管发明46年以来,这一数量已经增长了8倍。

再来审视一下这种序列性文体。你会看见科学家用逻辑步骤引领着你,一句接着一句,走上他讲故事的路径。他自得其乐,因而享受写作。

我引用了许多作家是如何描述物质世界的方方面面的,来表明他们首先都是以人的面貌出现:这些人在他们自己与专业之间以及与读者之间找到了共通的人性线索。无论你的专业是什么,都可以获得同样的融洽关系。序列性写作的原则适用于每一个领域,在那里,读者必须被护卫着越过这些艰涩的新领域。想一想所有这些领域:生物、化学、政治、经济、伦理、宗教,它们纠结在一起;还有艾滋病、堕胎、石棉、毒品、基因剪接、老年医学、全球变暖、医疗保险、核能、污染、有毒垃圾、类固醇、克隆、代理母亲,以及几十种其他问题。只有通过专家们清晰的写作,我们作为普通公民的外行,才能在这些所知甚少或者一无所知的领域,做出合理的选择。

下面以一个概述本章所有内容的例子来结束这一章。在阅读晨报上报道1993年国家杂志奖的内容时,我看到在高度受重视的奖项中,险胜重量级杂志如《大西洋月刊》、《新闻周刊》、《纽约客》、《名利场》的,是一份我从未听说过的杂志,叫《I.E.E.E谱系》。结果我发现,这是一本电气与电子工程师研究会的旗舰杂志,这个专业协会拥有32万会员。据该杂志编辑唐纳德·克里斯琴森说,这份杂志曾经充斥着积分符号和首字母缩写,其中的文章甚至对其他工程师来说都深不可测。“电气与电子工程师研究会有37个界限分明的不同学科,”该编辑说,“如果你不能用词语做清晰的描述,我们协会内部的成员也不能彼此沟通。”

在让32万工程师都能看懂这份杂志的同时,克里斯琴森也使它能让普通读者看得懂。格伦·佐皮特撰写的获奖文章《伊拉克如何以逆向工程制造核弹》就证明了这一点。这是我所读过的调查报道类文章中最好的,它是服务于普及公共知识的非虚构写作。

文章用类似侦探小说的结构,描述了国际原子能机构是如何千方百计监察伊拉克人依照秘密计划几乎制造出了原子弹,并解释了为何说他们离完成已经如此接近。因此这篇文章既是科学史作品又是政治文件,而且至今仍是炙热文件,因为伊拉克人所进行的研究——并且预计将持续到萨达姆·侯赛因倒台——违反了国际原子能机构所公布的规定;核材料中许多是从不同的工业国家以非法途径得到的,包括美国。《I.E.E.E谱系》中的这篇文章突出解释了一种电磁同位素分离技术。这项技术被巴格达南部阿尔图维撒的研究中心所使用:

电磁同位素分离计划不但震惊了国际原子能机构,也震惊了西方情报机构。利用这种技术,铀同位素流通过电磁作用在真空容器中被转移方向。这个容器与相关设备被称为卡留管。稍重的铀238同位素比铀235同位素转移得慢些,就是这个细微的差别被用于分离出可裂变的铀235。然而,“理论上很有效的过程在实践中却是一件非常、非常麻烦的事,”莱斯利·索恩说。他最近刚从国际原子能机构行动组现场活动干事的职务上退休。事实上,一些铀238同位素始终混合于铀235,而且同位素流很难控制。

好,这已经很清楚了。但是这个过程为何如此麻烦?同位素流为何难以控制?这位作者会帮忙解释的。他决不会忘记上一段把读者留在哪儿了,也不会忘记他们下面都想知道什么。

这两种同位素材料聚集在杯状的石墨容器中,但是它们在两个容器中的聚集会由于给予电磁的能量和电磁的温度的微小变化,而发生巨大的离散。因此在实践中,该材料趋向于四处飞溅在容器的内壁上,这就要求在每十几个小时的操作后清洗容器。

这就是人们所谓的麻烦。但是这个过程的确可行吗?

为此人们需要成百上千的磁体以及上亿瓦特的功率。比如在曼哈顿计划中,田纳西橡岭的Y-12电磁同位素分离设施耗用的电能比加拿大全国还多,还要用上全美国的银储备,后者用于缠绕所需的许多电磁体(铜在别处用于战争目的)。主要是由于这些问题,美国科学家认为没有哪个国家会转向电磁同位素分离技术来生产相对大量的原子武器所需的浓缩材料……

发现伊拉克人的电磁同位素分离计划,带有像间谍小说一样许多精彩纷呈的喜剧性。第一条线索据说来自于被图维撒伊拉克军队所劫持的人质衣服上。人质被释放之后,他们的衣服由情报专家作了分析。专家们发现了极微小的核材料样品,而其中的同位素浓缩物只有在卡留管中才能产生……

“突然间我们发现了活恐龙,”国际原子能机构行动小组的副主任迪米特里厄斯·佩里克斯这样说。

甚至在如此的高科技中,作者也从未丢失人的因素。这不是一个有关“科学”的故事,而是有关人从事“科学”的故事——一伙偷偷摸摸制造炸弹的人和一队高科技警察。像发现恐龙这样的比喻是含金量十足的,它是一种我们大家都能明白的比喻,甚至连孩子都知道恐龙已不复存在。

依照精彩侦探小说推理的必然结果,这篇文章所做的结论就是整个调查的要点:发现伊拉克“并没有限制自己生产武器级原材料,而且同时还在拼命地围绕这种材料制造一种可运载武器,进行一种称之为武器化的威慑性任务”。首先我们被告知任何想尝试这一任务的人都有什么选择:

原子弹的两种基本类型是外射装置和内爆装置。后者设计和制造起来要难得多,但用等量的可裂变材料却可以提供更高的爆炸当量。国际原子能机构调查人员没有发现证据说明伊拉克在积极研制外射装置,他们说,伊拉克显然在等筹集资金和资源研发内爆装置,甚至已经开始了相当高级的内爆装置设计。

何为内爆装置?接着读 :

在内爆装置中,裂变物质被普通炸药所产生的冲击波力量物理压缩,然后到达既定的一个点,中子被释放,引发超速的裂变连环反应——原子爆炸。因此,内爆装置的主要部分是点火系统、炸药的组合体以及核心。点火系统包括以真空管为主的被称为氪的高能爆炸部件。这些部件能够释放足以引爆普通炸药的能量。炸药组合体部分所包括的“镜片”可精确地将球状的内爆冲击波聚焦在裂变核上,其核心是一个中子起爆器。国际原子能机构已经收集到足够的证据,证明伊拉克人已经在这些领域的各个部分取得进展。

这一段说到压缩,正是线性紧凑写作的精华,依次解释内爆装置及其三个主要部分。但我们现在想知道,国际原子能机构的证据是如何收集到的呢?

1990年3月,新闻爆出伊拉克试图从加利福尼亚圣马科斯的CSI科技公司进口氪。在经过美国和英国海关18个月的“圈套”行动之后,两名伊拉克人在伦敦希思罗机场被捕。而在几年前,伊拉克已经从美国其他公司搞到武器级电容器,而且制造出了自己的电容器……

我用这个例子得到以下证明,或者说我让《I.E.E.E谱系》杂志为我提供了以下证明:一个如此复杂的科学题材只用少数几个很快就能解释清楚(比如氪)或查到(比如裂变)的术语,就能用漂亮的语言写得如此清晰有力,那么任何题材也都可以被你们这些自认为害怕科学的作者,以及自认为害怕写作的科学家,写得清晰有力。