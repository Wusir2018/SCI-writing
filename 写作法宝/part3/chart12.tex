\chapter{写地点:游记}
了解了如何写人之后,就该知道如何写地点。人和地点是多数非虚构作品得以成立的两个支柱。每一件事总要发生在某处,读者想知道那个某处是什么样的。

在少数情况下,你只需要一段或两段文字来描写事件的背景。但更常见的是,你需要再现整个居民区或城镇的气氛,以便让你所讲述的故事富有质感。在某些情况下,比如游记本身——你以粗犷的笔调回忆自己如何乘小船穿越希腊诸岛,或者 在落基山背包旅行——描述性的细节会是主要内容。

对地点的描述不管占多大比重,都似乎相对容易。但令人沮丧的真相是,这其实很难,因为正是在这个领域,多数作家——包括专业和业余的——写出了不仅是最差的作品,而且简直就是糟糕透顶的作品。这种糟糕透顶的作品与作者性格上的某些严重缺陷并无关系。相反,它是作者的热情这一美德所造成的。没人能像刚旅行回家的人那样很快就招人烦。这种人特别享受自己的旅行,非得把一切都告诉大家不可——但我们并不想听这“一切”。我们只想听一部分,听是什么使他的旅行与众不同?听他能告诉我们有什么我们还不知道的?我们不需要他描述在迪斯尼乐园的每一趟乘车游玩,或者告诉我们大峡谷有多么壮观,或者威尼斯有运河。但假如有一次他在迪斯尼的乘车游玩遇到了阻碍,或假如有人坠人了壮观的大峡谷,那才是值得听的。

我们去了某地,感到自己是到达那里的第一人,为此我们浮想联翩,这些都是很自然的事,也很合理。这些感觉与想法促使我们不断四处旅行,验证自己的经验。有谁能参观伦敦塔而不想到亨利八世的妻子们,或者参观埃及而不被金字塔宏伟的规模和悠久的历史所感动?但那些地方其实早就被无数人光顾过。作为作者,你必须勒紧主观主义这条缰绳——你作为旅行者常常被新景色、声音、气味所感动——但为了读者,你必须保持观察的客观性。记录下你旅途中的一切会使你兴趣盎然,因为是你在旅行,但那一切也同样会使读者兴趣盎然吗?不会。仅仅汇集大量的细节来激发读者的兴趣是行不通的。每一个细节都必须有重要意义。

另一大陷阱是风格问题。在非虚构作品中,作者最容易滥用词语、无病呻吟。你会局促不安地在对话中使用大量形容词,如“神奇的”、“斑驳的”、“玫瑰色的”、“传奇式的”、“飞快的”。这些都是常见的流行语。在一天的观光中,有一半的景点都很“别致”,特别是风车房和廊桥;这些都被确认为具有“别致”性的特征。坐落在山区(或丘陵地带)的城镇被描绘为“依偎”在那里——我几乎没读到过在山区有哪个城镇不是“依偎”在那里——乡村被小路间隔得“星罗棋布”,更合意的描述是这些地方处于“半遗忘”状态。在欧洲,你一醒来就听见马车咯噔咯噔地沿着“饱经历史沧桑”的河边行驶;你似乎听见羽毛笔的嚓嚓声。这是一个新旧相遇的世界——旧的与旧的决不相遇。在这样的世界里,毫无生气之物焕发生机:店面微笑、楼房夸耀、废墟示意,就连烟囱顶都唱起古老的迎宾歌。

游记体裁还是一种富有弹性风格的写作,其中的词语严格审视起来并无微言大义,或者说其含义因人而定,如“吸引人”、“迷人”、“浪漫”。写出“这座城市自有其吸引力”并说明不了什么。除了女子仪态学校校长,又有谁会来界定“迷人”呢?还有“浪漫”又如何界定?这些都是观者眼中的主观概念。日出对一个人可能很浪漫,而对另一个人则可能很落寞。

那么你该如何克服如此惊人的差异而写好地点呢?我的建议可以缩减为两个原则——一个是风格原则,另一个是内容原则。

首先,在风格方面,要精心选词。如果一个词语很容易到手,要特别小心审视,它可能就是无数的陈词滥调之一。这些词语被紧密地编入游记写作之中,你需要花特别的力气才能避免。另外还要抵制滥用光彩夺目的抒情词语来描绘神奇的瀑布。这么做最好的效果也只会使你听起来做作,不像你自己,而最坏的效果则是自以为是。因此要尽力找鲜活的词语和意象。将文绉绉的词语留给诗人吧,将此类词语留给愿意要的人吧。

至于内容,也要极为审慎地选择。假如你描写海边,不要写“海岸岩石遍布”,或“时而海鸥飞过”。海岸本来就常常是岩石遍布,而且上方常有海鸥飞过。去掉每一个此类家喻户晓的事实:不要描写大海有波浪,沙滩是白色的。选择的细节要具有重要性。这些细节也许对你的叙述很重要,也许与众不同,或丰富多彩,或滑稽可笑,或赏心悦目,但一定要保证这些细节有效用。

让我来举几个不同作家的例子,他们的个性完全不同,但所选的细节却具有很相似的力量。第一个例子是琼·迪迪翁的一篇文章,题目是《金色梦想的梦想家》。文章报道了一起发生在加利福尼亚州圣贝纳迪诺谷耸人听闻的案件。在该文前面一段,作者就好像驾车带我们驶离都市文明,来到一段人迹罕至的公路。就在那里,露西尔·米勒的大众牌汽车不明原因地起了火:

这就是祈祷电话\footnote{一种宗教组织的热线电话,以吸引教徒参加宗教活动。}好打书难买的加州。在这里,人们梳着蓬松隆起的头发,穿着卡普里裤,女孩们一生的愿望就是能穿上中长款的白色婚纱;与此同时,这里的离婚事件也层出不穷,有金伯利、谢丽、黛比、提华纳等离婚案,许多人又回归美发学校重新学着打扮自己。“我们当时就是疯疯癫癫的孩子,”他们毫不后悔地这么说,一心展望未来。未来在这个金色国度总是一片灿烂,因为没人记得过去。在这里,热风吹拂,陈旧的生活方式不再相干,离婚率是全国平均离婚率的两倍,每38个人中就有一个生活在拖车房里。在这里,人们从 四面八方来到了自己的终点站,有的漂泊自寒冷的北方、遥远的过去,远离自己陈旧的生活方式。在这里,人们积极地寻找新的生活方式,在他们唯一所知的电影和报纸中寻觅。露西尔·玛丽·麦克斯韦·米勒就是这种新生活方式所浓缩成的纪念碑。

先想象一下榕树街,因为那里正是事发地点。去榕树街的路从圣贝纳迪诺沿着66号路的山丘大道一段向西行驶:经过圣菲道岔区和小睡汽车旅馆;再路过一个叫灰泥圆锥帐篷19号的汽车旅馆,“睡印第安棚屋——让钱花得超值”;路过丰塔钠赛车城和拿撒勒派丰塔钠教堂,还有歌舞厅休息点;路过凯泽钢厂,穿过库卡芒加,出去就到了位于66号路和玛瑙大道交汇处的卡普凯餐馆——那里有酒吧和咖啡馆。从意思是“禁海”的卡普凯,沿玛瑙大道往北,可以看见小区的彩旗在疾风中猛 烈飘动。“半英亩牧场!快餐吧!石灰华大厅!95美元搞定。”这是一条令人心狂神迷之路,是新加利福尼亚残渣余孽的处所。但再过一段路,各式招牌在玛瑙大道渐渐稀疏,那里的房屋也不再有春季房地产业主房屋的亮丽色彩,而是色彩斑驳的低矮平房;那里的住户种几架葡萄,养几只鸡。之后,山丘变陡,路开始爬坡,平房也很少见,四周一片荒寂,路面凹凸不平,两边长满桉树和柠檬树——这里就是榕树街。

仅两段描述,我们不但感觉到印第安灰泥圆锥帐篷、临时住房、借来的夏威夷式浪漫,而且还感觉到在此地落脚的人们矫饰的生活,以及其可悲可叹的不稳定性。所有细节——数字、名称、招牌——都各行其职。

描述具体细节也是约翰·麦克菲散文的支柱。他写的《进入乡间》这部有关阿拉斯加的书——从他众多别具匠心的书中只选一部——其中有一部分专门描写为州政府选择新地点的可能性。麦克菲只用几句就使我们明白了现州府所在地的问题。无论是作为居住地还是作为立法者制定合理法规的处所,这个地方都是个问题:

在现今的朱诺,行人尽管低头、奋力向前,还是会受风的阻碍而寸步难行。沿街有栏杆,参众议员们可以借此把着上班。在过去的几年里,风速表接二连三地被安装在城市上方的山梁上,测出的风速可达每小时200英里。这些装置都没能逃过一劫。塔古风将这些风速表的指针吹到了刻度的尽头,随后将其摧散。这儿的天气倒不总是那么糟糕,但朱诺市基本上是在这样严酷的气候条件下形成的,因而这里的街区很紧凑,建筑互相毗邻,街道呈狭窄的欧式风格,依着山坡、面对咸水湾而建……

哈里斯就是在那两年里(在阿拉斯加州参议院期间)忽然有了搬迁州府的动议。研究这一动议始于1月份,持续了至少三个月,哈里斯渐渐得到一种他称为“完全被隔绝的感 觉。你陷入困境,人们接触不到你,你在笼子里,你每天都要向难对付的游说者辩解,而且每天都是同一伙人。所发生的一切需要更多的宣传解释工作”。

这座城市如此远离美国普通民众的生活体验,其特性显而易见。对立法者来说,一个可能性是将州府搬到安克雷奇。在那儿人们至少不会感到自己是在一个封闭的城市里。麦克菲在这一段描述中提取了最本质的东西,无论在细节还是比喻上都恰到好处:

几乎所有美国人都知道安克雷奇,因为安克雷奇是城市涨出其边界的那部分,曾驱逐过桑德斯上尉。先建设,后文明。安克雷奇时不时以开拓之城的名义得到大家的谅解。但安克雷奇并不是边塞之城。它与其环境毫不相干。它随风而来,是一粒美国的种子。安克雷奇整齐划一,就像一架大饼干成形切割刀从埃尔帕索这个地方切出来的一样。安克雷奇位于代顿纳海滩背海处的奥克斯纳德中心特伦顿的北部边缘。它很紧凑,是速成的阿尔布开克市。

麦克菲所做的是捕捉朱诺市和安克雷奇市的意义所在。作为游记作者,你的主要任务是找到你所描写之地的中心意义。几十年来,无数作者试图把握密西西比河,捕捉这条贯穿美国神圣中部的雄伟通道的精髓,而且他们在描述中常常带有《圣经》般愤慨的语气。但是没有人能像乔纳森·拉邦那样对这一流域作出如此简明的描述。他的创作灵感来自于重访受洪灾的中西部各州后的体验。以下是文章的开头:

从西部飞往明尼阿波利斯,你会把这看成是一件具有神学内涵的事。明尼苏达州大片平坦的农庄散落于规整的网格之中,空阔得就像方格纸那样令人惊讶。每一条沙砾铺的路,每一条沟渠,都是沿着城镇测量系统中的经纬线设计的。农庄是方的,田地是方的,房屋是方的;假如你把人们头顶上方的屋顶掀掉,会看见各家各户坐在方形房间正中央的方形桌子旁。大自然已被剥夺、修剪、钻探、惩罚、压制在这个直角形、直向思维的路德教区域。这里让你渴望看见桀骜不驯的曲线或者不规则的景象,看到田野里斑驳的色彩,在那里农人不经意间放任玉米和大豆混种。

但在这趟飞行途中并没有不经意的农人。下面的地貌一览无余,仿佛等待上帝检查,就像一幅巨大的广告,显示出人们严守的操行。它说明,下面没有任何嬉戏、不正常的情况;我们都是清清白白正直的百姓,是去天堂的合适人选。

然后大河走进视野,一条宽阔的蛇形影子顺从地在棋盘上爬行。密西西比河弯弯曲曲地流淌,一片片黑色沼泽,一座座绿色雪茄形状的岛屿,令人迷惑不解。这一切看起来就好像被有意置于此地,目的是教导中西部敬畏上帝的人们,告诫他们大自然固执不化的本性。大自然就像约翰·加尔文的坏脾气一样,它以野兽的姿态呈现在大陆中央。

在河上生活的人们赋予密西西比河不同的性别。他们对此绝非随心所欲,而几乎总是赋予自己的性别。“你最好尊重这条
河,不然他会吃了你,”闸门工大声说。“她可怕着呢,她在这一带就吞噬了好多人,”午餐台边的女服务员这么说。诗人艾略特将这条河比喻为我们的内心世界(相对于海在我们周围),这是在向我们揭示日常生活中有关密西西比河的一个事实。看见浑浊翻腾的河水,人们的确会想到这体现了自己内心的矛盾冲突。人们向旁人吹嘘这条大河的恣意汪洋,它巨大的滋事与破坏力,还有它所造成的洪灾与溺人事件。此时,他们的嗓音里有一种弦外音:“我内心也有那么做的冲动……我知道那种感觉。”

身为非虚构作者,有谁比生活在美国更幸运呢?这个国家在不断地变换花样、创造奇迹。无论你所写的地方在城市还是乡村、东部还是西部,每一个地方都有与其他地方不同的面貌、人口,以及一套文化习俗。要发现这些与众不同的特点。下面三段描述了美国几个很难区分的地方。然而,在每一个例子中,作者都给出了许多细节,使我们感到身临其境。第一个片段选自杰克·埃格洛斯的《去见迪克与简的途中:波多黎各人的心路历程》,描写的是作者童年在纽约拉美社区的经历。在那里不同族裔的“领地”可以共存于同一个街区:

每一个教室里有十个孩子不会说英语。黑人、意大利人、波多黎各人在教室里关系良好,但大家都知道彼此不能去其他人的区域。有时我们在自己的区域也不能太随意活动。在109街,从路灯往西是拉丁王牌区,从路灯往东是塞纳卡人区,我就属于这个“俱乐部”。不会说英语的孩子号称为海老虎,这个称号出自一首西班牙流行歌曲。“海老虎”与“海鲨鱼”是两艘船,从圣胡安驶抵纽约,从岛上运过来很多很多移民。

居民区有界限。第三大道以东为意大利区。第五大道以西为黑人区。南面,在103街有一座小山,当地人称之为库尼山。当你登上山顶,奇怪的事就发生了:美国从此开始,因为从山的南部开始是“美国人”生活的区域。迪克和简没有死,他们还好好地活着,在更好的街区里活着。

每当我们一群波多黎各孩子决定去杰弗逊公园泳池游泳,我们就知道会冒险干仗,被意大利人揍。每当我们去哈莱姆的拉米拉格罗萨教堂,我们知道也会冒险干仗,被黑人揍。而当我们翻过库尼山,我们会冒险看别人的白眼,鄙视的目光,以及来自警察的审问,比如“你们在这个街区干吗?”或者“你们这些孩子怎么不回自己的地儿呆着去?”。

什么自己的地儿?哥们儿,我写过关于美国的作文。尽管我不会打网球,难道我就无权享用中央公园网球场?我就不能看迪克打球吗?难道这些警察不为我服务吗?

让我们从那里再到东得克萨斯的一座小城,也就是阿肯色州界对面的一座城镇。普鲁登斯·马金托什描写该城的一篇文章刊登在《得克萨斯月刊》上。我喜欢这本杂志,因为该文的作者和其他得克萨斯的作家在其中以活灵活现的风格,带我这么一个曼哈顿中城的居民遍访了该州的每一个角落。

我逐渐意识到,在成长的过程中我一直以为是得克萨斯特点的,其实属于整个南方。人们所珍视的得克萨斯神话其实与我所在州的那个部分没有多大关系。我认识多花狗木、无患子、紫薇、含羞草,但这里没有得克萨斯羽扇豆或者红扁萼花。虽然四州博览会和牧人竞技会在我所在的城里举办,我却从未学会骑马。我从未认识过什么真正戴牛仔帽穿牛仔靴的人,即使认识这样穿戴的人,他们也只是穿着玩而已。我只认识一些农场主,他们的农场被称为某某老人农场之类的,而不是大 门上立有牛标记的牧场。城里的街道被命名为木、松、橄榄以及大道,但没有瓜达卢佩和拉瓦卡之类的名字。

再往西走是加利福尼亚莫哈维沙漠的穆拉克机场,那里是美国极为艰苦荒芜的地方,但正适合美国空军使用。上一代空军就是在这里开始实验如何突破声障的。汤姆·沃尔夫在《太空英雄》中的前几章对此有精彩描述。

这里看起来就像是化石地貌,这是由于长久以来的地质变迁所遗留下来的。到处都是巨大的干涸湖床,最大的是罗杰斯湖。除了三齿蒿之外,这里唯一的植被是短叶丝兰,形成一派弯曲成畸形的植物景象,看起来像仙人掌和日本盆景植物的杂交。这些植物有石化深绿的颜色和变形得厉害的枝干。傍晚,短叶丝兰的轮廓显露在石化的荒原上,就如同陈旧的梦魇。夏天,这里的温度一般可达华氏110度,干涸的湖床布满沙砾,掀起的暴风和沙尘暴就好像直接来自电影中的异域军团。晚上,温度会降至接近零度,12月份开始下雨,干涸的湖泊会满上几英寸深的水,某种腐臭的史前虾会从沼泽中向上爬,海鸥会从一百多英里外的海洋越过山脉飞过来,吞吃这些蠕动的小返祖虾。人们得亲眼看见才会相信……

风将几英寸深的水横着湖床吹来吹去,湖面变得绝对平滑。春天湖水蒸发,太阳烘烤大地,河床就成为自然界最棒的降落场地,而且也是最大的,可以允许降落误差达几英里。这对于在穆拉克机场从事的这项航空事业来讲是求之不得的环境。

在穆拉克,除了风、沙、风滚草、短叶丝兰之外,只有两个并排的匡西特活动房屋样式的飞机棚,几个汽油加油泵,单一的混凝土跑道,几个油毡纸棚屋,还有几顶帐篷,此外别无他物。军官们呆在标为“营房”的棚屋里,级别低一点儿的军人呆在帐篷里,整晚挨冻,整天挨烤。每一条通往营区的路都有一个由士兵把守的岗哨。就在这样一个被上帝遗弃之地,军队所从事的是研发超音速飞机和火箭推进式飞机。

我称这种写作形式为游记写作,但要练习这种写作并不意味着你非得去摩洛哥或蒙巴萨才行。去当地的购物中心,或者保龄球馆,或者日托中心。但无论你写什么地方,要去够次数,区分出那里别具一格的特色。一般来讲,那种特色就是那个地方和住在那里的人的某种结合。如果你去的是本地保龄球馆,其特色就是那里的气氛和常客的一种混合。如果你去的是外国城市,其特色就是那里古老文化和现住居民的一种混合。要尽力去发现。

英国作家V.S.普里切特是这种发现技艺的大师,他是最优秀、最多才多艺的非虚构作家之一。下面看看他从伊斯坦布尔的造访中能挤出点儿什么:

伊斯坦布尔所激发的想象力震撼了大多数游客。苏丹王的形象在我们头脑中挥之不去。我们或多或少期望看见他们仍浑身宝气地盘腿坐在卧榻之上。我们记得圣城的故事。事实上,伊斯坦布尔除了其氛围外,并没有什么辉煌。这座城市建在 山丘上,有陡坡,有鹅卵石街道,人群熙熙攘攘……

商店大多卖布、衣服、袜子、鞋子,希腊商人跑出店外,摊开布匹向过客兜售,土耳其人则被动地等客上门。搬运工喊叫,人人都喊叫,你被马头撞上,被驮着的床上用品撞到一边,而穿过这一切,你又看见一种在土耳其的奇妙景象——一个神情严肃的年轻人拎着一个由三条链子吊着的铜盘,在盘子正中心有一小杯红茶。他没洒一滴,行动自如地穿过混乱的人群,将茶送给坐在店前台阶上的老板。

人们发现在土耳其有两种人:一种是拎着的,另一种是坐着的。论坐得悠闲、精到、美妙,无人比得过土耳其人,他身体的每一寸都坐着,就连他的脸都坐着。他坐在那里,就好像经历了几代宫廷峰的宫殿中苏丹王所享有的坐的艺术。他再喜欢不过的是邀请你在他的店里或办公室一起坐坐,同时还有十几位其他坐客:他客气地问你几个问题,有关你的年龄、婚姻、孩子的性别、亲戚多少、在哪儿住、如何生活,然后像其他坐客一样,你清一清嗓子,说话音量之大超过在里斯本、纽约或谢菲尔德所听到的任何声音,随后同大家一起静默。

我喜欢“就连他的脸都坐着”这样的词语——短短几个词,所表达的意思却极具想象力,令人惊叹,同时很好地说明了土耳其人的特性。我要是再去土耳其,绝对不会注意不到那里的坐客。仅用一种迅捷的洞察力,普里切特就抓住了整个民族的特性。这是描写好其他国家的基本写作要素。要从非物质性中提炼出重要的特性。

英语(正像普里切特所提醒我的那样)历来擅长游记这种独特的文体形式——这种文章更着重于某个地点从作家那里提取 了什么,而不是作家从那个地点提取了什么。新景观能够激发思想,不然作者在头脑中可能想不到这些。如果游历在扩大,它应该不仅扩大我们对哥特式教堂的观瞻或者法国人酿酒方法的了解。它应该激发群星璀璨般大量的思想,有关男男女女如何工作、嬉戏、养儿育女、敬神拜神、生生死死。在阿拉伯半岛有一批痴迷沙漠的英国学者探险家,如T.E.劳伦斯、弗雷亚·斯塔克、威尔弗里德·塞西杰。他们选择生活在贝都因人的部落中,在其所撰写的书籍中,大部分的神奇力量来自于他们在如此严酷苛刻的环境中对于生存的反思。

因此,当你写地点时,要尽量地尽其材。但假如把创作过程倒过来,就让地点让你人尽其才吧。虽然梭罗只出城一英里,却写了美国人所能写出的最丰富的一本游记《瓦尔登湖》。

最后,无论如何,给地点带来生机的是人的活动:人的活动赋予地点性格。四十年后,我依然记得詹姆斯·鲍德温在《下一次是火》一书中动感十足的描述。他在书中描写了自己作为一所哈莱姆教堂儿童传道员的情景。我至今依然有礼拜日早晨身处教堂里的感觉,因为鲍德温在自己的作品中超越了简单的描述,而进入由声音、节奏、共同信仰、共同情感相结合的更高的文学领域:

教堂里一派激动人心的场面。过了好长时间我才从这激动中回过神来,但在最直觉、最内心的层面,我却从未真正回过神来,也将永远不会。从没有如此的音乐,如此的圣剧,演绎出圣人欢庆、罪人呻吟、铃鼓疾敲,所有人的声音汇合在一起,向神圣的主高唱赞歌。我从未体验过如此的情怀,它发自这些多肤色的、憔悴的、带有某种胜利和容光焕发的面容;大家发自内心深处的声音来自对主的仁慈的极端渴望,可触可见,永不间断。我从未见过如此火一般的热情和激动毫无征兆地充溢整个教堂,使教堂就像莱德贝利\footnote{莱德贝利(Leadbelly,即Huddie William Ledbetter,1888-1949):美
国黑人民谣及布鲁斯传奇音乐家,以强有力的嗓音与粗犷的吉他弹奏闻名,为民谣确立了标准。}和很多其他人所证实的那样“震颤”。从那以后,我再也没有感受过如此巨大的力量和荣耀。有时候在布道中间,我忽然感觉到自己已经真的奇迹般地高高举起人们所说的“主的旨意”——这时教堂和我融为一体。他们的痛苦和欢乐就是我的,我的也是他们的——他们大喊“阿门!”“哈利路亚!”“是啊,主!”“赞美他的名字!”“传道吧,兄弟!”这些声音持续不断,拍击着唱独角戏的我,直到我们都平等如一,在圣坛下痛苦并快乐地扭动身体,唱歌跳舞,大汗淋漓。

决不要怕写一个自己认为已经被别人写尽的地方。只要你写了,那就是你的地方。我就给自己定了这么一项挑战,决定写一本书,题目是《美国的景点》,其中有15个最受欢迎的热门景区已经成为美国标志性景点,或者说强有力地代表了美国人的理想与精神。

我要写的景点中有九个是超级偶像:拉什莫尔山、尼亚加拉瀑布、阿拉莫、黄石公园、珍珠港、弗农山庄、康科德与莱克星顿、迪斯尼乐园、洛克菲勒中心。还有五个地方体现了对美国的独特看法:马克·吐温童年的故乡密苏里州的汉尼拔,他借此创造了密西西比河和理想童年的双重神话;内战结束之地阿波马托克斯;莱特兄弟发明飞机之地基蒂霍克,它象征着美国人发明创造的天赋;德怀特·D·艾森豪威尔的大草原故乡堪萨斯的阿比林,它象征着美国小城镇的价值观;纽约州边远村庄肖托夸,美国的自我完善与成人教育理念大多孵化于此。我所写的圣地中只有一处是新的:玛雅·林民权纪念地,在阿拉巴马州的蒙哥马利,纪念在南方民权运动中被杀害的人们。除了洛克菲勒中心,我之前从未参观过以上其他的地方,对那里的历史也一无所知。

我的方法不是问在拉什莫尔山景点抬头凝视的游客“你感觉到什么?”我知道他们会说什么,比如一些主观的印象(如“简直不可思议!”),因此这些信息对我没用。相反,我到这些地方去问那里的管理员:你觉得为什么每年有两百万人到拉什莫尔山来?或者三百万人到阿拉莫来?或者一百万人到康科德桥来?或者几十万人到汉尼拔来?所有这些人都在追寻什么?我的目的是进入每一个地方,了解其自身的意图,去发现每一个地方自身的努力目标,而不是我可能希望或者想要这个地方怎么样。


通过采访当地的男男女女——公园管理员、园长、图书管理员、商人、老居民、得克萨斯共和国姊妹协会、弗农山庄女士协会的女士们——我进入了最丰富的矿藏之一,这里等待着作者去发掘、寻找美国:在当地工作的人习以为常的健谈满足了其他人的需求。 下面就是三个地方的管理员告诉我的:

拉什莫尔山:“下午,当阳光将阴影投入那个凹陷处,”一位叫弗雷德·班克斯的管理员说,“你会感到那四位伟人的眼睛直勾勾地看着你,不管你挪到哪里都是如此。伟人们一直看入你的心里,想知道你在想什么,使你感到内疚:‘你尽职了吗?’”

基蒂霍克:“来到基蒂霍克的人中有一半与航空有某种联系,他们在寻找事物的根,”主管安·奇尔德雷斯说。“我们得定期替换某些威尔伯·莱特和奥维尔·莱特的照片,因为他们的脸都被磨掉了——参观者想要触摸他们。莱特兄弟是普通人,受教育程度只有高中水平,但他们在很短的时间里,用很少的资金,做出了非同寻常的事。他们大获成功,我觉得他们改变了我们的生活方式,‘我会这么受鼓舞并这么勤奋地工作来创造这么伟大的东西吗?’”

黄石公园:“参观国家公园是一项美国家庭传统,”管理员乔治·鲁宾逊说,“人人皆知的公园就是黄石了。但还有一个藏而不露的原因。我认为人们有一种天生的需求,要与他们所进化而来之地重新建立某种联系。我在这里注意到一种最紧密的联系,那是最年轻的人和最年长的人之间的联系。他们都更靠近自己的原始地。”

这本书中带有强烈感情色彩的内容,都是由我让别人述说提供的。我无须增强其感情度或爱国度。要小心强度的分寸。假如你在写神圣之地或意义重大之地,那就把强度留给别人。我到达珍珠港之后不久明白了一件事。日本于1941年12月7日炸沉亚利桑那战舰之后,人们继续让这艘战舰每天漏一加仑油。在我后来采访那儿的主管唐纳德·马吉时,他回顾说,接这份工作时,他扭转了一项禁止45英寸高的孩子参观亚利桑那纪念舰的官僚规定。那项规定认为,小孩们的行为会“负面地影响”其他游客的感受。

“我不认为孩子太小就不能理解那艘船象征着什么,”马吉告诉我,“如果他们看见漏油——如果他们看见那艘船还在流血 ,他们是会记住它的。”