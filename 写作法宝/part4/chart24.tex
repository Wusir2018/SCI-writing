\chapter{尽力写好}
我有时会被问到,能否回忆起有那么一刻意识到自己要当作家。其实那种炫目的时刻并没有出现过,我只知道我想过要为报社工作。但是我可以指出自己早年所继承下来的一系列人生态度,它们从我父母双方的家族通过不同的途径继承到我这里,一直引导着我。

我母亲喜爱好作品,她在报纸和书籍中经常能找到这样的作品。她常从报纸上剪下专栏和文章,其中的优雅用语或智慧、或是对生活的独到见解都使她愉悦。由于母亲的影响,我很小就知道,好的写作可以出现在任何地方,甚至在名不见经传的报纸中,重要的是写作本身,而不是其载体。因此,我总是按照自己的标准尽力写好,而从未改变自己的风格去迎合读者,不管他们数量多少或者受教育程度有多高。我母亲还是一位具有幽默感和乐观精神的女性。这些都是写作中的润滑剂,就像在生活中一样,有幸拥有这些的作者能以更多的自信开始新的一天。

原本我不是非得要当作家。我父亲是商人。他的祖父于1848年大移民期间从德国来到美国,带来了制造虫胶清漆的配方。他在曼哈顿北边上城崎岖不平的空地上建起一座小房子当做工厂——现在是第十大道第59街——开始建立商行,称作威廉·津瑟公司。我仍保存着那个地方充满田园风光的照片:坡地向下通往哈德逊河,其中唯一的生物是一头山羊。公司在那里呆到1973年,后来搬到了新泽西。

对一种生意来说,一直由同一个家庭经营并在曼哈顿同一个街区延续一个多世纪,这实属罕见。我作为男孩,摆脱不了要继承家业的唠叨,因为我在家中是第四个孩子,且是唯一的男孩;父亲命中注定先有三个女儿。那是“黑暗年代”,让女儿像儿子一样做生意或比他们做得更好,这种想法二十年之后才能被接受。我父亲是热爱自己生意的人。当他谈论自己生意的时候,我从未感到他将其看做是赚钱的买卖,而是一种艺术,需要用想象力和最好的材料来完成。他对质量抱有极大的热情,不能容忍二等品。他从未进商店找过便宜货。他为自己的产品要价高,因为他用最好的材料制造产品,公司因而生意兴隆。未来都为我准备好了,父亲期盼着有一天我会同他一起干。

但不可避免的,他等到的是与期盼不同的一天。战后返家不久,我去了《纽约先驱论坛报》,不得不告诉父亲我不能继承家业了。他以自己惯有的大度接受了这个消息,并祝我在自己选择的领域一切顺利。我解放了,无须去完成别人的期待,而那些期待对我并不合适。不论成功与否,我都可以自由行事。

只是到后来我才意识到,在自己的旅途中,我带上了父亲赐予的另一种禀赋:一种刻骨的信念,即质量是最上乘的奖赏。我也从未进过任何商店找便宜货。虽然在家里母亲才是文艺人士——像爱收集东西的喜鹊一样爱好收藏书籍,热爱英语语言,善写书信——但是从商业领域,我吸收到了匠人的职业道德。随着年岁的增长,我发现自己不停地修改我已经修改过的作品,决心要比所有为此版面竞争的对手都写得好,而此时,我内心听见的却是父亲谈论虫胶清漆的声音。

除了尽可能写好之外,我还想尽可能写得有意思。我曾告诉有志向的作者,他们应该把自己部分地看做是艺人,但他们不愿意听——这个词儿带有的是狂欢、杂耍、小丑的意味。但要想成功,你就必须使自己的作品比别人的更有趣,从而让自己的作品在报纸或杂志中脱颖而出。你必须找到某些方法来将自己的写作行为提升到一种娱乐上。一般来讲,这意味着给读者一种惊喜。很多方法都能完成这项任务:幽默、轶事、悖论、意料之外的引语、有利的事实、奇异的细节、迂回的路数、典雅的措辞。这些似乎是娱乐的成分,但却构成你自己的“风格”。当我们说自己喜欢某些作的风格时,其实是指自己喜欢他们付诸笔端的个性。作家就是邀请我们与其同行之人,如果要在两个旅伴中选一个,我们一般会选那些努力使旅途开心的人。

与医学或其他科学不同,在写作领域,人们并不会突然宣布有什么新的发现。我们不必怕在晨报上读到,在如何写好一句清晰的句子方面有了突破——这样的信息自从钦定版《圣经》时代就有了。我们知道动词比名词更有力,主动动词比被动动词更好,短词语和短句子比长的更好读,具体细节比模糊抽象的内容更容易理解。

显然,这些规则也不总是不折不扣的。维多利亚时代的作家有乐于装饰的品位,他们不认为简洁是一种美德,而许多现代作家,比如汤姆·沃尔夫,就突破樊笼,将一贯繁复的语言转变为正能量的来源。这样技术高超的杂技家却不多见,不过多数非虚构作家在坚持简洁清楚的准则方面能做得很好。我们也许可以运用电脑这样的新技术来减轻写作负担,但总的来讲,我们知道自己需要什么。我们都在用同样的词汇和同样的原则写作。

那么,写好的界线在哪里呢?花大力气掌握本书所讨论的工具,就能得到百分之九十的答案。可以增加几点天赋的东西,比如乐感好的耳朵、节奏感、对词语的感觉,但最终的优势同样适用于所有其他竞争性行业。如果你想要比别人写得好,就必须首先在主观上想比别人写得好。你必须对自己技巧上最微小的细节都拥有无比的自豪感。而且你必须诚心诚意地捍卫自己所写的作品,抗击各类中间人——编辑、代理人、出版商——其眼光可能与你不同,而其标准可能也没有你的高。有太多的作者被威逼而苟安于现状,埋没了自己的才华。

我总是感到,自己的“风格”——把自己对自己的认识审慎地表现在纸上——就是我的主要市场财富,这一资本可以把我同其他作家区别开来。因此我从不让任何人来修补我的文章,一旦提交了文章,我就会竭力保护它。有好几家杂志编辑曾告诉过我,在他们所认识的作者中,我是唯一在得到稿酬以后还在乎自己文章情况的人。多数作者不会与编辑计较,因为他们不想惹恼编辑,他们对自己的作品能出版已经感激不尽,同意自己的风格,也就是自己的个性,被当众侵犯。

而捍卫自己所写的,则是你仍活着的标志。我在这个问题上是一个顽固派,我会为每一个分号而战。但是编辑们可以容忍我,因为他们看出来我是认真的。事实上,比起被我吓跑的撰稿机会,我的顽固反倒给我带来更多的机会。编辑们一旦有异乎寻常的写作任务就经常会想到我,因为他们知道我会以异乎寻常的审慎来完成。他们还知道能按时收到文稿,而且文稿会准确无误。记住,非虚构写作这一行所包括的不只是写。它还意味着可靠。编辑们会正当地放弃靠不住的作者。

这就使话题落到了编辑上。编辑是朋友还是敌人——是解救我们摆脱罪恶的诸神,还是践踏我们诗意灵魂的游手好闲之徒?就像世间其他造物一样,他们也都各不相同。我一想起那十几位编辑,心里就充满感激,他们通过以下诸多方式使我的写作更加锐利:改变焦点和重心,或者质询语气,或者发现逻辑或结构上的弱点,或者建议一种不同的开头,或者当我在几条路径面前难以抉择时让我向他们述说,或者删减各种各样多余的内容。有两次我曾删掉书中的一整章,因为编辑告诉我那一章没有必要留着。但最重要的是,我记得那些好编辑的慷慨大度。他们对于作者与编辑共同努力完成的项目充满热忱。他们对于我能够做好的信心让我不断前行。

一个好编辑带给作品的是作者已失去许久的客观眼光,编辑改进文稿的方法也是无止境的:修剪,构形,使文字清晰,调整几百处时态、代词、地点与语气的不统一之处,注意所有可能引起歧义的句子,将拖沓的长句分成短句,在作者误入歧途时将其引回到主路上,在作者因忽视连接词而丢失读者的地方牵线搭桥,对判断与品位问题发出质疑。编辑之手还必须是隐形的。无论他用自己的词语增加了什么,都不应该听起来像编辑词语,而是像作者自己的词语。

对于所有这些挽救行为,怎样热切地感激编辑都不为过。不幸的是,编辑也可能造成很大的伤害。通常的破坏形式有两种:改变风格和改变内容。先看看风格问题。

好编辑最喜欢的就是他几乎无须触碰的作品。差编辑老是想修补别人的作品,以忙碌来证明自己没有忘记语法和用词的细枝末节。他是个咬文嚼字的家伙,在路上到处寻找缝隙,而不是欣赏景致。他常常压根儿就不会想到,作者是在用耳朵写作,努力想要达到一种声音或节奏效果,或者只是在玩文字游戏。作者最惨淡的时刻之一,便是意识到编辑不明白自己努力要做之事的意义所在之时。

我记得许多此类令人沮丧的大暴露。有一件较轻的事件,涉及我写的一篇和“访问艺术家”项目有关的文章。这个项目把一些画家和音乐家带到一组经济萧条的中西部城市。我是这样描写的:“这些地方不像是被这许多访问艺术家所访问的城市。”\footnote{“They don’t look like cities that get visited by many visiting artists.”这里visited和visiting起到部分重复的特殊效果,有点儿像中国的绕口令。}当校样回来,这句话变成:“这些地方不像是在这许多访问艺术家日程之中的城市。”\footnote{“They don’t look like cities that are on the itinerary of many visiting artists.”编辑修改后,该句就变得平淡无奇了。}一点小变化?对我则不然。我用了重复,因为这是我喜欢的方法,它能带给读者惊喜,让他们一句话读到中间时有新鲜感。但编辑记得的却是用同义词来代替重复词这一规则,他是在改正我的错误。我打电话表示抗议,他却感到很吃惊。我们争执了很久,谁也不肯让步。最后他说,“你真的很在乎这个吗?”我的确很在乎,因为每一次这样的侵害都会引发另一次,作者必须立场鲜明。我甚至从杂志社买回自己的文章,因为不能接受他们对我文章的修改。假如你允许自己的特色被编辑掉,你就会丧失自己一项主要的优点。你也会丧失贞操。

作者与编辑的理想关系应该是一种商谈与信任的关系。编辑常常会做修改来澄清一句含糊的句子,结果无意中丢失了要点——一个事实或一个细微差别,作者有理由将其包括在内,但编辑却不了解。在这种情况下,作者应该要求将其要点还原。编辑也应该同意帮这个忙。但他也有坚持自己的权利,来修改任何不清楚的地方。清晰度是每一位编辑要向读者保证的。编辑决不应该让自己都不明白的东西印出来。假如他都不明白,至少还会有另一个人也不明白,那就会有越来越多的人不 明白了。简言之,这个过程应该是由作者和编辑共同审校文稿,找到每一个问题的答案,来最好地服务于终稿。

这一过程中的问题可以通过见面解决,也可以通过电话解决。不要让编辑以距离或忙碌为借口,未经你的同意修改你的作品。“我们已经到截止期了”,“我们已经晚了”,“同你联系的人病了,不在”,“我们上周有一次大调整”,“我们的新社长才上任”,“文稿放错地方了”,“编辑在度假”——这些令人烦躁的词语掩盖了许许多多的低效率和过错。出版界一个不愉快的变化是谦恭态度受到侵蚀,而这曾是该行业的惯例。特别是杂志编辑,他们对于整个出版程序已经变得漫不经心,而这些本应是顺理成章之事:通知作者作品收到,以合理的速度读稿,告诉作者稿件是否可用,如果不用立即退稿,如果文章需要修改就同作者建设性地商榷,发给作者校样,保证作者及时得到稿酬。作者往往处于弱势地位,他们须反复打电话了解自己文章的状况,索取稿费,还要应对其他类似的窘境。

现在盛行的理念是,这样的“谦恭”只是虚饰而已,因此可以不要。事实正好相反,谦恭的行为举止与这一行业是有机的整体。它们是荣誉的准则,维系着整个行业,忘记它们的编辑几乎就是在拿作者的根本权利当儿戏。

当编辑超越风格或结构的改变而进入内容这一神圣领域,这种傲慢最具伤害力。我经常听见自由撰稿人说,“当我拿到杂志找自己的文章时,简直连自己认都认不出来了。他们整个又重写了一个开头,让我说自己并不相信的话。”这可是最大的罪过——篡改作者的观点。编辑通常会做作者允许他们做的事,特别是当时间有限时。作者出于谦卑,允许编辑修改自己的作品以达到自己的目的。而每一次的妥协都在提醒编辑,作者可以被当做雇来的帮工。

但最终,作者的服务目的必须是自己的。你所写的只能是你自己的,而不是别人的。尽可能发挥自己的才能,然后用你的生命捍卫它。只有你才知道自己的才能能走多远,编辑不知道。写得好意味着相信自己的写作,相信自己,勇于冒险,敢于与众不同,推动自己创造卓越。只有你迫使自己写好,你才能写好。

我最喜爱的一个审慎的作者的定义来自乔·迪马吉奥,但他并不知道那是他下的定义。迪马吉奥是我所见过的最伟大的棒球运动员,但他看起来是最轻松的一个。他在外场能跑到超远的距离,运动起来步伐矫健优雅,总是比球提前到达,使得最难的接球看起来都像是家常便饭,甚至当轮到他击球,奋力击球之时也看起来不费劲。我惊叹于他看起来不费吹灰之力的优雅,而他所做的只有通过日复一日的努力才能达到。有一位记者曾问他,始终都能把球打得那么好,是如何做到的。他回答说:“我总在想,看台上至少有一个人从未看过我打球,我不想让他失望。”