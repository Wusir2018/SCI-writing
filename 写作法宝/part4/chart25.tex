\chapter{附录一 用法}\footnote{本部分内容原为第六章,由于主要涉及英语写作中语言的用法,因此放在附录中供读者参考阅读。——编者注}
所有这些有关好词坏词的谈论将大家带人一个灰色但重要的领域,被称为“用法”。什么是好用法?什么是好英文?什么新造的词可以用?谁又是裁判?用“O.K.”O.K.吗?

之前我提到大学生与校方纠缠不休(hassling)的事件,在上一章我将自己形容成词语迷(word freak)。这里就有两个非常新的词语。“hassle”既是动词又是名词,意思是叫某人不好过,或叫人不好过的行为。任何人由于填不好35-BX表而被反复责难都会同意,这个词听起来恰如其分。“freak”的意思是热心者,其中少不了一种沉湎的气氛,比如称某人是jazz freak(爵士迷),或chess freak(国际象棋迷),sun freak(日光浴迷),不过那也许会大增我用词的机会,去形容一个有马戏穿插节目就必看的人为freak freak(超级迷)。

不过我很高兴接受这两个用法。我并不把这些词当做俚语,或用引号引上来显摆我在搬弄青少年文化中的俚语,而且比谁懂的都多。这些都是好词儿,我们需要它们。但我不接受“notables”、“greats”、“upcoming”,还有许多其他新的来客。这些都是廉价词,我们不需要。

为什么一个词好而另一个词却廉价?我给不出答案,因为用法并没有固定的界限。语言是织物,每周都有变化,增加新线,去掉旧线,甚至词语迷们也味词语的可行度而战,经常以完全主观的依据,如品位,来下结论(比如认为“notables”太俗)。这还是没有解决谁是品鉴家的问题。

60年代一部崭新的词典《美国传统词典》的编辑们也遭遇到这一问题。他们召集了一个“用法专家组”来帮助评估找上门来的新词和不可靠的结构。哪些词应该请进来,哪些不顺耳的应该赶出去?该小组由104名男女成员组成——他们大多是作家、诗人、编辑和教师——都以喜爱语言和讲究用法而著称。我也是该小组的一员,在后来的几年里,我不断地接到问卷。我会接受“finalize”(结束)和“escalate”(升级)这两个词吗?你对“It’s me”(是我)感觉如何?我会允许“like”(像……)作为连词吗——就像很多人都那么用一样(like so many people do)。“mighty”(强大的)这个词用在“mighty fine”(好极了)中如何?

我们被告知,在词典里我们的观点会被单独列在“用法注释”中,这样读者就会看到我们的投票情况。问卷还为我们可能感到非要评述的部分留有空间——小组成员对这个机会非常有积极性,这在词典出版后,我们的评述公布于众时可以看出。真实激情高涨。“上帝啊,不!绝不行!”当被问起“to author”当动词用时,芭芭拉·W·塔奇曼这样大叫。学术界并不像维护语言纯洁者面对此类语言淤泥时那样动怒。我同塔奇曼声明的观点一致,绝不能让“author”作为动词合理合法,就像我同意刘易斯·芒福德的观点一样,“good”作为副词只应该“留作欧内斯特·海明威独有的特征”。

但是词语用法的守护人假如只保护语言的规范性,那他们只做了一半的工作。任何一个傻蛋都能决断,后缀“wise”,比如在“healthwise”(在健康方面)中,听起来就“傻气”(doltwise),或者“rather unique”(怪怪的,怀孕了)并不比rather pregnant(有身孕)更贴切。另一半的工作是欢迎任何能带来力度和色彩的新词语,助力语言的生长。因此我很高兴我们百分之九十七的人投票同意接受“dropout”(退出,退学)这个洁净、生动的词,但只有百分之四十七的人接受“senior citizen”(老年公民),这个词是典型的来自社会学领域矮胖的新侵人者,一个非法外国人现在成为未登记的居民。我也很高兴我们接受了“escalate”(升级)这个词。这种词语发明我一般不喜欢,但越南战争赋予了它确切的含义,完全带有一种表示错误的弦外之音。

我很高兴我们完全接受了以前的词典嘲笑为“太口语”化的各种富有活力的词语,如形容词“rambunctious”(喧闹的),动词“trigger”(触发)和“rile”(使恼火),名词“shambles”(混乱)和“tycoon”(巨头),还有“trek”(跋涉),后者由百分之七十八的人通过,意思是艰难的路程,比如“通勤者每天去曼哈顿的跋涉”(the commuter’s daily trek to Manhattan)。此词原来是荷兰南非人的词语,用于表示布尔人用牛车迁移时的艰难旅程。但我们的专家小组显然感到曼哈顿通勤者每天的跋涉不次于艰难的旅程。

然而还是有百分之二十二的专家不愿意让“trek”溜进一般用法中。这显示了我们专家小组投票方法的好处——把大家的观点公之于众,有疑问的作者可以参照行事。因此,有百分之九十五投票反对“myself”(我本人)在诸如“He invited Mary and myself to dinner”(他邀请玛丽和我本人共进晚餐)中的用法,贬抑该词为太“prissy”(谨慎)、“horrible”(可怖),是“genteelism”(雅语),这样就可以警示那些不想谨小慎微、面目可憎或故作文雅之人。正像瑞德·史密斯所说的那样,“‘myself’是那些蠢货的庇护所,他们从小就被教唆‘me’是个令人生厌的词儿”


另一方面,只有百分之六十六的专家小组成员拒绝用动词“to contact”(联系)这个曾经被认为是不雅之词,而只有一半人反对分裂不定式和动词“to fault”(找茬)和“to bus”(乘巴士)。因而,假如你决定主动找校董事会要孩子乘校车(to bus)去另一个城市,那么只有百分之五十的读者会找你茬儿(to fault),说你的用法是错的。假如你联系(contact)校董事会,那你就会另冒百分之六十六的风险被认为用词错误。我们对词语用法明确快捷的准则由优秀图书《审慎的作者》的作者西奥多·门·伯恩斯坦说得很清楚:“我们应该用方便与否来检测词语的用法。那个词真正填补了需要吗?如果是的话,那就特许它使用。”

所有这些都证明了词典编撰家已知的事实:词语用法的法则是相对的,倾向于法则制定者的品位。专家组成员之一凯瑟琳·安·波特称“O.K.”这个词“可恶、粗俗”,并声称她一生从未说过这个词,而我坦诚地承认我说过“O.K.”这个词。“most”(大多数)这个词在“most everyone”(几乎每个人)的用法中被艾萨克·阿西莫夫讥讽为“可爱的乡下话”,而被维古尔·汤普森欢迎为“好的英语习语”。“regime”指任何政府当局,如“the Truman regime”(杜鲁门当局),这个用法得到专家组几乎每个人的赞同,“dynasty”(王朝)一词也一样。但这些词却引来雅克·巴尔赞的勃然大怒,他说,“这都是些专业术语,你们这些该死的历史盲!”我也许会给“regime”一个“O.K.”。但受巴尔赞所谓不准确的斥责,我现在倒觉得这个词看起来像个新闻体词。有一个我嗤之以鼻的词是“personality”(明星),比如在“电视明星”(TV personality)中。但是现在我却怀疑那是否就是唯一的词,来表示一大群为出名而出名的人——也许没有别的词可以替代。加博尔姐妹\footnote{加博尔姐妹(The Gabor Sisters):匈牙利裔美国三姐妹女演员,玛格达(Magda,1915—1997)、扎·扎(Zsa Zsa,1917— )、伊娃(Eva,1919—1995),除了演艺生涯外,以再婚次数多闻名。}真正都干了什么呢?

最终的问题是,何为“正确”用法?我们没有国王建立国王英语,我们只有总统英语,但我们并不想要。《韦氏词典》长期以来都是传统与信念的捍卫者,其1961年的第三版却以宽容的姿态搅浑水,争辩说只要有人用,几乎任何词语都可以,并指出,“ain’t”\footnote{该词为美语口语,可代替助动词be,have的否定式,并且不分时态形式,实际常为文化低的阶层使用。}这个词“在美国大部分地区,许多受过良好教育的人士口头都用。”

《韦氏词典》到底在何处培育了这些受过良好教育之人,这我是不(“ain’t)清楚。不过口头语言比书面语言要松散,这却是事实,而《美国传统词典》恰当地将语言的两种形式问题交给了我们。我们通常允许口语习语用法,但在书面语中由于其随意性而禁止使用。我们充分意识到“笔头语言最终一定要依从口头语言,”正像塞缪尔·约翰逊所言,今天的口头废话也可能成为明天的笔头金子。分裂不定式,或者在句尾加介词的用法日渐被接受这一事实证明,正式句法结构并不能永远坚守阵地,以抗击说话者用更舒服的方式表达同样的意思一而且也不应该。我认为句尾加了介词的句子也可以是个好句子。\footnote{作者这句话就以介词of结尾,来证明其可接受性。原句: I think a sentence is a fine thing to put a preposition at the end of.}

我们专家组发现,甚至在单词范围内,其正确与否的标准都有所不同。我们大多数都投票反对将“cohort”(美口语:同伴)作为“colleague”(同事)的同义词,除非是用调侃的语气用该词。因而一位教授在院系例会上就不说同其cohorts在一起,但在学院聚会的非正式场合就可以说众多cohorts在一起,大家戴着滑稽的帽子。我们反对用“too”(太)作为“very”(很)的同义词,比如:“His health is not too good。”(他的健康不太好。)谁的健康呢?但我们同意该词用于讽刺或幽默,比如:“He was not too happy when she ignored him.”(当她忽视他的时候,他不太高兴。)

这些似乎是微不足道的区别,实则不然。这些用法提醒读者,人们对用法的细微差别是敏感的。用“too”替换“very”形成赘语,比如:“He didn’t feel too much like going shopping.”(他不太想去购物。)但在上一段里富有挖苦性的例子对林·拉德纳却别有意味。该词用法增加了一丝讥讽,不然该句就不存在这种弦外之音了。

幸运的是,我们专家组的评议形成了一种模式,并且提供了一种至今仍有用的准则。我们对于新词与词组的接受度最终还是开明的,但在语法上是保守的。

拒绝像“dropout”这样完美的词,或者对无数的词与词组每天都进入正确用法之门熟视无睹都是愚蠢的。新词与词组的诞生借助于科技之风,以及商业、体育和社会变化之风,如:“outsource”(外包),“blog”(博客),“laptop”(笔记本电脑),“mousepad”(鼠标垫),“geek”(极客),“boomer”(核潜艇),“Google”(谷歌),“iPod(苹果牌随身听),“hedge fund”(对冲基金),“24/7”(每天24小时,每周7天),“multi-tasking”(多重处理),“slam dunk”(扣篮),以及成百上千的其他词语。同时我们也不该忘记60年代反正统文化群体发明的短词,他们以此来反击当权者自以为是的繁庸言词,如:“trip”(吸毒幻觉),“rap”(说唱),“crash”(垮台),“trash”(废话),“funky”(时髦的),“split”(离开),“rip-off”(偷窃),“vibes”(共鸣),“downer”(镇静药),“bummer”(懒汉)。假如简明是获奖的标准,这些词就是赢家。接受这些一夜间进入语言的词汇的唯一问题是,这些词的消失也常常同样突然。60年代晚期“所发生的事件”(happenings)现今“不再发生”(np longer happen),“绝了”(out of sight)的用法已“不见踪影”(out of sight),甚至连“巨好”(awesome)这样的词都冷却下来。关注词语用法的作者必须时刻从僵死的词语中辨别出新出现的词语。

至于在我们用法专家组持保守态度的领域,我们坚持语法中大多数常规性的区别——“can”(能) 与“may”(可以),“fewer”(更少:修饰可数名词)与“less”(更少:修饰不可数名词),“eldest”(排行老大)与“oldest”(最老的)等——我们强烈反对用法中经常性的错误,坚持认为“flout”(轻视)还是不能表示“flaunt”(夸耀),无论多少作者轻视(flout)规则,夸耀(flaunt)自己的无知都不行。“fortuitous”的意思还是“accidental”(偶然性);“disinterested”的意思还是“impartial”(不偏不倚);“infer”(推断)的意思不是“imply”(暗示)。我们在此的动机是对于语言精确美的热爱。不正确的用法会使你失去你最想赢得的读者。弄清楚下列词的区别:“reference”(参考)与“allusion”(典故),“connive”(默许)与“conspire”(密谋),“compare with”(比较)与“compare to”(比拟)。如果你非得用“comprise”,请正确使用。该词的意思是“include”(包括),如:晚餐包括肉、土豆、沙拉和甜点(Dinner comprises meat, potatoes, salad and dessert.)。

“我选用的总是合乎语法的用法,除非听起来做作,”玛丽安·穆尔如此解释道。这也是我们专家组最终所采取的立场。我们并非迂腐的学究,因此也不会阻挠语言不断保持其词语的新鲜度。但这并不意味着我们必须接受每一个笨拙而庸俗不堪的词语。

与此同时,战斗仍在继续。至今我仍收到《美国传统词典》发来的选票,征求我对新语言用法的意见,比如动词“definitize”(Congress definitized a proposal”,国会确定了一项提案),名词“affordables”(支付得起之物),俗语“the bottom line”(底线),以及误用之词“into”(He’s into backgammon and she’s into jogging,他爱好十五子游戏,而她爱好慢跑)。

当今行话充斥着我们的生活和语言,这不需要专家组都看得出来。卡特总统签署过行政命令,指示联邦法规一定要“简洁、清晰地”撰写。克林顿总统的司法部长珍妮特·雷诺敦促全国的律师们用“短小、久远、大家都懂的词”——如“right”(对)与“wrong”(错)以及“justice”(正义)等——来替换“大量的法律行话”。公司聘用了顾问以避免文件晦涩难懂,甚至保险业也在努力改写保险单,使用不那么糟糕的英语向我们说明当遭难时如何获得赔偿。这些努力能有多大效果,我不敢肯定。然而,令人欣慰的是,我们看见有众多的监察员就像克努特(Canute)\footnote{克努特国王(King Canute,约985-1035):丹麦、英国、挪威国王,曾将宝座置于海边,下令海浪停止,不许沾湿他的脚和王袍,结果不奏效,因而感叹王位之虚无,上帝之伟力。}一样站在海滩上,竭力抵抗着海浪。这也是所有认真的作者应该做的——看每一片冲刷上来的新漂浮物,问“我们需要它吗?”。

我记得有人第一次问我,“How does that impact you?”(那对你的影响如何?)我一直认为除了在牙科,“impact”是名词。后来我开始遇见“de-impact”(消除影响),一般用于消除与经济窘境有关的影响。名词一夜间竟变成了动词。我们瞄准目标,我们获取真相(We target goals and we access facts)。火车乘务员通知火车将不靠站(Train conductors announce that the train won’t platform)。机场门上的标记告知人们该门有报警装置(A sign on an airport door tells me that the door is alarmed)。公司在缩小规模(Companies are downsizing)。扩大业务是正在进行的努力中的一部分(It’s part of an ongoing effort to grow the business)。“ongoing”(正在进行中)成了行话,其主要用途就是提高士气。假如老板告诉我们这是一项正在进行中的项目,我们就会以更多的热情投入每天的工作;假如他们为正在进行中的需要已经瞄准了资金,我们就会心甘情愿地为单位多奉献。否则我们可能成为无积极性的猎物。

我可以继续列下去,例子足以填满一本书,但那可不是我想要大家读的。我们仍需要解答以下的问题:什么是好的用法?一个有助的办法是区分行话与用法。

我会说,比如“prioritize”(按优先顺序排列)是行话——是听起来比“rank”(排列)更重要的华而不实的新动词——“bottom line”(底线,关键)是用法,是借自簿记领域的比喻词语,传达某种可见的意象。商业人士都知道,底线至关重要。假如有人说,“The bottom line is that we just can’t work together”(关键是我们不能在一起工作),我们知道他是什么意思。我不怎么喜欢这个词组,但关键(the bottom line)是它已在此驻留。

新用法也伴随着新的政治事件而到来。正像越南战争提供了“escalate”(升级)一词的用法,水门事件提供了一系列与阻碍和欺骗的意思相关的词汇,包括“deep-six”(销毁),“launder”(掩饰),“enemies list”(仇敌名单),以及其他由“gate”后缀构成的丑闻事件(比如,“Iran-gate”“伊朗门事件”)。在理查德·尼克松任职期间,“launder”变成了肮脏之词,这一讽刺真是恰到好处。今天当我们听说有人掩饰资金的来源以及资金的路径,这个词具有了精准的含义。该词短小、生动,是我们所需的。我接受“launder”和“stonewall”(阻碍)这两个词的用法,但我并不接受“prioritize”和“disincentive”(负刺激因素)的用法。

我愿意就区分好英语与行业英语提出类似的准则。其区别就像诸如“printout”(打印输出)与“input”(输入)之间的差异。一份printout特指计算机打出来的东西。在计算机出现之前,并不需要该词,但现在需要。但它只是用于专业领域。而“input”却不然,它本来被造出来用于描述输入计算机的信息,而今却用于各行各业,从饮食到哲学话语(“我想倾听你对上帝是否真正存在这个问题的输入意见。”“I’d like your input on whether God really exists.”)

我并不想给予别人我的输入意见(input),然后得到他人的反馈(feedback);但我愿意奉献我的看法,然后倾听别人对这些看法是怎么想的。对我来说,好的用法包括用现已存在的好词语来清楚、简洁地向别人表达自我——而这些好词语几乎总是存在的。你可以说这就是我用言辞表达人际关系的方式。

※ that和which:\footnote{原为第九章“零零碎碎”的内容,请读者参与阅读。——编者注}

任何人试图在不到一个小时之内解释“that”和“which”的用法,都是自找麻烦。亨利·沃森·福勒在《现代英语用法词典》中将其分成25栏。我只需两分钟,也许创世界纪录了吧。以下(我希望)是你需要记住的大部分:

总是用“that”,除非它会使意思发生歧义。注意,在编辑审慎的杂志中,如《纽约客》,“that”的用法独占鳌头。我提出这一点是因为大家仍普遍相信,作为学校、大学的遗留物,“which”更正确、更可接受、更具书卷气。其实不然。在多数情况下,“that”才是你自然而然所说的,因此也是你该写的。

如果你写的句子需要用逗号来表达其确切意思,那它也许需要用到“which”。“which”服务于某种特别的辨认功能,不同于“that”。(A)“Take the shoes that are in the closet.”(拿壁橱里的鞋)其意思为:拿壁橱里的鞋,而不是床下的鞋。(B)“Take the shoes, which are in the closet.”这里只牵涉一双鞋的问题,这个“which”用法告诉你鞋在哪儿。注意逗号在B中是必要的,而在A中是不必要的。

“which”更常用于细致地描述、辨别、定位、解释,不然就是用于修饰逗号之前的词语:

The house, which has a red roof,带红屋顶的房子

The store, which is called Bob’s Hardware,那间被称为鲍勃五金的店铺

The Rhine, which is in Germany,德国境内的莱茵河

The monsoon, which is a seasonal wind,一种季节性的风,季风

The moon, which I saw from the porch,我从阳台上看见的月亮

以上就是我要说的,我想这些是你在写好非虚构作品前需要知道的,此类作品要求整齐排列信息。 。