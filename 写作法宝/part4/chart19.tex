\chapter{自己的声音}
我写过一本有关棒球的书,还写过一本有关爵士乐的书,但我从未想过一本用体育英语写,另一本用爵士乐英语写。这两本我都尽力用最好的英语写,用我惯常的风格写。虽然题材上大相径庭,但是我要读者知道他们所倾听的是来自同一个人。其他作家也会以他们自己的方式写书。作为作家,无论我写什么,我的商品只是我自己,而你的商品也只是你自己。不要改变自己的声音去迎合自己的题材。培养一种读者一读文字就能认出的声音,这种声音不但在富于乐感的字里行间令人愉悦,而且还能避免任何降低其品味的音调:轻浮、屈就以及迂腐。


先从风趣开始。

有一种写作风格听起来特别放松,你会觉得是在听作家与你交淡。E.B.怀特可能是其最佳典范,当然我还会想起其他一些具有这类风格的大师——詹姆斯·瑟伯、V.S.普里切特、刘易斯·托马斯。我偏向这种风格,因为这是我一直在努力实践的。一般人认为用这种风格写起来轻而易举,但事实正好相反。看起来举重若轻的风格其实是通过艰苦的努力和不断的润色才能达到的。语法与句法都钉得严丝合缝,这样写出来的文字是作者尽了最大力气的。

下面是E.B.怀特一篇文章的开头,颇具代表性:

我9月中旬同一头病猪过了几天几夜,现在感到有什么驱使我非得讲一讲这段时间是怎么过的不可,特别是因为最后猪死了,我活了,其实事情很容易会是相反的结果,那样就没人来讲这事儿了。

这句话特别随意,我们会想象自己就坐在怀特在缅因家中的阳台上。怀特坐在摇椅里,抽着烟斗,词句就从他那说书人的嗓音中滚落。但是再看一看这句话。其中的一切并非偶然,而是训练有素的写作行为。语法规范,用词简朴、准确,节奏富有诗人的秉性。这就是最佳的举重若轻的风格:这种有条不紊的创作,以其散发出的温暖使我们释然。作者听起来很自信,他并非要竭力讨好读者。

无经验的作者正是缺乏这一点。他们以为要达到一种随意的效果,所要做的就只是“百姓化”——就像老好人贝蒂或者鲍勃隔着后院的栅栏聊天。他们想做读者的朋友,他们急于要显出平易近人的样子,结果连文字都不好好写了。他们写出来的文字风格轻浮。

一个轻浮的作者会如何处理E.B.怀特与猪在一起时的孤独时光呢?他可能会这样写:

曾熬夜看护过一头病肉猪吗?信不信由你,那可得缺老鼻子觉了。那事儿我早在9月份干了三宿,我那另一半还以为我发飙了呢。(只是开个玩笑,帕姆!)着实地讲,那整桩子事儿有点儿叫我郁闷。因为,你知道,结果那头猪死了,死在了我前头。跟你说实话,当时我自己感觉身体并不太好,所以我觉得当时很可能是人而不是那头老肥猪先翘辫子了。你可以拿救命钱打赌,猪先生可不会为这事儿去写一本书!

没有必要费力找出诸多理由来说明这篇东西为何这么糟糕。它太粗糙、太土气、太啰嗦、太讨好(我已经不再读那些老说“你知道”的作者的文章了),这简直就是鄙视语言。但这种轻浮的风格最可悲之处是,它比好语言还难读。作者试图减轻读者在阅读旅途中的烦劳,但结果却在沿途四处设置障碍:粗鄙的俚语、拙劣的句子、浮夸的说教。E.B.怀特风格的作品读起来容易得多。他知道语法作为语言工具几百年流传下来并非偶然;这些工具是读者需要并发自内心想要的后盾。人们不断地阅读E.B.怀特或者V.S.普里切特,因为他们写得就是好,不可不读。但假如读者认为你在降低档次对他们说话,他们就不会再读你写的东西了。没人愿意被照顾。

要抱着对最典范的语言崇敬的态度写作,要对得起最佳的读者。假如你有用轻浮的文风写作的冲动,朗读一下你写的东西,看你是否喜欢自己的声音。

找到读者喜欢的音调与风格主要设计品味的问题。这么说无济于事——品位这种特质太不可捉摸,甚至不可界定。但是一旦遇见,就能认出来。穿衣有品味的女人使我们愉悦,她有能力经过一番综合的打扮,不但给人以时尚感和惊喜,而且恰到好处。她知道什么效果好,什么不好。

对作家和其他艺术家来讲,知道不做什么是品位的重要组成部分。两位爵士钢琴家可能技艺相当,有品位的一位会善用每一个音符来倾诉他或她的故事,没品位的那位则会用涟漪与装饰音来侵浸我们。有品位的画家信任自己的眼晴,知道什么该出现在画布上,什么不该。无品位的画家呈现给我们的风景不是太漂亮,就是太杂乱,或者太艳丽,反正有点那个。有品位的平面设计师知道少即是多,设计是文字的奴仆。没有品位的设计师则会将文字扼杀于背景的色彩与线条以及虚饰之中。

我意识到自己面对的是一种主观的东西,对一个人美丽的东西对另一个人却可能是庸俗之作。品位也随时代变化,昨日迷人之物今日被嘲笑为蹩脚货,但明日又可能重新时髦起来,又被认定为迷人。那我为什么还提这件事呢?就是要提醒你这样的事情确实存在。品位是一股贯穿于整个写作过程的暗流,你应该意识到它的存在。

其实,有时候品位也是看得见的。每一种艺术形式都有一种实质性的核心,能够冲破变幻莫测的时间的束缚。帕提侬神庙的建筑比例之中一定有什么内在的东西使人愉悦,西方人继续让两千年前的希腊人为其设计公共建筑,任何人只要在首都华盛顿转一转就会很快发现这一点。巴赫的赋格曲具有超越时间的优雅,它根植于超越时间的数学定律。

写作也有诸如此类的标杆吗?不多。写作是每个人个性的表达,我们一旦遇上什么,便知道自己是否喜欢。同样,通过知道省略什么,我们会收获很多,比如套话。假如作者生活于幸福的无知之中,不知道套话是死亡之吻,假如他翻遍每一块石头并不遗余力地使用套话,我们可以推断,他缺乏把握语言新鲜感的天赋。面对新鲜与陈腐,他会毫不犹豫地选择陈腐。他的声音是庸才的声音。

套话散布在我们四周的空气中,并不容易识别,就像老朋友们伸手相助,随时提供简短的比喻形式为我们表达复杂的意思。这就是话语最初如何成为套话的,甚至连审慎的作家在初稿中都会用到不少。但之后,我们有机会清除掉套话。套话是你在稿件后续的改写和朗读中,能够不断听出来的问题中的一项。要注意套话听起来多么难以区分,使你注定满足于使用同样的陈词滥调,而不是努力使用自己的新颖词语。套话是品位的敌人。

要超越个人的套话,将重点扩展到更宽广的语言用法上。同样,新鲜感至关重要。有品位者选择令人惊诧、有力、精准的词语,无品位者滑入轻浮的校友杂志中班级纪事之类的俗语——在这样的世界中,权威人士是领袖人物或者当政者。“领袖人物”到底有什么问题呢?并没有什么问题,但同时也是问题所在。有品位者知道,对权威人士,最好他们是什么就称呼什么:官员、行政主管、主席、总裁、主任、经理。无品味者特意用俗气的同义词,这更加剧了不精准这一不利因素。那么到底公司里哪些头儿才是领袖人物呢?无品味者用“第n个”,还有“数不清”,或者“到此为止”\footnote{这三个词原文为:“umpteenth”,“zillions”,“period”,皆为俚语,相当于汉语中含义相近的口头语。}等,比如“她说她不想再听到那件事了。到此为止”。

但最终,品位是一种综合素质,不可分析:有品位的耳朵可以听出一瘸一拐的句子与轻快跳跃的句子之间的差异;品位赋予人直觉,使人知晓何时一个随意、俗气的词语变成规范的句子后,不但听起来对头,而且似乎就是不可避免的选择。(E.B.怀特就是平衡此举的大师。)这意味着品位是可以学会的吗?是,也不是。完美的品味,就像完美的音高,是上帝赐予的天赋。但其中的一部分还是可以学得的。关键在于要研读有此天赋的作家的作品。

不要犹豫,要模仿其他作家。模仿对于任何艺术或技艺的学习者,都是创造过程的一部分。巴赫和毕加索也不是一蹦出来就成为成熟的巴赫与毕加索,他们也需要样板。对于写作尤其如此。找到你感兴趣的领域中最好的作家,朗读他们的作品。将他们的声音和品位沁入你的耳朵——还有他们对语言的心态。不要担忧模仿他们会令你失去自己的声音和自我。很快你就会脱掉那层皮,成为你本应该成为的自我。

通过阅读其他作家,你还可以投身于一个能丰富自己的更悠久的传统之中。有时候你还会开发出雄辩之才,或者民族记忆之源,给你的写作增添深度,而这光靠自身是达不到的。让我以间接方法来举例说明我的意思。

一般来讲,我不会通过读州政府官员发布的公告,来确定哪一年中的重要日子的确为那一年中的重要日子。但在1976年,我当时在耶鲁教书,康涅狄格州州长埃拉·格拉索想出一个令人愉快的主意,重新发布40年前由威尔伯·克罗斯撰写的感恩节公告。她称之为“雄辩之杰作”。我常想雄辩的风采是否已从美国人的生活中消失,或者我们是否还将其看作是一种值得为之努力奋斗的目标。因此我研究了克罗斯州长的言辞,看看它们是如何经受住时间的考验的。时间就是对于先辈们辩才的严酷裁判。我欣喜地发现我同意格拉索州长的看法。这的确是一件大师级作品:

在此季节更替之际,时间久远到已被忘却,耐寒的橡树叶在风中窸窣作响,霜冻给空气增添了一丝凉意。黄昏早早到来,温馨的夜晚随着猎户座俄里翁的脚跟渐渐拉长。这似乎是一段不错的时光,我们大家聚在一起,赞美我们的创世主与保护人,他以我们不知的方式,再次将我们带到一年的终了。为了遵守这一风俗,我将9月26日礼拜四,确定为公共感恩节日,祝福我们一直共同享有的好运,祝福赐予我们所热爱的州为大地上受优待的地区——祝福众生所拥有的舒适:来自土壤的丰收哺育着我们,各行各业劳作的丰硕成果维持着我们的生活——祝福所有这一切,这一切就像呼吸对于身体一样珍贵,这一切激发起人们心中的信念,滋养与增强人们的言与行;祝福荣誉高于一切;祝福以坚定的勇气与热忱在漫漫征途上追寻真理;祝福同胞们彼此赠予自由与公正并自由地享受这一切;祝福无上的和平之荣耀与仁慈降临我们的土地——我们以庄严隆重的仪式,再次聚会,来庆祝我们家园的收获节日。让我们以诚心,谦卑地接受这些祝福。

格拉索州长加了一个跋,号召康涅狄格州的市民们“重振清教徒在新大陆的第一个严酷的冬天所表现出来的勇于奉献与担当的精神”,而且当晚我还特意在脑子里观望了一下猎户座俄里翁。我很高兴被提醒自己生活在大地上受优待的地区之一,也很高兴被提醒和平并非是唯一让我们为之感恩的无尚荣耀。语言在此优雅地用于公共福祉,因此同样需要我们感恩。杰弗逊、林肯、丘吉尔、罗斯福以及阿德莱·斯蒂文森的文辞节奏向我滚滚而来。(而艾森豪威尔、尼克松以及布什父子的文辞节奏却没有滚来。)

我将感恩节公告张贴在公告栏上让学生赏析。从他们的评论中,我发现有几个学生以为我是在开玩笑。由于了解我对简洁的癖好,他们猜想我是将克罗斯的言辞当作了过分华丽的代表。

这件事给我留下几个问题。我将威尔伯·克罗斯的散文体推荐给年轻人,这种语体是向民众沟通的一种方式,而这一代人难道就从未领略过此中语言的高贵性吗?自从1961年约翰·F·肯尼迪的就职演说以来,我就再也想不出还有任何一种这样的尝试。(马里奥·科莫和杰西·杰克逊也许能部分地恢复我的信心。)这是伴随电视长大的一代人,在电视中画面胜于言辞——言辞实际上贬值了,只是用来闲聊,而且常被误用,音也经常发错。这也是伴 随音乐长大的一代——歌与节奏主要是用来听和感觉的。空气中充斥着这么多噪音,有哪一个美国孩子在接受倾听训练吗?有哪一个人在关注一句结构讲究的句子中所体现的庄重感吗?

我的另一个问题提出了一个更微妙的谜题:区别雄辩与浮夸的界限是什么?我们为什么会受到威尔伯·克罗斯言辞的感召,而觉得大多数政客和政府官员硬塞给我们的是华而不实的花哨与炫耀,他们的演讲只会使我们昏昏欲睡?

其中的部分原因带我们回到品味问题上。耳朵对语言敏感的作家会寻求鲜明的意向,而避免陈腐的言辞。平庸的作者恰恰在寻求这些陈词滥调,以为他能用所谓检验过的真实的硬通货丰富自己的思想。答案的另一部分在于其简明性。经得起时间检验的作品一般用词短小精悍、强劲有力;沉闷的词是那些由三、四或五个音节构成的词,多数来自拉丁语词源,许多以拉丁语后缀结尾,表示一种模糊的概念。在威尔伯·克罗斯的感恩节公告中,没有四音节的词,只有十个三音节的词,其中三个还是他非用不可的专有名词。注意州长的公告中并没有含糊的用词,而只有诸如树叶、风、霜冻、空气、夜晚、大地、舒适、土壤、劳作、呼吸、身体、公正、勇气、和平、土地、仪式、家园等词汇。这些都是从正面意义上最朴实的词语——抓住了季节的韵律和生活的节奏。同时还要注意,所有这些词都是名词。动词之后,朴实的名词是最有力的工具,它们能引起情感的共鸣。

但最终,雄辩流动于更深的层面。它以一切尽在不言中感动着我们,在我们已经知晓的读物、宗教以及传统的内容中触发回响。雄辩邀请我们的自我那一部分加入到交流之中。林肯的演讲回响起詹姆士一世钦定版《圣经》的神韵,这绝非偶然。他从儿时就对此几乎谙熟于心,他自己浸淫于其言辞语调之中,他的正式用语更趋向于伊丽莎白时代的特点,而非美国式。他的第二届任职演说回响起《圣经》的用语和对《圣经》的阐释:有人竟敢请求公正的上帝给予帮助,从别人脸上的汗水中拧出自己的口粮,这可真是咄咄怪事,但却让我们不加裁判,我们也不受裁判。此句的前面部分借自《创世记》中的一个比喻,后部分改自《马太福音》中一条著名的训诫,而“公正的上帝”来自《以赛亚书》。

如果说这个演讲对我的影响大于所有其他美国官方文件,那不只是因为我知道林肯五周后被暗杀了,或者因为我感动于他的和解呼吁在达到顶点时的所有痛苦,这一和解对于国人来讲有百利而无一害;还因为林肯运用了某些西方人有关奴隶制最古老的教诲——仁慈与审判。对于像他一样在《圣经》的教诲下成长的人们,当他们在1865年听到他的演讲时,能感受到他的言辞所承载的坚定的弦外之音。即使到了21世纪,林肯的理念中那种久远而几乎难以理解的愤怒也很难让人忽略,那就是,上帝也许有意让内战继续,“直到由奴隶们两百五十年全无报酬的劳作所堆积的财富最终全部沉没,直到鞭子抽打出来的每一滴血最终由刀剑汲取的另一种血得到偿还,就像三千年前所说的那样,必须说,‘上帝的审判既是真实的也是公正的。’”

威尔伯·克罗斯的感恩节公告同样回响着深入我们骨髓的真谛。对于季节的更替与大地的丰收,我们赋予了自己强烈的情感。谁没有怀着敬畏之心遥望过射手座呢?对于“漫漫征途上追寻真理”与“赠予自由与公正”这样的民主进程,在这个我们已经赢得了许多人权而仍有许多使我们无措的国家,我们奉献自己,追寻自己的真理,接受自己的馈赠与给予。克罗斯州长并没有占用我们的时间解释这些进程,我为此向他表示感谢。我厌恶一个平庸的演说家动用太多的陈词滥调来告诫我们太多,而滋养我们太少。

因此,当你讲述自己的故事时,记住过去的功用。在具有地域性和民族性根基的写作中,包括南方写作、非裔美国人写作、犹太裔美国人写作等,感动我们的是比叙事者古老得多的声音,以非同寻常的丰富节奏向我们述说。托妮·莫里森是最雄辩的黑人作家之一,她曾说:“我记得伴随我长大的人们的语言。语言对他们至关重要,所有力量都在语言之中。其中有优雅与比喻,还有很庄重的成分,以及充满《圣经》意蕴的成分。人们的习惯是,当有重要的事情要说时,假如你来自非洲,你就会用寓言的形式,否则就会用另一种层次的语言表达。我就想这样运用语言,因为我的感觉是,黑人小说之所以其有黑人的特点,并非因为它是我写的,或者因为其中有黑人,或者因为它事关黑人。重要的是风格。它具有特定的风格。它具有无可替代的风格。我无法描述,但我可以创作出来。”

追求你自己传统中那些无法取代的风格吧。拥抱它,它就可能引你走向生动与雄辩。