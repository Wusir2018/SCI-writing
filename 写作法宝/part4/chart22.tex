\chapter{作者的抉择}
这本书是有关抉择的——无数的抉择,一个接一个进入写作的每一次实践。有一些事关重大(“我应该写什么?”),而另一些则可能小到一个小小的用词。但所行这些抉择都十分重要。


前面一章是有关重要抉择的:形式、结构、紧凑、中心以及意图等问题。这一章是有关细小的抉择的:在组织一片长文的过程中需要成百上千次抉择。我觉得呈现出这些抉择是如何作出的对你会有帮助。我会用一篇自己的文章作为标本进行解剖。

学会如何组织一篇长文就像学会如何写出一个清晰、悦人的句子一样重要。你必须时刻记住写作是线性的和连续的,逻辑是将其合在一起的黏合剂,从一句到另一句、一段到另一段、一节到另一节,必须保持张力,而且叙事性——精湛的、老套的讲故事方式——是牵引读者前行但不被他们发现这一条绳索的要素,如若不然,你所有的清晰、悦人的句子都会七零八落。读者唯一应该注意的是你为自己的旅程制定了一个合理的计划,其中的每一步都是必然的。

我有一篇文章叫《来自廷巴克图的新闻》,刊登在《康泰纳仕旅人》上。虽然这是一个作家针对一个具体问题的解决方法,但它说明了一些适用于所有与非虚构作品写作相关的问题。我对该作品作了注释,以说明在这个过程中我所作出的抉择。

对于任何文章的写作来说,最难的抉择是如何开头。开头必须用引人入胜的想法抓住读者,然后每一段都要牢牢地保持读者的兴趣,逐渐增加信息量。信息的重要性在于能让读者产生足够的兴趣,这样他们就会坚持走完全程。文章的开头可以短至一段,也可以需要多长就有多长。当所有该做的工作都完成了,你就会知道开头可以结束了,然后你可以放松语气,继续你的叙述。下面是文章的第一段,能够引发读者深入思考——我希望这种想法是读者之前从未想到的。

我到达廷巴克图时,印象最深的是那里的街道用沙子铺成。我突然意识到沙子完全不同于泥土。每一座城镇始于土道,而后居民增多,环境改善,土道就逐渐铺成了沙道。但是沙子代表着颓败。用沙子铺街道的城市是一座天边的城市。

注意这五句话非常简洁,都是质朴的陈述句。每句话包括一个想法——只有一个。读者每次只能处理一个想法,而且只能以线性方式处理。作者所遭遇的多数问题是竭力让一句话做太多的事儿。
绝不要怕将一个长句拆成两个甚或三个短句。

这当然就是我为什么在那里的原因:廷巴克图是人们寻找天边的终点站。在仅以名字就吸引游客到来的六七个地方中——巴里岛和塔希提岛、撒马尔罕和非斯、蒙巴萨和澳门——没有一个能在它所表达的遥远性方面比得上廷巴克图。我惊讶地发现,在听说我去过那里之后,有那么多人都认为廷巴克图不是个真地方,或者即使认为是,也想不出那地方到底会在哪儿。他们只熟悉那是一个词儿——在表达几乎不可企及的同义词中是最生动的一个,歌词作者用这么一个天赐的小玩意儿来押韵,而且用来比喻一个堕入爱河的男孩要走多远才能追上根本追不上的女孩。但作为一个真实的地方——廷巴克图当然是“久已逝去”的非洲王国之一,就像所罗门国王的宝库一样,当维多利亚探险者去寻找时,发现它其实并不存在。

这一段的第一句由上一段的最后一句衍生而来,读者没有机会从文本游离出去。之后这一段话有一个目的:进一步认可读者对于廷巴克图的已知情况——或者一知半解的情况。这样,文章欢迎读者作为一位旅伴,与作者带着同样的情感上路。文章同时增加了某些信息——不是生硬的实事,而是悦人的传说。

下面这一段到了实打实的部分——不能再推迟。注意在这三句话中挤进了多少信息:

这久已逝去的廷巴克图还是被找到了,不过实在经历了异常的艰难险阻之后——苏格兰人戈登·莱恩于1826年,法国人勒内·卡耶于1828年——对于他们所付出的艰辛,这些人一定会感觉遭到残酷的嘲讽。16世纪的旅行家利奥·阿弗里卡纳斯曾经描述过这座十万人的传奇城市,一个拥有180所穆斯林学校、2万多名学生的学术中心,可如今已变成荒凉之地,唯有土房依旧,昔日的辉煌和人口早已不复存在,而它仍存在的原因只有其独一无二的地理位置,此外它还是横穿撒哈拉驼队所走之路的重要交叉口。非洲多数的贸易,特别是北方的食盐和南方的黄金,都是在廷巴克图完成的。

廷巴克图的历史和其出名的原因就介绍到此。有关这座城市的过去和重要性,杂志读者也只需要知道这些。不要给杂志读者超过其所需的信息。假如你想多讲,写一本书或者写给学术期刊吧。

现在你的读者还想知道什么?在每一句话之后要问自己这个问题。在此他们想知道的是:我为何要去廷巴克图?我此行的目的是什么?下面这一段就解答了这个问题。同样,将上一句的线索拉紧:我来到廷巴克图是为了看那些骆驼商队。我们六人看见《纽约时报》周日版发布的一则两周旅游广告——不知是聪明还是愚蠢——于是就签了单。这次旅游是由一家专门做西非业务的法国小旅行社组织的。(廷巴克图在马里,属于前法国苏丹。)旅行社的办公地点在纽约市,为了避开人群,我周一早上第一件事就是去了那里。我问了一些寻常的问题,得到了一些寻常的回答——黄热病疫苗、霍乱疫苗、防疟疾药丸、不要喝当地的水——然后得到一本小册子。

除了解释旅行的来龙去脉,这一段还做了一件事:引入作者的性格和口吻。在游记写作中,千万不要忘记自己是导游。只叫读者上路还不够,你必须带他们上你的路。要让他们对你有认同感,认同你的希望和看法。这意味着叫他们了解你是谁。“聪明还是愚蠢”这个词语唤起游记文学中一个熟悉的形象:游客可能被看做是傻瓜或小丑。另一个脱口而出的妙语是“避开人群”\footnote{beat the crowd,为英语成语,有一定趣味性。}。我用此语只是为了自娱。严格地讲,在第四段才说廷巴克图在哪儿为时太晚。但在此之前我找不到合适的地方来提这个,不然会拆散开头的组织结构。

下面是第五段:

“有机会参观一年一度的阿扎雷盐商队到达廷巴克图,这真是一生一次的疯狂举动。”旅游手册这样开始。“想象这样的图景:上百头骆驼驮着珍贵的大盐块(内陆西非的土著人称其为白金)凯旋进入廷巴克图这个拥有7000名居民的古老而神秘的半沙漠半城市地区。兴高采烈的游牧者们赶着驼队行进了一千多英里,穿过了撒哈拉沙漠,最终得以部落舞蹈和户外欢宴来庆贺长途跋涉的结束。作为酋长的客人,你可以在沙漠的帐篷里过一夜。”

这是一个很有代表性的例子,显示出作者如何让别人为其做有效的工作——一般来讲,用他们的词语会比作者本人的更能说明问题。旅游手册在此不但告知所许诺的旅游项目,而且其语言本身也有娱乐性,它使你进入促销者自吹自擂的窗口。盯住那些滑稽与利己的引语,并怀着感激之情加以运用。下面是开头部分的最后一段:

好了,这就是我这种类型的人喜欢的旅行,但不一定是我这种类型的人喜欢的散文,而且结果也成了我妻子和其他四个人喜欢的旅行。从年龄上看,我们的跨度从中午到享受医疗费减免的年纪。我们其中五位来自曼哈顿中城,一位是寡妇,来自马里兰,大家都养成了去天边旅行的终身习惯。诸如威尼斯和凡尔赛之类的名字在我们早期的旅行记录中压根儿没冒过头,甚至连马拉喀什或者卢克索或者清迈都不算在内。我们谈的是不丹和婆罗洲、西藏和也门,还有摩鹿加群岛。现在——赞美真主——我们终于到了廷巴克图。我们所期盼的骆驼商队就要到来了。

开头到此结束。写这六段所花的时间相当于文章其余部分的所有时间,但我终于花大力气将其各就各位,充满信心地把文章展开了。也许别人可以写出更好的开头,但是我尽力了。我感到与我同在的读者会坚持到底。

对于措辞的选择,其重要性不亚于对结构的抉择。平庸是好文章的敌人,挑战在于不要写起来人云亦云。在开头必须交代的一个细节是我们四个人年龄多大。开始我写的是诸如“我们都在五六十岁”这样的句子,也算过得去。但只足过得去就乏味了。有没有什么办法将这一细节推陈出新呢?好像没有。最终,仁慈的缪斯给了我灵感,我选择了“医疗减免制度”\footnote{Medicare,美国政府为65岁以上老人设置的医疗减免制度。}——因此就有了“从中年到享受医疗费减免的年纪”这样的词语。如果你耐心寻找,就能找到合适的名称和比喻来将沉闷但必要的细节描述得栩栩如生。

甚至写出“威尼斯”和“凡尔赛”都花费了很长时间。原本我写的是“诸如伦敦和巴黎之类的名字在我们早期的旅行记录中压根儿就没冒过头”,但这没意思。我竭力在想其他一些家喻户晓的首都。罗马和开罗?雅典和曼谷?都好不到哪儿去。也许头韵能起作用——读者喜欢所有能满足他们对韵律和节奏偏好的感觉。马德里和莫斯科?特拉维夫和东京?太玄乎了。于是我不再想首都了,而是竭力想旅游热点城市。威尼斯(Venice)跳人我的脑中,我很高兴找到这个词,人人都去威尼斯。还有什么别的城市以V开头?只有维也纳,但它同威尼斯在诸多方面都太靠近。最后我改变思路,从旅游城市转到旅游景点,在头脑中从主要城市向外扇形扩散,在这一次次的远足过程中突然就碰上了凡尔赛(Versailles)。一天的工作就这样结束了。

往下我需要一个比“出现”更新鲜的词语。我要一个能传达意象的活跃动词。没有一个寻常的同义词对路。最终我想起了“冒头”——一个简单得令人发笑的词。但这个词再恰当不过了:它所描述的图景是一个东西时不时冒出水面。这样就只剩一个抉择了。什么样的稍微偏远一点的旅游景点,对于签约去廷巴克图的我们六个人是寻常之事?我最终选定三个地方——卢克索、马拉喀什和清迈——在50年代我第一次去的时候,这些地方相当具有异国风情。而眼下,这些地方已今非昔比了。航空旅行时代使它们像伦敦和巴黎一样家喻户晓。

写这句话总共用了近一个小时,但我丝毫没有抱怨。相反,见此句入其位给了我很大的愉悦感。在写作中,抉择无大小,都值得花大量时间做好。一个句子出来,在文中起到恰到好处的作用,就足以奖励你一丝不苟的劳作,对此你和读者都会心领神会。

注意在开头的结尾处有一个星号。(也可以空一行。)这个星号是一个指示标。它向读者宜布,你以某种方式组织了文章,一个新的阶段即将展开——也许是时间顺序的改变,比如倒叙,或者题材的变化,或者重点与语气的变化。在此,经过一个高度紧凑的开头之后,星号可以使作者深呼一口气,再重新开始,这时可以用更悠闲的节奏继续叙述:


要去廷巴克图,首先需要从纽约乘飞机到位于象牙海岸的首都阿比让,再从那里乘机到其北部邻居马里的首都巴马科。不像青翠的象牙海岸,马里很干燥,其南半部主要是由尼日尔河滋养,而其北半部则完全是沙漠。廷巴克图的确就是旅行者往北穿越撒哈拉沙漠之前的最后一站,或者旅行者往南去的第一站——在几周的灼热和干渴之后,人们所渴望的地平线上的一个点。

由我们六个人组成的旅行团中,没人对马里有多少了解,或者对那里有什么期待——我们的想法都关注在我们的集结地,即盐商队到达之地:廷巴克图,而没有关注我们到达那里需要跨越的这个国家。出乎预料的是,我们立即被此地吸引。马里沉浸在色彩之中:漂亮的人们身着图案迷人的服饰,市场上有鲜亮的水果和蔬莱,孩子们露出奇迹般的笑容。作为一个赤贫的国家,马里人口众多。树木成行的的巴马科市以其能量与信心使我们感到兴奋。

第二天我们起了个大早,乘面包车行驶了十个钟头——这辆车曾有过好日子,但也好不了太多——终于到了圣城杰内。杰内是中世纪位于尼日尔河上的一个贸易与伊斯兰学术中心,早于廷巴克图,并在荣耀程度上与其争锋。如今要到杰内去只能乘小渡船。我们在难以言状的公路上颠簸,希望在天黑前赶到,巨大的黏土制清真寺尖塔与角楼看上去像远处的沙漠城堡,似乎在不断后退,逗弄着我们。当我们最终到达那里,清真寺看起来仍像沙漠城堡——好像孩子们在沙滩上建造的一座别致堡垒。从建筑学角度看(后来我得知),它属于苏丹风格。这么多年来,沙滩上的孩子们一直在以苏丹风格建造堡垒。黄昏时分徜徉在杰内古老的广场上是旅途中的一次高潮。

之后两天同样丰富多彩。其中一天是进入多贡人的居住区,后来返回时又从那里出来。多贡人生活在高原上,外人不易进入,他们以泛灵论文化和宇宙观为人类学家所重视,同时也以面具与雕塑为艺术品收藏家所珍视。我们在那里花了几个钟头在村子四处爬上爬下,观看了面具舞蹈。我们瞥见这个社会的时间太短暂,它远非这么简单。第二天是在莫普提度过的,那是尼日尔河上一座充满活力的城镇,我们都大为惊喜,只是离开得太早。但我们在廷巴克图约定了日子,而且还包了一架飞机到那里去。

显然关于马里还有很多要说,把它们挤进这四段话远远不够——有许多学术书籍论述过多贡文化以及尼日尔河流域的居民。但这篇文章写的不是马里,而是追寻骆驼商队。因此,针对该文的大范围必须确定。我的抉择是尽快穿过马里——以最少的语句解释我们所取的路径,以及我们所停留之处的重要性。

在这样的时刻,我问了自己一个非常有用的问题:“这篇文章真正要讲的是什么?”(而不仅仅是“这篇文章讲的是什么?”)你对克服了种种困难所收集到的材料自然格外珍惜,但假如那不是你决定要讲述的中心,这些材料就不能成为你收进文章的理由。这就要求有一种近乎受虐狂的自律性。对于丧失这许多材料的唯一安慰就是你其实并没有完全丧失它们;它们存留在你的文章中,看不见摸不着,但读者可以感知得到。读者总是能感觉到你对于自己题材的了解多于付诸笔端的内容。

回到“但我们在廷巴克图约定了日子”:

我去旅行社咨询的时候,最让我担心的是日期的精确性问题。我问旅行社的负责人,她如何能确定盐商队会在12月2日到达。牵引骆驼的游牧者在我的印象中可不是按照时间表行事之人。我对充满活力的骆驼和旅行社颇为乐观,我妻子则不以为然,她确信我们在廷巴克图会被告知盐商队来过又走了,或者更可能的是,什么时候来根本就没信儿。旅行社的人笑话了我的问题。

“我们同商队保持着密切的接触,”她说。“我们派侦查员进入沙漠。如果他们报告商队要晚几天到达,我们会调整你们在马里的日程安排。”这对我来讲挺有道理——乐观精神能让一切都有道理——现在我上了飞机,那架飞机比林白\footnote{查尔斯·林白(Charles Lindbergh,1902-1974):第一个完成独自飞越大西洋的美国飞行家。当时的飞机比较简陋狭窄。}的飞机大不了多少。我们往北飞向廷巴克图,所飞越之地极为贫瘠,往下看不到任何有人居住的迹象。而与此同时,成百头骆驼驮着巨大的盐块向南移动过来见我。甚至就在此时,部落酋长们正在思忖如何在他们沙漠上的帐篷里接待我。

前面两段都包含了一点幽默的笔触——开点儿小玩笑。这些努力还是为了自娱自乐,但同时也是有意要保持第一人称的叙述角度。游记和幽默作品的最古老特点之一就是叙事者永远是轻信者。适当运用这一特点,让自己上当受骗——或者做彻头彻尾的蠢货——会给读者带来优越感这种巨大的快感。

我们的驾驶员驾驶飞机在廷巴克图上空转了一圈,让我们俯瞰整个城市,我们远途跋涉就是要看这座城市。整座城市四处散落着大片泥土建筑,看起来像是被长久废弃,就像影片《万世流芳》结尾处的新津德尔要塞一样死寂,当然电影中下面无人存活。撒哈拉在不断向前侵占,横跨中部非洲造成称为萨赫勒的干旱地带,并早已推过廷巴克图,使其成为沙漠上的孤岛。我因恐惧而感到震颤,我可不想被撂在这块被遗弃之地。

这一段提到电影《万世流芳》,是为了引发读者对于所描述事件的联想。这样就通过好莱坞使得廷巴克图颇具传奇色彩。通过唤起人们对新津德尔要塞命运的想象——布赖恩·唐莱扮演那个施虐狂法国军团司令,他将士兵的尸体重新放在堡垒里立起来各就各位——我显露出自己对这一电影题材的钟爱,同时也与兴趣相投的影迷们建立起联系。我所追求的是共鸣,这种共鸣可以起到强烈的情感效果,而作家单靠自己是无法完成的。

两个词——“震颤”(tremor)和“遗弃”(forsaken)——花了我一点儿时间才找到。我在《罗格同义词词典》中找到“forsaken”这个词,我肯定自己以前从未用过这个词。能在同义词中见到这个词,我喜出望外。这个词是耶稣基督最后所用的词语之一(这就关系到共鸣的问题),它表达了几乎不能再多的孤独与被遗弃感。

当地导游在机场接我们,他是图阿雷格人,叫穆罕默德·阿里。对于旅行迷来说,他看上去叫人放心——假如有任何人称得上拥有撒哈拉的这个地区,那就是图阿雷格人。他们是桀骜不驯的柏柏尔族裔,既不屈服于阿拉伯人,也不肯向后来的法国殖民者低头。法国人横扫北非,但后来退出沙漠地区,使那里成为了独立领地。穆罕默德·阿里身着图阿雷格人传统的蓝袍,他有一张黝黑、聪慧的面孔,棱角分明,有某些阿拉伯人的相貌特征,身手矫健,显示出他性格的一部分。后来得知,他十几岁时曾随父亲踏上前往麦加的哈吉朝圣之路(许多图阿雷格人后来都皈依伊斯兰教),并且在阿拉伯半岛和埃及呆了七年,学习英语、法语和阿拉伯语。图阿雷格人有自己的语言,叫塔马奇克语,字母写起来很复杂。

穆罕默德·阿里说他得先带我们去廷巴克图的警察局检查护照。这种面谈我在电影里看多了,因 此也就不会感到不适。我们坐在像是地牢的房间里,接受两名全副武装的警察的询问。在不远处的一间牢房里,我们可以看见有一个大人和小孩在睡觉。这时我又有一刹那的闪念——这次是电影《四根羽毛》中的英国士兵长期被囚禁在乌姆杜尔曼的情景。当我们重新出来,那种压抑感还迟迟没有散去。穆罕默德·阿里带我们走过荒凉的城区,例行公事般地向我们介绍了几处“景点”:大清真寺、市场、三座有纪念标牌的破烂房子,表明莱宁、加利耶还有德国探险家海因里希·巴斯曾在此居住。我们没有见到任何其他游客。

同样,这个《四根羽毛》的典故就像影片《万世流芳》一样,会让所有知道这部电影的人胆战心惊地认出其出处。这部电影来源于一次真实的战役——基奇纳的军队溯尼罗河而上,远征去报复马迪军队曾击败了戈登将军——这样的事实给予这句话一种恐怖的锋芒。显然在撒哈拉前哨,阿拉伯的正义远非仁慈。

此处的星号标记出气氛的变化。它其实告诉读者:“廷巴克图就描述到此。现在我们将进人故事的正题:找骆驼商队。”在一篇长而复杂的文章中,如此划分不但有助于读者跟随你的线路图,同时也可以去除写作过程中的某些焦虑,使得你能够将材料劈成可处理的分块,每次处理一块。这样整个任务就会显得不那么可怕,惊恐也会被赶走。

在阿扎雷旅店,我们看起来像是唯一的客人,我问穆罕默德·阿里在廷巴克图有多少游客准备迎接盐商队。

“六个人,”他说,“你们六个人。”

“可是……”我心里有点儿什么让我说不下去。我换了一个方式。“我不明白‘阿扎雷’这个词是什么意思。为什么叫阿扎雷盐商队?”

“那是法国人用过的词儿,”他说,“当时他们组织了商队,每年都把所有的驼队召集起来完成驮盐的旅程,时间大约在12月初。”

“那他们现在怎么办?”几个人异口同声地问。

“是这样,当马里获得独立的时候,他们决定让商人们在任何他们需要的时候将盐商队带进廷巴克图。”

马里于1960年获得独立。结果我们是在廷巴克图等待一桩已经有27年未曾进行的事件。

这最后一句话像一颗小小的炸弹掉进故事里。但这已经是不证自明了,只需用事实说话,无须评论,因此我没有加任何感叹号来提醒读者这是不同寻常的时刻。那样做会毁了读者自我发现的乐趣。相信你自己的材料。

我妻子对此以及其他一些事情并不感到惊讶。我们冷静地看待这个信息:旅游老手坚信他们能通过某种途径找到骆驼商队。我们的主要反应是说得此事很逗乐,广告中的真实性标准被如此厚颜无耻地漠视。穆罕默德·阿里对于旅游手册中所提供的虚夸许诺一无所知。他只知道自己受雇来带我们去看盐商队,并告诉我们第二天早上去寻找驼队,而且会在撒哈拉过一个晚上。他说12月初通常是商队开始到达的时间。但他并没有说任何有关酋长帐篷之事。


这里比较仔细选择的单词有:“标准”(canons)、“厚颜无耻”(brazenly)、“虚夸”(gaudy)、“提供”(tendered)。这些词生动而准确,但并不冗长、花哨。最可喜的是,这些词都是读者恐怕始料未及的,因而也受欢迎。酋长帐篷那句话指的是旅游手册中的一个词语,是另一个小笑话。段落结尾处这些“关键语”推动读者进入下一段,并使他们保持良好的精神状态。

早上,无垠的天际似乎传来理性的声音,我妻子说除非我们乘两辆车去,否则她不会进入撒哈拉。这样,我高兴地看到有两辆陆虎吉普车在旅店外等我们。有一个男孩正在用自行车打气筒给其中一辆车的车胎打气。我们当中的四个人挤进一辆陆虎的后座,穆罕默德·阿里坐在前座,紧靠司机。第二辆陆虎带着另外两名游客和两个被描述为“学徒”的男孩。没人说他们在学什么徒。

这里有另一个令人惊叹的事实,无须任何修饰——给轮胎打气——然后在结尾处开一个小玩笑。

我们径直开出去,便进入撒哈拉沙漠地带。这里的沙漠像无边无际的棕黄色地毯,没有任何道路的痕迹。离这里最近的大城镇就是阿尔及尔。正在此时,我强烈地感觉到在天边,有一个声音说,“真是疯狂。你为什么要这么干?”但是我自己知道为何。我是在追寻的征途上,这样就可以回溯我第一次邂逅那些由英国“沙漠怪人”撰写的书籍中的时代——这些独行者包括查尔斯·道蒂、理查德·伯顿爵士、T.E.劳伦斯、威尔弗雷德·塞西格等,他们生活在贝多因人之中。我总是好奇,他们过的是一种什么样的苦行生活?它有何种魔力使那些英国人如此着迷?

这里给人以更多的联想。提到道蒂和他的同胞可以提醒读者,这里的沙漠也有其书面文献,而且毫不逊色于其电影文献。这又增加了我所要承载的一项感情包袱,读者有权去了解这些。

下面段落中的第一句话紧接着上一段的问题,以此结束前一段:

现在我开始去了解。我们在沙漠上行驶,穆罕默德·阿里偶尔给司机打个手势:往右一点儿,往左一点儿。我们问他怎么知道要往哪儿走。他说他可以靠沙丘辨别出路来。可沙丘看起来全一样。我们又问需要坐多久车才能看见盐商队。穆罕默德·阿里说他希望不超过三四个小时。我们继续行驶。仅凭我肉眼凡胎,前面几乎什么实物也看不出来。但过了一会儿,这几乎是乌有之地却变成了实物——整个沙漠成为一个实体。我竭力想把这一事实汇入我的新陈代谢系统之中。它诱惑我进入一种接受状态,我完全忘却了我们为何身处此地。


突然间,司机向左来了个急转弯,停了下来。“骆驼,”他说道。我使劲睁大自己习惯于都市景观的眼睛,但什么也没看见。后来远处的景物才开始清晰起来:一个有四十匹骆驼的商队以雄伟壮观 的节奏向廷巴克图开来,骆驼商队这样的情形已经有上千年的历史,它们从北部的陶代尼走了二十天把盐驮过来。我们开到离商队一百码处——不能再近,穆罕默德·阿里解释说,因为骆驼是很敏感的动物,很容易被任何“陌生”的东西惊吓。(我们显然是陌生客。)他说骆驼总是在深夜被带进廷巴克图卸盐,那时全城空寂无人。这就是“胜利进入”。

那景象真是惊人,比有组织地进入更富有戏剧性。商队的茕茕孑立是每一个曾经穿越撒哈拉的商队所独有的。骆驼彼此间勾连在一起,似乎步凋一致,像“火箭女郎”\footnote{Rockettes,美国火箭女郎舞蹈团,成立于1925年,以舞蹈动作精准整齐闻名。}波浪式的节奏一样精准无比。每一头骆驼驮着两大板盐,一边绑一板。盐板看起来像被弄脏了的白色大理石。盐板(后来我在廷巴克图市场测量过)3.5英尺长,1.5英尺高,4/3英寸厚——大致是一头骆驼能驮的最大尺寸和重量。我们坐在沙地上观看商队,直到最后一头骆驼消失在沙丘后。

现在的语调落为直接叙述——一句陈述句接着另一句。唯一难以抉择之处涉及“茕茕孑立”(aloneness)这个词儿,因为它不是我所惯用的词儿——太“诗情画意”。但最终我还是确信没有其他词儿可以起到相同的效果,便不情愿地将其保留下来。

此时已经是正午,太阳酷热。我们爬回陆虎继续向沙漠深处行驶,直到穆罕默德·阿里找到一棵树,所投下的阴影正好够我们五个纽约客和一个马里兰来的寡妇容身。我们在那儿一直呆到4点,在野外吃午餐,眼睁睁瞪着映得一切都发淡的景色,打盹。阴影随着太阳移动,我们也跟着移动。两位司机整个午休期间都在修补汽车,甚至似乎是在拆卸其中一辆陆虎的引擎。一个游牧人不知从何而来,停下来问我们是否有奎宁。另一个游牧人也不知从何而来,停留片刻,聊了一会儿。后来我们看见两名男子穿过沙漠朝我们走来,在他们后面……难道这就是我们第一次目睹的海市蜃楼吗?是另一个盐商队,长长的有五十头骆驼,在天空中勾画出轮廓来。上帝知道他们从多远就看见了我们,那两名男子离开商队前来拜访。其中一位是老者,大笑着过来。他们同穆罕默德·阿里坐下来,打听廷巴克图的最新消息。

这里最难的一句话是描述修补陆虎的司机。我想让这句话同其他句子一样简洁,但里面要有一点点惊讶——带一点挖苦与幽默。除此之外,我在此的目的就是尽量简洁地讲述余下的故事:

四个小时就这样不知不觉地过去了,就好像我们滑入了一个不同的时区,撒哈拉时区,而在下午接近傍晚,太阳的热度开始减退,我们回到陆虎里,惊讶地发现那辆车还能开,于是便开始横越撒哈拉,驶向穆罕默德·阿里所称的我们的“营地”。在我的想象中,那宣称自己为营地的地方,即使没有酋长的帐篷,至少也得有一顶帐篷。当我们终于停下来后,发现那营地与我们一整天驶过的其他地方惊人地相似。不过那里的确有一棵小树。有一些贝多因妇女蹲在树下——一身黑色打扮,脸用面纱遮挡着——穆罕默德·阿里将我们安置在她们旁边的沙地上。


妇女们一见到我们就往后缩——几个外国白人突然被卸在她们中间。她们彼此紧紧地靠在一起,看起来就像一排带状物。显然穆罕默德·阿里碰巧为自己的游客发现了“地方色彩”,看见后就停了下来,之后的事儿就全靠我们自己了。我们只能坐在那里,竭力表现出友好的姿态。但我们还是清楚地意识到自己是入侵者,而且我们的表情恐怕同我们的内心一样不舒服。我们在那儿坐了好一会儿,这时那排黑色带状物才慢慢地分开,变成四位妇女、三个孩子和两个赤身裸体的婴儿。穆罕默德·阿里离开去了什么地方,似乎不想与那些贝多因人有任何关系;也许作为图阿雷格人,他把那些人视作沙漠贱民。


反倒是那些贝多因人的善意让我们放松下来。其中一位妇女撩低面纱,露出电影明星般的微笑——洁白的牙齿和黑亮的眼睛显示在美丽的脸庞上——她在随身携带的物品中翻找着,拽出一个毯子和一个草垫,拿过来让我们坐在上面。我记得所有相关的书籍都说在沙漠上没有入侵者这回事儿,任何人的出现都在预料之中。之后不久,两名贝多因男子从沙漠上过来,同家人汇合,我们才看出其家庭成员包括两个男人,每个男人有两个妻子,还有他们各自的孩子。年长一点的丈夫脸庞刚毅俊朗,他见到两位妻子,温柔地拍拍她们的头,像是祝福,然后在离我不远处坐下。其中一位妇女给他端去晚饭——一碗小米饭。他马上将那碗饭送给我。我婉言谢绝了,但他的举止我不会忘记。我们一同坐着,相互为伴,默不作声,他吃着饭。这时孩子们也凑过来认识我们。太阳落下,圆月升起,朗照在撒哈拉沙漠上。

与此同时,我们的司机在两辆陆虎车旁铺开毯子,用沙漠柴火生起火。我们在自己的毯子上重新组合了一番,望着星星出现在沙漠的上空,吃了点儿鸡肉之类的晚饭,准备就寝。卫生间设施都是临时的——各自解决。我们被提醒过撒哈拉的晚上很冷,已经随身带了毛衣。我穿上毛衣,在毯子里蜷缩起来,沙地的硬实感稍稍减轻了一些,很快就在莫大的寂静之中睡着了。一个小时后,我被同样莫大的喧闹声吵醒——那家人牵来自己的山羊和骆驼准备过夜。之后一切复归于寂静。

早晨我发现毯子旁边的沙地上有爪印。穆罕默德·阿里说有一只豺曾来打扫过我们昨晚的剩饭——我想起来,晚饭时吃的鸡一定剩了不少。可我什么也没听见。我忙于做梦,梦见自己是沙漠枭雄阿拉伯的劳伦斯。

[结尾]

写文章的一个关键抉择是在何处结尾。故事本身常常会告诉你要在何处结尾。而这篇文章的结尾并不是我原来头脑中计划的。因为我们此次旅行的目标是要找到一支盐商队,因此我设想一定要看完历史久远的盐交易的一整个循环:描述我们是如何回到廷巴克图,看见盐在市场上卸下、买卖。但我越接近后面,就越不想写。结尾部分赫然耸立在眼前,要结束反倒成了苦差,对我和读者都无乐趣可言。

突然,我想起来,自己对于本次旅行的实际情形的描述并没有任何限定。我无须重建所有一切。我所叙述的真正高潮并不一定非得是看见盐商队,而是要找到生活在撒哈拉的人们横亘古今的好客之情。我们遇到一个游牧家庭,他们几乎一无所有,却主动请我们共享晚饭,这在我生活中,很少有什么时刻能与之相比。也没有什么其他时刻能如此生动地体现出我来到沙漠所要寻找的,以及所有那些英国人所撰写的——生活在天边所具有的高尚情操。

当你从自己的材料中领悟到这样的道理——当你的故事告诉你,无论后来发生了什么,一切足以就此结束——马上寻找门口出来。我出来得很快,只短暂停留,以保证叙述的统一性丝毫无损:保证开始这次旅行的作者兼导游自始至终都是同一个人。其中嬉戏般地提及劳伦斯确保了第一人称的叙述口吻,包裹进众多联想,使这次旅程有了一个圆满的结局。意识到我可以就此打住是一种绝好的感觉,不只因为我的劳作结束了——字谜猜出来了——还因为文章的结尾感觉对头。这个抉择是正确的。

作为附言,我还想提出最后一种抉择。这关乎非虚构作者为自己创造运气的需要。我常用于激励自己不断前行的一句话是“登上飞机”。我一生中两次最刻骨铭心的时刻,其结果都来自于登上飞机,这与我准备撰写《米切尔与鲁夫》一书有关。首先我与音乐人威利·鲁夫和德怀克·米切尔一同去了上海,他们在上海音乐学院向中国听众介绍了爵士乐。一年后我又同鲁夫去了威尼斯,听他晚上在圣·马克教堂用法国圆号演奏格里高利圣歌,当时没有其他人在那里研究这种激发了威尼斯音乐学派灵感的声学特点。在以上两次机会中,鲁夫对于主办方是否允许他演奏也并没有把握,我当时决定前往也可能白费了我的时间和金钱。但是我登上了飞机,而且那两篇原载于《纽约客》的长文也许是我的两篇最佳作品。我登上了飞机前往廷巴克图,去寻找骆驼商队,同样也是碰运气,不知能否实现;我登上了飞机前往布雷登顿,采访春训,不知道在那里会受到欢迎还是回绝。我的书《写而学》的诞生是因为一个陌生人的电话。电话提出了一个非常有意思的教育问题,于是我便登上飞机前往明尼苏达,去进一步了解情况。

登上飞机将我带到全美国、全世界不同寻常的事件之中,至今我仍旧如此。这并非表明我去机场之时不诚惶诚恐,其实我总是如此——这是写作中必要的部分。(略带诚惶诚恐的心态能给写作一种棱角。)但当我回家之时,总是满载而归。

作为非虚构作家,你必须登上飞机。如果有题材使你感兴趣,要抓住不放,假使在另一个镇、另一个州或者另一个国家,都要去。题材不会主动来找你。

先决定你自己想做什么,然后下定决心去做,最后去完成它。