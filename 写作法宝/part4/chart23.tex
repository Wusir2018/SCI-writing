\chapter{写家史及回忆录}
我所知道的最悲哀的一句话是“我真希望问过母亲那件事”。或者父亲,或者祖母,或者祖父。每一位父母都知道,我们的孩子对我们令人惊叹的人生并不像我们那样感到惊叹。只有当他们有了自己的孩子——第一次感到自己不断增长的年龄所带来的刺痛——才突然想更多地了解自己家族的传统以及所有积累的轶事和传说。“我父亲常常讲述的那些美国来的故事到底是什么?”“我母亲在中西部长大的那个农场到底在哪里?”

作者是记忆的保管员,这就是本章的内容:如何给自己的人生和自己所出生的家庭留下某种记录。这种记录可以有多种形式。它可以是正式的回忆录——那种文学建构式的审慎行为。或者也可以是一种非正式的家族史,写出来告诉自己的儿女和孙子孙女们有关他们出生的家庭情况。它可以是口述历史,摘自用录音机录下来的父母或祖父母的讲述,他们因为太老或病得太厉害,不能写了。或者可以是任何你想要的东西:某种历史与回忆的混合体。无论是什么,它都是一种重要的写作。记忆随着主人的故去而消逝,这太常见了,而时间总是不够用,这也同样太常使我们感到惊诧。

我父亲是个商人,他并没有文学抱负,却在晚年写了两部家史。这对于一个没有多少自娱天赋之人是再好不过的差事了。在公园大道高高在上的公寓里,父亲坐在自己钟爱的绿色皮制扶手椅中,写下来他自己这边家族的历史——津瑟家族与沙尔曼家族——回溯到19世纪的德国。后来他又写了由他祖父于1849年创建于西59街的家族虫胶清漆生意历史。他用铅笔写在黄色的标准拍纸簿上,从不停下来修改——当时或后来都是如此。他对任何让他必须重新审视或慢下来的事业都没有耐心。在高尔夫球场,他会边向球走,边估量球局,随手从袋子里拎出球棒,一靠近球就抬起球棒击球,几乎都不停步。

父亲写完两部家史,让人打好字,油印好,然后用塑料封皮装订好。他亲自签名,给三个女儿一人一份,三位女婿一人一份,还有我,我妻子,还有15个孙子孙女,其中几个还不识字。我喜欢这些人各自都有一份的做法,这样做是认定他们每个人都是家族故事中平等的一员。这些孙子孙女中到底有几个花时间看了这些家族史,我不知道。但我肯定一定有人看了,我愿意设想这15份家族史现在都收藏在从缅因到加利福尼亚各处的房子里,等待下一代人的青睐。

父亲的所作所为使我深深感到,这就是家史的样板,它并不渴求更多的东西,要将其出版的想法不会出现在他的头脑中。有许多令人信服的理由说明写作与出版无关。写作是一种强大的探索机制,其中的一种自足之处是学会同自己的人生故事相处。另一种是闯过人生中最艰难困苦的难关——迷茫、悲痛、病困、痴迷、失望、失败——找到理解和安慰。

父亲的两部家史的影响在我身上稳步增长。开始我觉得自己对这两部家史并没有表现出应有的慷慨,也许我对于父亲能够轻而易举地完成在我看来很难的写作过程并不以为然。但几年后,我却发现自己在细读家史,以便让自己想起久已失去的亲戚,或者久已失去的有关纽约地理的事实,而且每次阅读,我都越发羡慕这样的写作。

其中最要紧的一个是语气问题。我父亲不是作家,他从不担心如何找到自己的“风格”。他只是按照自己说话的方式写,而现在每当我读到他的语句,都会听见他的个性与幽默、他的习语与用法,许多都是20世纪初他大学岁月的回响。我也听见他的诚实。他对血缘关系并不多愁善感,而是将某个叔伯简洁地评价为“二流货”,或对某个表兄评价为“从未取得什么重大成绩”,我对此报以会心的微笑。

当你写自己的家史时,记住这一点。不要刻意当“作家”。现在我意识到,比起我来,我父亲是一位更自然的作家,我总是瞎折腾,吹毛求疵。就做你自己,这样读者会追随你。假如你刻意要写作,那么读者就会跳船逃跑。你的产品就是你自己。回忆录和个人历史的关键环节在于如何处理好你与你记忆中的体验与情感的关系。


父亲在其家史中并没有避开童年时代最大的创伤:父母婚姻的突然终结,当时他和弟弟鲁道夫还是小孩子。他们的母亲是自立更生的德国移民H.B.沙尔曼的女儿。沙尔曼十几岁就随1849年的淘金者乘大篷车去了加利福尼亚,结果在路上同母亲和姐妹走散。他们的母亲弗里达·沙尔曼继承了父亲强烈的自信与抱负,而当她与德裔美国朋友圈子里前途无量的小伙儿威廉·津瑟结婚时,把津瑟看做是自己文化志向上的知己。他们常常在晚上出没于音乐会、歌剧院,还举办音乐沙龙。但后来证明这位前途无量的丈夫显然并没有这类渴望。家对他来说就是晚餐后靠在椅子上睡觉的地方。

可想而知,他的懒散对于年轻的弗里达·沙尔曼是多么痛苦的警醒。我熟知已是老太婆的她还不停地赶着去卡内基音乐厅,在钢琴上弹奏贝多芬和布拉姆斯,去欧洲旅行学外语,督促我父亲、我的姐妹和我不断提高文化修养。她要实现自己在婚姻中破碎的梦想的愿望从未动摇过。但她却有着德国式的强烈嗜好,喜欢斥责人。她81岁去世,当时身边无人,她的朋友都被训斥跑了。

多年前我曾写过祖母,那是为《五个童年》一书写的回忆录。我在其中描述了自己孩提时代所熟悉的祖母,我夸赞她充沛的精力,但同时也指出她在我们生活中所造成的困难。书出版后,我母亲却为婆婆说话,而其实祖母让母亲的生活很不容易。“你奶奶其实是很腼腆的,”她说,“她很想让别人喜欢自己。”也许是吧,但实情却是处在我母亲的印象与我的之间。可祖母对我来说就是如此。那就是我所记得的实情,因此我就这么写。

之所以提到这一点,是因为回忆录作者经常问的一个问题是:我应该以孩子时候的视角写,还是以现在作为成人的视角写?我认为,最有力的回忆录是那些能够保存好记忆中时间与地点的统一性的作品,比如拉塞尔·贝克的《长大》,或者V.S.普里切特的《门前的出租车》,或者吉尔·克尔·康韦的《库伦来时路》,这些作品都回顾了作为儿童或青少年在为生活而抗争的成人世界中的情形。

但是如果你选择另一条路———以老年时期更睿智的视角,去写自己年轻的岁月——这样的回忆录会有自己独特的路数。一个好例子是《年轻时代的诗人》,在其中艾琳·辛普森回忆了她与第一任丈夫约翰·贝里曼的早期生活,还有他那些著名的自我毁灭型的诗友们,包括罗伯特·洛厄尔和德尔莫尔·施瓦兹。她当时作为新娘还太年轻,不理解这些诗人的守护神为何。当她已是老太婆,在回忆录中重访那段岁月时,她早已成为作家,而且还是执业的心理分析诊疗师,于是她便运用心理诊疗知识,创造出一幅不可多得的美国主要诗歌流派人物肖像。但它们是两种不同类型的写作。选其一即可。

我父亲的家史向我讲述了有关他母亲婚姻的细节,而我在写自己的回忆录时并不知晓。现在我知道了这些事实,可以理解是祖母的失望使她变成后来的自己。假如我今日重写家史,会在记述中增加她一生所奋力探求的日耳曼式“狂飙突进”\footnote{storms and stresses,指20世纪60年代晚期至80年代早期德国文学与音乐领域的变革,是文艺形式从古典主义向浪漫主义过渡的时期,代表人物有歌德和席勒。在此指充满激情的个性。}。(我母亲的家族来自新英格兰的扬基人——诺尔顿家族和乔伊斯家族——其家族成员生活太太平平,没有情感上的大起大落。)我也将在记述中增加父亲一生的遗憾,那就是位于他一生中心的巨大黑洞。在父亲的两部家史中都没怎么提及祖父,更没有原谅;所有的同情都给了遭受委屈的年轻离婚女人,以及她一生的坚忍不拔。

但父亲身上某些最具吸引力的品质——风度、幽默、敏捷、蓝眼睛中最蓝的光泽——一定是来自津瑟那一头,而不是来自沉思默想、棕色眼睛的沙尔曼家族。我总是感觉被剥夺了对失踪的祖父有更多了解的权利。每当我向父亲问起有关祖父的事情,他就岔开,无可奉告。当你写自己的家史时,要做一个记录天使,记录下你的后代可能想知道的一切。

这使我想起另一个回忆录作者常问的问题:如何对待我所写之人的隐私呢?我应该去掉可能冒犯或伤害自己亲人的内容吗?我的姐妹们会怎么想?

不要事先就担忧这个问题。你的第一项任务就是按照自己的记忆,将故事写下来——现在就写。不要回头看都有什么亲人在关注你。说自己想说的,畅所欲言,开诚布公,完成任务。然后再考虑隐私问题。如果你只为自己家人写家史,那也就没有法律或道德上的义务将其给外人看。但如果你心里有更广泛的读者——想要邮给朋友看或者出版一本书——你可能要给亲人看其中提到他们的页码。这是起码的尊重,没人愿意在书中由于自己被提到而大吃一惊。这也给了他们时间让你去除某些段落——你可能同意也可能不同意这么做。

最后,这是你写的故事——你是完成所有任务的人。假如你的姐妹对你的回忆录不能苟同,她可以写自己的回忆录,她的与你的也同样有效,无人可以垄断彼此共享的过去。有一些亲人希望你并没有说过某些你已经说了的事情,特别是当你暴露了这些不太可爱的家庭成员的特性时。但我相信,在更深层次上,多数家庭都想记录下每一位家庭成员为成为一家人所做出的种种努力,无论这种努力有多少瑕疵,而且他们会祝福你,感谢你承担起这项任务。但前提是你开诚布公,而无非分之想。

非分之想为何?让我带你回到90年代回忆录热的年月。在那个年代之前,回忆录作者对自己难以启齿的经历和想法都罩上面纱,社会对某些礼仪仍有共识。之后脱口秀擅自登台,羞耻之心被甩出窗外。突然间,没有什么记忆中的片段被认为是太卑劣的,也没有什么家庭被认为是太不正常的,一切都通过有线电视、杂志和书籍被快速传播出去以挑逗大众。结果是出现了雪崩般大量的回忆录,而写这些东西比精神疗法好不了多少。作者们用此形式沉溺于自我暴露和自我怜悯,来痛击任何对他们不公之人。写作不再入时,而抱怨盛行。

但如今已无人记得那些书籍,读者不会与抱怨产生共鸣。不要用回忆录来发泄过去的伤痛、清算旧账,请到别处去排解那些气愤。90年代出版的回忆录中我们仍记得的,是那些充满爱和原谅的书籍,比如玛丽·卡尔的《说谎者俱乐部》、弗兰克·迈考特的《安吉拉的骨灰》、托拜厄斯·沃尔夫的《这个男孩的生活》,以及皮特·哈米尔的《醉酒人生》。虽然他们所描述的童年是痛苦的,但作家们对自己年轻的自我同对自己的长辈们一样严苛。他们想让我们知道,我们并非受害者。我们的家人避免不了犯错误,我们无怨无悔地生存下来,继续走自己的人生。写回忆录对他们来说已变成一种自我愈合的行为。

写回忆录对你同样可以是一种愈合。如果你真诚地以自己的爱心与你生活中所遇之人的爱心来交流和写作,那么无论他们曾给你造成多大的痛苦,或者你给他们造成了多大的痛苦,读者都会与你的感情旅程产生共鸣。

现在来探讨较难的部分:如何组织这团棘手的东西。多数人在开始写回忆录时,都为该放多少进去束手无策。放入什么?剔除什么?从哪儿开始?在哪儿结束?如何组织故事?成千上万个过往的片段呈现在他们眼前,这对他们如何从中强行组织某种顺序提出了挑战。由于这种焦虑,许多回忆录会写到一半便搁置多年,甚至干脆就不写了。

有什么解决办法呢?

你必须做出一系列递减的抉择。例如,在一部家史中,其中一个大抉择就是只写这个家族的一支。家族是非常复杂的组织,特别是当你回溯几代人的历史时更是如此。可以决定写你母亲这边,或者你父亲这边,但不要同时写两边。可以以后再回到另一边,为其另立一个写作计划。

切记你是回忆录的主人——导游。你必须为自己要讲的故事找到一种叙述轨道,绝不能失控。这意味着在你的回忆录中要去除许多不必出现的人,如兄弟姐妹。

在我的回忆录写作课上,有一位女士想写她在密歇根长大时住过的一座房子。她母亲已去世,房子也卖了,她和父亲以及十位兄弟姐妹打算在房子那里会面,处置里面的家具。她认为这项写作任务有助于她理解自己在天主教大家庭中的童年。我同意她的计划——这是一个完美的回忆录框架——于是我问她打算如何进行。

她说她计划从采访父亲和兄弟姐妹开始,弄清他们对于那座房子的记忆。我问她要写的故事是否是他们的故事。她说不是他们的,是她自己的。如果是这样,我说,采访所有这些兄弟姐妹几乎完全是浪费时间和精力。到了此时,她才开始关注所要讲述内容的合适结构,准备好心态来面对这座房子以及对它的记忆。我为她节省了几百个小时的时间,不必再采访和转写那些不相干的内容并将其插入自己的回忆录中。这是你的故事。你只需要采访家庭成员中对家事具有特别看法的人,或者那些能够以事例开解你解决不了的谜团的人。

这儿有来自另一门课的一个例子。

有一位叫海伦·布拉特的年轻犹太女士,急于撰写有关她父亲作为大屠杀幸存者的经历。她父亲14岁时从波兰自己家的村子里逃脱——是逃出来的少数犹太人之一——千辛万苦逃到了意大利,又来到新奥尔良,最后到了纽约。现在他已经80岁,他女儿想请他一起回访那个波兰村子,这样她就能够了解他的早期生活,写他的经历。但老人恳求着拒绝了。他身体太弱,而且过去对他太痛苦了。

于是2004年她独自去了那里。她在那里记录、照相,还同村子里的人交谈。但她还是找不到足够的事实,来让自己对父亲的经历做出恰如其分的描述,这使她非常苦闷。她绝望的心情笼罩着整个班级。

有好一会儿,我都想不出什么来教导她。最后,我说,“这不是你父亲的故事。”

她看了我一眼,我至今还记得她的眼神,因为她明白了我说的意思。

“这是你的故事,”我告诉她。我指出无人能掌握充分的事实来重建她父亲的早期生活,就连大屠杀专家也做不到。在欧洲,犹太人的过去有太多都被毁掉了。“如果你写自己对于父永过去的探索,”我说,“你也就描述出他的生平和他的传统。”

我看见她如释重负。她向我们微笑了一下,那表情我们从未见过,她说她要马上开始写这个故事。

课程结束了,没人交来作业。我给她打电话,她说还在写,需要更多的时间。后来有一天,一份24页的文稿寄了过来。文稿的名称是“回家”,描述了海伦·布拉特去普莱斯纳的朝圣之旅,那是波兰东南一个地图上都没有的小乡镇。“整整65年,”她写道,“我是这个小镇自1939年以来迎来的布拉特家成员中的第一位。”渐渐地,她让镇上的人认识了自己,并发现父亲的许多亲戚——祖父母、叔叔婶婶——仍被那里的人记得。有一位老人说,“你真像你奶奶海伦,”这时她有“一种胜过一切的安全和平和的感觉”。

下面是她故事的结尾:

我回家后,父亲和我一起度过了整整三天。他仔细观看了我拍摄的四个小时录像的每一分钟,就好像那是一部杰作。他想听我路途中的每一个细节:我遇见了谁、我去了哪里、我看见了什么、我喜欢吃那里的什么食物、不喜欢吃什么、我被招待得怎么样。我向他保证自己受到了张开双臂的欢迎。虽然我仍没有家族成员的照片来告诉我他们都长得什么样,但如今我在头脑中已经有了他们性格的图画。我在那里被完全陌生的人如此善待,这一事实反映出我的祖父母在居民中所赢得的尊重。我交给父亲好几箱他老朋友的信件和礼物:波兰伏特加酒、地图、带相框的照片,以及普莱斯纳的绘图。

随着我的讲述,父亲看起来像一个幸福的孩子,等待着打开自己的生日礼物。他眼中的忧伤也消失了,看起来兴高采烈,忘平所以。当他在我的录像上看见自家的宅地,我预计他会哭泣,他的确哭了,但那是欢乐的眼泪。他似乎非常自豪,于是我问他,“爸爸,你这么自豪地在看什么?是你的房子吗?”他回答说,“不,是你!你成了我的眼睛、耳朵和腿了。谢谢你完成了这次旅程。它使我感到就好像自己亲自去了那里。”

我最后有关递减的忠告可以用这个词来概括:关注小处。不要在你的过去,或者在你家族的过去四处搜寻,一定要找到自己认为是“重要”而值得包括在回忆录中的片段。寻找独立成章、至今仍在记忆中栩栩如生的小事。如果你仍记得这些事,那是因为它们包含了普遍的真谛,读者能从自己的生活中认出它们来。

这一忠告结果成了我重要的一课。这一课是我通过2004年撰写一本叫《写你的人生》一书得来的。此书是我自己人生的回忆录,但在写作过程中,我也停下来解释了我所做出的一些缩减和组织材料的抉择。我从不觉得自己的回忆录必须包括发生在我身上的所有重要事情,而这却是老人们坐下来总结自己人生旅途时所面临的共同诱惑。我的回忆录中有许多章节是有关一些细小片段的。这些片段客观地讲并不“重要”,但它们对我自己却很重要。正是因为它们对我重要,才能与读者产生情感上的共鸣,触及对他们也重要的一种普遍真谛。

回忆录中有一章是关于一种机械棒球游戏机的。这个游戏我曾与孩提时代的朋友查理·威利斯玩过成千上万个小时。此章开头解释说我于1983年在《纽约时报》上写过一篇文章,描述青少年如何着迷于那种游戏。我说我母亲在我入伍期间,一定是把我的游戏机给扔掉了。“但在朦胧的记忆中,我看见了‘狼獾’这个词儿。‘狼獾’对于我,就如同‘玫瑰花蕾’对于公民凯恩\footnote{出自美国电影《公民凯恩》,其中提到“玫瑰花蕾”这一名称,它对电影中人物凯恩以及导演兼编剧奥逊·威尔斯(Orson Wells,1915-1985)都具有特别而神秘的意义。}一样——几乎是一种不可复得的模糊印迹。我提到的这款游戏机,也许有人能在阁楼、地下室或车库里找到它,如果那样,我会马上乘下一班飞机到那里去——查理·威利斯也会这么做。”


没过几天我就收到来信,寄信人是曾经拥有这款玩具的人,他们记得自己也曾与儿时的伙伴不停地玩这款游戏。最后一封邮戳地是阿肯色州的布恩维尔,我简直不敢相信邮寄人的地址竟然是:“狼獾玩具公司”。信的邮寄人是威廉·W·莱伦,销售副总经理。“我们已于1950年停止生产‘奖旗赢家’游戏机,”他说,“但我在公司的陈列馆里四处找了找,发现我们还有一副。假如你碰巧在附近,我倒愿意同你玩几轮。”

我从来也没去过布恩维尔,但1999年比尔·莱伦退休到了康涅狄格,有一天他给我打了个电话。他说他从狼獾公司买下了那最后一副“奖旗赢家”游戏机,他想知道我是否还有兴趣玩。几天后他就来到了我在纽约的办公室,打开了我60多年未见的游戏机。

那家伙可真漂亮,我盯着它发亮的绿色金属内场,手指尖仍然可以感到球棒的触感,和从前一样,把缠紧弹簧的球棒向后拉,等待回击投球。我还能感觉到那写着“快”、“慢”的按钮,一边一个,可以用不同的速度投球。比尔和我将游戏机拎起来放在小块地毯上,开始玩起来——两个七十多岁的老头儿面对面跪着,每半局站起来换一次场地。外面太阳已经落下,莱克星顿大道上空天色已晚,但是我们谁也没注意到。

这是一个非常特别的写作题材,可能没有几个人拥有一副机械棒球游戏机,但每个人都有自己儿时最喜欢的玩具,或者游戏,或者娃娃。我就有这样一个玩具,而且在我人生的后半段,这个玩具又被带回给我,这就不由得与读者产生联系,他们也会愿意再握一握自己儿时喜爱的玩具,或是游戏,或是娃娃。他们所认同的并非棒球游戏,而是这一游戏的理念——一种普遍的理念。假如写回忆录时担心你的故事不够宏大,不能让别人感兴趣,记住这一点:在你记忆中挥之不去的小事能引起共鸣。要相信这些小事。

《写你的人生》中另外一章写的是“二战”期间服兵役的事。同我这一代人中的多数男人一样,在回忆中,那场战争是我人生中一个关键的经历。但在回忆录中,我并没有写战争本身。我只是讲述了一段我们的部队在卡萨布兰卡登陆后,行军穿越北非的故事。我和战友们登上一列火车,这列火车由叫做“四十一八”的破旧木车厢连成,如此称谓是因为法国人在“一战”期间用这样一节车厢运送四十个人或者八匹马。法文“QUARANTE HOMMES OU HUIT CHEVAUX”(四十个人或八匹马)的钢印仍在上面。

有六天的时间,我都坐在车门敞开的车厢里,我的双脚掠过摩洛哥、阿尔及利亚、突尼斯。这是我最不舒服的一次火车旅行——同时也是最棒的。我不能相信自己是在北非。我是一名备受东北部盎格鲁-撒克逊裔白人新教徒爱护的子孙,在我的成长和教育中,没人提到过阿拉伯人。现在突然间,我处于一个一切都新鲜的土地上——每一种景象、声音以及气味。我将在这片充满异国风情的土地上度过的八个月是一种浪漫情怀的开始,而且从未冷却。这段时光将把我塑造成一个非洲和亚洲以及其他遥远文化国度的终身旅行者,而且将永远改变我对世界的看法。

记住:你最大的故事常与其重要性有关,而非题材——不是你在某个场合做了什么,而是那个场合如何影响了你,如何将你塑造成后来的你。

至于如何将你的回忆录组织起来,我的最终告诫还是:关注小处。将人生的经历分成可控的片段来处理。不要开始就预见成品——发誓要建构宏伟大厦。这只能让你焦虑。

下面是我的建议。

周一早上坐到书桌前,写下在你记忆中仍然栩栩如生的事情。不必太长,三五页即可,但要有开头和结尾。把记下的片段放在文件夹里,继续日常生活。周二早上,重复这么做。周二所记的片段不必与周一的相关联。记下记忆所唤起的任何内容,一旦开始写作,你的潜意识便开始呈现你的过去。

这样保持两个月,或三个月,或六个月。写自己的“回忆录”,开始时不要没耐心——在写之前,它只是一种想法,需要耐心才能完成。然后有一天,从夹子里拿出所有的记录,把它们摊到地板上。(地板常常是作者最好的朋友。)通读一遍,看这些记录都向你讲述了什么,有什么条理浮现出来。它们能告诉你回忆录要讲什么以及不讲什么。它们能告诉你什么首要、什么次要,什么有意思、什么没意思,什么富有情感,什么重要,什么不同寻常,什么滑稽,什么值得追寻和扩展。你开始瞥见自己的叙事结构以及要走的路。

接下来,你所要做的一切就是把各个片段组织起来。