\chapter{愉悦、恐惧与信心}
在孩提时代,我从未想过长大后要以写作为生,甚至当作家。我只是想为报社工作,我想加人的报纸是《纽约先驱论坛报》。我每天早上读这份报纸,喜爱它所传达的愉悦感。每一位为之工作的人——编辑、作者、摄影师、化妆师——都心情愉快。报上的文章常常有特别的典雅、人性化以及幽默的气息——作者与编辑们喜欢将自己的某种才气传达给读者。我觉得他们就是在为我办报。做其中的一名编辑或作者是我对终极美国梦的想法。


这一梦想成真是在我“二战”回国之后,那时我在《纽约先驱论坛报》的队伍中找到了一席之地。我有一个信念,那就是,愉悦的感觉对于作者或者出版社都是无价之宝,我至今仍与当初将这一观念灌输给我的人们同室工作。最棒的记者以满腔热情与充沛精力写作,最棒的评论家与专栏作家,如维吉尔·汤普森、瑞德·史密斯,以雄辩与个人见解中快乐的信心写作。在“拆分页”上——也就是报纸的第二版第一页,当时报纸只有两版——美国最受尊重的权威沃尔特·李普曼的政论专栏就登载在H.T.韦伯斯特的单幅漫画之上。韦伯斯特是漫画系列《胆小鬼》的创作者,也是美国的名人。我喜欢在同一页呈现出两种风格迥异的作品这种逍遥自在的大气。没人想到要将韦伯斯特挤到漫画栏目中。此二人都是巨匠,都是同一个等式的一部分。

在这些快乐的人之中,有一位叫约翰·奥赖利的本地新闻部记者。他因冷面孔的滑稽报道而受人们喜爱,他能使稀奇古怪的想法带上某种严肃的抢先报道的味道。我记得他每年都写灯蛾幼虫的文章,据说根据那种毛毛虫的棕色和黑色条纹的宽窄,可以预测这个冬天是寒冷还是温暖。每年秋天,奥赖利都会同摄影师纳特·费恩驱车到熊山公园。纳特因拍摄贝比·鲁斯告别扬基体育场而获普利策奖并因此闻名。他们俩在那里观察选中的灯蛾幼虫过马路。他的文章以模仿科学与博物馆考察的风格写就,自命不凡,神气十足。报纸总是在第一页底部的一个三栏标题之下登载这篇报道,还带有一张灯蛾幼虫的照片,但虫子的条纹并不清晰。来年春天,奥赖利会写一篇续文,告诉读者灯蛾幼虫预测得是否准确,其实即便不准,也没有人会责备他——或者责备虫子们。其用意就是要逗大家开心。

从那时起,我就将愉悦感当成我作为作家和编辑的信条。写作是很孤独的工作,我尽力使自己保持愉悦的心情。如果在写作过程中遇到什么有意思的事,我就写进去给自己找乐子。如果我自己觉得有意思,我想其他人也会觉得有意思,对我来说似乎这一天的工作就很值得。即使有一些读者并没感觉到什么乐子,我也不在乎;我知道大众中相当一部分人根本就没有幽默感——并不知道这世上还有人在想方设法给他们找乐子。

我在耶鲁教书的时候,请了幽默作家S.J.佩雷尔曼与学生交谈,其中一名学生问,“当幽默作家都需要具备什么品质?”他回答说,“需要胆大妄为,精神饱满,兴高采烈,其中最重要的是胆大妄为。”然后他说:“必须要让读者感到作者感觉良好。”这句话就像罗马焰火筒在我脑中炸响:它说明了愉悦的整个过程。然后他补充说:“即使作者并非感觉如此,也要这么做。”这句话同样带给我很大的震动,因为我知道佩雷尔曼的人生所遭遇的苦难超过一般的失意与艰辛。但他每天都到打字机旁让字符跳起舞来、他怎么会不感觉良好呢?他是使出了浑身解数。

作家在写作之时必须使自己充满活力,其程度不亚于演员、舞者、画家以及音乐家。有一些作家以强大的能量流推动着我们——诺曼·梅勒、汤姆·沃尔夫、托妮·莫里森、威廉·F·巴克利、小亨特·汤普森、大卫·福斯特·华莱士、戴夫·埃格斯——我们以为只要他们一动笔,词语就会源源流淌。没人能想到他们每天早上在打开开关之时所使出的力气。

你也需要打开开关。没人会为你打开它。

不幸的是,同样强大的负能量流——恐惧——在作怪。大多数美国人在小时候心中都有对写作的恐惧,一般都在学龄期间,而且从未完全摆脱过这种恐惧。那张白纸或者电脑空屏等着我们用美妙的词语来填充,但也能使我们束手无策,什么词语也写不出,或者写出来也是不怎么美妙的词语。假如我把写作当做一项任务——当天必须完成的任务——而不是一种愉悦,看到屏幕上出现一团糟,我也会很沮丧。我唯一的自我安慰是可以明天、后天、大后天再对付那些糟糕的句子。每一次修改,我都尽力将自己的性格投人到材料之中。

对于非虚构作家来讲,也许最大的恐惧是担心不能成功地完成任务,而虚构写作则另当别论,因为虚构作家所写的是他们自己创造的世界,而且那种难以捉摸的风格也是他们自己的发明(如托马斯·品钦和唐·德利洛),所以我们无权告诉他们,“那样做不对。”我们只能说,“那不适合我。”非虚构作家可没有这种喘息的机会。他们必须一切都有凭据:对事实、所采访的人物、事件的地点,以及该地所发牛的事件本身。而且他们还得对自己的技艺以及所有夸大其辞与杂乱无章的危险负责,比如失去读者、使读者困惑、使读者厌倦、不能使读者从头读到尾。在每一次报道不实和每一个技艺失误上,我们都可以说,“那样做不对。”

那么你如何才能击败所有这些不满与失败呢?树立信心的一个办法是写自己感兴趣并且在乎的题材。诗人艾伦·金斯伯格是我请到耶鲁与学生交流的另一位作家,他被问起是否有一个确切的时间,是他有意识地决定要当一名诗人的。金斯伯格说,“那并不是一个选择,而是一种认识。我当时二十八岁,有一份市场调查的工作。一天,我告诉自己的心理医生,说我真正想做的是辞掉工作专门写诗歌。心理医生就说,‘干吗不呢?’我说,‘美国心理分析学会会说什么呢?’他却说,‘这无界线问题。’于是我就辞了工作。”

我们绝不知道这对于市场调查领域损失会有多大。但是对于诗歌,这可是一个重要时刻。无界线问题:这对作家是好建议。你可以顺着自己的路线走。瑞德·史密斯在一次同行体育记者的葬礼上发表了这样的悼词:“死亡没什么了不起。生存是戏法。”我羡慕瑞德·史密斯的原因之一是,他以优雅与幽默写了55年的体育,不向那种认为他应该写一些“正经”事儿的压力屈服,而许多体育记者都毁在了这种压力之下。他在体育报道中找到了自己想做、喜爱做的事,而正因为他的正确选择,与许多写“正经”题材的作家相比,他说出了美国价值观中更重要的事情,而那些作家写的题材则正经到了无人看得懂的地步。

生存是戏法。写得有趣的作家一般都是那些保持了自己的趣味的人们。这几乎是当作家的全部要点。我用写作为自己的生活增添乐趣,并提供给自己后续教育。如果你写自己喜欢知道的事情,你的喜悦之情就会体现在所写的内容里。学习具有激励作用。

这并不意味着当你进入陌生领域之时会泰然自若。作为非虚构作家,你会一次又一次被抛入不同的专业领域,而且你会担心自己不具备将故事带给读者的资格。每次在开始一个新项目前我都会感到焦虑。我去布笛登顿写有关棒球的《春季训练》一书之时,就有这样的感觉。虽然我一辈子都是棒球迷,但却从未写过体育报道,从未采访过一位专业运动员。严格来讲,我并没有这方面的资历。我拿着笔记本靠近的每一个人——经理、教练、球员、裁判员、星探——都可能问,“你还写过其他与棒球有关的东西吗?”但没人问。他们没问,因为我有另一种资历:真诚。这些人很清楚,我真正想了解的是他们是如何工作的。当你进人新的领域,需要增加信心之时,记住这一点:你最好的资历就是你自身。

同时要记住,你的写作任务不一定像你想的那么狭窄。它经常会最终触及你的经历或教育中意想不到的角落,使你能以自己的力度扩展写作的内容。每减少一分对事物的陌生感都能减少你的恐惧感。

1992年的一件事就证明了这一点,那一年我接到《奥杜邦》编辑的电话,问我是否愿意为该杂志写一篇文章。我说我不行。我是第四代纽约人,我的根深扎在水泥里。“那对我,对你,或对《奥杜邦》都不合适,”我告诉编辑。我从未接受过自己觉得不适合的任务,所以很快就告诉编辑他们应该找其他人。《奥杜邦》的编辑回答说——就像好编辑应该做的那样——他确信我们可以共同做点什么。几周后,他打电话说杂志社决定找人重新写一篇有关罗杰·托里·彼得森的文章。彼得森使美国成为护鸟之国,他的《鸟类田野指南》自1934年起就一直是畅销书。我感兴趣吗?我说我对鸟类不甚了解。我能确切辨认出来的唯一一种鸟就是鸽子,它常光顾我在曼哈顿的窗台。

我需要同我所写的人感觉到一种通灵。有关彼得森的写作任务并非我所求,而是来寻我的。我所写的几乎每一篇人物评析都与我所熟知并且喜爱的作品有关:那些富有创意的人物,比如漫画家奇克·杨(《勃朗戴》)、歌词作家哈罗德·阿伦、英国演员彼得·塞勒斯、钢琴家迪克·海曼,以及英国游记作家诺曼·刘易斯。我多年来与他们相伴并以此为乐,我对他们的感激是我坐下来写作的力量来源。如果你想让自己的写作传达愉悦,那就写你敬重的人。用写作来进行诋毁和诽谤对于作家和写作对象都是毁灭性的。

然而后来发生的一件事改变了我对《奥杜邦》邀请的想法。我碰巧看了一部由美国公共广播公司录制的电视纪录片《鸟的欢庆》。该片概括了罗杰·托里·彼得森的生平和工作。那部影片极富美感,让我很想更多地了解这个人。引起我注意的是彼得森84岁时还在全力以赴地工作,每天画四个小时,而且到世界各地的鸟类栖息地拍摄鸟的照片。这着实使我感兴趣。鸟儿并非我描述的对象,我的对象是百折不挠之人。我记得彼得森住在康涅狄格的一座小镇上,离我们家夏天常去的地方不远。我开车过去就能见到他。假如气氛不对头,除了一加仑汽油,也并没有什么损失。我告诉《奥杜邦》的编辑,我想以非正式的方式尝试一下——“拜访罗杰·托里·彼得森”,而不是写什么重要的人物传略。

当然,其结果还真成了一篇重要的传略,有4000字长,因为当我一看到彼得森的工作室,我就意识到,把他当做鸟类学家——像我一直认为的那样——错失了理解他生活的核心。其实他首先是艺术家,是他作为画家的技艺,使得他将自己对于鸟的知识为千百万人所理解,并且确立了他作为作家、编辑以及自然资源保护者的威信。我询问了他早期的教师和导师——重要的美国画家如约翰·斯隆、埃德温·迪更生——以及伟大的鸟类画家詹姆斯·奥杜邦和路易斯·阿加西·富尔特斯对他的影响。于是,我写的故事成了艺术的故事、教导的故事,同时也是鸟儿的故事,涉及我的诸多兴趣。这也是一个人历经艰险的故事,80多岁的彼得森的日程安排对于50岁的人来说都是很大的压力。

这个例子对于非虚构作家的启示是:对自己的写作任务思路要宽。不要一想到《奥杜邦》的文章就非得严格限制在自然上,或者《汽车与驾驶员》的文章就非得严格限制在车上。要将你的题材的界限推广开来,看看它能将你引向何处。要将你自己的生活带入其中。只有你写了,那个故事才能成为你自己的版本。

彼得森的故事也有了我的版本,它刊登在《奥杜邦》上不久,我妻子就在家里的自动回复电话机里发现一条录音,“请问是写自然的威廉·津瑟吗?”她觉得这事儿挺逗,的确如此。但事实上,我的文章被鸟类研究团体接受为对彼得森的权威描述。我提这件事是要给所有非虚构作家以信心:要掌握技巧。如果你掌握了这一行当的工具——采访与合理组织的基本功——而且如果你将自己的常识与人情味带入写作任务中,你就可以写任何题材。

然而,人们还是很难不被专家的一技之长所吓倒。你会想,“这个人对自己的领域了如指掌,我太无知了,采访不了他。他会觉得我很笨。”他对自己的领域了如指掌的原因是,那是他的领域,而你只是一个杂家,你在努力使他的工作为公众所理解。这意味着要激发他澄清对他而言显而易见的表述,因为他以为这些表述对别人也同样显而易见。要相信自己的常识,以此来分清读者都需要知道什么,而不要怕问无知的问题。假如专家认为你无知,那是他的问题。

你的验证方法应该是:专家的第一个回答充分吗?通常都不充分。我在签约对彼得森的领域作第二次探寻时,领悟到了这一点。里佐利艺术书籍出版社的一位编辑打电话给我,说出版公司在策划休闪版的“罗杰·托里·彼得森的艺术与摄影”画册,里面有上百张彩色插图,需要8000字的文本介绍。作为撰写彼得森的新权威,我被应邀撰写介绍。我成了彼得森的新权威,这事儿还真挺逗。


我告诉编辑我给自己定过规矩,绝不对一件事写两遍。我曾尽力细心地写过《奥杜邦》的文章,不会再重来一遍。不过我欢迎编辑拿到该文的版权并把它重印在他的画册里。他同意我的建议,但条件是我再写4000字,把它们天衣无缝地编进去,来着重介绍彼得森作为艺术家和摄影家的创作方法。

这听起来挺有意思,于是我又找到彼得森,问了他一系列新问题,比我写《奥杜邦》时问他的更专业。《奥杜邦》的读者想要了解彼得森的生活,而现在我的读者想要了解这位艺术家是如何创作艺术作品的,因此我的问题直奔创作过程与技巧。我以绘画开始。

“我称自己的作品为‘混合媒介’,”彼得森告诉我,“因为我的主要目的是教育。我可以先用透明的水彩开始,然后继续用水粉画,然后上一层保护性的丙烯,然后在保护层上再用丙烯或是蜡笔,或者彩色铅笔,或者铅笔,或者墨水——什么材料都行,只要能完成我想要的。”

从以前的采访中,我得知彼得森开始时的回答并不充分。他是瑞典移民的儿子,沉默寡言,不善于夸夸其谈。我问他目前的技巧与以往他用的方法有何区别。

“目前我在观望,”他说。“我在尽量增加细节,而不丧失简洁的效果。”然后他又停下不说了。

但是他为何在人生这么晚的时刻感到需要增加更多细节?

“多年来,那么多人熟悉了我直截了当的鸟的图画,”他说,“他们现在想要看到更多的细节:羽毛的形态,或者更三维的感觉。”

我们谈论了绘画之后,转到了摄影。彼得森回忆起他曾拥有过的每一台摄鸟的照相机,从13岁起用的玻璃感光底片和皮腔普里莫9型相机,到最终用上广受称道的现代技术如自动对焦和闪光灯补光。我不是摄影师,从未听说过自动对焦和闪光灯补光,但是我只要肯暴露自己的无知,就能了解这些为什么会有帮助。自动对焦:“你只要能在取景器中看到鸟儿,相机就会完成剩下的一切。”闪光灯补光:“胶片所看到的远不如人眼。人眼可以看见阴影中的细节,而闪光灯补光则可以使相机捕捉到细节。”

但技术只是技术而已,彼得森提醒我。“许多人认为好设备会使一切变得容易,”他说。“他们错误地认为相机可以解决一切。”他自己知道这是什么意思,但是我需要知道相机为何不能解决一切问题。当我用“为什么不可以呢?”和“还有其他什么呢?”紧逼着问他时,我得到不仅一个答案,而是三个:

“作为摄影师,你将自己的眼睛和结构带入整个过程以及对温度的感受中——比如,你不会在大中午摄影,也不会在早上和晚上拍摄。你还要特别在意光线的特性,多云的天气有利于摄影。对动物的知识也大有帮助,可以预测鸟儿要做什么。你可以预测到鸟儿集中捕食这样的活动,此时鸟儿会捕食小群巡游的鱼类。集中捕食对于摄影师很重要,因为鸟类所做的一个基本事情就是进食,而这时它们对你有耐心的时间长得多。事实上鸟儿在这种情况下经常忽略你的存在。”

我们的对话就这样往下进行,专家先生同无知先生,直到我汲取到自己觉得有趣的想法。“我走到一半又回到奥杜邦的路上,”彼得森说——真有意思——“我感觉到由于环保运动所发生的变化。”他回忆说,在他小时候,每一个有弹弓的小孩都会打鸟儿,许多鸟类被打尽,或者几近被猎人灭绝。这些猎人屠杀鸟类只是为了获其羽毛,或者卖给餐馆,或者为了寻乐。好消息是,他活了足够长的时间,看到了许多鸟类在幸免于灭绝之后,又重新恢复了数量,这有赖于人们现在都在积极参与保护鸟类以及它们的栖息地。然后他说:“人类对鸟的态度也改变了鸟对人类的态度。”

真有意思。作为一个作家,我对自己说过那么多遍“真有意思”而深有感触。假如你发现自己在这么说,要注意顺从自己的嗅觉。相信自己的好奇心,将其与读者的好奇心连接起来。

彼得森说鸟对人类也改变了态度是何意呢?

“乌鸦变得更加驯服,”他说。“海鸥增加了——它们是垃圾场的清洁队。小燕鸥开始在商场顶棚筑巢;几年前在密西西比州的戈蒂埃,有几千对儿小燕鸥在欢唱河商场屋顶筑巢。嘲鸫特别喜欢呆在商场里——它们喜欢那里的植物,特别是野蔷薇,其果实大小正好方便这些鸟儿吞咽。它们还喜欢商场里熙熙攘攘的景象——它们就呆在那里,指挥交通。”

我们在彼得森的工作室里谈了几个小时。这间工作室是一个艺术与科学的小前哨——有画架、颜料、画笔、绘画、照片、地图、相机、照相设备、部落面具,还有一架架工具书和期刊——在我访问结束的时候,他送我出去,我说,“我都看全了吗?”经常的情况是,当你将铅笔放到一边之后,才在告别之时的闲聊里得到最好的材料。被采访之人在尽力使自己的生平对一个陌生人有益于观瞻之后,会放松下来,想起某些需补充的要点。

当我问他我是否都看全了的时候,彼得森说,“你想看看我的鸟类收藏吗?”我说当然想看。他带我顺着外面的楼梯来到储藏室的门前,打开锁,领我进入一间满是柜子和抽屉的地下室——这些装备是储藏科学用品所司空见惯的,让人回想起那些小学院里从未被现代化的博物馆。达尔文可能就用过此类抽屉。

“我在这下面有两千件标本用于研究,”他告诉我。“其中多数标本有上百年历史,但仍然有用。”他打开一个抽屉,拿出一只鸟儿,让我看标签,上面标着:橡树啄木鸟,1882年4月10日。“想想看!这只鸟儿有112岁了”他说。他打开其他一些抽屉,轻轻地举着其他几个维多利亚晚期的标本让我看——这使人联想到格罗弗·克利夫兰总统的任期。

里佐利出版社的这本画册出版于1995年,其中有令人惊叹的绘画和照片,而彼得森在一年后就去世了。在观察到世界上9000种鸟类中“略多于4500种”之后,他的追求终于结束了。我是否享受为这两篇文章所花的时间呢?我说不准。彼得森的性格太阴郁,采访他的乐趣不大。但我享受完成一项超出正常经历的复杂任务的满足感。我还自己捕获了一只鸟儿,而当我将彼得森收拾好,放入抽屉,与其他收集起来的标本放在一起时,我在想:真有意思。