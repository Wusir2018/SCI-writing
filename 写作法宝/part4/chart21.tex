\chapter{终稿的严酷性}
我在曼哈顿的新学院大学教了许多年的写作课,学员在课上经常告诉我,他们有一篇文章的构思特别适合《纽约时报》,或者《体育画报》,或者其他某个杂志。这是我最不想听到的。他们已经可以想象出自己的文章付印了:标题、总体结构、照片,还有他们最渴望的署名。现在他们所必须做的就只剩下写了。

这种对于终稿的偏执给作者造成许多问题,使他们的注意力游离于所有必须要在前期做出的决定,包括文章的形式、语气以及内容。这是很典型的美国人的问题。我们的文化崇拜获胜的结果:联赛冠军、考试高分。教练雇来是要打赢比赛的,教师的好坏在于能否将学生送进最好的大学。在这个过程中不那么耀眼的收获——学问、智慧、成长、信心、学习面对挫败——没有给予同样的尊重,因为这些打不了分。

对作者来讲,赢分才是筹码。专业作家在写作研讨会上最常被问到的问题是,“我怎么才能卖出我的作品?” 这是我唯一不想回答的问题,部分原因是我不够格——我不清楚当今市场上编辑们在寻找什么,虽然我希望自己知道。但是主要原因是,我对于教作者如何销售没有兴趣。我想教他们如何写。如果程序对路,产品会自然产出,销售也会跟进。

这就是我在新学院大学所开的写作课程的前提。社会科学研究新学院\footnote{新学院的全称。}由开明的学者们创建于1919年,自那时起,该学院一直是这个城市最活跃的学习中心之一。我愿意在这里教课,因为我对其历史作用始终抱有认同感:提供信息来帮助目标明确的成人拓展自己的生活。我喜欢乘地铁上夜课,成为那些人当中的一员,匆匆忙忙地走进走出、上课下课。

我选择“人物与地点”作为课程的名称,因为它们一起构成说明文的核心。通过集中探讨和练习以下两个要点,我认为可以教给非虚构作家需要知道的多数内容:如何定位作者在某处所写的东西,如何使生活在那个地方的人谈论该地的特色——或者曾经具有的特色。

但是我还想做一项试验。作为编辑和教师,我发现非虚构写作中最少被传授和最易被忽略的技能是如何组织一篇长文:如何完成这个拼字游戏。写作课总是教学生如何写好一个清楚的陈述句。但是如果叫他们尝试更宽泛的东西——一篇文章或者一本书——结果他们写的句子会像弹子一样撒得满地都是。每一位面对一篇长文的编辑,都熟悉那种无法逆转的混乱所带来的沮丧时刻。作者眼睛盯着终点线,从不多想如何去跑这场比赛。

我想知道,是否有什么办法能将作者从对写作的最终结果的痴迷中拽开。突然间我想出一个极端的主意:我要教一门不要求写作的写作课。

在第一次课上,我的班有二十几个成年学生,之后一直如此,从二十几岁到六十几岁,多数为女性。其中有几个是郊区小报和电视台以及行业杂志的记者,但主要是做普通工作的人,他们想学习如何用写作来使生活更有意义:弄清他们此时此刻是谁,他们曾经是谁,以及他们所出生于其中的传统是什么。

我用第一节课让大家作介绍,并且解释写人物和地点的一些基本原理。最后我说:“下周我要你们来时告诉大家一个对你重要并且你想写的地方。告诉大家你为何想写这个地方,以及你想如何写。”我从来都不是那种喜欢朗读学生习作的教师,除非写得特别好。人们对于自己所写的内容过于敏感脆弱。但我猜想他们对于自己只是在思考的东西不至于太敏感。这些思考尚未转化到神圣的纸上,总是可以随时变更,或者重新组织,或者否决掉。尽管如此,我还是不知结果会如何。

一周后,第一位发言者是一位年轻女士,她说她想写在第五大道北面的教堂,该教堂最近起了一场大火。虽然教堂已经重新启用,但墙壁都熏黑了,木头也烧焦了,闻起来还有烟味儿。这位女士感觉这一切都让她很不安,她想弄清这场大火对于她这个教区的居民以及这个教堂都意味着什么。我问她计划写什么。她说她可能会采访那里的牧师,或者风琴师,或者消防队员,也许还有教堂司事,或者唱诗班指挥。

“你给了我们五篇弗朗西斯·克斯·克莱恩的好作品,”我告诉她。克莱恩是《纽约时报》的一位记者,他写本地专题,充满人文情怀,文笔流畅。“但是这些对你,对我,对这门课都不够。我要你再深入一些。我要你在你自己和这个你要写的地方之间找到某种联系。”

这位女士问我自己心中酝酿有何作品。我说我不会提出什么建议,因为开这门课的一个想法就是让大家集思广益,想出可能的解决办法。但既然她是我们的第一个实验品,我就试一下。“在接下来的几周里,你去教堂的时候,”我说,“只是坐在那里,想一想这场大火。三到四个礼拜日之后,教堂就会告诉你那场大火意味着什么了。”然后我说:“上帝会让教堂告诉你那场大火意味着什么。”

教室里学生们露出些许惊讶的表情。美国人一提到宗教就神经兮兮的。但学生明白我是认真的,而且从那一刻起,他们也认真对待我的想法。每个星期他们都邀请我们其他人进入他们的生活,向我们讲述某个使他们感兴趣或触动他们情感的地方,并想方设法决定如何写这个地方。我会用每次课的前半段时间教授技巧并且阅读一些非虚构作家的段落,这些作家已经解决了学生们挣扎着要解决的问题。另一半时间是我们的实验:一个分析作者组织材料问题的解剖台。

最大、最要紧的问题是紧凑性:如何从大量混杂的经历、情感记忆中提炼出一种连贯的叙事形式。“我想写一篇有关艾奥瓦州小城镇消失的文章,”以为女士告诉我。她向我们描述她小时候曾在祖父母的农场生活过,自那时起,中西部的生活结构遭到了很大的侵蚀。这是一个有关美国的好题材,作为社会史很有价值。但是没人能就艾奥瓦州小城镇消失写出一篇像样的文章,这样会太笼统,缺乏人情味。作者只能写艾奥瓦州的一个小城镇,借此来述说更大的故事,甚至仅在这一个城镇中,她还得进一步缩小故事的范围:一家商店,或者一户人家,或者一位农场主。我们在课上论及不同的方法,这样作者逐渐将组织故事的思路缩小到更人性化的程度上。

我惊叹于学生们的摸索常常会在突然间揭示出恰当的路径,在座的人都觉得切实可行。一位男士说他想写一篇有关他曾经生活过的城镇的文章,想尝试一种可行的方法:“我可以写X。”可是这个X甚至对他自己来说都并不有趣,缺乏特点,Y和Z也是如此,P、Q、R也是如此。于是作者继续在自己的生活中打捞碎片,后来几乎是偶然地,他撞到了M,一段忘却已久的记忆,似乎不重要但无可争议地真实,其中的一件事囊括了他最初想要写这个城镇的全部理由。“这就是你要写的故事,”班里的好几个人逗这么说,而且的确如此。这门课给了学生充分的时间来找到自己的题目。

从直觉入手来释放题材,正是我进入学生的新陈代谢中想要他们做的。我告诉他们,如果他们真的写出文章来,我会很高兴去读,即使在课程结束后寄给我都可以,但那并非是我的主要兴趣。我主要对过程感兴趣,而非产品。起初这使他们感到不安。这里是美国——大家所要求的不只是要写得有效,还要有结果,这也是国民的权利。好几个人悄悄地来到我跟前,几乎是偷偷摸摸地,就好像向我泄漏的是什么不光彩的秘密。他们说,“你知道,这是我所上的唯一不以市场为导向的写作课。”这句话叫人沮丧。但过了一段时间,学生们发现不必赶截止日期解放了他们。截止日期这只恶魔在他们本科和研究生在学期间一直都贪得无厌(“星期五交论文”)。他们放松下来,开始喜欢考虑用不同的方法到达想要去的地方。其中一些方法有效,而另一些则不好用。拥有失败的权利具有解放性的力量。

偶尔我也向写作坊的中小学教师描述此课程。我倒不特别期待他们觉得这与他们所教的年龄段 的学生有多大关联性——青少年的记忆和情感不如成年人丰富。但他们总是催促我提供更多的细节。我问他们为何这么感兴趣,他们回答说,“你给了我们一个新的时间表。”他们这么回答的意思是,传统的学期短论文的作业方式可能是教师们遵循得太想当然、太久的传统。他们开始反思,留作文作业时要给学生们更多的空间,并且要依照不同的标准来批改。

我的课堂教学法——找到一个特别的地方来描写它——只是一种教学手段。我真正的目的是给作者以新的心态,他们可以在之后可能尝试的写作项目中应用,为写作旅程留出足够的时间。对于我的一个30多岁的律师学生来说,这个旅程花了他三年时间。1996年的一天,他打电话给我,说他好不容易完成了那个题材的写作,终于可以交稿了,而那个写作计划中的组织问题是他早在1993年班上提出来的。愿意看看吗?

结果到来的是一份350页的手稿。我得承认我心里的某个部分不想收到一份350页的手稿。但我心里更大的一个部分很欣慰,我当初启动的程序终于沿着自己的路径有了结果。同时我也很好奇,想看看这位律师是如何解决他的问题的,因为这些问题我记得很清楚。


他告诉我们他想写的地方是康涅狄格州郊区的一座城镇,他在那里长大,而他的主题是足球。他儿时在校队踢球,与其他五个同他一样热爱足球的男孩结下了亲密的友谊,他想写那种亲密的体验,和他对足球提供给他这种体验的感激之情。我说这是一个好题材:一份回忆录。

这种关系非常紧密,律师继续说,这六位在东北部已成为中年专业人士的男人仍保持联系——它们继续定期见面——他还像写这段经历,以及他对这种长久友谊的感激。我说这也是好题材:一篇好随笔。

但还有更多。这位律师还想写当今足球的状况。他记忆中的这项体育的肌理已被社会变迁所侵蚀。他说其中丧失了许多可贵之处,球员不再在更衣室换球衣,他们在家里就穿上队服,然后开车去球场,又开车回家。律师的想法是回到他的母校作志愿教练,然后写今昔的反差。这又是一个好题材:一篇调查报告。

我喜欢听律师的故事。我被带入一个我一无所知的世界,他对这一世界的感情很吸引人。但我也知道他会逼疯自己,我这么告诉他。他不能将所有的故事都挤在一个小屋顶下,他只能选一个结构完整的故事。结果,他还是将所有的故事挤到一个屋顶下了,但这幢房子必须大幅扩充,而这个工作花了他三年。

在我读了他称为《我们人生之秋》的手稿之后,他问我稿件是否够递交出版社的水平。我告诉他还不够,稿件还要修改一遍。也许之前他只是不想再费那个力气。他想了一会儿,然后说,既然已经走了这么远,他愿意再试一把。

“但即使根本出版不了,”他说,“我也很高兴完成了这个稿件。我真是迫不及待地想告诉你这对我曾经多么重要——把足球在我人生中意味着什么写下来多么有收获。”

我想起两个最终的词。一个是追求,另一个是意图。

追求是讲故事的一种最古老的主题,一种我们从不厌倦听到的信念行为。回顾起来,我注意到我班上的许多学生,一旦布置给他们思考一个对他们重要的地点的作业,他们都会用这个作业来追求比地点本身更深的东西:一个意义、一个想法、过去的一件小事。结果是,对于这群陌生人来讲,班上总有一股温暖的活力。(有些班甚至还会聚会。)学生发起的每一次追求都在我们自己的追寻和渴望中找到回响。寓意:每一次你用追求或朝圣的形式讲故事,你就会在这游戏中领先。读者在读你的作品时,会带着自己的联想为你做一部分你的工作。



意图是我们在写作中希望成就的,称为作者的灵魂。我们可以通过写作来肯定、来庆贺,或者来揭露、来诋毁;选择权是我们自己的。诋毁长久以来一直是新闻报道的一种模式,奖赏窥探者和受雇记者以及隐私的冒犯者。但没人能迫使我们写我们不想写的东西。我们因此得以保持自己的意图。非虚构作家经常忘记,并没有人要求他们默认媚俗的作品,为那些有自己日程安排的杂志编辑承担写垃圾文章的任务——去卖商业产品。

写作与性格相关联。如果你的价值观可靠,你的文章也就可靠。这都始于意图。想清楚你想做什么,想如何做,以人性与人格来寻找自己的方法,完成文章的写作。在这之后,你的文章才有卖点。