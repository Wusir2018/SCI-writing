\chapter{风格}
作者努力写出简洁的句子,但总有那肿胀的怪物埋伏在暗处捣乱。有关写作初期的告诫就谈到此。

“但是,”你也许会说,“如果我去除你认为是赘语的所有部分,如果我将每一句话都剥到只剩骨头,那我还能剩下什么?”这个问题问得合理。简洁推到极致也可能会指向一种不比“狄克喜欢珍妮”和“看斯波特跑”这样的句子更复杂多少的风格。

我将首先从木匠手艺的层面回答这个问题。然后涉及更大的问题,即作者是谁,如何保护作者的身份。

很少有人意识到自己写得多么糟糕。没人向他们指出有多少过剩或含糊的词语溜进自己的写作风格中,以及它们如何阻碍了作者想说的话。假如你给我一篇八页的文章,我叫你删减到四页,你会叫喊说那做不到。然后你回家去做,结果会好得多。之后便是难的部分:删减到三页。

要点是你必须先将自己的写作拆开,然后再搭建起来。你必须知道基本工具是什么,以及其预设的作用是什么。再拿那个木匠手艺比喻为列,首先必须锯好木头,然后钉钉子。之后,你如果有雅兴,再切修棱角、添加别致的顶部。但千万不要忘记你是在练习一项基于一定原则之下的技能。如果钉子不牢,房子就会坍塌。如果动词不牢、句子结构摇晃,句子就会分崩离析。

我得承认,有一些非虚构作家,如汤姆·沃尔夫和诺曼·梅勒,他们建构起了不起的房屋。但这些作家花了许多年来学习技艺,最终才搭建起神奇的塔楼和空中花园,使我们这些从未梦想过如此装饰的人惊叹不已。他们知道自己在做什么。没人一夜间就成为汤姆·沃尔夫,就连他本人也不能。

那么首先要学会钉钉子。如果你所建房屋既结实又好用,就该对具简洁之力感到满意。

但你没有耐心去成就一种“风格”——去修饰简朴的词语,这样读者就会把你认作一种特殊的人。你只会寻觅花哨的比喻、华而不实的形容词,就好像“风格”是什么能在风格店买到的东西,是可以用装饰漆那鲜亮的颜色装饰词语的东西。(装饰漆是装修工用的彩色漆。)风格店并不存在,风格对于写作中的人是有机体,就像头发是他自己身体的一部分,或者假如他秃顶,那么就是他身体缺乏的那部分。试图添加风格就像加假发在秃头上。瞧第一眼时,之前秃顶的人看起来年轻,甚至还帅气,但瞧第二眼时——看假发时人们总会瞧第二眼——那人看起来就不对劲了。这里的问题并非是他看起来没有梳理好,他梳理得很好,我们真得敬佩假发匠人的高超工艺,问题是他看起来不像自己了。

这个问题是那些特意装点自己文体的作家的通病。你丧失的是使你自己独一无二之处。读者会看出你是否在装腔作势。读者要的是,与他们交谈之人听起来是真挚的。因此,写作的一个基本准则是:做你自己。

然而,没有什么准则比这一条更难遵循。这要求作者做到两件事,而按其新陈代谢的本能来说,这些都是难以做到的。他们必须放松,必须有信心。

叫作者放松就像在检查疝气时叫人放松一样;至于信心,看,他多么僵硬地坐着,眼睛直勾勾地盯着等待他造词儿的电脑屏幕。看,他多么频繁地站起来找吃的或喝的。作者会想方设法躲避写作实践。我可以证明我在报社工作期间,作为记者,我每小时去饮水机的次数大大超过身体对水的需求。

如何拯救作者出苦海呢?很不幸,并没有拯救的办法。我只能安慰大家说并非只有你的遭遇如此。有些日子好过一些,另一些日子则难熬到让你绝望,不再想写作。大家都有过这些日子,而且还会有更多这样的日子。

当然,最好还是尽量减少难熬的日子。这就使我回到如何放松这个问题上来。

假如你就是那位坐下来准备写作的作者。你想到你写的文章必须具有一定的长度,不然就不会显得重要。你想到文章印出来有多么庄严。你想到那么多人阅读你的文章。你想到文章必须具有重重的权威性。你想到文章的风格必须炫目。怪不得你会浑身发紧:你在忙于想着自己甚至还未开始的文章写完之后所要承担的巨大责任。而你发誓要对这一任务称职,于是四处找大词,找那些如果你不是想刻意给人突出印象就根本想不到的词,并且一头栽进去。

第一段是一场灾难——整段话成了似乎来自机器的一系列笼统词语的组合。人是不会那样写的。第二段也好不了多少。但是第三段开始有点儿人情味,而到了第四段你开始听起来像自己了。你开始放松了。令人称奇的是,编辑会常常删掉文章的前三四段,甚至前几页,而始于作者听起来开始像自己的部分。前几段的问题不只是缺乏人情味和浮夸华丽,而是根本就没说什么,只是刻意地要写出一个花哨的前言。作为编辑,我总是在寻找说类似以下这样话的句子:“我永远不会忘记那一天……”这时我就想,“啊哈!真人说话啦!”

作者用第一人称写作显然是最自然的。写作是把两个人之间密切的交往付诸笔端,写作保持人情味才会顺畅。因此我敦促人们用第一人称写作,用“我”或“我们”。但大家表示抗拒。

“说我认为什么,或者我感觉什么,那个我是谁啊?”他们问。

“不说你认为什么,那么你又是谁呢?”我回答他们,“只有一个你。没有任何其他人想的和感觉的一模一样。”

“可是没人会在乎我的观点,”他们说,“那样会让我觉得太突出了。”

“如果你对他们讲述有趣的事,他们会在乎的,”我说,“而且要用自然而然的词语讲述。”

然而,要作者用“我”并不容易。他们认为必须先赢得袒露自己情感和思想的权利,不然就会显得自以为是,或者不庄重。此类恐惧也影响了学术界,因而出现了学术性的“某人/有人”的用法(“有人不能苟同于莫尔特比博士对于人类状况的观点”),或者无人称的用法(“希望费尔特教授的专著会理所当然地拥有更广泛的读者”)。我可不想见用“有人/某人”这样的人——很乏味。我要的是对自己的话题有激情的教授来向我诉说该话题为何使他着迷。

我意识到写作中有广大领域不允许用“我”。报纸不要第一人称代词“我”用于新闻报道中;许多杂志也不要它出现在文章中;商业、社会机构不要它用在大量发送到美国家庭的报告中;大学不让把“我”用在学期论文或学位论文中;还有英语教师也不赞成用第一人称代词,除非是作为书面语的“我们”(“我们在麦尔维尔对于白鲸的象征用法中看到……”)。以上那些禁用是合理的。报纸文章就应该由客观报道的新闻构成。在学生没经过一番挣扎学会从其内在优点和外在评论评价一部作品之前,教师不希望学生避重就轻地发表意见——“我觉得哈姆雷特挺愚蠢”。我也理解教师的这些想法。“我”这个词有可能变成某种自我陶醉和自我逃避的工具。

然而,我们的社会已经变得害怕袒露自己的心声。向我们发送宣传品、寻求支持的社会机构听起来惊人地相似,但所有这些机构——医院、学校、图书馆、博物馆、动物园等——当然是由那些有不同梦想和展望的男男女女创建和管理的。这些人都在哪儿?在所有那些非人称的被动语态句子中,如“已经采取行动”和“重点已经确定”,很难瞧见他们的真面目。

即使是在不允许用“我”的时候,也有可能传达一种个性化的我的意思。政治专栏作家詹姆斯·赖斯顿在专栏写作中并没用“我”,但我却很清楚他是什么样的人,对其他很多随笔作家和记者,我也可以这么说。好作家在字里行间是看得见的。如果不允许你用“我”,写作时至少要用“我”思考,或者第一稿用第一人称写,然后再把“我”去掉。这样能预热你的非人称风格。

风格与心理绑在一起,写作的心理机制根深蒂固。我们表达自己的特有方式,或由于“作者心理阻滞”而没能表达好自己的原因,部分埋藏于潜意识心理中。作者心理阻滞的种类就像作家种类一样多,我也没有捋清它们的意图。这是一本薄书,我的名字也不叫西格蒙德·弗洛伊德。

但是我也注意到避免“我”的新理由:美国人在言辞上不愿意冒任何风险。我们的上一辈领袖们直言不讳自己的立场和信仰,现今的领袖们却千方百计地巧用词语逃避这一宿命。看看他们如何在电视采访中拐弯抹角,就是不立场鲜明。我记得有一次福特总统向一行来访的商人保证他的财政政策会奏效。他说:“每月我们所看见的是不断增亮的云彩,此外别无他物。”我对此的理解是那云彩还相当黑。福特的句子含义模糊不清,等于什么也没说,却给他的选民打了镇静剂。

后来的当局者们也并没有起色。国防部长卡斯帕·温伯格在1984年评价波兰危机时说:“有继续严重关注的余地,而且形势仍然严重。严重的形势持续越长,严重关注的余地也就越多。”老布什总统被问及他有关突击步枪问题的立场时说:“有不同的群体认为可以禁止某种枪支。我不在其列。我是在那些深刻关注者之列。”

不过我的全能冠军当属70年代身为四任主要内阁成员的艾略特·理查森。很难确定从他那模棱两可的句子宝库中选哪一句,还是看这句吧:“但是呢,均衡地来讲,我认为少数族裔及妇女维权行动还是取得了一定的成果。”\footnote{原文如下: And yet, on balance, affirmative action has, I think, been a qualified success.}一句13个单词的话中就有5个词含义模糊。在现代公共话语中,我给它最空泛句子一等奖,但与其媲美的还有他分析如何消除生产线工人单调乏味的句子: “这样呢,最后,我形成一个坚定的信念,我在开始提到过:就是这个问题太新,无法最终判断。”


那就是坚定的信念吗?像摇摇晃晃来回摆动的年迈拳击手这样的领袖不能鼓舞信心——也不配鼓舞信心。作家也是如此。推销你自己,你的题材就会发挥自己的吸引力。相信自我身份,相信自己的见解。写作是一种自我行为,你得承认这一点。要全力以赴使自己不断向前。