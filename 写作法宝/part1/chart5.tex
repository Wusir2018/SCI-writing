\chapter{读者}
在你面临如何保护自我身份的问题之后,很快会出现另一个问题:“我在为谁写作?”

这是一个基本问题,也有一个基本的回答:你是在为自己写作。不要预想谁是你广大的读者群。没有那样的读者群——每一位读者都与众不同。不要猜想编辑要出版什么样的东西,或者你认为国人有情绪读什么。编辑和读者并不知道想读什么,他们是读了之后才知道的。不过,他们一直在寻找新东西。

假如你心血来潮想制造点儿幽默,不要担心读者是否能“明白”。假如你在写的过程中感到那个幽默逗人,就放进去。(幽默总能取出来,但只有你才能放进去。)你写作主要是取悦自己,如果你写起来感到一种享受,你也会给值得为之写作的读者带来快乐。假如你失去呆头呆脑的凡夫俗子们的喜爱,那你本来就不需要他们。

这似乎自相矛盾。我开始警告说读者是不耐烦的鸟儿,栖息在干扰或困顿的危险边缘。现在我说你必须为自己写作,不要为读者是否跟随而烦恼。

我谈的是两个不同的问题。一个是技能,另一个是态度。第一个是掌握确切技能的问题。第二个是你如何运用技能表达个性的问题。

在技能方面,如果失去读者是由于糟糕的写作技艺,那是不可原谅的。如果读者读了一半你的文章而打瞌睡是由于你疏忽了某些技巧性的细节,那就是你自己的错了。但在大问题上,诸如读者是否喜欢你,或者喜欢你说的,或喜欢你是怎样说的,或同意你说的,或认同你的幽默感或你对生活的展望,你一刻也无须对读者感到烦恼。你就是你,他就是他。你要么与读者合得来,要么就合不来。

这似平还是自相矛盾。你如何能小心翼翼地想着不失去读者,同时还能不在乎读者的看法呢?我向你明确说明这是两码事。

首先,努力掌握好工具。简化、修剪,全力以赴谋篇。把这想成一种机械行为,这样很快你造的句子就会变得更简洁。这一活动决不会变得像刮胡子或抹沐浴露那么机械,你总得思考运用各种各样工具的方式。但至少你的句子是建在扎实的原则基础之上的,而你失去读者的机会也会大为减小。

然后将另一个问题看做一种创造性的活动:如何表达你是谁。放松,说你自己想说的。既然风格就是你是谁,你只需要忠实于自己,这样就会发现自己的风格逐渐从日积月累的赘语和瓦砾下 冒出来,每天变得更具特色。也许这种风格需要几年才能落实为你自己的风格、你自己的声音。就像需要时间发现自我,作为一个人,使自己独具风格也需要时间,而且你的风格随着年龄的增长会发生变化。

但是不论在什么年龄,写作时都要做你自己。许多老作家仍然以他们在二三十岁时的热忱写作,显然他们的想法仍年轻;另一些老作家无边际地漫谈,不断重复自己,其风格预告他们已经成了惹人厌唠叨鬼。许多大学生写作时就像他们是枯燥乏味的30年老校友。决不要在写作中说你平常交谈时不习惯说的。假如你平常不习惯说“确实”或“再者”,也不习惯称别人为 “个体”(“他是一位出色的个体”),就请不要那样写。

一起来看看下面几位作家是如何将自己的热情和妙想付诸笔端并以此为乐,而不在乎读者足否有共鸣的。第一个节选自《母鸡 :赏析》,是怀特于1944年第二次世界大战的高峰撰写的:

鸡在城里人那儿不总享有一种荣耀的地位,虽然我注意到鸡蛋是在源源不断地产出。眼下母鸡 受宠。战争神化了她,使其成为后方的宠儿,在会议桌上受到赞美,在每个吸烟车厢里受到表扬。她那少女般的气质和奇特的习惯成为激动的农业专家们的话题,而昨天母鸡对他们来说还很陌生,毫无荣耀和诱惑性。

我个人对母鸡的感情早在1907年就开始了,自那时起,无论顺境还是逆境,我对她始终忠实有加。我们俩的关系并不容易保持。最初,我还是个孩子,住在精心规划的市郊,我得算计邻居和警察;我的鸡就像一家地下报纸那样,得严密地看护好。后来,我长大成人,住在乡下,我得算计城里的老朋友,他们大多数只把母鸡看做是滑稽剧中突然出现的喜剧性道具……他们的鄙视只会增添我对母鸡的忠诚。我做到了忠贞不渝,就像新郎对待受到自己家人公开嘲讽的刚过门的新娘一样。现在该我来面带微笑了。我听着城里人咯咯叫着高谈阔论,他们突然间拿起母鸡作为社交话题,空气中充斥着他们新发现的痴迷和知识,还有新罕布什尔红鸡与怀恩多特花边鸡的相对迷人之处。从他们异常兴奋的惊叹和赞扬声中,你会觉得那些母鸡就是昨天在纽约郊区孵出来的,而不是在遥远过去的印度丛林里。

对养母鸡的人来说,一切有关家禽的传说都激动人心,无限神奇。每年春天我都会静下心来拿出我的农场日志,以同样痴迷的面部表情,阅读有关如何准备育雏暖房的古老故事……

有人在写我根本不感兴趣的题目,然而我却十分喜爱这篇文章。我喜欢其风格的简洁美。我喜欢其韵律,喜欢其意想不到但又令人耳目一新的词语“神化”、“诱惑”、“咯咯叫”,还有那些细节,如怀恩多特花边鸡和育雏暖房。但我主要喜欢的是,这儿有一个人毫无掩饰地向我讲述他与家禽始自1907年的恋爱史。文章充满人性与温存,三段之后我对这位母鸡爱好者是何许人也就相 当了解了。

再选一位风格与怀特大相径庭的作家。此人嗜好奢华词语,并不神化简洁句。但他们在坚持个人观点不动摇和想什么说什么方面堪称兄弟。此人就是H.L门肯。他在1925年夏天报道了臭名昭著的“猴子审判案”,被审判者是一位名叫约翰·斯科普斯的年轻教师,因为他在田纳西州自己的教室里教了进化论。以下是报道的节选:

在田纳西州代顿一个炎热的夏日,他们审判了离经叛道的斯科普斯,但我是自觉自愿去到那里的,因为我很想见识一下仍在活动中的福音基督教派的情形。在共和国的大城市里,尽管虔诚的圣徒们不断努力,该教派的发展还是停滞不前,内在活力消耗殆尽。就连主日学校的学监们都偷偷地从收音机里听爵士乐,摇晃起防火的腿\footnote{fire-proof legs,宗教隐喻,指教徒需要赴汤蹈火才能通过炼狱到达天堂。};他们的学生正进入青春期,荷尔蒙剧增,也不再报名去非洲做传道工作,而是和人拥抱、亲吻起来。虽然那伙人试图处斯科普斯以极刑,但我发现甚至在代顿也有一股反律法主义的味道。村里的九座教堂在礼拜日都半空,院子里满是杂草。只有两三个当地牧师运用他们怪异的神学知识还维持着运转;其余牧师不得不为邮寄裤子生意接单,或者在附近的草莓地里干活儿;我听说还有一位做了理发匠……进村正好十二分钟,我被一位基督徒拉住,带我去喝坎伯兰牧场酒,那是一种半玉米酒、半可乐的混合酒。那酒对我就像一剂难喝的药,但我却发现代顿的先知们兴致勃勃地一饮而尽,喝完抓抓肚皮、翻翻眼珠子。他们都热衷于“创世记”,但他们的脸喝得太红,显然不属于禁酒者之列,而一旦有女孩轻盈地走过大街,他们会不约而同地把手伸向自己的领结,就好像自己应该同那些好谈情说爱的电影明星为伍。

这就是纯粹的门肯,充满电涌般的动力,同时又显得漫不经心。翻开他的书,几乎在每一页,他所述说的都一定会激怒其同胞自诩的虔诚。美国人将自己的英雄、教堂、教化之法——特别是禁酒令——都沐浴在神性之中,而这些对门肯只是一口永不干枯的虚伪之井。他向政客和总统投掷的最重磅炸药——如他所描绘的“大天使伍德罗·威尔逊\footnote{伍德罗·威尔逊(Woodrow Wilson,1856-1924):美国总统(1913-1921),推进民主改革,颁布禁酒令,经历第一次世界大战,后获诺贝尔和平奖。}”,仍烧灼纸页——至于基督教信徒和神职人员,他们绝对是以江湖骗子和蠢货的面目出现。


门肯能够摆脱20年代种种异端邪说似乎是个奇迹;在那个年代,英雄崇拜已成为美国的宗教,基要派圣经地带\footnote{the Bible Belt,美国南部、中西部基督教基要派教徒多的地带}的人们自以为是的愤怒从东到西四处渗透。他不但摆脱了那个年代,而且成为他那代人最受尊重、最有影响力的记者。他对后来非虚构作家的影响不可估量,而且至今其专题文章似乎仍清新依旧,就像是昨天才写的。

门肯大受欢迎的秘密——除了他令人炫目的美语用法之外——就是他在为自己写作,而毫不在乎读者会怎么想。人们倒不一定需要分享他的偏见,欣赏他兴高采烈、尽情的表达。但门肯从不胆小怕事或含糊其辞;他不向读者叩头或讨好任何人。当这样的作家需要勇气,但受尊重、有影响的记者正是诞生于这样的勇气。

向前移到我们自己的时代,下面是一本叫《如何在本地生存》的节选。该书的作者詹姆斯·赫恩登在其中描述了自己在加利福尼州当初中老师的经历。在美国萌芽的有关教育的所有严肃书籍中,对我来讲,赫恩登的书最好地捕捉到了教室里的实情。他的风格与众不同,但他的声音真挚。下面是该书的开头:

我还是以皮斯顿开始吧。可以把皮斯顿描绘成一个红头发、中等身材、胖胖的初二学生,他的突出特点就是固执。无须细说,马上显而易见:皮斯顿不想做的,皮斯顿不做;皮斯顿想做的,皮斯顿做。

这倒不是什么大问题。皮斯顿主要想画画,画怪物,在油印纸空白处画设计图,然后打印出来,偶尔写恐怖故事——一些孩子称他为食尸鬼——而当他不想做这些事时,就在走廊里溜达,偶尔(我们听说)检查女生的卫生间。

我们有一些小冲突。有一次我叫所有学生坐下,听我要说的话——有关他们在走廊里的行为举止。我允许他们在走廊里来回走动,但不要大声喧哗,这要靠大家的自觉(我打算这么说),我不想从其他老师那里听到此类事情。问题是首先得坐下——我坚持每个学生先坐下,然后我再说话。皮斯顿一直站着。我又重新发令。他并不理睬。我指着他说我在跟他说话。他表示他听见我的话了。随后我问他究竟为何不坐下。他说他不想坐。我说我就是要他坐下。他说那对他不起作用。我说无论如何必须坐下。他说为什么?我说因为我是这么说的。他说他不坐。我说,看着我,我要你坐下听我说我要说的。他说他在听呢,我会听但我不坐下。

唉,这就是学校有时会发生的事。作为教师的你会对一件事鬼迷心窍——我是受伤的一方,通常向学生赐予闻所未闻的自由,而他们通常是在那儿得寸进尺。喝咖啡时间回到办公室,听某人说你们班的谁和谁未经允许在走廊里闲逛,向我班里的孩子做鬼脸、打手势,当时我正在上有关埃及那一课的最重要部分——这可不是啥乐事——应该允许你说某种固执己见的话,大多数学生也会听,坐下来听你说,但偶尔总会有人跟你拧劲,拒绝屈从于没有必要的事……我们怎么会弄到这步境地?我们应该扪心自问。

在同一句中使用“ain't”和“tendentious”\footnote{原句中同时用了“ain't”(俚语)、“tendentious”(正式用语),因而故意造成文体反差。},引用而不用引号,这样的作者当然知道自己在做什么。这种看起来毫无匠心然而却充满匠心的风格,对于赫恩登的写作目的是理想的手法。这种风格避免了充斥于诸多写作中的矫揉造作,而这些作者的工作也是很有价值的;它还为更丰富的幽默和常理提供了更广阔的空间。赫恩登听起来是一位好老师,我喜欢与他为伍。但最终他还是在为自己写作:一个人的读者。

“我在为谁写作?\footnote{Who am I writing for?通常用语}”本章开始的这个问题惹恼了一些读者。他们想让我用更正式的文体“我在为何人写作?\footnote{Whom am I writing for?正式用语。}”但我说不出口。那不是我的风格。
