\chapter{赘语}
与赘语作斗争就像同杂草作斗争一样,作者总是稍稍滞后。五花八门的新赘语一晚上就冒出来,到中午就成了美国话语的一部分。看看尼克松总统的助手约翰·迪恩在水门事件电视听证会那天的创举吧。第二天美国人人都说“就在当前这个时间点上”,来代替“现在”。

看看披挂在动词后面但并无必要的介词吧。我们不再“领导”委员会。我们“领导起”委员会。我们不再“面对”问题。当我们能“空闲得出来”几分钟,我们“针对问题面对”它。也许你会说,那都是细枝末节,不值得烦扰。但这的确值得烦扰。写作的改进同我们能去除的赘语数量成正比。“空闲得出来”中的“得出来”就不应该有。仔细审查付诸笔端的每一个词,你会发现毫无目的的词语数量惊人。

举形容词“私人/个人的”为例,如:“我的一个私人朋友”、“他的个人感受”,或者“她的私人医生”等。这些只是成百上千可去除词语的典型代表。用“私人朋友”来区分“生意朋友”,结果使语言和友谊都贬值了。某人的感受就是那个人的个人感受——那就是“他的”的含义。至于私人医生,指的也就是被叫进突然病倒的女演员化妆室的男医生或女医生而已,这样该演员就不必由剧院指派的那种公事公办的医生来诊治了。有朝一日我倒愿意看见那人变成“她的医生”。医生就是医生,朋友就是朋友。其余皆赘语。

赘语即佶屈聱牙的词语,它排挤掉简明的同义词。甚至在约翰·迪恩之前,业界人士已经停止说“现在”。他们说“在当下”(“我们所有的操作员在当下都在帮助顾客”),或者说“在目前的这段时间里”,或者说“不久很快”(意思是“一会儿”)。其实这个意思都可以用“现在/马上”来表示目前的时间(“现在我可以见他了”),或者用“现今”来表示历史上的现在(“现今物价高涨”),或者简单用动词进行时(“在下雨”),而没有必要说“在目前的这段时间里我们正在经历降雨”。


“经历/感受”这个词是最糟糕的赘语之一。甚至连牙科医生都会问你是否感受到疼痛。而假如椅子上坐的是他自己的孩子,他就会说,“疼吗?”简而言之,他就会是他自己。他通过在职业角色中用浮夸的词语,使自己不但听起来更重要,而且还钝挫了事实中痛苦的一面。此类语言同样用于航空乘务员演示当机舱内用尽空气时氧气罩会如何落下。“假如飞行器一旦最终遭遇不太可能发生的紧急情况,”乘务员如此开始——其用语本身就够夺人氧气的了,大家都已经准备好了任何灾难的发生。

赘语是乏味的委婉语,贫民窟成为社会经济萧条区、垃圾清扫工成为废物处理人员、城市垃圾场成为减量单位。我想起比尔·莫尔丁的卡通,描述有两个流浪汉在货车车厢里。其中一人说,“我开始只是个简单的盲流,可现在是绝对失业了。”赘语是政治正确性的极端表现。我见过一则男孩夏令营广告,组织者的目的是提供“个别关注给那些极少数与众不同的孩子”。

赘语是公司官方语言,用以掩盖其错误。当DEC公司裁减3000个岗位时,其告示并没提及解雇,而是称之为“不情愿之措施”。在空军发射的火箭爆炸失事后,火箭被称为“提早撞击地面”。当通用汽车公司关闭一家工厂时,那是“与产量计划相关的调整”。破产的公司是处于“负现金流状态”。

赘语是五角大楼的语言,称侵略是“加强保护性反应打击”,对其庞大预算需求的解释是为了“反威慑力”。正像乔治·奥威尔在《政治与英语语言》一文中指出的那样——该文写于1964年,但在柬埔寨、越南以及伊拉克战争期间经常被引用——“政治性演讲与写作主要是为不可辩护者辩护……因而政治性语言不得不由委婉语、设问句以及纯粹的云山雾罩式的模糊语构成”。奥威尔警告说,赘语不只是恼人,而且是致命的工具。这在最近几十年间的美国军事冒险中得以证明。在乔治·布什任总统期间,伊拉克的“平民伤亡”成立“间接破坏”。

语言伪装在亚历山大·黑格将军任里根总统国务卿期间达到新高峰。在黑格之前,没人想到会说“在此成熟之机的关键时刻”来表示“现在”。他告知美国人民与恐怖主义作斗争可以用“有意义的制裁性强制手段”,中程核导弹正处于“关键性漩涡之中”。至于公众对此心存的忧虑,他的意思是“交给艾尔”,但他实际说的是:“我们必须将此推到公众关注度更低的分贝。我认为在这一领域的情况没多少学习曲线可得。”

我可以继续从各行各业引出例证——每个行业都有不断增长的行话库,它们扬起尘埃,迷住大众的眼晴。但都列出来会很烦人。在此提出来的目的是引起大家注意:赘语是写作的敌人。因此要警惕并不比短词强的长词:assistance/help(帮助),numerous/many(许多),facilitate/ease(促进),individual/man or woman(个人),remainder/rest(剩余),initial/first(首先),implement/do(实施),sufficient/enough(足够的),attempt/try(试图),referred to as/called(被称为),还有成百上千更多的词。警惕所有含糊的时髦新词:paradigm(范例)与parameter(参数),prioritize(优先考虑)与potentialize(使成为潜力)。这些都是窒息写作的杂草。能写“与什么人交谈”,就不要写“与什么人对话”。不要写“与什么人协调配合”。

同样阴险的还有人们用于解释自己打算如何解释的各种词组:“我也许可以补充”、“应该指出的是”、“使人有兴趣注意的是”。假如你也许可以补充,就补充吧。假如有什么应该指出,就指出吧。假如使人有兴趣注意,就使它有趣好了。当有人说“那会使你感兴趣吗”之时,我们对下面所说的到底是什么不都会感到疑惑不解吗?不要胀大本无须胀大之事,如:可能除外的情形是/除外,由 于这样一个事实/因为,他完全缺乏这样一种能力/他不能,直到那样一个时刻/直到,目的就是/为了⋯⋯

有什么一眼就能认出赘语的办法吗?有一个办法,我在耶鲁的学生发现其行之有效。我会在一篇文章中用括号括上任何一个无用的成分。通常情况下只有一个词被括上:动词后赘的不必要的介词(命令起来),或者同动词意思一样的副词(高兴地笑),或者描述已经很清楚的事实的形容词(高高的摩天楼)。我的括号经常括上削弱句子力度的小小修饰语(有一点儿),或者诸如“在某种意义上”之类的毫无意义的词组。有时候我的括号会括上整个句子——就是那种基本上重复前一句,或者读者无须知道,或读者自己能明白的句子。大多数初稿可以砍掉一半而不损失任何信息或作者的语气。

我括上学生的浮夸词而不划掉这些词的原因,是避免冒犯其视若神明的散文体。我要完好无损地保留他们的句子,以便他们自己分析。我对学生说,“我也可能错,但我认为去掉这个地方并不影响意思。你自己决定。读一下不带括号的部分,看是否通顺。”在学期的前几周,我发还给学生的文章都是括号。整段整段都被括上。但很快学生们就学会在心里给赘语加上括号,到了期末,他们的文章就几乎没有赘语了。现今,当初的许多学生已成为专业作家,他们对我说,“我至今还能看见你的括号——它们会跟我一辈子。”

你也可以培养同样的眼力。从文章里找出赘语,无情地修剪。对所有可以去除的部分都要心存感激。重新审校你写的每一句话。每一个单词都起新作用吗?有没有什么地方虚夸、做作或者赶时髦?你是否只是因为自己觉得写得漂亮,而对那无用的部分不能割舍?

简洁,再简洁。