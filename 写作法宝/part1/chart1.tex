\chapter{交往}
康涅狄格州的一所学校曾经举办过一次“专注于艺术日”,并问我是否愿意去谈一谈专业写作。到了那里我才发现另一位报告人也在受邀之列——布洛克\footnote{布洛克(Brock)在此为双关语,既是姓氏,也有“讨厌鬼、坏蛋”的意思。——译者注。全书脚注除特别说明,均为译者注释。}医生(我如此称呼他)。他是一位外科医生,最近开始写作,向杂志社卖了几篇故事。他要谈一谈业余写作。这样我俩搭成一组,坐下来面对一群学生、教师和家长,个个都急于了解我们光彩的工作。

布洛克医生身着鲜艳的红上衣,看起来有点儿波希米亚风格,就像作家应有的穿戴那样。第一个 问题是问他的。当作家的感觉是什么样的?

布洛克医生说写作真是其乐无穷。劳累一天从医院回家,他会直奔自己的黄色本子,将一天的紧张情绪都写掉,真是字如泉涌,容易。我却说写作不容易,并非其乐无穷。写作是一件艰难而孤独的差事,很少字如泉涌。

下一个问布洛克医生的问题是修改是否重要。他说绝对不用。“任其发展好了,”他告诫大家。发展成任何句子形式都是作者最自然的反应。我却说修改才是写作的本质所在。我指出专业作家反复修改句子,然后修改全篇。

“遇上写作不顺的时候你怎么做?”有人问布洛克医生。他说那他就停一天,等到感觉好了再写。我却说专业作家必须定一个每日的写作计划,而且必须坚持完成计划。我说写作是技能,不是艺术,由于缺乏灵感而逃避技能的人是自欺欺人。这样的人是穷途末路。

“那当你感到沮丧或不愉快时又怎么办呢?”一名学生问。“不会影响你的写作吗?”

也许会吧,布洛克医生回答道。去钓鱼,去散步。也许不会吧,我说。假如你的工作是每天写作,你就该像做其他工作一样,学会完成每天的写作。

一名学生问我们是否觉得多交往文学界的人很有用。布洛克医生说他很享受自己作为文人的新生活。他还说他自己好几篇故事被出版商和经纪人拿到作家和编辑聚会的曼哈顿饭店的午餐会上。我说专业作家是孤独的劳作者,他们很少见其他作家。

“你在作品中用象征吗?”一名学生问我。

“我尽量避免,”我回答道。我一直都抓不住小说、剧本或是电影中的微言大义,至于舞蹈和哑剧,我也从未搞清楚其中的含义。

“我太喜欢象征了!”布洛克医生惊叹道,然后又兴致盎然地描述起自己是如何将象征编织进作品中的。

一上午就这样进行着,大家都很受启发。最后布洛克医生告诉我他对我的回答非常感兴趣——他从未想过写作竟然会艰难。我告诉他我对他的回答同样也感兴趣——我也从未想过写作竟然会容易。也许我应该像他那样操手术刀才是。

至于学生们,大家都觉得我们叫他们简直不知所措。但比起我们单独讲的话,实际上我们提供给他们更宽广的写作过程的一瞥。原因是并没有任何一种“正确”的方法来做这种个性化的工作。有各式各样的作家和各式各样的方法,任何能帮助你表达的方法,对你就是正确的。有人白天写作,有人晚上写作。有人需要安静,有人打开收音机。有人用笔写,有人用电脑写,还有人通过录音写。有人第一稿一气呵成,然后再修改,也有人再三斟酌第一段之后才能写第二段。

但所有人都敏感,都紧张。他们受难以抗拒的力量驱使,将自己的某些部分付诸笔端,但他们并不是什么出来就自然地写什么。他们是坐下来从事一种文字工作,出现在纸上的自我大不如坐下来写作的自我那么放松。问题是要发现紧张之下的真正自我。

最终,任何作家要销售的产品并非所写的东西,而是自我。我经常发现自己饶有兴趣地阅读从未想过会有兴趣的话题——也许是某种科学探索。吸引我的是作家对自己领域的热忱。他是如何被吸引进这一领域的?他对此负载的情感是什么?他的生活因此是如何改变的?要了解一位曾在瓦尔登湖独处的作家,并不需要自己也去瓦尔登湖独处一年。

这是一种个人交往,它是好的非虚构写作的核心。本书将探讨基于此种交往而产生的两项最重要的特点:人文与温情。好的写作具有一种吸引读者一段接一段读下去的生气,而这并不是什么使作者“个性化”的把戏问题。这是如何运用语言创造出最大限度清晰度和力度的问题。

这些原则能教出来吗?恐怕不能。但大多数是可以学出来的。

