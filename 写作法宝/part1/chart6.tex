\chapter{词语}
有一种可以称为新闻体的写作,在任何风格的写作中,它的存在都将扼杀文章的新鲜度。这是报纸以及像《人物》之类的杂志通用的语言——一种由廉价词语、编造词语以及陈词滥调构成的混合体。这类词语广泛流行,作者想不用都难。你必须坚决抵制这些词语,否则你写出来就会像个雇佣文人。如果你不培养一种对词语的尊重和对语义细微差别着迷的兴趣,就不可能成为留有印记的作家。我们的语言拥有丰富有力而灵活的词汇。要花时间发掘,找到自己需要的词汇。

何为“新闻体”?那是用其他词性的词语临时补缀在一起的被子。形容词用作名词(“greats“大人物,“notables”名人),名词用作动词(“to host”主办),或者名词削掉一部分构成动词(“enthuse”充满热情,“emote”强烈地表现),或者名词加后缀构成动词(“beef up”强化,“put teeth into” 全身投入于)。在这个词语世界里,大人物是“famed”(著名的),他们的同事是“staffers”(职员),未来总是“upcoming”(即将到来),有人总是“firing off”(发射)外交照会或看法。串在一起如下:在美国多年来已经没有人再发照会、备忘录或电报了。著名外交官康多莉扎·赖斯邀请了外国名流来振作国务院高级职员的士气,她坐下来发射出一串串照会。这些发射出去的照会总是以义愤的态度和坐立的姿势发的。发射用的武器为何物,我从未弄清楚过。

下面是一本著名新闻杂志里的一篇文章,读起来累人的程度无可比拟:

去年二月,便衣巡警弗兰克·赛尔皮科敲开一个布鲁克林海洛因毒品贩子嫌疑人的家门。当门开了一条缝,赛尔皮科硬挤进去,当头就遭遇一颗22毫米口径手枪子弹打进脸部。他竟然活了下来,但头部仍有碎片嗡嗡作响,造成眩晕和永久性左耳失聪。同样痛苦的是,有人怀疑他被其他警察故意立为射击的靶子。原因是35岁的赛尔皮科独自一人在掀起一场四年的战争,反对习以为常、流行肆虐的腐败。他和其他同事认定那腐败充斥于纽约市警察局。他的所作所为现在正像冲击波一样传遍纽约最精英级别的……虽然调查委员会即将到来的报告的影响还有待感受,赛尔皮科没有多少希望……

即将到来的报告还有待感受,因为报告只是即将到来;至于永久性失聪,下结论还为时过早。而什么使得那些嗡嗡作响的碎片嗡嗡响呢?目前只有赛尔皮科的头会嗡嗡作响。但除了逻辑上的懒惰用法外,这篇报道读起来累人的原因是,作者失误于除了用最眼前的陈词滥调外别无他词。“硬挤进去”,“当头就遭遇”,“打进脸部”,“独自一人掀起战争”,“腐败充斥于”,“像冲击波一样传播”,“纽约精英”——这些单调乏味的词语构成了最平庸的写作。我们知道这类文章中会有什么,其中毫无给人以惊喜的惊人之语或微妙的面貌。我们落在了平庸的雇佣文人手里,而且会立即发现这一点。于是我们就不往下读了。

不要使自己陷入如此境地。避免的唯一办法是特别小心用词。假如你发觉自己在写“有人近来经受了一场病痛来袭”或者在写“某个行业一直在经受下滑”,那么问问你自己他们如何能“享受”这些\footnote{enjoy是“享受、享有”的意思,后跟正面意义的名词,前面两句的“经受”用的也是“enjoy”一词,作者用此来表示有些词常被误用}。留意其他作家选词的方式,对你自己从大量的词汇中选取的词眼光要挑剔些。写作中的比赛不是看快,而是看独创性。

养成习惯,不但要读今天作家写的,还要读前辈大师从前所写的。写作学自模仿。如果有人问我是怎么学会写作的,我会说我学写作是通过阅读那些在做我也想做之事的男男女女们的作品,同时竭力琢磨出他们是如何实现的。但要按照最佳样板来培养自己。不要以为文章登在了报纸杂志上就是好文章。马虎大意的编辑在报纸上司空见惯,这经常是由于时间紧张,而那些惯用陈词滥调的作者所面对的编辑们,看了太多这类词,已经失去了辨别能力。

还要养成使用词典的习惯。我最喜欢用的方便词典是《韦氏新世界词典》大学第二版。当然如同所有词语迷一样,我还拥有更厚的词典,在查特别词语的时候,使我受益匪浅。如果你对词义有疑问,查一查。查清词源,注意词根衍生出来的种种有趣的分支。看看该词是否有你所不知的意思。掌握词与词之间构成同义词的细微差别。哄骗、劝诱、奉承和诱骗之间都有什么区别?给自己弄一本同义词词典。

别瞧不起那本鼓鼓的词语集合库《罗格同义词词典》。人们很容易将其看做是一本滑稽可笑的书。例如,查“恶棍”,你会被与“流氓、无赖”有关的词语所淹没,只有词典编撰家才能魔法般地再现几个世纪的相关词汇,如“不公正”、“精神错乱”、“堕落”、“流氓作风”、“放荡”、“脆弱”、“罪恶滔天”、“恶名昭著”、“猥亵”、“腐败”、“邪恶”、“坏事”、“再犯”,还有“罪恶”。你会找到“恶棍”与“地痞”、“歹徒”与“犯罪分子”、“堕落者”与“坏蛋”、“流氓”与“强盗”、“骗子”与“饭桶”、“无赖”与“流氓”、“悍妇”与“荡妇”。你会找到与其搭配的形容词、副词和动词来描述犯错误者是如何犯错误的,还有交叉参照引向其他腐化与罪恶的灌木丛。有《罗格同义词词典》在身边是刺激活跃记忆的最好朋友。它节省你在大脑的纹络系统中寻找的时间,去找到就在你嘴边的词——光在嘴边可不行。《罗格同义词词典》对作者来说就好比韵书对歌词作者,能够提醒你、帮你选词,你应该充满感激地用它。假如你发现流氓(scalawag)和饭桶(scapegrace)这两个词,想知道彼此的区别,那么就去查词典。

还要记住,在你选词、串词的同时,要注意词语听起来如何。这似乎荒唐:读者是用眼晴阅读。但事实上,他们能听见自己在阅读,其程度比你所想的要大得多。因此,韵律与头韵之类的事对句子至关重要。前面一段就是一个典型的例子——它也许不是最好的,但毫无疑问是最眼前的。显然,我喜欢组合一些词语,如ruffians和riffraff、hooligans和hoodlums,比起只提供一个简单 的词语单来,读者也更喜欢我这样组合。他们不只是喜欢词语的组合,而且还欣赏为了娱乐大家所做的努力。然而,他们并不是用眼睛欣赏。他们是用内在的耳朵听词语。

E.B.怀特的《风格的要素》一书,每位作者都应该一年读一遍。怀特在其中做出了令人信服的说明。他建议对那些存留了一两个世纪的词语,尽量重新组合一下,比如托马斯·潘恩的名句“这是考验人的灵魂的时代”:

这样的时代考验人的灵魂。

生活在这个时代是多么考验人啊!

这是考验人的时代,考验的是灵魂。

这是关乎灵魂的,是考验人的时代。

潘恩的词语像诗,其他四句像燕皮粥,这就是创作过程非凡的神秘之处。好的散文作家必须是半个诗人,总是听自己所写的。怀特是我最钟爱的风格文体学家之一,因为我意识到自己是与一位注重语言节奏和音调的人在一起。我(用耳朵)喜爱他用词语构成的图案,变成一个一个句子。我尽力猜想他是如何改写句子,重新组合,使句子结束时仍留有余音的;他如何选一个词而不用另一个词,目的是追求一种情感分量。比如就像“宁静”(serene)和“寂静”()之间的区别——一个非常轻柔,另一个却令人吃惊地不安,因为有n和q这两个不寻常的音。

这种对于声音和韵律的考量应该包括在你的所有写作之中。假如你写的句子都以同样慢腾腾的步伐行进,甚至连你自己都发觉这是致命的弱点,但不知如何治愈,那就朗读出来。(我完全凭耳朵写作,在将自己的作品公布于世之前总是通读一遍。)这样你就会听见问题处在何处。看看自己能否获得某些文体上的变化,方法是通过颠倒句子结构,或替换上新颖、奇特的词,或改变句子长度,这样句子就不会听起来像是出自同一台机器。偶尔一个短句能发挥巨大的作用。它会长久萦绕于耳。

记住,词语是你所拥有的唯一工具。学会有独创性地、谨慎地用词。同时也要记住:有人在那儿倾听。